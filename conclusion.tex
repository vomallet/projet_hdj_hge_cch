% ============================================================
% CONCLUSION GÉNÉRALE
% ============================================================

\titleformat{\subsection}[runin]
  {\bfseries\color{APHPblue}}
  {}
  {0pt}
  {}

\begin{spacing}{1.25}

La consolidation des activités d’hôpital de jour digestif, interventionnel et addictologique au sein d’un plateau unique constitue un levier structurant de transformation pour le GHU AP--HP.Centre. Elle répond de manière opérationnelle aux enjeux contemporains du virage ambulatoire, dans un contexte durable de contraintes capacitaires, de tension sur les ressources humaines et d’augmentation continue des besoins spécialisés.
\\
\\
Le modèle proposé repose sur une organisation transversale, lisible et sécurisée, permettant d’absorber une activité ambulatoire complexe tout en limitant le recours à l’hospitalisation conventionnelle. La graduation des prises en charge ambulatoires, fondée sur l’intensité réelle de surveillance requise, assure une allocation efficiente des ressources médicales et paramédicales, sans compromis sur la qualité ni sur la sécurité des soins.
\\
\\
Au-delà de l’optimisation organisationnelle, le plateau HDJ mutualisé constitue un outil structurant de coordination des filières digestives et interventionnelles. Il favorise l’intégration des compétences pluridisciplinaires, la standardisation des parcours et la fluidification des interfaces entre consultations, plateaux techniques et hospitalisation complète. Cette architecture renforce la lisibilité de l’offre ambulatoire et l’attractivité médico-scientifique du site.
\\
\\
Sur le plan médico-économique, le dispositif présente un profil autosoutenable, fondé sur des hypothèses prudentes de montée en charge et un dimensionnement compatible avec les capacités physiques projetées. Il positionne le site Cochin comme un centre expert régional de l’ambulatoire digestif, en cohérence avec les orientations stratégiques de l’AP--HP, les recommandations des sociétés savantes et les standards internationaux d’organisation des parcours complexes.
\\
\\
L’ensemble de ces éléments justifie pleinement l’engagement institutionnel dans la mise en œuvre progressive de ce plateau HDJ mutualisé, qui constitue une réponse robuste, évolutive et immédiatement opérationnelle aux besoins actuels et futurs du territoire.

\end{spacing}
