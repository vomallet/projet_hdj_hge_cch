% ============================================================
% CONCLUSION GÉNÉRALE
% ============================================================

\section*{Conclusion générale}
\addcontentsline{toc}{section}{Conclusion générale}

\begin{spacing}{1.25}

La consolidation des activités d’hôpital de jour digestif, interventionnel et addictologique au sein
d’un plateau unique constitue un levier structurant d’efficience pour le GHU AP--HP.Centre.
L’unification organisationnelle permet une augmentation substantielle de la capacité de prise en
charge, une amélioration mesurable de l’accessibilité aux parcours spécialisés et une réduction
documentée des hospitalisations conventionnelles évitables, dans un contexte durable de tension
capacitaire et de ressources humaines contraintes.

La spécificité du projet repose sur une organisation graduée des prises en charge ambulatoires,
distinguant trois niveaux de complexité et d’intensité de surveillance. Cette segmentation permet
d’allouer les ressources médicales et paramédicales au plus près des besoins médicaux réels,
tout en évitant un surdimensionnement en lits hospitaliers. Les prises en charge nécessitant une
surveillance continue sont concentrées sur des lits médicalisés, tandis que les traitements
intermédiaires sont réalisés sur des fauteuils dédiés.

Les parcours d’évaluation multidisciplinaire et de coordination — notamment en hépatologie
métabolique, cirrhose compensée, addictologie et certaines prises en charge MICI — sont organisés
au sein d’un espace non allongé de type \emph{lounge}. Inspiré des standards de confort et de
fluidité des salons premium, cet espace accueille des patients autonomes ne nécessitant ni
alitement ni surveillance continue, mais requérant un temps médical et paramédical structuré,
séquentiel et à forte valeur ajoutée. Cette organisation optimise les flux, améliore l’expérience
patient et constitue un levier majeur d’efficience capacitaire.

À régime de croisière, l’activité consolidée du plateau est estimée entre \textbf{4\,700 et 5\,000
séances d’hôpital de jour par an}, pour des \textbf{recettes annuelles proches de 3\,M€}. Ce modèle
positionne le site Cochin comme un centre expert régional de l’ambulatoire digestif, en cohérence
avec les orientations stratégiques de l’AP--HP, les recommandations des sociétés savantes et les
standards internationaux d’organisation des parcours complexes.

\end{spacing}
% ============================================================
% PRINCIPES DE CAPACITÉ ET NIVEAUX DE PRISE EN CHARGE
% ============================================================

\subsection*{Principe général de dimensionnement}

Le dimensionnement du plateau repose sur une distinction fonctionnelle entre trois niveaux
d’accueil ambulatoire, définis selon l’intensité de surveillance requise et la durée des prises en
charge. Les hypothèses retenues correspondent volontairement à une \textbf{phase de démarrage
prudente}, permettant une montée en charge ultérieure sans modification structurelle.

\begin{itemize}[leftmargin=1.1cm]
    \item \textbf{Niveau~1~: lits médicalisés} — surveillance continue, patient allongé.
    \item \textbf{Niveau~2~: fauteuils de soins} — surveillance infirmière directe, durée intermédiaire.
    \item \textbf{Niveau~3~: espace non allongé de type lounge} — patients autonomes, coordination
    multidisciplinaire sans besoin de lit.
\end{itemize}
% ============================================================
% TRADUCTION EN CAPACITÉ PHYSIQUE — HYPOTHÈSES CONSERVATRICES
% ============================================================

\subsection*{Traduction en capacité physique (hypothèses conservatrices)}

La traduction des hypothèses de flux en capacité physique repose sur des hypothèses
volontairement conservatrices, intégrant les aléas organisationnels (annulations, no-show),
les contraintes de ressources humaines, la variabilité des durées de prise en charge et les
indisponibilités ponctuelles des plateaux techniques.

Le plateau est supposé fonctionner sur une base de \textbf{44 semaines réellement opérées par an},
à raison de \textbf{5 jours par semaine}, soit \textbf{220 jours ouvrés/an}.

Les hypothèses de réalisation retenues sont :
\begin{itemize}[leftmargin=1.1cm]
    \item \textbf{Lits médicalisés} : 1 patient/lit/jour avec un coefficient de réalisation de \textbf{0.85}.
    \item \textbf{Fauteuils de soins} : 2 patients/fauteuil/jour avec un coefficient de réalisation de \textbf{0.80}.
    \item \textbf{Fauteuils lounge} : 2 patients/fauteuil/jour avec un coefficient de réalisation de \textbf{0.75}.
\end{itemize}

La capacité annuelle attendue est calculée selon la formule :
\[
HDJ/an = 220 \times \big(
N_L \times 1.0 \times 0.85
+ N_F \times 2.0 \times 0.80
+ N_{Lg} \times 2.0 \times 0.75
\big).
\]

Sur cette base, un dimensionnement conservateur compatible avec une cible de
\textbf{4\,700 à 5\,000 HDJ/an} est le suivant :
\begin{itemize}[leftmargin=1.1cm]
    \item \textbf{6 lits médicalisés},
    \item \textbf{6 fauteuils de soins},
    \item \textbf{5 fauteuils lounge} (espace non allongé de type \emph{salon premium}).
\end{itemize}

Ce calibrage correspond à une capacité annuelle théorique de :
\[
220 \times (6\times1.0\times0.85
+ 6\times2.0\times0.80
+ 5\times2.0\times0.75)
= \textbf{4\,884 HDJ/an}.
\]
% ============================================================
% TABLEAU SYNTHÉTIQUE — CAPACITÉ PHYSIQUE, ACTIVITÉ ET RECETTES
% ============================================================

\begin{sidewaystable}[p]
\centering
\renewcommand{\arraystretch}{1.25}
\rowcolors{2}{APHPsoft}{white}

\begin{tabular}{
p{4.6cm}
>{\centering\arraybackslash}p{1.6cm}
>{\centering\arraybackslash}p{3.0cm}
>{\centering\arraybackslash}p{3.0cm}
>{\centering\arraybackslash}p{3.4cm}
>{\centering\arraybackslash}p{3.4cm}
}
\toprule
\textbf{Type de capacité} &
\textbf{N (unités)} &
\textbf{Hypothèse de flux (conservatrice)} &
\textbf{Capacité annuelle estimée} &
\textbf{Activités principales concernées} &
\textbf{Recettes annuelles estimées \newline (ensemble du sous-plateau)} \\
\midrule
Lits médicalisés &
\textbf{6} &
1 patient/lit/jour $\times$ 0.85 &
$\approx$ 1\,122 HDJ/an &
PBH, cirrhoses avancées, chimiothérapies digestives prolongées,
radiologie interventionnelle post-acte &
$\approx$ 1.5--1.7 M€ \\
Fauteuils de soins &
\textbf{6} &
2 patients/fauteuil/jour $\times$ 0.80 &
$\approx$ 2\,112 HDJ/an &
Fer IV, albumine seule, biothérapies courtes,
chimiothérapies simples &
$\approx$ 0.6--0.7 M€ \\
Fauteuils lounge (non allongé) &
\textbf{5} &
2 patients/fauteuil/jour $\times$ 0.75 &
$\approx$ 1\,650 HDJ/an &
Hépatométabolique, évaluation des cirrhoses,
addictologie, parcours multidisciplinaires &
$\approx$ 0.6--0.8 M€ \\
\midrule
\textbf{Total plateau HDJ} &
— &
— &
\textbf{$\approx$ 4\,884 HDJ/an} &
— &
\textbf{$\approx$ 3 M€} \\
\bottomrule
\end{tabular}

\caption{Dimensionnement physique du plateau d’hôpital de jour selon des hypothèses
volontairement conservatrices. Le \emph{lounge} constitue une capacité assise premium,
non assimilée à des lits, permettant l’optimisation des flux sans surdimensionnement
hospitalier.}
\end{sidewaystable}
