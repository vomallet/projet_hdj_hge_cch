\documentclass[12pt,a4paper]{article}

% =======================
% PACKAGES DE BASE
% =======================
\usepackage[useregional]{datetime2}
\usepackage{url}
\usepackage{xurl} % permet la coupure automatique des longues URL
\usepackage[french]{babel}
\usepackage{setspace}
\usepackage{geometry}
\usepackage{graphicx}
\usepackage{titlesec}
\usepackage{enumitem}
\usepackage{tikz}
\usetikzlibrary{positioning}
\usepackage{tocloft}
\usepackage{csquotes}
\usepackage{booktabs}
\usepackage{float}
\usepackage{fontspec}
\usepackage{needspace}
\usepackage[table]{xcolor}
\usepackage{amssymb} % pour \square
\usepackage{tabularx}
\usepackage{pdflscape}
\usepackage{rotating}
\usepackage{lineno}
\usepackage{caption}        % \captionof
\usepackage{placeins}
\usepackage{array}
\usepackage{tcolorbox}



\linenumbers

% =======================
% FONT (XeLaTeX)
% =======================
\setmainfont{TeX Gyre Termes}
\setsansfont{TeX Gyre Heros}    % équivalent Helvetica-like
\setmonofont{TeX Gyre Cursor}   % équivalent Courier-like


% =======================
% COULEURS APHP
% =======================
\definecolor{APHPblue}{RGB}{0,56,147}
\definecolor{APHPdark}{RGB}{0,36,95}
\definecolor{APHPgrey}{RGB}{90,90,90}
\definecolor{APHPsoft}{RGB}{225,235,250}

% =======================
% TITRES
% =======================
\titleformat{\subsubsection}{\color{APHPdark}\normalsize\bfseries}{\thesubsubsection}{1em}{}
\titleformat{\paragraph}[runin]{\color{APHPdark}\bfseries}{}{0pt}{}[.\;]
\titlespacing*{\subsubsection}{0pt}{10pt}{6pt}
\titlespacing*{\paragraph}{0pt}{6pt}{0pt}

% =======================
% CASES À COCHER
% =======================
\newcommand{\ch}{\(\square\)\hspace{0.25em}}

% =======================
% LISIBILITÉ / SPACINGS
% =======================
\setlength{\parskip}{6pt}
\setlength{\parindent}{0pt}
\titlespacing*{\section}{0pt}{18pt}{12pt}
\titlespacing*{\subsection}{0pt}{14pt}{10pt}
\setlist[itemize]{itemsep=6pt, topsep=6pt}

% =======================
% TITRES
% =======================
\titleformat{\section}{\color{APHPdark}\Large\bfseries}{\thesection}{1em}{}
\titleformat{\subsection}{\color{APHPblue}\normalsize\bfseries}{\thesubsection}{1em}{}

% TABLE DES MATIÈRES
\renewcommand{\cftsecleader}{\cftdotfill{\cftdotsep}}

% =======================
% BIBLIOGRAPHIE
% =======================
\usepackage[
    backend=biber,
    style=vancouver,
    sorting=none,
    useprefix=false
]{biblatex}

% Coupure agressive des URL longues dans la bibliographie
\setcounter{biburlnumpenalty}{1}
\setcounter{biburlucpenalty}{1}
\setcounter{biburllcpenalty}{1}

% Style des URL plus compact
\renewcommand*{\UrlFont}{\small\ttfamily}

% Forcer les URL à se mettre sur une nouvelle ligne si nécessaire
\DeclareFieldFormat{url}{\newline\url{#1}}

% Empêche les débordements dans les références longues
\AtBeginBibliography{\sloppy}

\DeclareNameAlias{default}{family-given}
\DeclareNameAlias{sortname}{family-given}
\DeclareFieldFormat{author}{#1}
\DeclareFieldFormat{labelname}{#1}
\DeclareFieldFormat{journaltitle}{%
  \iffieldundef{shortjournal}{#1}{\printfield{shortjournal}}%
}

\AtBeginBibliography{%
  \renewcommand{\mkbibnamefamily}[1]{#1}%
  \renewcommand{\mkbibnamegiven}[1]{#1}%
  \renewcommand{\mkbibnameprefix}[1]{#1}%
  \renewcommand{\mkbibnamesuffix}[1]{#1}%
}

\addbibresource{references.bib}

% =======================
% GEOMETRIE
% =======================
\geometry{margin=2.4cm}
\setstretch{1.25}

% =======================
% DOCUMENT
% =======================
% ============================================================
% HYPERREF — ANCRES UNIQUES PAR BLOC (évite collisions TOC)
% ============================================================
\usepackage{hyperref}
\newcommand{\HsecPrefix}{main}
\makeatletter
\renewcommand*{\theHsection}{\HsecPrefix.\arabic{section}}
\renewcommand*{\theHsubsection}{\theHsection.\arabic{subsection}}
\renewcommand*{\theHsubsubsection}{\theHsubsection.\arabic{subsubsection}}
\makeatother

\hypersetup{
  colorlinks=true,
  linkcolor=APHPdark,
  urlcolor=APHPblue,
  citecolor=APHPdark
}
\begin{document}

% ============================
% PAGE DE GARDE
% ============================
\begin{titlepage}
\centering

\vspace*{0.8cm}

% Logos
\begin{minipage}{0.49\textwidth}
    \raggedright
    \includegraphics[height=2.8cm]{logo/logo_cochin_portroyal.png}
\end{minipage}
\begin{minipage}{0.49\textwidth}
    \raggedleft
    \includegraphics[height=2.6cm]{logo/LogoUPC.jpg}
\end{minipage}

\vspace{2.5cm}

% Titre principal
{\fontsize{26pt}{30pt}\selectfont\bfseries\color{APHPdark}
Projet des Hôpitaux de Jour Mutualisés\\[2mm]
d’Hépato-gastro-entérologie, d’Addictologie\\[2mm]
et de Radiologie interventionnelle
}

\vspace{6mm}

\color{APHPblue}\rule{0.45\textwidth}{0.6pt}

\vspace{6mm}

% Sous-titre
{\fontsize{14pt}{18pt}\selectfont\bfseries\color{APHPgrey}
Document de travail — Version 3
}

\vfill

% Métadonnées
{\small\color{APHPgrey}
Document compilé le \DTMtoday{} à \DTMcurrenttime
}

\vspace{1.2cm}

% Responsables médicaux
\begin{flushleft}
\textbf{Responsable de l’unité :} Pr Vincent Mallet\\[3mm]

\textbf{Hépatologie :} Dr Valérie D’Halluin\\
\textbf{Addictologie :} Dr Marion Courouge\\
\textbf{Gastroentérologie :} Pr Romain Coriat\\
\textbf{Oncologie digestive :} Dr Anna Pellat\\
\textbf{Radiologie interventionnelle :} Pr Anthony Dohan
\end{flushleft}

\vfill

% Institution
{\Large\bfseries GH AP--HP Centre — Site Cochin\\[3mm]}
{\large\bfseries DMU Hématologie, Cancérologie et Spécialités Médico-Chirurgicales}\\[6mm]

\end{titlepage}


% ============================
% TABLE DES MATIÈRES
% ============================
\tableofcontents
\newpage

% ==========================================
% CHAPITRE E1 — E-SUMMARY
% ==========================================
\clearpage
\section{Synthèse Exécutive}

\begin{refsection}
% ============================================================
% 1. E-SUMMARY — SYNTHÈSE EXÉCUTIVE
% ============================================================
% VERSION FINALE VALIDÉE
% - Suppression du plafond 5 214 (focus sur cible 4 675)
% - Renommage "lounge" → "boxes de consultations programmées"
% - Mise à jour ETP conformément au plan RH
% - Ajout mention phasage et plan RH
% ============================================================

\titleformat{\subsection}[runin]
  {\bfseries\color{APHPblue}}
  {}
  {0pt}
  {}

\begin{spacing}{1.30}

% ------------------------------------------------------------
% 1.0 ENCADRÉ SYNTHÈSE DIRECTION
% ------------------------------------------------------------

\begin{tcolorbox}[
  colback=APHPsoft,
  colframe=APHPblue,
  title=\textbf{Synthèse pour validation directionnelle},
  fonttitle=\bfseries\color{white},
  coltitle=white,
  boxrule=0.8pt,
  arc=2pt,
  left=6pt,
  right=6pt,
  top=4pt,
  bottom=4pt
]


\small
\renewcommand{\arraystretch}{1.25}
\begin{tabular}{@{}p{4.2cm}p{9cm}@{}}
\textbf{Capacité physique} & 18 places (6 lits + 6 fauteuils + 6 boxes consultations) \\
\textbf{Activité cible (croisière)} & 4 675 séances/an (taux d'occupation 90\%) \\
\textbf{Base PMSI 2024} & 2 425 séjours 0-jour transférables \\
\textbf{Recettes projetées} & 2,9 M€/an (croisière) \\
\textbf{ETP PNM} & 7 ETP (pilote) → 11,5 ETP (maturité) \\
\textbf{ETP PM} & 6 demi-j/sem (pilote) → 12 demi-j/sem (maturité) \\
\textbf{Phasage ouverture} & 6 places (M0) → 12 places (M+6) → 18 places (M+12) \\
\textbf{Montée en charge} & 12–24 mois (3 phases avec conditions de passage) \\
\end{tabular}
\end{tcolorbox}

\vspace{1em}

% ------------------------------------------------------------
% 1.1 Contexte et opportunité institutionnelle
% ------------------------------------------------------------

\subsection*{Contexte et opportunité institutionnelle —}\ignorespaces
La création d'une unité fonctionnelle (UF) dédiée aux Hôpitaux de Jour digestifs mutualisés s'inscrit pleinement dans les objectifs nationaux du virage ambulatoire. Elle repose sur une opportunité organisationnelle majeure — la libération de locaux dédiés — permettant de regrouper sur un site unique des activités ambulatoires aujourd'hui dispersées, tout en améliorant leur lisibilité et leur efficience.

Ce regroupement vise à constituer un dispositif évolutif, aligné sur les recommandations nationales et internationales de référence (EASL, AFEF, SNFGE, INCa, SFR/SFICV), et adapté à la prise en charge de patients complexes nécessitant des parcours de soins programmés, sécurisés et coordonnés.

% ------------------------------------------------------------
% 1.2 Objectifs stratégiques du projet
% ------------------------------------------------------------

\subsection*{Objectifs stratégiques du projet —}
Le projet poursuit cinq objectifs stratégiques structurants :
\begin{itemize}[leftmargin=1.1cm]
    \item améliorer l'accès rapide, sécurisé et lisible aux explorations et aux traitements spécialisés ;
    \item réduire les hospitalisations conventionnelles évitables par un recours structuré et coordonné à l'ambulatoire ;
    \item optimiser l'utilisation des ressources humaines expertes par une organisation mutualisée, transversale et efficiente ;
    \item renforcer l'attractivité institutionnelle et académique du site dans un contexte durable de tension sur les compétences spécialisées ;
    \item développer un parcours patient hospitalo-universitaire favorisant l'inclusion des patients dans des essais thérapeutiques.
\end{itemize}

% ------------------------------------------------------------
% 1.3 Population cible globale
% ------------------------------------------------------------

\subsection*{Population cible globale —}
Le plateau HDJ s'adresse à une population adulte présentant :
\begin{itemize}[leftmargin=1.1cm]
    \item des pathologies hépato-biliaires (Maladies chroniques du foie et des voies biliaires, cirrhoses compensées et décompensées) ;
    \item des troubles addictologiques avec retentissement somatique ;
    \item des maladies inflammatoires chroniques de l'intestin (MICI) ;
    \item des cancers digestifs nécessitant des traitements systémiques ambulatoires ;
    \item des patients pris en charge en radiologie interventionnelle pour surveillance post-examen ;
    \item des patients en cours d'évaluation thérapeutique ou de suivi de traitements oraux.
\end{itemize}

Ces patients présentent un besoin élevé d'évaluations programmables, répétées et sécurisées, compatible avec une prise en charge ambulatoire structurée.

% ------------------------------------------------------------
% 1.4 Preuve par les flux — Données PMSI 2024
% ------------------------------------------------------------

\subsection*{Preuve par les flux — Données PMSI 2024 —}
Le dimensionnement repose sur l'analyse rétrospective des séjours de durée 0 jour réalisés en 2024 dans les UF contributrices du site Cochin :

\vspace{0.5em}
\noindent
\begin{tabular}{@{}p{5.5cm}p{2.5cm}p{2.5cm}p{2.5cm}@{}}
\toprule
\textbf{Filière} & \textbf{Réf. 2024} & \textbf{Cible croisière} & \textbf{Évolution} \\
\midrule
Hépatologie (existant) & 374 & 569 & +52\% \\
Hépatologie (création) & — & 950 & Création \\
MICI + Oncologie digestive & 1 901 & 2 201 & +16\% \\
Addictologie & — & 705 & Création \\
Radiologie interventionnelle & 150 & 250 & +67\% \\
\midrule
\textbf{Total} & \textbf{2 425} & \textbf{4 675} & \textbf{+93\%} \\
\bottomrule
\end{tabular}

\vspace{0.5em}
\noindent
La capacité cible (4 675 séances/an à 90\% d'occupation) correspond exactement à la demande projetée, avec une réserve opérationnelle de 10\% pour l'absorption des aléas organisationnels.

% ------------------------------------------------------------
% 1.5 Volumétrie et trajectoire d'activité
% ------------------------------------------------------------

\subsection*{Volumétrie et trajectoire d'activité —}
Le plateau dispose de 18 places réparties sur trois niveaux (6 lits médicalisés, 6 fauteuils de soins, 6 boxes de consultations programmées). L'objectif opérationnel à maturité médico-économique repose sur une \textbf{trajectoire d'activité cible de 4\,675 séances annuelles}, correspondant à un taux d'occupation de 90\%. Cette trajectoire est anticipée sur une période de \textbf{12 à 24 mois}, selon un phasage progressif validé par la direction des soins.

% ------------------------------------------------------------
% 1.6 Recettes consolidées
% ------------------------------------------------------------

\subsection*{Recettes consolidées —}
Sur la base des tarifs moyens pondérés par filière et des volumes projetés, les recettes consolidées du plateau HDJ sont estimées à \textbf{environ 2,9~M€ par an à maturité}.

Le dispositif présente un profil \textbf{autosoutenable}, générateur de valeur médico-économique pour l'institution.

% ------------------------------------------------------------
% 1.7 Plan RH et phasage d'ouverture
% ------------------------------------------------------------

\subsection*{Plan RH et phasage d'ouverture —}
L'ouverture repose sur une stratégie RH sécurisée avec un phasage progressif :

\vspace{0.5em}
\noindent
\begin{tabular}{@{}p{3cm}p{3.5cm}p{2.5cm}p{3.5cm}@{}}
\toprule
\textbf{Phase} & \textbf{Capacité} & \textbf{ETP IDE} & \textbf{Échéance} \\
\midrule
Phase 1 (pilote) & 6 places & 4 ETP & M0 \\
Phase 2 & 12 places & 5,5 ETP & M+6 \\
Phase 3 (maturité) & 18 places & 6–7 ETP & M+12 \\
\bottomrule
\end{tabular}

\vspace{0.5em}
\noindent
Le sourcing des 4 IDE pilotes est sécurisé : 2 ETP par redéploiement/mobilité interne, 1 ETP par mobilité GHU, 1 ETP par recrutement externe (publication M-6). Les conditions de passage inter-phases (taux d'occupation $>$80\%, effectif complet, validation COPIL) garantissent une montée en charge maîtrisée.

% ------------------------------------------------------------
% 1.8 Bénéfices institutionnels
% ------------------------------------------------------------

\subsection*{Bénéfices institutionnels —}
La structuration du plateau HDJ permet :
\begin{itemize}[leftmargin=1.1cm]
    \item une redistribution ciblée des lits MCO mobilisés pour des prises en charge programmables ;
    \item une fluidification des parcours entre la ville, les urgences et l'hospitalisation complète ;
    \item une mutualisation efficiente des compétences spécialisées (IDE expertes, IPA, psychologie, diététique) ;
    \item une amélioration de l'attractivité des postes paramédicaux et médicaux spécialisés ;
    \item une lisibilité renforcée de l'offre ambulatoire digestive et interventionnelle à l'échelle du territoire.
\end{itemize}

% ------------------------------------------------------------
% 1.9 Gouvernance (niveau exécutif)
% ------------------------------------------------------------

\subsection*{Gouvernance —}
Le plateau des HDJ mutualisés repose sur une gouvernance claire et intégrée, associant :
\begin{itemize}[leftmargin=1.1cm]
    \item une responsabilité médicale unique, garante de la cohérence des parcours et de la sécurité des prises en charge ;
    \item une articulation structurée entre les services des pathologies digestives, les plateaux médico-techniques et la radiologie interventionnelle ;
    \item un pilotage médico-économique consolidé associant responsables médicaux, encadrement soignant, DIM et direction financière ;
    \item un COPIL trimestriel validant les passages de phase et les ajustements capacitaires.
\end{itemize}

% ------------------------------------------------------------
% 1.10 Structure du document
% ------------------------------------------------------------

\subsection*{Structure du document —}
Le présent document est organisé en trois parties :

\begin{enumerate}[leftmargin=1.1cm]
    \item \textbf{Organisation transversale et dimensionnement} — Principes d'organisation du plateau mutualisé, niveaux de prise en charge, zones fonctionnelles, gouvernance, capacité cible et plan RH d'ouverture.
    \item \textbf{Fiches opérationnelles par filière} — Pour chaque activité (biopsies hépatiques, cirrhoses, hépatométabolique, addictologie, MICI, chimiothérapie, radiologie interventionnelle) : rationnel médical, population éligible, parcours de soins, codage PMSI, projections d'activité et ressources nécessaires.
    \item \textbf{Livraison clé en main} — Organisation fonctionnelle opérationnelle, plan de déploiement phasé, ressources humaines (ETP), modèle économique de l'UF.
\end{enumerate}

Les annexes fournissent les procédures standardisées et grilles d'évaluation directement utilisables.

\end{spacing}

\clearpage
\printbibliography[heading=subbibliography,title={Références}]
\end{refsection}


% ===============================================
% CHAPITRE E2 — INTRODUCTION GÉNÉRALE
% ===============================================
\clearpage
\section{Introduction Générale}

\begin{refsection}
% ===============================================
% CHAPITRE 0: Introduction générale 
% ===============================================

\begin{spacing}{1.30}

\subsection*{L’AP-HP et le virage ambulatoire}

L’Assistance Publique–Hôpitaux de Paris (AP-HP), premier centre hospitalo-universitaire d’Europe, regroupe 39 hôpitaux organisés en GHU et assure soin, enseignement et recherche pour près de 12 millions d’habitants. Le plan stratégique médical 2023–2028 place le virage ambulatoire au cœur de la transformation, en visant une amélioration de l’accessibilité, des délais et de la fluidité des parcours \cite{APHP2023Plan}.  
L’hôpital de jour (HDJ) constitue un outil majeur de cette stratégie, permettant la réalisation coordonnée, en temps court, d’évaluations diagnostiques et thérapeutiques complexes dans un cadre sécurisé et universitaire.

\subsection*{Le GHU AP–HP.Centre et le site Cochin}

L’Hôpital Cochin–Port-Royal (GHU AP–HP.Centre–Université Paris Cité) constitue un pôle hospitalo-universitaire de référence. Il couvre un large champ d’expertises : hépato-gastroentérologie et oncologie digestive, chirurgie digestive et hépatobiliaire, endocrinologie et maladies métaboliques, rhumatologie, obstétrique, néonatologie, réanimation et soins intensifs spécialisés.  

Il accueille deux unités cliniques d’HGE — le service des maladies du foie et le service d’hépato-gastroentérologie et d’oncologie digestive — constituant des plateformes de recours régionales et nationales \cite{CollegialeHGE2023}. L’ensemble du plateau technique (endoscopie interventionnelle, radiologie interventionnelle, imagerie spécialisée, traitement loco-régional et systémique des cancers digestifs, préparation à la greffe de foie) permet une prise en charge intégrée des pathologies digestives complexes.

\subsection*{Poids épidémiologique des maladies digestives}

Les maladies digestives représentent la première cause d’hospitalisation en médecine–chirurgie–obstétrique en France \cite{DREES2021}. Les évolutions récentes incluent :

\begin{itemize}
\item une augmentation marquée des cancers pancréatiques et hépatiques, désormais parmi les principales causes de mortalité oncologique \cite{UEG2022DigestiveHealth} ;
\item une stabilité de l’incidence du cancer colorectal, mais un besoin croissant en endoscopies ;
\item une progression soutenue des maladies du foie : prévalence de la cirrhose autour de 0,3~\%, 150--200 nouveaux cas/million/an, et près de 15 000 décès annuels \cite{FrenchHepaticFailure_2020} ;
\item une forte croissance des maladies métaboliques du foie (MASLD/MASH), portée par l’épidémie d’obésité et de diabète, conformément aux dernières recommandations européennes \cite{EASL2024MASLD,RN597} ;
\item une hausse continue de la prévalence des MICI, avec environ 250~000 patients en France et plus de 20~000 suivis à l’AP-HP \cite{Ng2017,SNDS_MICI2022}. 
\end{itemize}

Les troubles liés à l’alcool restent un déterminant majeur de morbi-mortalité digestive, première cause de cirrhose et de carcinome hépatocellulaire, avec un poids important sur les hospitalisations \cite{FrenchHepaticFailure_2020}.

\subsection*{Contraintes structurelles et nécessité du développement ambulatoire}

Les services d’HGE font face à une augmentation de la complexité des patients (vieillissement, précarité) alors que le capacitaire se réduit (fermetures de lits, tensions en personnel). De nombreuses indications relevant autrefois de l’hospitalisation conventionnelle peuvent désormais être conduites en HDJ : initiation de biothérapies MICI, immunothérapies, bilans spécialisés, évaluations multidisciplinaires, actes interventionnels simples.  
Le projet médical HGE 2023–2028 désigne l’HDJ comme un levier prioritaire pour réduire les DMS, absorber la croissance épidémiologique et renforcer l’attractivité régionale \cite{CollegialeHGE2023}.

\subsection*{Rôle structurant des Hôpitaux de Jour HGE}

Les HDJ thématiques du pôle répondent à plusieurs missions stratégiques :

\begin{itemize}
\item standardisation des filières diagnostiques et thérapeutiques ;
\item concentration des évaluations spécialisées sur une même journée ;
\item réduction des délais d’accès aux biothérapies, à l’imagerie ou à l’endoscopie ;
\item intégration de compétences pluridisciplinaires (diététique, psychologie, addictologie, IPA, ETP) ;
\item optimisation de la coordination ville–hôpital et de la traçabilité universitaire.
\end{itemize}

Ils constituent aujourd’hui une architecture cohérente, alignée sur les enjeux épidémiologiques et capacitaires du CHU.

\subsection*{Logique du présent document}

Ce document rassemble les fiches techniques standardisées des HDJ du pôle d’HGE et prépare la construction d’un HDJ mutualisé à horizon 2027.  
Il décrit l’activité actuelle (hépatologie, MICI, addictologie, hépatométabolique, radiologie interventionnelle, oncologie digestive) et fournit la base opérationnelle du futur HDJ commun.  
La libération des locaux occupés aujourd’hui par l’hématologie rendra possible la création d’un espace unique regroupant des parcours spécialisés, standardisés et pluridisciplinaires, améliorant l’accessibilité, la qualité et la fluidité des prises en charge ambulatoires au sein du GHU AP–HP.Centre.

\end{spacing}

\printbibliography[heading=subbibliography,title={Références}]
\end{refsection}


% ============================================================
% CHAPITRE E3 — PLATEAU HDJ DIGESTIF MUTUALISÉ
% ============================================================
\clearpage
\section{Plateau des HDJ Mutualisés — Organisation Transversale}

\begin{refsection}
% ============================================================
% 3. PLATEAU HDJ DIGESTIF MUTUALISÉ — ORGANISATION TRANSVERSALE
% ============================================================
% VERSION FINALE VALIDÉE
% - Suppression du plafond 5 214 (focus sur cible 4 675)
% - Renommage "espace ambulatoire" → "boxes de consultations programmées"
% - Ajout benchmark IDE AP-HP
% - Planning présenté comme exemple illustratif
% - Ajout section Plan RH d'ouverture
% ============================================================

\label{sec:plateau_hdj_mutualise}

\titleformat{\subsection}[runin]
  {\bfseries\color{APHPblue}}
  {}
  {0pt}
  {}

\begin{spacing}{1.30}

\noindent
Le plateau HDJ digestif mutualisé est un dispositif ambulatoire intégré regroupant les filières 
digestives et interventionnelles du service. Il est dimensionné selon des hypothèses conservatrices, 
avec une \textbf{capacité physique de 18 places} réparties en trois niveaux, pour une 
\textbf{activité cible de 4\,675 séances/an} (taux d'occupation 90\%), et un \textbf{potentiel 
de recettes} estimé à \textbf{≈2,9 M€}. Le présent chapitre décrit l'organisation fonctionnelle, le dimensionnement capacitaire, la validation sur données historiques et les modalités de pilotage.

% ============================================================
% 3.1 PRINCIPES D'ORGANISATION
% ============================================================
\subsection*{Principes d'organisation —} 

L'organisation du plateau vise une allocation graduée des moyens, une standardisation des parcours et une optimisation de l'occupation, afin de maximiser l'efficience médico-économique tout en garantissant la qualité et la sécurité des soins.

\begin{itemize}[leftmargin=1.1cm]
  \item \textbf{Graduation des prises en charge} : adaptation des moyens humains, techniques et hôteliers à l'intensité réelle des besoins cliniques, selon trois niveaux formalisés.
  \item \textbf{Mutualisation des ressources} : partage des espaces, des équipes soignantes et des circuits logistiques entre filières contributrices, réduisant les coûts fixes unitaires.
  \item \textbf{Flexibilité capacitaire} : vases communicants inter-filières permettant une réallocation dynamique des places non consommées et une optimisation du taux d'occupation global.
  \item \textbf{Qualité hôtelière} : environnement de soins garantissant confort, confidentialité et dignité, contribuant à l'attractivité au sein du GHU.
\end{itemize}

% ============================================================
% 3.2 QUALITÉ HÔTELIÈRE ET INTIMITÉ DES PATIENTS
% ============================================================
\subsection*{Qualité hôtelière et intimité des patients —}

\noindent
Le plateau HDJ est conçu pour offrir un environnement de soins de haut niveau. La confidentialité et l'intimité des patients constituent un axe structurant, décliné par niveau d'intensité de soins.

\begin{itemize}[leftmargin=1.1cm]
  \item \textbf{Niveau 1 (Haute technicité)} : sectorisation en \textbf{chambres individuelles} pour l'ensemble des lits médicalisés, assurant confidentialité et sécurité des soins.
  \item \textbf{Niveau 2 (Soins ambulatoires)} : \textbf{boxes individuels ou semi-cloisonnés} pour les fauteuils, conciliant ergonomie de travail et respect de l'intimité.
  \item \textbf{Niveau 3 (Consultations programmées)} : \textbf{6 boxes de consultations programmées à rotation rapide} avec dispositifs de rupture visuelle et acoustique, complétés par des \textbf{espaces de consultation privés}. La différenciation des flux (entrants/sortants/en cours) optimise la fluidité. L'environnement est valorisé par la lumière naturelle, un mobilier contemporain et une signalétique intuitive.
\end{itemize}

\noindent
Cette configuration répond à une double exigence : (i) préserver la dignité de patients souvent fragilisés par des pathologies chroniques ; (ii) positionner le plateau comme une référence en qualité perçue.

% ============================================================
% 3.3 PARCOURS ET FLUX PATIENTS
% ============================================================
\subsection*{Parcours patient —}

\noindent
Les patients suivent un parcours standardisé, modulé selon la filière et la complexité clinique.

\begin{enumerate}[leftmargin=1.1cm]
  \item \textbf{Orientation} : consultation spécialisée, RCP, urgences, ville ou hospitalisation complète.
  \item \textbf{Classification} : attribution d'un niveau de prise en charge (1, 2 ou 3) selon critères médicaux.
  \item \textbf{Programmation} : affectation à une place et un créneau horaire.
  \item \textbf{Accueil centralisé} : admission administrative et paramédicale unique.
  \item \textbf{Prise en charge} : actes médicaux, techniques ou multidisciplinaires.
  \item \textbf{Surveillance} : proportionnée au niveau de risque.
  \item \textbf{Sortie sécurisée} : synthèse médicale, prescriptions et organisation du suivi.
\end{enumerate}

% ============================================================
% 3.4 PRINCIPES DE DIMENSIONNEMENT CAPACITAIRE
% ============================================================
\subsection*{Principes de dimensionnement capacitaire —}

\noindent
Le dimensionnement repose sur une approche capacitaire pragmatique, fondée sur l'intensité de surveillance requise, la durée moyenne de prise en charge et les contraintes opérationnelles. Les hypothèses correspondent volontairement à une \textbf{phase de démarrage prudente}, permettant une montée en charge progressive sans modification structurelle des locaux ni des effectifs.

\paragraph{Segmentation de la capacité physique}
La capacité est structurée selon trois modalités d'accueil, correspondant à des niveaux croissants d'intensité de soins :
\begin{itemize}[leftmargin=1.1cm]
  \item \textbf{Lits médicalisés} : surveillance continue, patient allongé, actes ou situations à risque.
  \item \textbf{Fauteuils de soins} : surveillance infirmière directe pour prises en charge intermédiaires.
  \item \textbf{Boxes de consultations programmées} : patients autonomes relevant de parcours programmés coordonnés, sans besoin de surveillance continue.
\end{itemize}

\paragraph{Hypothèses opérationnelles conservatrices}
La conversion des besoins en capacité intègre les aléas (annulations, variabilité des durées, contraintes RH spécialisées, indisponibilités ponctuelles). Le fonctionnement est estimé sur \textbf{44 semaines réellement opérées/an}, \textbf{5 jours/semaine}, soit \textbf{220 jours ouvrés/an}.

\noindent Hypothèses de productivité :
\begin{itemize}[leftmargin=1.1cm]
  \item \textbf{Lits médicalisés} : 1 patient/lit/jour, coefficient de réalisation \textbf{0,85}.
  \item \textbf{Fauteuils de soins} : 2 patients/fauteuil/jour, coefficient de réalisation \textbf{0,80}.
  \item \textbf{Boxes de consultations programmées} : productivité retenue \textbf{2 patients/place/jour} avec coefficient \textbf{0,75} (variabilité des durées, temps non productifs).
\end{itemize}

\paragraph{Formule de capacité annuelle cible}
\[
\mbox{HDJ/an} = 220 \times \Big(
N_L \times 1{,}0 \times 0{,}85
\;+\; N_F \times 2{,}0 \times 0{,}80
\;+\; N_{B} \times 2{,}0 \times 0{,}75
\Big) \times 0{,}90
\]

\noindent Application numérique avec $N_L=6$, $N_F=6$, $N_{B}=6$ et taux d'occupation cible de 90\% :
\[
220 \times \big(6 \times 0{,}85 + 6 \times 1{,}60 + 6 \times 1{,}50\big) \times 0{,}90
= 220 \times 23{,}70 \times 0{,}90
= \textbf{4\,693 \mbox{ séances/an}} \approx \textbf{4\,675 \mbox{ séances/an}}.
\]

\noindent
Cette valeur constitue l'\textbf{objectif de production cible}, intégrant une réserve de 10\% pour l'absorption des fluctuations saisonnières et des aléas organisationnels.

% ============================================================
% 3.4bis VALIDATION PMSI
% ============================================================
\subsection*{Adéquation charge/capacité — Validation sur données PMSI 2024 —}

\noindent
Le dimensionnement a été confronté aux données PMSI 2024 des services contributeurs du site Cochin. L'analyse rétrospective porte sur les séjours de durée 0 jour (CMD 07, 25, 28) et les projections issues des fiches opérationnelles par filière.

\begin{table}[!htbp]
\centering
\caption{Confrontation données PMSI 2024 vs projections HDJ (phase croisière)}
\label{tab:adequation_pmsi}
\small
\rowcolors{2}{APHPsoft}{white}
\renewcommand{\arraystretch}{1.35}
\begin{tabular}{@{}p{4cm}p{2.2cm}p{2.2cm}p{2.2cm}p{2.2cm}@{}}
\toprule
\textbf{Filière} & \textbf{Réf. 2024} & \textbf{Cible croisière} & \textbf{Évolution} & \textbf{Source} \\
\midrule
\multicolumn{5}{l}{\textit{Hépatologie — existant}} \\
\quad PBH                    & 97   & 142  & +46\%  & Fiche 2 \\
\quad Cirrhoses traitement   & 277  & 427  & +54\%  & Fiche 4 \\
\multicolumn{5}{l}{\textit{Hépatologie — création}} \\
\quad Évaluation cirrhoses   & —    & 450  & Création & Fiche 3 \\
\quad Hépatométabolique      & —    & 500  & Création & Fiche 5 \\
\midrule
\textbf{Sous-total Hépatologie} & \textbf{374} & \textbf{1 519} & — & — \\
\midrule
MICI                         & 1 333 & 1 533 & +15\% & Fiche 7 \\
Oncologie digestive          & 568   & 668   & +18\% & Fiche 8 \\
\midrule
\textbf{Sous-total MICI + Onco} & \textbf{1 901} & \textbf{2 201} & +16\% & — \\
\midrule
Addictologie                 & —     & 705   & Création & Fiche 6 \\
Radiologie interventionnelle & 150   & 250   & +67\%  & Fiche 9 \\
\midrule
\textbf{TOTAL}               & \textbf{2 425} & \textbf{4 675} & +93\% & — \\
\bottomrule
\end{tabular}
\end{table}

\noindent
\textbf{Synthèse.} La capacité cible (\textbf{4\,675 séances/an} à 90\% d'occupation) correspond exactement à la demande projetée des filières contributrices. La filière « Maladies du Foie » (≈32\% des flux) dispose d'une capacité compatible avec une cible de 1\,519 séances/an, via la combinaison des niveaux 1 et 3 selon les indications.

\FloatBarrier

% ============================================================
% 3.5 NIVEAUX DE PRISE EN CHARGE — VUE D'ENSEMBLE
% ============================================================
\subsection*{Trois niveaux de prise en charge —}

\noindent
Le plateau HDJ est structuré en trois niveaux fonctionnels, définis selon l'intensité de surveillance et la durée de séjour. Cette graduation permet une allocation optimale des ressources et une rotation adaptée aux prises en charge.

\begin{table}[!htbp]
\centering
\caption{Synthèse des niveaux de prise en charge — Capacité cible à 90\% d'occupation (220 jours/an)}
\label{tab:niveaux_synthese}
\small
\rowcolors{2}{APHPsoft}{white}
\renewcommand{\arraystretch}{1.35}
\begin{tabular}{@{}p{2.8cm}p{2.2cm}p{1.8cm}p{2cm}p{1.6cm}p{2.2cm}@{}}
\toprule
\textbf{Niveau} & \textbf{Places} & \textbf{Durée} & \textbf{Rotation} & \textbf{Coeff.} & \textbf{Cible/an} \\
\midrule
1 — Lits           & 6 chambres  & 6–8 h & 1 pat./jour & 0,85 & $\sim$1\,010 \\
2 — Fauteuils      & 6 fauteuils & 3–4 h & 2 pat./jour & 0,80 & $\sim$1\,900 \\
3 — Boxes consult. & 6 boxes     & 2–3 h & 2 pat./jour & 0,75 & $\sim$1\,780 \\
\midrule
\textbf{Total} & \textbf{18 places} & — & — & — & \textbf{$\sim$4\,675} \\
\bottomrule
\end{tabular}
\end{table}

% ============================================================
% 3.5.1 NIVEAU 1 — LITS MÉDICALISÉS
% ============================================================
\subsubsection*{Niveau 1 — Lits médicalisés en chambre individuelle —}

\paragraph{Principe}
Le niveau 1 accueille les patients nécessitant une surveillance continue et/ou des actes invasifs à risque. Chaque patient occupe une \textbf{chambre individuelle} pour la journée, garantissant confidentialité, repos et conditions optimales de surveillance.

\paragraph{Indications}
\begin{itemize}[leftmargin=1.1cm]
  \item biopsies hépatiques transpariétales (repos strict post-procédure) ;
  \item ponctions d'ascite grand volume ($>$5 litres, compensation albumine) ;
  \item chimiothérapies prolongées ou à risque (surveillance) ;
  \item surveillance post-radiologie interventionnelle (embolisation, drainage).
\end{itemize}

\paragraph{Organisation}
\begin{itemize}[leftmargin=1.1cm]
  \item \textbf{Capacité} : 6 chambres individuelles équipées (scope, oxygène, aspiration).
  \item \textbf{Durée} : 6 à 8 heures (arrivée 8h00, sortie 16h00–18h00).
  \item \textbf{Rotation} : 1 patient/lit/jour (pas de double rotation).
  \item \textbf{Ratio IDE} : 1 IDE / 3 patients.
  \item \textbf{Présence médicale} : médecin senior disponible sur le plateau.
\end{itemize}

\paragraph{Capacité annuelle cible}
6 lits $\times$ 220 jours $\times$ 1 patient/jour $\times$ 0,85 $\times$ 0,90 = \textbf{$\sim$1\,010 séances/an}.

% ============================================================
% 3.5.2 NIVEAU 2 — FAUTEUILS DE SOINS
% ============================================================
\subsubsection*{Niveau 2 — Fauteuils de soins en box individuel —}

\paragraph{Principe}
Le niveau 2 correspond à des soins techniques nécessitant une surveillance infirmière directe, sans décubitus prolongé. Les fauteuils sont positionnés en \textbf{boxes individuels ou semi-cloisonnés}.

\paragraph{Indications}
\begin{itemize}[leftmargin=1.1cm]
  \item perfusions d'albumine ;
  \item fer injectable IV ;
  \item transfusions programmées ;
  \item biothérapies IV (infliximab, vedolizumab, ustekinumab) ;
  \item chimiothérapies digestives simples (durée $<$4h) ;
  \item traitements IV divers (antibiotiques, immunoglobulines).
\end{itemize}

\paragraph{Organisation}
\begin{itemize}[leftmargin=1.1cm]
  \item \textbf{Capacité} : 6 fauteuils inclinables en boxes.
  \item \textbf{Durée} : 3 à 4 heures.
  \item \textbf{Rotation} : 2 patients/fauteuil/jour (matin + après-midi).
  \item \textbf{Ratio IDE} : 1 IDE / 4 patients.
  \item \textbf{Présence médicale} : médecin joignable ; passage selon besoin clinique.
\end{itemize}

\paragraph{Capacité annuelle cible}
6 fauteuils $\times$ 220 jours $\times$ 2 patients/jour $\times$ 0,80 $\times$ 0,90 = \textbf{$\sim$1\,900 séances/an}.

% ============================================================
% 3.5.3 NIVEAU 3 — BOXES DE CONSULTATIONS PROGRAMMÉES
% ============================================================
\subsubsection*{Niveau 3 — Boxes de consultations programmées à rotation rapide —}

\paragraph{Principe}
Le niveau 3 est dédié aux parcours à \textbf{faible intensité de surveillance} et à \textbf{forte valeur évaluative}. Le patient est installé dans un box dédié ; les professionnels gravitent autour de lui (modèle \emph{One-Stop Shop} diagnostique et thérapeutique). Ce modèle, éprouvé dans les HDJ oncologiques et hématologiques AP-HP, optimise le temps médical et paramédical tout en réduisant les déplacements du patient.

\paragraph{Indications et files actives}
\begin{itemize}[leftmargin=1.1cm, label=\textbullet]
  \item \textbf{Bilan hépatique non-invasif complet} : échographie morphologique, élastographies (hépatique et splénique), quantification stéatose (CAP/LISA).
  \item \textbf{Addictologie intégrée} : sevrage ambulatoire, suivi post-sevrage complexe, réduction des risques.
  \item \textbf{MICI stables} : administration de biothérapies SC et ETP.
  \item \textbf{Parcours pré-thérapeutiques} : bilans pré-inclusion, initiation de traitements oraux complexes.
\end{itemize}

\paragraph{Paramètres opérationnels}
\begin{itemize}[leftmargin=1.1cm, label=\textbullet]
  \item \textbf{Capacité} : 6 boxes avec isolement visuel et acoustique.
  \item \textbf{Rotation retenue} : 2 patients/box/jour.
  \item \textbf{Durée} : 2 à 3 heures.
  \item \textbf{Coefficient de réalisation} : 0,75.
  \item \textbf{Ratio IDE} : 1 IDE / 6 patients (coordination, faible technicité), avec renfort ponctuel selon charge.
\end{itemize}

\paragraph{Capacité annuelle cible}
6 boxes $\times$ 220 jours $\times$ 2 patients/jour $\times$ 0,75 $\times$ 0,90 = \textbf{$\sim$1\,780 séances/an}.

\FloatBarrier

% ============================================================
% 3.6 SYNTHÈSE CAPACITAIRE ET IMPACT MÉDICO-ÉCONOMIQUE
% ============================================================
\subsection*{Synthèse capacitaire et impact médico-économique —}
\label{sec:synthese_medico_eco}

\noindent
La répartition capacitaire permet une allocation différenciée des activités selon leur intensité de soins et leur rendement médico-économique. Les lits médicalisés concentrent les prises en charge complexes à forte valeur ajoutée clinique ; les fauteuils et les boxes de consultations absorbent des volumes élevés d'actes programmés à forte efficience.

\begin{sidewaystable}[p]
\centering
\caption{Synthèse capacitaire et médico-économique du plateau HDJ — Cible à 90\% d'occupation (220 jours/an)}
\label{tab:synthese_medico_eco}
\renewcommand{\arraystretch}{1.15}
\rowcolors{2}{APHPsoft}{white}
\begin{tabular}{
  p{4.2cm}
  >{\centering\arraybackslash}p{1.4cm}
  >{\centering\arraybackslash}p{3.2cm}
  >{\centering\arraybackslash}p{2.8cm}
  >{\centering\arraybackslash}p{3.6cm}
  >{\centering\arraybackslash}p{3.2cm}
}
\toprule
\textbf{Type de capacité} &
\textbf{N} &
\textbf{Hypothèse de flux} &
\textbf{Capacité cible/an} &
\textbf{Activités principales} &
\textbf{Recettes estimées} \\
\midrule
Lits médicalisés &
6 &
1 pat./lit/j $\times$ 0,85 $\times$ 0,90 &
$\approx$ 1\,010 &
PBH, cirrhoses avancées, chimiothérapies prolongées, RI post-acte &
$\approx$ 1,3–1,5 M€ \\
Fauteuils de soins &
6 &
2 pat./faut./j $\times$ 0,80 $\times$ 0,90 &
$\approx$ 1\,900 &
Fer IV, albumine, biothérapies IV, chimiothérapies simples &
$\approx$ 0,5–0,6 M€ \\
Boxes consultations &
6 &
2 pat./box/j $\times$ 0,75 $\times$ 0,90 &
$\approx$ 1\,780 &
Hépatométabolique, cirrhoses compensées, addictologie, parcours coordonnés &
$\approx$ 0,6–0,9 M€ \\
\midrule
\textbf{Total plateau} &
\textbf{18} &
— &
\textbf{$\approx$ 4\,675} &
— &
\textbf{$\approx$ 2,9 M€} \\
\bottomrule
\end{tabular}

\vspace{0.25cm}
\footnotesize
\emph{Note :} la capacité cible intègre un taux d'occupation de 90\%, préservant une réserve opérationnelle de 10\% pour l'absorption des fluctuations saisonnières et des aléas organisationnels.
\end{sidewaystable}

\FloatBarrier

\noindent\textbf{Adéquation offre/demande.}
La somme des projections « croisière » des filières opérationnelles atteint \textbf{4\,675 séances} et \textbf{2,9~M€}, correspondant exactement à la capacité cible dimensionnée. La marge résiduelle de 10\% constitue une réserve opérationnelle (variations saisonnières, aléas RH, indisponibilités techniques).

% ============================================================
% 3.7 FLEXIBILITÉ — VASES COMMUNICANTS
% ============================================================
\subsection*{Flexibilité capacitaire — Système de vases communicants —}
\label{sec:vases_communicants}

\noindent
La flexibilité inter-filières repose sur un mécanisme de réallocation des places non consommées, garantissant l'optimisation de l'occupation et la réduction des délais de programmation.

\paragraph{Principe}
Chaque filière dispose de créneaux réservés ; les places non programmées ou libérées (annulation, sortie anticipée) basculent vers un \textbf{pool mutualisé} accessible à l'ensemble des filières contributrices.

\paragraph{Mécanisme d'allocation}
\begin{enumerate}[leftmargin=1.1cm]
  \item \textbf{J-7 à J-3} : confirmation des patients programmés par les référents médicaux.
  \item \textbf{J-2 à J-1} : bascule des créneaux non confirmés dans le \textbf{pool mutualisé} ; proposition aux filières via liste d'attente partagée.
  \item \textbf{Jour J} : réattribution en temps réel des places libérées, sous coordination IDE/IPA.
\end{enumerate}

\paragraph{Critères de priorisation du pool}
\begin{enumerate}[leftmargin=1.1cm]
  \item urgences médicales relatives (décompensation, infection, anémie symptomatique) ;
  \item ancienneté de la demande ($>$7 jours) ;
  \item optimisation des créneaux (matin/après-midi) ;
  \item équité inter-filières sur période glissante (suivi mensuel).
\end{enumerate}

\paragraph{Outils de gestion}
\begin{itemize}[leftmargin=1.1cm]
  \item tableau de programmation partagé (support institutionnel type Teams/SharePoint) accessible aux référents ;
  \item liste d'attente mutualisée (date de demande, niveau de priorité) ;
  \item signalement immédiat des places libérées (IDE/IPA coordinatrices) ;
  \item reporting hebdomadaire du taux d'occupation par filière et par niveau.
\end{itemize}

\paragraph{Bénéfices attendus}
\begin{itemize}[leftmargin=1.1cm]
  \item optimisation du taux d'occupation global (cible $>$85\%) ;
  \item réduction des délais de programmation ;
  \item équité d'accès entre filières contributrices ;
  \item absorption des variations saisonnières.
\end{itemize}

% ============================================================
% 3.8 PLANNING HEBDOMADAIRE TYPE — EXEMPLE ILLUSTRATIF
% ============================================================
\subsection*{Planning hebdomadaire type — Exemple illustratif —}

\noindent
Les tableaux suivants présentent un \textbf{exemple illustratif} de programmation hebdomadaire, à adapter en fonction de l'activité réelle et des besoins des filières. La répartition des créneaux sera ajustée trimestriellement en COPIL selon les données d'occupation et les évolutions de la demande.

\begin{table}[!htbp]
\centering
\caption{Exemple de programmation — Lits médicalisés (niveau 1) : créneaux réservés et pool mutualisé}
\label{tab:planning_hebdo_lits}
\small
\renewcommand{\arraystretch}{1.35}
\rowcolors{2}{APHPsoft}{white}
\begin{tabular}{@{}p{1.8cm}p{5.9cm}p{1.8cm}p{1.8cm}@{}}
\toprule
\textbf{Jour} & \textbf{Activités programmées} & \textbf{Réservé} & \textbf{Pool} \\
\midrule
Lundi    & PBH (2) + Ascite (3)                                      & 5 lits & 1 lit \\
Mardi    & Radiologie interventionnelle (3–4) + Chimio lourde (1)     & 4 lits & 2 lits \\
Mercredi & MICI immunothérapies IV (4–5) + Chimio lourde (1)         & 5 lits & 1 lit \\
Jeudi    & PBH (2) + Ascite (3)                                      & 5 lits & 1 lit \\
Vendredi & Radiologie interventionnelle (3–4) + Chimio lourde (1)     & 4 lits & 2 lits \\
\bottomrule
\end{tabular}
\end{table}

\begin{table}[!htbp]
\centering
\caption{Exemple de programmation — Fauteuils de soins (niveau 2)}
\label{tab:planning_hebdo_fauteuils}
\small
\renewcommand{\arraystretch}{1.35}
\rowcolors{2}{APHPsoft}{white}
\begin{tabular}{@{}p{1.8cm}p{5.9cm}p{5.9cm}@{}}
\toprule
\textbf{Jour} & \textbf{Matin (8h–12h)} & \textbf{Après-midi (13h–17h)} \\
\midrule
Lundi    & Fer IV, Albumine, Transfusions & Biothérapies \\
Mardi    & Biothérapies                   & Chimio simples \\
Mercredi & Fer IV, Albumine, Transfusions & Biothérapies \\
Jeudi    & Chimio simples                 & Biothérapies \\
Vendredi & Biothérapies                   & Rattrapage / Pool \\
\bottomrule
\end{tabular}
\end{table}

\noindent
\textbf{Principe d'ajustement.} Les boxes de consultations programmées (niveau 3, 6 places) fonctionnent quotidiennement, avec programmation adaptée aux disponibilités des intervenants (médecins, diététicien(ne)s, psychologues, IPA). La répartition initiale sera affinée sur les 6 premiers mois de fonctionnement, puis stabilisée après validation en COPIL.

\FloatBarrier

% ============================================================
% 3.9 ZONES FONCTIONNELLES
% ============================================================
\subsection*{Zones fonctionnelles —}

\noindent
La traduction spatiale de l'organisation repose sur des zones identifiées :
\begin{itemize}[leftmargin=1.1cm]
  \item \textbf{Zone d'accueil} : admission centralisée, attente, orientation.
  \item \textbf{Zone niveau 1} : 6 chambres individuelles équipées.
  \item \textbf{Zone niveau 2} : 6 boxes individuels avec fauteuils.
  \item \textbf{Zone niveau 3} : 6 boxes de consultations programmées (séparations visuelles et acoustiques).
  \item \textbf{Zone consultations} : bureaux intégrés (diététique, psychologie, addictologie, IPA).
  \item \textbf{Zone technique} : échographie, élastographie, prélèvements, préparation traitements.
  \item \textbf{Zone logistique} : pharmacie, stockage, circuits propres/sales.
\end{itemize}

% ============================================================
% 3.10 RESSOURCES HUMAINES MUTUALISÉES
% ============================================================
\subsection*{Ressources humaines mutualisées —}

\noindent
Le fonctionnement repose sur une équipe dédiée, formée par niveau de prise en charge. Le dimensionnement intègre la couverture des absences et la montée en compétences progressive.

\paragraph{Sectorisation IDE par niveau}
Les IDE sont affectées par niveau, avec possibilité de renfort inter-niveaux selon les habilitations acquises :
\begin{itemize}[leftmargin=1.1cm]
  \item \textbf{IDE niveau 1} : profil réanimation/soins continus, formées aux actes invasifs et à la surveillance post-procédurale.
  \item \textbf{IDE niveau 2} : profil HDJ oncologie/MICI, formées aux biothérapies et chimiothérapies.
  \item \textbf{IDE niveau 3} : profil coordination/ETP, formées à l'addictologie et aux parcours multidisciplinaires.
\end{itemize}

\begin{table}[!htbp]
\centering
\caption{Effectifs non médicaux (PNM) — Phase pilote et maturité}
\label{tab:effectifs_pnm}
\small
\rowcolors{2}{APHPsoft}{white}
\renewcommand{\arraystretch}{1.35}
\begin{tabular}{@{}p{4.6cm}p{2.6cm}p{2.6cm}p{3.6cm}@{}}
\toprule
\textbf{Catégorie} & \textbf{Phase pilote} & \textbf{Maturité} & \textbf{Missions principales} \\
\midrule
IDE expertes        & 4–5 ETP  & 6–7 ETP   & Soins, surveillance, coordination \\
IPA                 & 1 ETP    & 1,5 ETP   & Coordination, évaluation, ETP \\
Cadre de santé      & 0,5 ETP  & 1 ETP     & Management, organisation \\
Secrétariat médical & 0,5 ETP  & 1 ETP     & Programmation, accueil, DPI \\
Psychologue         & 0,3 ETP  & 0,5 ETP   & Soutien, addictologie \\
Diététicien(ne)     & 0,3 ETP  & 0,5 ETP   & Bilans nutritionnels, ETP \\
\midrule
\textbf{Total PNM}  & \textbf{$\sim$7 ETP} & \textbf{$\sim$11,5 ETP} & — \\
\bottomrule
\end{tabular}
\end{table}

\begin{table}[!htbp]
\centering
\caption{Vacations médicales hebdomadaires (PM)}
\label{tab:effectifs_pm}
\small
\rowcolors{2}{APHPsoft}{white}
\renewcommand{\arraystretch}{1.35}
\begin{tabular}{@{}p{6.6cm}p{3.0cm}p{3.0cm}@{}}
\toprule
\textbf{Spécialité médicale} & \textbf{Phase pilote} & \textbf{Maturité} \\
\midrule
Hépatologie (PBH, cirrhose)        & 2 demi-j/sem & 4 demi-j/sem \\
MICI (biothérapies)                 & 2 demi-j/sem & 3 demi-j/sem \\
Addictologie                        & 1 demi-j/sem & 2 demi-j/sem \\
Oncologie digestive                  & 1 demi-j/sem & 2 demi-j/sem \\
Radiologie interventionnelle        & —            & 1 demi-j/sem \\
\midrule
\textbf{Total vacations/semaine}    & \textbf{6 demi-j} & \textbf{12 demi-j} \\
\bottomrule
\end{tabular}
\end{table}

\paragraph{Plan de formation à la polyvalence}
La sectorisation par niveau n'exclut pas une montée en compétences croisée permettant les renforts ponctuels ; un plan sur 12 mois est prévu avant ouverture.

\begin{table}[!htbp]
\centering
\caption{Plan de formation croisée — Compétences transversales IDE}
\label{tab:formation_ide}
\small
\rowcolors{2}{APHPsoft}{white}
\renewcommand{\arraystretch}{1.35}
\begin{tabular}{@{}p{4.6cm}p{3.1cm}p{2.5cm}p{2.5cm}@{}}
\toprule
\textbf{Compétence} & \textbf{Public cible} & \textbf{Durée} & \textbf{Validation} \\
\midrule
Surveillance post-RI (embolisation, drainage) & IDE niveaux 2–3 & 3 jours & Attestation RI \\
Chimiothérapies digestives                    & IDE niveaux 1–3      & 5 jours & Habilitation URC \\
Gestion des sevrages (CIWA, anxiolyse)        & IDE niveaux 1–2   & 2 jours & Attestation addicto \\
Biothérapies IV (perfusion, EI)               & Toutes IDE             & 2 jours & Attestation MICI \\
Surveillance post-PBH                         & IDE niveaux 2–3    & 1 jour  & Compagnonnage \\
\bottomrule
\end{tabular}
\end{table}

\noindent\textbf{Principe :} chaque IDE maîtrise son niveau principal et \textbf{au moins une compétence transversale} permettant le renfort inter-niveaux. L'affectation quotidienne tient compte des habilitations validées.

\paragraph{Ratios de sécurité}
Les ratios IDE/patients sont différenciés par niveau :
\begin{itemize}[leftmargin=1.1cm]
  \item \textbf{Niveau 1} : 1 IDE / 3 patients (surveillance renforcée post-acte).
  \item \textbf{Niveau 2} : 1 IDE / 4 patients (soins techniques standards).
  \item \textbf{Niveau 3} : 1 IDE / 6 patients (coordination, faible technicité), ajustable selon charge et complexité.
\end{itemize}

\paragraph{Benchmark institutionnel}
Ce dimensionnement est aligné sur les ratios observés dans les HDJ oncologiques et hématologiques 
de l'AP-HP (Pitié-Salpêtrière, Saint-Louis, Cochin-Hématologie), qui fonctionnent avec des ratios 
comparables pour des niveaux de technicité équivalents. \textbf{Le modèle a été validé par la direction 
des soins du GHU lors de la phase de cadrage}, après analyse comparative des effectifs IDE dans les structures HDJ de même dimension et de même technicité au sein de l'AP-HP.

\noindent
En configuration maximale (18 patients simultanés), l'effectif requis est compatible avec le dimensionnement (6–7 ETP IDE) permettant la couverture des absences.

\FloatBarrier

% ============================================================
% 3.10bis PLAN RH D'OUVERTURE
% ============================================================
\subsection*{Plan RH d'ouverture —}
\label{sec:plan_rh_ouverture}

\noindent
L'ouverture du plateau HDJ repose sur une stratégie RH sécurisée, articulée autour de trois axes : (i) identification des ressources existantes redéployables, (ii) recrutement ciblé, (iii) phasage progressif de l'ouverture permettant une montée en compétences maîtrisée.

\paragraph{Origine des IDE pilotes (phase 1 — 4 ETP)}
\begin{table}[!htbp]
\centering
\caption{Sourcing des 4 IDE pilotes — Phase d'ouverture}
\label{tab:sourcing_ide}
\small
\rowcolors{2}{APHPsoft}{white}
\renewcommand{\arraystretch}{1.35}
\begin{tabular}{@{}p{3.8cm}p{1.5cm}p{4.5cm}p{3.5cm}@{}}
\toprule
\textbf{Origine} & \textbf{ETP} & \textbf{Profil} & \textbf{Statut} \\
\midrule
HDJ Oncologie Cochin (redéploiement partiel) & 1,0 & IDE expérimentée chimio/biothérapies & Accord cadre obtenu \\
Service Maladies du Foie (mobilité interne) & 1,0 & IDE expérimentée PBH/ascite & Candidature identifiée \\
HDJ MICI Saint-Antoine (mobilité GHU) & 1,0 & IDE expérimentée biothérapies IV & Contact en cours \\
Recrutement externe (poste publié) & 1,0 & IDE diplômée, formation prévue & Publication M-6 \\
\midrule
\textbf{Total phase pilote} & \textbf{4,0} & — & — \\
\bottomrule
\end{tabular}
\end{table}

\paragraph{Calendrier de recrutement — Phases 2 et 3}
\begin{table}[!htbp]
\centering
\caption{Calendrier de montée en charge RH IDE}
\label{tab:calendrier_rh}
\small
\rowcolors{2}{APHPsoft}{white}
\renewcommand{\arraystretch}{1.35}
\begin{tabular}{@{}p{2.5cm}p{2.5cm}p{2.5cm}p{5.5cm}@{}}
\toprule
\textbf{Phase} & \textbf{Échéance} & \textbf{ETP cumulés} & \textbf{Actions} \\
\midrule
Pré-ouverture & M-6 à M-3 & — & Publication postes, entretiens, sélection \\
Phase 1 (pilote) & M0 & 4 ETP & Ouverture 6 places, formation sur site \\
Phase 2 & M+6 & 5,5 ETP & Extension 12 places, recrutement 1,5 ETP \\
Phase 3 (maturité) & M+12 & 6–7 ETP & Extension 18 places, ajustement si besoin \\
\bottomrule
\end{tabular}
\end{table}

\paragraph{Phasage de l'ouverture capacitaire}
L'ouverture progressive permet de sécuriser la montée en compétences et d'ajuster les effectifs en fonction de l'activité réelle.

\begin{table}[!htbp]
\centering
\caption{Phasage de l'ouverture — Capacité et activité}
\label{tab:phasage_ouverture}
\small
\rowcolors{2}{APHPsoft}{white}
\renewcommand{\arraystretch}{1.35}
\begin{tabular}{@{}p{2.5cm}p{3.5cm}p{2.5cm}p{3.5cm}p{2.0cm}@{}}
\toprule
\textbf{Phase} & \textbf{Capacité ouverte} & \textbf{ETP IDE} & \textbf{Activité cible} & \textbf{Échéance} \\
\midrule
Phase 1 (pilote) & 6 places (2 lits + 2 fauteuils + 2 boxes) & 4 ETP & $\sim$1\,500 séances/an & M0 \\
Phase 2 & 12 places (4 lits + 4 fauteuils + 4 boxes) & 5,5 ETP & $\sim$3\,100 séances/an & M+6 \\
Phase 3 (maturité) & 18 places (6 lits + 6 fauteuils + 6 boxes) & 6–7 ETP & $\sim$4\,675 séances/an & M+12 \\
\bottomrule
\end{tabular}
\end{table}

\paragraph{Conditions de passage inter-phases}
Le passage d'une phase à la suivante est conditionné à :
\begin{enumerate}[leftmargin=1.1cm]
  \item \textbf{Taux d'occupation $>$80\%} sur la phase en cours (moyenne sur 2 mois).
  \item \textbf{Effectif IDE complet} pour la phase suivante (recrutement validé ou en poste).
  \item \textbf{Absence d'incident grave} (EI de niveau 3 ou plus) non résolu.
  \item \textbf{Validation COPIL} sur proposition du responsable médical et du cadre de santé.
\end{enumerate}

\paragraph{Plan de contingence}
En cas de difficulté de recrutement :
\begin{itemize}[leftmargin=1.1cm]
  \item \textbf{Option 1} : mobilisation temporaire du pool IDE de remplacement du GHU (convention existante).
  \item \textbf{Option 2} : contrat CDD de 6 mois pour sécuriser la transition, avec objectif de CDisation.
  \item \textbf{Option 3} : ajustement du phasage (maintien phase intermédiaire jusqu'à recrutement effectif).
\end{itemize}

\noindent
\textbf{Point d'attention.} Le recrutement du poste externe (1 ETP) constitue le point critique du calendrier. La publication anticipée (M-6) et l'attractivité du projet (polyvalence, technicité, qualité hôtelière) constituent des leviers favorables, confirmés par l'intérêt exprimé lors des premières consultations informelles avec des IDE du GHU.

\FloatBarrier

% ============================================================
% 3.11 SCHÉMA FONCTIONNEL
% ============================================================
\subsection*{Schéma fonctionnel du plateau —}

\noindent
Le schéma illustre le parcours patient depuis l'orientation jusqu'à la sortie, avec classification par niveau et pool mutualisé, ainsi que la gestion des complications.

\vspace{0.8cm}

\begin{center}
\begin{tikzpicture}[
  node distance=1.2cm,
  box/.style={
    rectangle, rounded corners=3pt,
    draw=APHPdark, thick,
    text width=5.2cm, minimum height=1.1cm,
    align=center, fill=APHPsoft
  },
  levelbox/.style={
    rectangle, rounded corners=3pt,
    draw=APHPblue, thick,
    text width=3.2cm, minimum height=0.9cm,
    align=center, fill=white
  },
  arrow/.style={->, thick, APHPdark}
]
\node[box] (orientation) {Orientation \\ Consultation / Ville / Urgences};
\node[box, below=of orientation] (classif) {Classification patient \\ Niveau 1, 2 ou 3};
\node[box, below=of classif] (accueil) {Accueil centralisé \\ Admission unique};

\node[levelbox, below left=1.5cm and -0.5cm of accueil] (n1) {Niveau 1 \\ 6 chambres};
\node[levelbox, below=1.5cm of accueil] (n2) {Niveau 2 \\ 6 fauteuils};
\node[levelbox, below right=1.5cm and -0.5cm of accueil] (n3) {Niveau 3 \\ 6 boxes};

\node[box, below=3.5cm of accueil] (sortie) {Sortie sécurisée \\ Synthèse + Suivi};
\node[box, right=2.5cm of classif] (pool) {Pool mutualisé \\ Vases communicants};

\node[box, below right=2cm and 3cm of sortie, fill=red!10] (hospit) {HC\\ ou SAU};

\draw[arrow] (orientation) -- (classif);
\draw[arrow] (classif) -- (accueil);
\draw[arrow] (accueil) -- (n1);
\draw[arrow] (accueil) -- (n2);
\draw[arrow] (accueil) -- (n3);
\draw[arrow] (n1) -- (sortie);
\draw[arrow] (n2) -- (sortie);
\draw[arrow] (n3) -- (sortie);
\draw[arrow, dashed] (pool) -- (accueil);

\draw[arrow, red, dashed] (n1) to[bend right=20] node[midway, right, font=\scriptsize] {Complication} (hospit);
\draw[arrow, red, dashed] (n2) to[bend right=10] (hospit);

\end{tikzpicture}
\end{center}

\begin{center}
\captionof{figure}{Schéma fonctionnel du plateau HDJ avec classification par niveau et pool mutualisé}
\label{fig:schema_fonctionnel}
\end{center}

\paragraph{Circuit de l'aval — Gestion des complications}
En cas de complication nécessitant une hospitalisation conventionnelle :
\begin{enumerate}[leftmargin=1.1cm]
  \item \textbf{Évaluation immédiate} par le médecin senior du plateau.
  \item \textbf{Contact du service d'aval} : Maladies du Foie, Gastroentérologie, SAU, Réanimation médicale si défaillance vitale.
  \item \textbf{Transfert sécurisé} avec dossier de liaison et transmissions IDE-IDE.
  \item \textbf{Traçabilité} : signalement EI si applicable, analyse en CODIR.
\end{enumerate}

\noindent
\textbf{Objectif :} taux de transfert non programmé vers HC/SAU $<$ 0,5\% des séances.

\FloatBarrier

% ============================================================
% 3.12 GOUVERNANCE
% ============================================================
\subsection*{Gouvernance —}

\noindent
Le plateau est placé sous une responsabilité médicale unique. La gouvernance s'appuie sur deux instances complémentaires.

\paragraph{Comité de pilotage stratégique (COPIL) — Trimestriel}
\begin{itemize}[leftmargin=1.1cm]
  \item \textbf{Composition} : responsable médical, référents filières, cadre supérieur, direction des soins, direction médico-économique.
  \item \textbf{Missions} : orientations stratégiques, arbitrages capacitaires, suivi des indicateurs, validation des passages de phase.
\end{itemize}

\paragraph{Comité opérationnel (CODIR) — Mensuel}
\begin{itemize}[leftmargin=1.1cm]
  \item \textbf{Composition} : responsable médical, cadre de santé, IPA, secrétariat, pharmacie.
  \item \textbf{Missions} : suivi opérationnel, gestion des flux, résolution des dysfonctionnements, analyse EI.
\end{itemize}

% ============================================================
% 3.12bis TABLEAU DE BORD
% ============================================================
\subsection*{Pilotage et qualité — Tableau de bord —}

\begin{table}[H]
\centering
\caption{Indicateurs de pilotage}
\label{tab:kpi}
\small
\rowcolors{2}{APHPsoft}{white}
\renewcommand{\arraystretch}{1.35}
\begin{tabular}{@{}p{3cm}p{8cm}p{2.2cm}@{}}
\toprule
\textbf{Domaine} & \textbf{Indicateurs} & \textbf{Fréquence} \\
\midrule
Activité          & Séances/filière/niveau, taux d'occupation, délais de programmation, annulations & Mensuel \\
Médico-économique & Recettes/filière, écart au budget, coût moyen/séance                           & Mensuel \\
Qualité/sécurité  & Complications, transferts non programmés, réhospitalisations à 30 j, satisfaction, EI & Trimestriel \\
RH                & Absentéisme, vacances de postes, heures supplémentaires, formations réalisées   & Trimestriel \\
\bottomrule
\end{tabular}
\end{table}

\paragraph{Règlement intérieur de l'unité}
Un règlement intérieur, validé par le COPIL avant ouverture, définit les règles d'arbitrage en cas de conflit de programmation :
\begin{enumerate}[leftmargin=1.1cm]
  \item \textbf{Priorisation médicale} : application de la grille de priorisation (cf.\ §3.7). En cas d'égalité, le médecin senior présent tranche.
  \item \textbf{Escalade} : en cas de conflit inter-filières, contact du responsable médical de l'UF (ou délégué) sous 15 minutes.
  \item \textbf{Traçabilité} : arbitrage consigné dans le DPI (motif, décideur, solution). Reporting mensuel en CODIR.
  \item \textbf{Recours} : litiges récurrents portés au COPIL trimestriel pour ajustement des créneaux.
\end{enumerate}

\noindent
Le règlement intérieur précise également : horaires d'ouverture, procédure d'annulation tardive (<24h), gestion des urgences relatives, circuit de signalement des EI.

\FloatBarrier

% ============================================================
% 3.13 MODÈLE ÉCONOMIQUE UF — SIMPLIFIÉ
% ============================================================
\subsection*{Modèle économique de l'unité fonctionnelle —}
\label{sec:modele_economique_uf}

\noindent
Le plateau HDJ est constitué en \textbf{unité fonctionnelle (UF) prestataire} au service des UF cliniques prescriptrices. Ce modèle, aligné sur les pratiques AP-HP standard, garantit une traçabilité comptable sans retraitement DIM complexe.

\paragraph{Principe d'imputation directe}
Les recettes sont imputées selon le médecin prescripteur :
\begin{itemize}[leftmargin=1.1cm]
  \item Chaque séance est rattachée à l'\textbf{UF du médecin prescripteur} (via le code UF de prescription dans le dossier patient) ;
  \item Les \textbf{recettes T2A} (GHS, suppléments, molécules onéreuses) sont affectées à l'UF prescriptrice ;
  \item Les \textbf{charges de fonctionnement} du plateau (personnel, consommables) sont mutualisées et réparties au prorata de l'activité réalisée.
\end{itemize}

\paragraph{Avantages du modèle}
\begin{itemize}[leftmargin=1.1cm]
  \item \textbf{Traçabilité native} : pas de retraitement comptable ni de clé de répartition à négocier ;
  \item \textbf{Incitation} : chaque UF clinique est incitée à programmer ses patients pour bénéficier des recettes ;
  \item \textbf{Simplicité} : absence de négociation annuelle de redistribution.
\end{itemize}

\paragraph{Périmètre}
L'imputation directe s'applique aux \textbf{recettes T2A} (GHS, suppléments, molécules onéreuses en sus). Les dotations MERRI, enveloppes MIGAC et recettes liées aux essais cliniques restent affectées selon les règles institutionnelles en vigueur.

\paragraph{Suivi}
Le DIM produit un reporting mensuel par UF prescriptrice (volume, recettes), consolidé trimestriellement en COPIL pour suivi de l'équilibre inter-filières.

\end{spacing}
\printbibliography[heading=subbibliography,title={Références}]
\end{refsection}

% ==========================================
% CHAPITRE E4 — DIMENSIONNEMENT
% ==========================================
% \clearpage
% \section{Dimensionnement et Capacité Cible des HDJ Digestifs Mutualisés}
% \begin{refsection}
% % ============================================================
% PRINCIPES DE DIMENSIONNEMENT CAPACITAIRE ET IMPACT FINANCIER
% ============================================================

\titleformat{\subsection}[runin]
  {\bfseries\color{APHPblue}}
  {}
  {0pt}
  {}

% ============================================================
% PRINCIPE GÉNÉRAL
% ============================================================

\subsection*{Principe général de dimensionnement capacitaire}

Le dimensionnement du plateau HDJ repose sur une approche capacitaire pragmatique, fondée sur l’intensité réelle de surveillance requise, la durée moyenne des prises en charge et les contraintes opérationnelles observées en pratique. Les hypothèses retenues correspondent volontairement à une \textbf{phase de démarrage prudente}, permettant une montée en charge progressive sans modification structurelle des locaux ni des effectifs.

Le dimensionnement présenté définit une \textbf{capacité annuelle maximale théorique}, constituant un plafond organisationnel et une réserve d’absorption des fluctuations d’activité, et non un objectif de production immédiat.

La capacité physique est structurée selon trois modalités d’accueil ambulatoire distinctes, correspondant à des niveaux croissants d’intensité de soins :
\begin{itemize}[leftmargin=1.1cm]
    \item \textbf{Lits médicalisés} : surveillance continue, patient allongé, actes ou pathologies à risque.
    \item \textbf{Fauteuils de soins} : surveillance infirmière directe pour prises en charge intermédiaires.
    \item \textbf{Espace non allongé de type \emph{lounge}} : patients autonomes relevant de parcours coordonnés, multidisciplinaires et programmés, sans besoin de surveillance continue.
\end{itemize}

Cette segmentation permet d’adapter finement les ressources physiques et humaines aux besoins cliniques, tout en évitant un surdimensionnement hospitalier en lits conventionnels.

% ============================================================
% HYPOTHÈSES OPÉRATIONNELLES
% ============================================================

\subsection*{Traduction en capacité physique (hypothèses opérationnelles conservatrices)}

La conversion des besoins en capacité physique repose sur des hypothèses volontairement conservatrices, intégrant :
\begin{itemize}[leftmargin=1.1cm]
    \item les aléas organisationnels (annulations, reports, \emph{no-show}) ;
    \item les contraintes de ressources humaines spécialisées ;
    \item la variabilité inter-individuelle des durées de prise en charge ;
    \item les indisponibilités ponctuelles des plateaux techniques.
\end{itemize}

Le fonctionnement du plateau est estimé sur une base réaliste de \textbf{44 semaines réellement opérées par an}, à raison de \textbf{5 jours par semaine}, soit \textbf{220 jours ouvrés/an}.

Les hypothèses de productivité retenues sont les suivantes :
\begin{itemize}[leftmargin=1.1cm]
    \item \textbf{Lits médicalisés} : 1 patient/lit/jour, coefficient de réalisation \textbf{0,85}.
    \item \textbf{Fauteuils de soins} : 2 patients/fauteuil/jour, coefficient \textbf{0,80}.
    \item \textbf{Fauteuils lounge} : capacité théorique maximale de 3 patients/fauteuil/jour. La productivité opérationnelle retenue est de \textbf{2 patients/fauteuil/jour}, avec application d’un coefficient de réalisation \textbf{0,75}, intégrant la variabilité des durées de séjour, les temps non productifs et les aléas organisationnels.
\end{itemize}

La capacité annuelle théorique est estimée selon la relation :
\[
HDJ/an = 220 \times \big(
N_L \times 1.0 \times 0.85
+ N_F \times 2.0 \times 0.80
+ N_{Lg} \times 2.0 \times 0.75
\big).
\]

% ============================================================
% DIMENSIONNEMENT CAPACITAIRE
% ============================================================

\subsection*{Dimensionnement capacitaire du plateau}

Sur cette base, un dimensionnement initial permettant une \textbf{capacité annuelle maximale comprise entre 4\,700 et 5\,000 séances d’HDJ} est le suivant :
\begin{itemize}[leftmargin=1.1cm]
    \item \textbf{6 lits médicalisés},
    \item \textbf{6 fauteuils de soins},
    \item \textbf{5 fauteuils lounge} (capacité assise non allongée).
\end{itemize}

Ce calibrage correspond à une capacité annuelle théorique de :
\[
220 \times (6\times1.0\times0.85
+ 6\times2.0\times0.80
+ 5\times2.0\times0.75)
= \textbf{4\,884 HDJ/an}.
\]

Cette capacité constitue un plafond organisationnel, cohérent avec une trajectoire d’activité projetée à maturité médico-économique inférieure, tout en laissant une marge d’absorption substantielle pour les fluctuations d’activité et l’évolution future des filières.

% ============================================================
% IMPACT MÉDICO-ÉCONOMIQUE
% ============================================================

\subsection*{Traduction médico-économique}

La répartition capacitaire permet une allocation différenciée des activités selon leur intensité de soins et leur rendement médico-économique. Les lits médicalisés concentrent les prises en charge complexes à forte valeur ajoutée clinique, tandis que les fauteuils de soins et l’espace \emph{lounge} absorbent des volumes élevés d’actes programmés à forte efficience organisationnelle.

\begin{sidewaystable}[p]
\centering
\renewcommand{\arraystretch}{1.25}
\rowcolors{2}{APHPsoft}{white}

\begin{tabular}{
p{4.6cm}
>{\centering\arraybackslash}p{1.6cm}
>{\centering\arraybackslash}p{3.0cm}
>{\centering\arraybackslash}p{3.0cm}
>{\centering\arraybackslash}p{3.4cm}
>{\centering\arraybackslash}p{3.4cm}
}
\toprule
\textbf{Type de capacité} &
\textbf{N} &
\textbf{Hypothèse de flux} &
\textbf{Capacité annuelle} &
\textbf{Activités principales} &
\textbf{Recettes annuelles estimées} \\
\midrule
Lits médicalisés &
6 &
1 patient/lit/jour $\times$ 0.85 &
$\approx$ 1\,122 &
PBH, cirrhoses avancées, chimiothérapies prolongées, RI post-acte &
$\approx$ 1.5–1.7 M€ \\
Fauteuils de soins &
6 &
2 patients/fauteuil/jour $\times$ 0.80 &
$\approx$ 2\,112 &
Fer IV, albumine, biothérapies courtes, chimiothérapies simples &
$\approx$ 0.6–0.7 M€ \\
Fauteuils lounge &
5 &
2 patients/fauteuil/jour $\times$ 0.75 &
$\approx$ 1\,650 &
Hépatométabolique, cirrhoses compensées, addictologie, parcours coordonnés &
$\approx$ 0.6–0.8 M€ \\
\midrule
\textbf{Total plateau} &
— &
— &
\textbf{$\approx$ 4\,884} &
— &
\textbf{$\approx$ 3 M€} \\
\bottomrule
\end{tabular}

\caption{Synthèse capacitaire et médico-économique du plateau HDJ selon des hypothèses conservatrices. La capacité présentée correspond à un plafond organisationnel, la trajectoire d’activité cible étant volontairement inférieure. L’espace \emph{lounge} constitue une capacité assise non allongée optimisant les flux sans création de lits supplémentaires.}
\end{sidewaystable}

% \printbibliography[heading=subbibliography,title={Références}]
% \end{refsection}

% ==========================================
% CHAPITRE E5 — CONCLUSION GÉNÉRALE
% ==========================================
\clearpage
\section{Conclusion Générale}

\begin{refsection}
% ============================================================
% CONCLUSION GÉNÉRALE
% ============================================================

\section*{Conclusion générale}
\addcontentsline{toc}{section}{Conclusion générale}

\begin{spacing}{1.25}

La consolidation des activités d’hôpital de jour digestif, interventionnel et addictologique au sein
d’un plateau unique constitue un levier structurant d’efficience pour le GHU AP--HP.Centre.
L’unification organisationnelle permet une augmentation substantielle de la capacité de prise en
charge, une amélioration mesurable de l’accessibilité aux parcours spécialisés et une réduction
documentée des hospitalisations conventionnelles évitables, dans un contexte durable de tension
capacitaire et de ressources humaines contraintes.

La spécificité du projet repose sur une organisation graduée des prises en charge ambulatoires,
distinguant trois niveaux de complexité et d’intensité de surveillance. Cette segmentation permet
d’allouer les ressources médicales et paramédicales au plus près des besoins médicaux réels,
tout en évitant un surdimensionnement en lits hospitaliers. Les prises en charge nécessitant une
surveillance continue sont concentrées sur des lits médicalisés, tandis que les traitements
intermédiaires sont réalisés sur des fauteuils dédiés.

Les parcours d’évaluation multidisciplinaire et de coordination — notamment en hépatologie
métabolique, cirrhose compensée, addictologie et certaines prises en charge MICI — sont organisés
au sein d’un espace non allongé de type \emph{lounge}. Inspiré des standards de confort et de
fluidité des salons premium, cet espace accueille des patients autonomes ne nécessitant ni
alitement ni surveillance continue, mais requérant un temps médical et paramédical structuré,
séquentiel et à forte valeur ajoutée. Cette organisation optimise les flux, améliore l’expérience
patient et constitue un levier majeur d’efficience capacitaire.

À régime de croisière, l’activité consolidée du plateau est estimée entre \textbf{4\,700 et 5\,000
séances d’hôpital de jour par an}, pour des \textbf{recettes annuelles proches de 3\,M€}. Ce modèle
positionne le site Cochin comme un centre expert régional de l’ambulatoire digestif, en cohérence
avec les orientations stratégiques de l’AP--HP, les recommandations des sociétés savantes et les
standards internationaux d’organisation des parcours complexes.

\end{spacing}
% ============================================================
% PRINCIPES DE CAPACITÉ ET NIVEAUX DE PRISE EN CHARGE
% ============================================================

\subsection*{Principe général de dimensionnement}

Le dimensionnement du plateau repose sur une distinction fonctionnelle entre trois niveaux
d’accueil ambulatoire, définis selon l’intensité de surveillance requise et la durée des prises en
charge. Les hypothèses retenues correspondent volontairement à une \textbf{phase de démarrage
prudente}, permettant une montée en charge ultérieure sans modification structurelle.

\begin{itemize}[leftmargin=1.1cm]
    \item \textbf{Niveau~1~: lits médicalisés} — surveillance continue, patient allongé.
    \item \textbf{Niveau~2~: fauteuils de soins} — surveillance infirmière directe, durée intermédiaire.
    \item \textbf{Niveau~3~: espace non allongé de type lounge} — patients autonomes, coordination
    multidisciplinaire sans besoin de lit.
\end{itemize}
% ============================================================
% TRADUCTION EN CAPACITÉ PHYSIQUE — HYPOTHÈSES CONSERVATRICES
% ============================================================

\subsection*{Traduction en capacité physique (hypothèses conservatrices)}

La traduction des hypothèses de flux en capacité physique repose sur des hypothèses
volontairement conservatrices, intégrant les aléas organisationnels (annulations, no-show),
les contraintes de ressources humaines, la variabilité des durées de prise en charge et les
indisponibilités ponctuelles des plateaux techniques.

Le plateau est supposé fonctionner sur une base de \textbf{44 semaines réellement opérées par an},
à raison de \textbf{5 jours par semaine}, soit \textbf{220 jours ouvrés/an}.

Les hypothèses de réalisation retenues sont :
\begin{itemize}[leftmargin=1.1cm]
    \item \textbf{Lits médicalisés} : 1 patient/lit/jour avec un coefficient de réalisation de \textbf{0.85}.
    \item \textbf{Fauteuils de soins} : 2 patients/fauteuil/jour avec un coefficient de réalisation de \textbf{0.80}.
    \item \textbf{Fauteuils lounge} : 2 patients/fauteuil/jour avec un coefficient de réalisation de \textbf{0.75}.
\end{itemize}

La capacité annuelle attendue est calculée selon la formule :
\[
HDJ/an = 220 \times \big(
N_L \times 1.0 \times 0.85
+ N_F \times 2.0 \times 0.80
+ N_{Lg} \times 2.0 \times 0.75
\big).
\]

Sur cette base, un dimensionnement conservateur compatible avec une cible de
\textbf{4\,700 à 5\,000 HDJ/an} est le suivant :
\begin{itemize}[leftmargin=1.1cm]
    \item \textbf{6 lits médicalisés},
    \item \textbf{6 fauteuils de soins},
    \item \textbf{5 fauteuils lounge} (espace non allongé de type \emph{salon premium}).
\end{itemize}

Ce calibrage correspond à une capacité annuelle théorique de :
\[
220 \times (6\times1.0\times0.85
+ 6\times2.0\times0.80
+ 5\times2.0\times0.75)
= \textbf{4\,884 HDJ/an}.
\]
% ============================================================
% TABLEAU SYNTHÉTIQUE — CAPACITÉ PHYSIQUE, ACTIVITÉ ET RECETTES
% ============================================================

\begin{sidewaystable}[p]
\centering
\renewcommand{\arraystretch}{1.25}
\rowcolors{2}{APHPsoft}{white}

\begin{tabular}{
p{4.6cm}
>{\centering\arraybackslash}p{1.6cm}
>{\centering\arraybackslash}p{3.0cm}
>{\centering\arraybackslash}p{3.0cm}
>{\centering\arraybackslash}p{3.4cm}
>{\centering\arraybackslash}p{3.4cm}
}
\toprule
\textbf{Type de capacité} &
\textbf{N (unités)} &
\textbf{Hypothèse de flux (conservatrice)} &
\textbf{Capacité annuelle estimée} &
\textbf{Activités principales concernées} &
\textbf{Recettes annuelles estimées \newline (ensemble du sous-plateau)} \\
\midrule
Lits médicalisés &
\textbf{6} &
1 patient/lit/jour $\times$ 0.85 &
$\approx$ 1\,122 HDJ/an &
PBH, cirrhoses avancées, chimiothérapies digestives prolongées,
radiologie interventionnelle post-acte &
$\approx$ 1.5--1.7 M€ \\
Fauteuils de soins &
\textbf{6} &
2 patients/fauteuil/jour $\times$ 0.80 &
$\approx$ 2\,112 HDJ/an &
Fer IV, albumine seule, biothérapies courtes,
chimiothérapies simples &
$\approx$ 0.6--0.7 M€ \\
Fauteuils lounge (non allongé) &
\textbf{5} &
2 patients/fauteuil/jour $\times$ 0.75 &
$\approx$ 1\,650 HDJ/an &
Hépatométabolique, évaluation des cirrhoses,
addictologie, parcours multidisciplinaires &
$\approx$ 0.6--0.8 M€ \\
\midrule
\textbf{Total plateau HDJ} &
— &
— &
\textbf{$\approx$ 4\,884 HDJ/an} &
— &
\textbf{$\approx$ 3 M€} \\
\bottomrule
\end{tabular}

\caption{Dimensionnement physique du plateau d’hôpital de jour selon des hypothèses
volontairement conservatrices. Le \emph{lounge} constitue une capacité assise premium,
non assimilée à des lits, permettant l’optimisation des flux sans surdimensionnement
hospitalier.}
\end{sidewaystable}

\printbibliography[heading=subbibliography,title={Références}]
\end{refsection}


% ============================================================
% ORGANISATION OPÉRATIONNELLE DES HDJ
% ============================================================
\clearpage
\phantomsection
\part*{Organisation Opérationnelle des HDJ}
\addcontentsline{toc}{part}{Organisation Opérationnelle des HDJ}

% --- Début bloc HDJ : reset numérotation + ancres hyperref uniques
\renewcommand{\HsecPrefix}{hdj}
\setcounter{section}{0}

\thispagestyle{empty}

\vspace*{4cm}
\begin{center}
{\Huge\bfseries\color{APHPdark} Hôpitaux de Jour Digestifs}\\[1.2cm]
\color{APHPblue}\rule{0.55\textwidth}{0.6pt}\\[1.2cm]
{\Large Organisation opérationnelle des HDJ}\\[0.6cm]
{\large Fiches standardisées de prise en charge ambulatoire spécialisée}
\end{center}

\clearpage

% ===============================================
% CHAPITRE O0: SOCLE MEDICAL AUX HDJ
% ===============================================
\section{Socle Médical Transversal des HDJ Digestifs}
\begin{refsection}
% ============================================================
% SOCLE MÉDICAL TRANSVERSAL — HDJ DIGESTIFS
% ============================================================
\titleformat{\subsection}[runin]
  {\bfseries\color{APHPblue}}
  {}
  {0pt}
  {}

%\section{Socle médical transversal des HDJ digestifs}
\label{sec:socle_medical}

\subsection*{Virage ambulatoire et parcours complexes}

Le développement des hôpitaux de jour digestifs s’inscrit pleinement dans le cadre du virage
ambulatoire, qui constitue un axe stratégique majeur des politiques de santé hospitalières.
Il répond à l’augmentation continue des pathologies chroniques, évolutives et complexes,
nécessitant des évaluations répétées, programmables et coordonnées, sans justification
d’hospitalisation conventionnelle prolongée dès lors qu’une organisation structurée,
sécurisée et multidisciplinaire est disponible.

Les HDJ permettent de concentrer, sur un temps court, des actes diagnostiques, thérapeutiques
et éducatifs à forte valeur ajoutée médicale, tout en assurant la continuité des parcours
ville–hôpital, la traçabilité universitaire des prises en charge et la sécurisation des circuits
de soins pour des patients souvent fragiles ou à risque.

\subsection*{Maladies hépatiques chroniques et métaboliques}

Les maladies hépatiques chroniques constituent un déterminant majeur de morbi-mortalité
digestive et représentent une part croissante de l’activité spécialisée. Les pathologies
métaboliques hépatiques (MASLD/MASH), les cirrhoses compensées et décompensées, ainsi que
leurs complications (hypertension portale, dénutrition, sarcopénie, complications rénales
et infectieuses) génèrent des besoins importants d’évaluations itératives et coordonnées.

Les recommandations internationales (EASL, Baveno~VII, AFEF) promeuvent des parcours
structurés reposant sur des outils non invasifs, l’imagerie spécialisée, l’évaluation
nutritionnelle et comportementale, et une stratification fine du risque. Ces approches sont
particulièrement compatibles avec une organisation en hôpital de jour, permettant
d’optimiser le suivi tout en limitant le recours à l’hospitalisation complète.

\subsection*{Maladies inflammatoires chroniques de l’intestin}

Les maladies inflammatoires chroniques de l’intestin (MICI) représentent des pathologies
chroniques à forte intensité de suivi, marquées par l’essor des biothérapies, des stratégies
de monitorage thérapeutique et des évaluations multidisciplinaires régulières. De nombreuses
étapes du parcours — initiation ou optimisation thérapeutique, surveillance biologique et
clinique, éducation thérapeutique, coordination paramédicale — relèvent désormais d’une
prise en charge ambulatoire spécialisée.

Les HDJ dédiés aux MICI permettent de sécuriser ces parcours complexes, de réduire les
hospitalisations évitables liées aux poussées ou aux effets indésirables, et d’améliorer la
réactivité thérapeutique dans un cadre standardisé et expert.

\subsection*{Oncologie digestive et traitements systémiques}

La prise en charge des cancers digestifs connaît une transformation profonde avec le recours
croissant aux chimiothérapies, immunothérapies et traitements ciblés administrés en
ambulatoire. Ces prises en charge nécessitent une organisation rigoureuse intégrant
évaluation clinique pré-thérapeutique, surveillance des toxicités, coordination étroite avec
les équipes d’oncologie et accès rapide au plateau technique.

Les HDJ d’oncologie digestive constituent aujourd’hui le cadre organisationnel de référence
pour concilier sécurité des soins, efficience hospitalière et amélioration de la qualité de vie
des patients, tout en optimisant l’utilisation des lits d’hospitalisation conventionnelle.

\subsection*{Addictologie et pathologies associées}

Les troubles liés à l’usage de l’alcool et des substances psychoactives constituent un facteur
majeur de morbi-mortalité digestive, métabolique et systémique. Ils interfèrent de manière
significative avec l’évolution des maladies hépatiques, oncologiques et métaboliques, et
complexifient les parcours de soins.

Leur prise en charge efficace repose sur des parcours intégrés associant évaluation
somatique, accompagnement addictologique, approche motivationnelle, éducation
thérapeutique et coordination pluridisciplinaire. Les HDJ addictologiques offrent un cadre
structuré et lisible pour ces évaluations complexes, favorisant l’adhésion des patients, la
prévention des complications et la réduction des hospitalisations non programmées.

\subsection*{Radiologie interventionnelle et actes programmés}

L’essor des techniques de radiologie interventionnelle mini-invasive a profondément modifié
les parcours digestifs et oncologiques, permettant la réalisation d’un nombre croissant
d’actes en ambulatoire. Ces procédures nécessitent une sélection rigoureuse des patients,
une préparation standardisée, une surveillance post-acte adaptée et une coordination étroite
entre cliniciens, radiologues et équipes soignantes.

L’intégration de la radiologie interventionnelle au sein d’un plateau HDJ mutualisé garantit
la sécurité des parcours, réduit les durées d’hospitalisation et améliore l’efficience
organisationnelle à l’échelle de l’établissement.

\subsection*{Rationalité médico-économique et organisationnelle}

La concentration des prises en charge digestives complexes en hôpital de jour permet de
réduire les hospitalisations conventionnelles évitables, d’optimiser l’allocation des
ressources médicales et paramédicales spécialisées, et d’améliorer la lisibilité des parcours
sur le territoire.

Ce modèle favorise une montée en charge progressive, une adaptation dynamique des
capacités (lits médicalisés, fauteuils de soins, espaces non allongés), et un pilotage
médico-économique fin, tout en maintenant un haut niveau de qualité, de sécurité et de
traçabilité des soins.

Sur cette base médico-organisationnelle commune, les sections suivantes déclinent,
sous forme de fiches opérationnelles standardisées, les principales activités d’hôpital
de jour digestives du site. Chaque fiche décrit le rationnel médical, la population éligible,
le parcours patient, les actes réalisés, les ressources mobilisées ainsi que les éléments
de codage et de projection d’activité, afin d’assurer une lecture homogène, comparative
et directement exploitable sur le plan organisationnel et médico-économique.

\medskip

\noindent\textit{Note méthodologique — données financières.}
Les montants financiers présentés dans ce document sont exprimés en tarifs T2A en vigueur
au 1\textsuperscript{er} mars 2025. Ils intègrent la majoration régionale Île-de-France (7~\%)
ainsi que la majoration dite «~Ségur~» (3,5~\%), conformément à la réglementation en vigueur.


\printbibliography[heading=subbibliography,title={Références}]
\end{refsection}

% ===============================================
% CHAPITRE 1: PBH
% ===============================================
\clearpage
\section{HDJ Biopsies du Foie}
\begin{refsection}
% ============================================================
% HÔPITAL DE JOUR — BIOPSIE HÉPATIQUE (PBH)
% ============================================================

\subsection{Rationnel médical}
\needspace{8\baselineskip}

\begin{spacing}{1.30}

La biopsie hépatique (PBH) demeure l’examen de référence pour le diagnostic, la classification et le suivi histologique de nombreuses maladies du foie. Malgré les progrès des outils non invasifs, elle reste indispensable dans plusieurs situations cliniques complexes où seule l’analyse tissulaire permet une caractérisation fine et reproductible des lésions.

Elle est essentielle dans les maladies hépatiques rares (hépatites auto-immunes, cholangites, granulomatoses), en particulier lorsque les présentations cliniques ou immunologiques sont atypiques, discordantes ou insuffisamment discriminantes. Dans les toxicités médicamenteuses, notamment celles induites par les immunothérapies (anti–PD-1/PD-L1, anti–CTLA-4) ou certaines thérapies ciblées, la PBH est un examen clé permettant de distinguer des lésions auto-immunes, granulomateuses, cholestatiques ou toxiques. Elle occupe une place centrale dans les algorithmes diagnostiques et décisionnels  \cite{Peeraphatdit_ImmuneHepatitis_2020, DeMartin_IOHepatitis_2020}.

La PBH est également incontournable pour évaluer la réponse aux traitements dans les maladies chroniques du foie. La régression de la fibrose et de la cirrhose constitue un critère pronostique majeur et un objectif essentiel des essais thérapeutiques contemporains. Aucune méthode non invasive ne permet aujourd’hui d’attester la réorganisation architecturale du foie : seule la PBH peut documenter de manière standardisée une réversion histologique, ce qui en fait l’examen de référence dans les études sur la MASLD/NASH et dans les essais visant à démontrer une amélioration de la fibrose \cite{Friedman_FibrosisRegression_2023}.

La PBH conserve une utilité en oncologie hépatique pour caractériser certaines lésions bénignes ou malignes lorsque la confirmation histologique conditionne la prise en charge. En maladies infectieuses, elle peut être indispensable pour rechercher des agents pathogènes difficiles à isoler. En hématologie (SMD, aplasie, LAM), la présence d’une fibrose constitue un facteur pronostique majeur avant allogreffe.

L’Hôpital de Jour fournit un cadre sécurisé : protocole décisionnel, guidage échographique (HLHJ003/HLHJ006), surveillance spécialisée, traçabilité, continuité ville–hôpital.

\end{spacing}

\clearpage

% ============================================================
% OBJECTIFS
% ============================================================

\subsection{Objectifs}
\needspace{6\baselineskip}

\vspace{0.8em}
\begin{center}
\fcolorbox{APHPdark}{APHPsoft}{
\begin{minipage}{0.95\textwidth}
\vspace{0.9em}

\begin{itemize}[leftmargin=1.1cm]
  \item obtenir un diagnostic histologique précis dans les maladies du foie complexes ou atypiques ;
  \item documenter les toxicités médicamenteuses, notamment sous immunothérapie ;
  \item évaluer la fibrose et la réversion structurale dans les protocoles thérapeutiques ;
  \item caractériser des lésions tumorales bénignes ou malignes lorsque cela conditionne la stratégie ;
  \item explorer des atteintes infectieuses hépatiques rares (hépatocultures) ;
  \item réaliser l’examen dans un cadre sécurisé, standardisé et traçable en HDJ.
\end{itemize}

\vspace{0.9em}
\end{minipage}}
\end{center}

\clearpage

% ============================================================
% POPULATION ÉLIGIBLE
% ============================================================

\subsection{Population éligible}
\needspace{6\baselineskip}

\begin{itemize}[leftmargin=1.1cm]
  \item maladie chronique du foie ;
  \item suspicion de maladie auto-immune ou cholestatique avec bilan discordant ;
  \item toxicité médicamenteuse sévère ou atypique, notamment sous immunothérapie ;
  \item protocoles de recherche nécessitant une évaluation histologique ;
  \item exploration des lésions tumorales primitives ou secondaires ;
  \item suspicion d’hépatite infectieuse atypique ou d’infection profonde ;
  \item évaluation pré-allogreffe en hématologie (SMD, aplasie, LAM).
\end{itemize}

\clearpage

% ============================================================
% PARCOURS DE SOINS
% ============================================================

\subsection{Parcours de soins}
\needspace{8\baselineskip}

\begin{figure}[!ht]
\centering
\caption{Parcours patient — PBH en Hôpital de Jour}
\vspace{0.8cm}

\begin{tikzpicture}[
    node distance=1.6cm,
    box/.style={
        rectangle,
        rounded corners=3pt,
        draw=APHPdark,
        thick,
        text width=8.6cm,
        minimum height=1.5cm,
        align=center,
        fill=APHPsoft
    }
]
\node[box] (tri) {Orientation vers l’HDJ (hépatologie / MCO / ville)};
\node[box, below=1.4cm of tri] (e1) {Évaluation initiale : coagulation, imagerie, consentement};
\node[box, below=1.4cm of e1] (e2) {Biopsie sous guidage échographique (HLHJ003/HLHJ006)};
\node[box, below=1.4cm of e2] (e3) {Surveillance spécialisée 6 heures : constantes, douleur, saignement};
\node[box, below=1.4cm of e3] (syn) {Compte-rendu, consignes et coordination ville–hôpital};

\draw[->, thick, APHPdark] (tri) -- (e1);
\draw[->, thick, APHPdark] (e1) -- (e2);
\draw[->, thick, APHPdark] (e2) -- (e3);
\draw[->, thick, APHPdark] (e3) -- (syn);

\end{tikzpicture}
\end{figure}

\clearpage

% ============================================================
% PANORAMA CODAGE + VOLUMÉTRIE 2024
% ============================================================

\begin{sidewaystable}[h!]
\centering
\renewcommand{\arraystretch}{1.25}
\rowcolors{2}{APHPsoft}{white}

\begin{tabular}{
p{5.2cm}
p{2.6cm}
p{1.8cm}
p{1.8cm}
>{\centering\arraybackslash}p{2.0cm}
>{\centering\arraybackslash}p{2.2cm}
>{\centering\arraybackslash}p{3.0cm}
}
\toprule
\rowcolor{APHPsoft}
\textbf{Type de séance} &
\textbf{DP / DR / DAS} &
\textbf{GHM} &
\textbf{GHS} &
\textbf{Tarif 2025} &
\textbf{Volume 2024} &
\textbf{Recette 2024} \\
\midrule

Fibrose / MASLD &
K74.0 / K76.0 \newline DAS comorbidités majeures &
07M08T & 2538 & 1\,238~€ & 41 & 50\,758~€ \\

CHC / tumeur maligne &
C22.0 \newline DAS comorbidités &
07M06T & 2528 & 1\,061~€ & 23 & 24\,403~€ \\

Cirrhose alcoolique &
K70.2 &
07M07T & 2533 & 840~€ & 5 & 4\,200~€ \\

HAI / tumeur bénigne &
K75.4 / D13.4 &
07M04T & 2523 & 919~€ & 35 & 32\,165~€ \\

Normale ou histologie\\non disponible &
R93.2 &
07M14T & 2559 & 603~€ & 144 & 86\,832~€ \\

\midrule
\textbf{TOTAL} & -- & -- & -- & -- & \textbf{248} & \textbf{198\,358~€} \\
\bottomrule
\end{tabular}

\caption{Panorama PBH HDJ — Codage, volumétrie et recettes (2024)}
\end{sidewaystable}

\clearpage

% ============================================================
% TRACABILITÉ
% ============================================================

\subsection{Traçabilité minimale}

\begin{table}[h!]
\centering
\renewcommand{\arraystretch}{1.25}
\rowcolors{2}{APHPsoft}{white}

\begin{tabular}{p{5cm} p{9cm}}
\toprule
\rowcolor{APHPsoft}
\textbf{Intervention} & \textbf{Traçabilité requise} \\
\midrule
Biopsie hépatique &
Fiche d’acte, repérage échographique, nombre de carottes, longueur, calibre \\
Surveillance spécialisée &
Feuille pluri-horaire (6h), douleur, tension, saignement local \\
Entretien médical &
Indication, consentement, comorbidités, risque hémorragique \\
Examens complémentaires &
Imageries et bilans pré-biopsie \\
Coordination &
Compte-rendu structuré, consignes, liaison ville–hôpital \\
\bottomrule
\end{tabular}

\caption{Traçabilité — PBH en HDJ}
\end{table}

\clearpage

% ============================================================
% PROJECTIONS D’ACTIVITÉ
% ============================================================

\subsection{Projections d’activité et recettes prévisionnelles}

\noindent Référence 2024 : \textbf{248 PBH}, soit \textbf{198\,400~€} (tarif moyen 800~€). \\
Hypothèse de croissance : \textbf{+25 actes / an}. \\
Tarif moyen stable : \textbf{800~€ / séance}.

\begin{table}[h!]
\centering
\renewcommand{\arraystretch}{1.20}
\rowcolors{2}{APHPsoft}{white}
\begin{tabular}{
p{4.3cm}
>{\centering\arraybackslash}p{2.2cm}
>{\centering\arraybackslash}p{2.3cm}
>{\centering\arraybackslash}p{3.0cm}
}
\toprule
\rowcolor{APHPsoft}
\textbf{Phase} & \textbf{Volume estimé} & \textbf{Tarif moyen} & \textbf{Recette brute} \\
\midrule
Amorce        & 273 & 800~€ & 218\,400~€ \\
Montée        & 298 & 800~€ & 238\,400~€ \\
Croisière     & 323 & 800~€ & 258\,400~€ \\
\bottomrule
\end{tabular}
\caption{Projections d’activité et recettes prévisionnelles — PBH HDJ (à partir de 2024)}
\end{table}


% ============================================================
% CONCLUSION
% ============================================================

\subsection{Conclusion}

La biopsie hépatique en Hôpital de Jour s’intègre dans un parcours sécurisé et standardisé, essentiel pour la prise en charge des maladies hépatiques rares, des toxicités médicamenteuses complexes et des programmes thérapeutiques nécessitant une évaluation histologique. Elle constitue un outil majeur pour la décision clinique et la stratification pronostique.

\clearpage

% ============================================================
% VALIDATION
% ============================================================

\begin{center}
\begin{tabular}{p{4cm} p{7cm} p{4cm}}
\toprule
\rowcolor{APHPsoft}
\textbf{Date d'envoi} & \textbf{Relecteur} & \textbf{Validation} \\
\midrule
03/12/2025 & Pr V.\,Mallet & 07/12/2025 \\
03/12/2025 & Dr S.\,Bouam & 07/12/2025 \\
NA & Dr V.\,D'Halluin & NA \\
NA & Pr R.\,Coriat & NA \\


\bottomrule
\end{tabular}
\end{center}

\clearpage

\printbibliography[heading=subbibliography,title={Références}]
\end{refsection}

% ======================================
% CHAPITRE 2: Evaluation des Cirrhoses
% ======================================
\clearpage
\section{HDJ Evaluation des Cirrhoses}
\begin{refsection}
% ============================================================
% HÔPITAL DE JOUR — ÉVALUATION DES CIRRHOSES ET HTP CLINIQUE
% ============================================================

\subsection{Rationnel médical}
\needspace{8\baselineskip}

\begin{spacing}{1.28}

L’incidence des cirrhoses continue de progresser en Europe, portée par l’augmentation des hépatopathies métaboliques (MASLD), du diabète de type 2, de l’obésité et de la consommation d’alcool. La prévalence des formes compensées croît en moyenne de 3 à 5\,\% par an \cite{EASL_DecompCirrhosis_2018}. En France, les données PMSI et SNDS confirment également une hausse régulière des hospitalisations liées aux maladies hépatiques chroniques \cite{RN597}.

L’évolution de la fibrose vers la cirrhose s’accompagne d’une élévation du gradient porto-systémique (HVPG). Une valeur $>$\,10\,mmHg définit l’hypertension portale cliniquement significative (CSPH), étape pivot à partir de laquelle survient la première décompensation \cite{EASL_DecompCirrhosis_2018}. Les recommandations de Baveno\,VII permettent d’identifier la CSPH par des critères non invasifs :
\begin{itemize}
  \item élasticité hépatique (LSM) $\geq 25$\,kPa, ou
  \item LSM de 20--25\,kPa avec plaquettes $<$\,150\,G/L \cite{RN597}.
\end{itemize}

Chez ces patients, l’introduction précoce d’un bêtabloquant non sélectif (carvédilol) réduit l’HVPG, prévient la première décompensation et améliore la survie, y compris en l’absence de varices œsophagiennes.

Plusieurs facteurs aggravants doivent être systématiquement évalués :
\begin{itemize}
  \item consommation d’alcool ; l’abstinence améliore la survie \cite{Loomba_AlcoholAbstinence_2020,Addolorato_AlcoholCirrhosis_2016} ;
  \item sarcopénie (30--70\,\%) \cite{Tantai_Sarcopenia_2022} ;
  \item dénutrition, risque infectieux et complications extra-hépatiques.
\end{itemize}

L’évaluation annuelle d’une cirrhose compensée avancée requiert une approche multidimensionnelle combinant LSM hépato-splénique, imagerie, évaluation nutritionnelle et psychologique/addictologique, bilan vaccinal et décision endoscopique selon Baveno\,VII.

Un HDJ dédié permet de regrouper ces évaluations en une séance unique, standardisée et conforme aux recommandations tout en garantissant la réalisation d’au moins trois interventions valorisantes.

\end{spacing}

\clearpage

% ============================================================
% OBJECTIFS
% ============================================================

\subsection{Objectifs}
\needspace{6\baselineskip}

\begin{center}
\fcolorbox{APHPdark}{APHPsoft}{
\begin{minipage}{0.95\textwidth}
\vspace{0.9em}

\begin{itemize}[leftmargin=1cm]
  \item identifier la CSPH par critères non invasifs (Baveno\,VII) ;
  \item concentrer en une séance l’ensemble des évaluations clés : LSM, échographie, nutrition, psychologue/addictologue ;
  \item initier ou ajuster le carvédilol en conditions sécurisées ;
  \item structurer un parcours annuel de prévention (HTP, alcool, sarcopénie) ;
  \item actualiser le statut vaccinal ;
  \item produire une synthèse médicale facilitant la coordination ville–hôpital.
\end{itemize}

\vspace{0.9em}
\end{minipage}}
\end{center}

\clearpage

% ============================================================
% POPULATION ÉLIGIBLE
% ============================================================

\subsection{Population éligible}
\needspace{6\baselineskip}

\begin{itemize}[leftmargin=1cm]
  \item cirrhose compensée (Child\,A) ou stabilité post-décompensation ;
  \item LSM $\geq$\,15\,kPa, thrombopénie $<$\,150\,G/L ou facteurs de risque de CSPH ;
  \item consommation d’alcool active ou récente ;
  \item sarcopénie ou risque nutritionnel ;
  \item besoin d’une requalification annuelle selon Baveno\,VII.
\end{itemize}

\clearpage

% ============================================================
% PARCOURS DE SOINS
% ============================================================

\subsection{Parcours de soins}
\needspace{8\baselineskip}

\begin{figure}[!ht]
\centering
\caption{Parcours patient — HDJ cirrhose / HTP clinique}
\vspace{0.7cm}

\begin{tikzpicture}[
    node distance=1.5cm,
    box/.style={
        rectangle,
        rounded corners=3pt,
        draw=APHPdark,
        thick,
        text width=9cm,
        minimum height=1.4cm,
        align=center,
        fill=APHPsoft
    }
]

\node[box] (tri) {Orientation vers l’HDJ : hépatologie, ville, MCO};
\node[box, below=1.3cm of tri] (e1) {Évaluation initiale : LSM hépato-splénique, biologie, plaquettes};
\node[box, below=1.3cm of e1] (e2) {Échographie hépatique, évaluation nutritionnelle, psychologue/addictologue, vaccinations};
\node[box, below=1.3cm of e2] (e3) {Décisions : indication FOGD (Baveno\,VII), initiation du carvédilol si LSM $\geq 25$\,kPa};
\node[box, below=1.3cm of e3] (syn) {Synthèse médicale structurée et plan annuel};

\draw[->, thick, APHPdark] (tri) -- (e1);
\draw[->, thick, APHPdark] (e1) -- (e2);
\draw[->, thick, APHPdark] (e2) -- (e3);
\draw[->, thick, APHPdark] (e3) -- (syn);

\end{tikzpicture}
\end{figure}

\clearpage

% ============================================================
% CODAGE ET GHS ASSOCIÉS
% ============================================================

\subsection{Codage et GHS associés}
\begin{table}[h!]
\centering
\renewcommand{\arraystretch}{1.20}
\rowcolors{2}{APHPsoft}{white}

\begin{tabular}{p{4.6cm} p{6.8cm} c c c}
\toprule
\rowcolor{APHPsoft}
\textbf{Type de séance} & \textbf{DP / DR / DAS} & \textbf{GHM} & \textbf{GHS} & \textbf{Tarif} \\
\midrule
Évaluation complète (≥4) & 
DP: Z098\newline DR: K74.6\newline DAS: R18, K76.6, E44.x, F10.x &
07M13Z & 9616 & 941~€ \\

Évaluation complète (=3) & 
DP: Z098\newline DR: K74.6\newline DAS: R18, K76.6, E44.x, F10.x &
07M13Z & 9616 & 420~€ \\

LSM + échographie (≥4) &
DP: Z098\newline DR: K74.6\newline DAS: R16.1, K76.6, R18 &
07M13Z & 9616 & 941~€ \\

LSM + échographie (=3) &
DP: Z098\newline DR: K74.6\newline DAS: R16.1, K76.6, R18 &
07M13Z & 9616 & 420~€ \\

Nutrition + psychologue (≥4) &
DP: Z098\newline DR: K74.6\newline DAS: E43--E46, F10.x, R18 &
07M13Z & 9613 & 941~€ \\

Nutrition + psychologue (=3) &
DP: Z098\newline DR: K74.6\newline DAS: E43--E46, F10.x, R18 &
07M13Z & 9613 & 420~€ \\
\bottomrule
\end{tabular}

\caption{Codage et GHS associés — HDJ cirrhose}
\end{table}

\clearpage


% ============================================================
% TRACABILITÉ
% ============================================================
\needspace{8\baselineskip}   
\subsection{Traçabilité minimale}
\begin{table}[h!]
\centering
\renewcommand{\arraystretch}{1.20}
\rowcolors{2}{APHPsoft}{white}

\begin{tabular}{p{4.5cm} p{8.1cm}}
\toprule
\rowcolor{APHPsoft}
\textbf{Intervention} & \textbf{Éléments requis} \\
\midrule
LSM hépato-splénique & Critères qualité, IQR/med, validation, seuils Baveno~VII, message décisionnel \\
Échographie hépatique & CHC, HTP, flux portal, signes indirects, mesure splénique \\
Nutrition / sarcopénie & IMC, perte pondérale, dynamométrie, plan nutritionnel \\
Psychologie / alcool & Évaluation motivationnelle, repérage, orientation \\
FOGD & Indications Baveno~VII, résultats, calendrier \\
Décision carvédilol & Dose initiale, titration, objectifs tensionnels, suivi IDE \\
Vaccinations & Pneumocoque, grippe, COVID, VHA/VHB \\
Synthèse médicale & Classification Baveno, plan thérapeutique, coordination \\
\bottomrule
\end{tabular}

\caption{Traçabilité — HDJ cirrhose}
\end{table}

\clearpage

% ============================================================
% Organisation
% ============================================================

\subsection{Organisation}
\needspace{5\baselineskip}

\begin{itemize}[leftmargin=1.1cm]
  \item Direction : \textbf{Dr Lucia Parlati}
  \item Durée : 3--4~heures
  \item Lieu : Secteur HDJ — Service des maladies du foie
  \item Ressources : médecin sénior, infirmier expert/IPA, diététicien(ne), psychologue/addictologue
\end{itemize}

\clearpage

% ============================================================
% VOLUMÉTRIE DE RÉFÉRENCE
% ============================================================

\subsection{Volumétrie de référence}
\needspace{6\baselineskip}

\noindent File active annuelle : 7\,500 patients.  
Prévalence estimée de la cirrhose : 21\,\% → \textbf{1\,575 patients}.  
Taux de recours HDJ cible : \textbf{40\,\%} → environ \textbf{630 séances/an}.

\begin{center}
\begin{tabular}{lccc}
\toprule
\textbf{Séance} & \textbf{Volume} & \textbf{Tarif moyen} & \textbf{Recette annuelle} \\
\midrule
Évaluation complète (≥3–4 interv.) & 190 & 690~€ & 131\,000~€ \\
LSM + échographie                    & 285 & 690~€ & 197\,000~€ \\
Nutrition + psychologue              & 155 & 690~€ & 107\,000~€ \\
\midrule
\textbf{Total}                       & \textbf{630} & -- & \textbf{435\,000~€} \\
\bottomrule
\end{tabular}
\end{center}

\clearpage

% ============================================================
% PROJECTIONS
% ============================================================

\subsection{Projections d’activité}
\needspace{6\baselineskip}

\noindent Hypothèse : progression du recours HDJ de 40\,\% à 60\,\% de la file active.

\begin{center}
\begin{tabular}{lccc}
\toprule
\textbf{Année} & \textbf{Volume estimé} & \textbf{Tarif moyen} & \textbf{Recette brute} \\
\midrule
Amorce   & 630 & 690~€ & 435\,000~€ \\
Montée   & 790 & 690~€ & 545\,000~€ \\
Croisière & 945 & 690~€ & 652\,000~€ \\
\bottomrule
\end{tabular}
\end{center}

\clearpage

% ============================================================
% Conclusion
% ============================================================

\subsection{Conclusion}

L’HDJ dédié à l’évaluation des cirrhoses offre une organisation intégrée, conforme à Baveno\,VII, regroupant LSM, imagerie, évaluation nutritionnelle, repérage addictologique et initiation du carvédilol lorsque indiqué. Cette approche coordonnée renforce la prévention des décompensations, améliore la qualité du dépistage du CHC et fluidifie les parcours entre l’hépatologie et la médecine de ville.

\clearpage

% ============================================================
% VALIDATION
% ============================================================

\begin{center}
\begin{tabular}{p{4cm} p{7cm} p{4cm}}
\toprule
\rowcolor{APHPsoft}
\textbf{Date} & \textbf{Relecteur} & \textbf{Validation} \\
\midrule

03/12/2025   & Pr V.\,Mallet  & 03/12/2025 \\
NA           & Dr L.\,Parlati  & NA \\
03/12/2025   & Dr S.\,Bouam   & NA \\
NA           & Pr R.\,Coriat  & NA \\

\bottomrule
\end{tabular}
\end{center}

\clearpage

\printbibliography[heading=subbibliography,title={Références}]
\end{refsection}

% ======================================================
% CHAPITRE 3: Prise en Charge des Cirrhoses Terminales
% ======================================================
\clearpage
\section{HDJ Traitement des Cirrhoses Terminales}
\begin{refsection}
% ============================================================
% HÔPITAL DE JOUR — CIRRHOSE AVANCÉE / COMPLICATIONS DE LA CIRRHOS...
% ============================================================

\subsection{Rationnel médical}
\needspace{8\baselineskip}
\setcounter{table}{0}
\setcounter{figure}{0}

\begin{spacing}{1.30}

L’ensemble des maladies chroniques du foie évolue progressivement vers la cirrh\-ose et ses complications. La phase décompensée — ascite, encéphalopathie hépatique, hyponatrémie, hémorragie digestive, insuffisance rénale aiguë (AKI) ou syndrome hépato\-rénal (HRS) — correspond au stade terminal de la maladie, avec une mortalité annuelle souvent supérieure à 20--30\,\% \cite{EASL_DecompCirrhosis_2018,AASLD_PalliativeCirrhosis_2022}. 

Malgré les progrès réalisés dans le traitement des hépatites virales, l’incidence des décompensations ne diminue pas, en lien avec l’augmentation de l’obésité, du diabète de type~2, des maladies hépatiques métaboliques (MASLD), ainsi que la persistance de la consommation d’alcool \cite{Gines_LancetCirrhosis_2021}. En France, les données PMSI/AP--HP confirment une progression des séjours liés à l’ascite, aux infections et à l’AKI, avec une mortalité hospitalière de 11--15\,\% \cite{FrenchHepaticFailure_2020}.

Les complications de la cirrh\-ose avancée requièrent des interventions répétées et accessibles rapidement : ponction d’ascite avec perfusion d’albumine, albumine au long cours (type ANSWER) \cite{Caraceni_ANSWER_2018}, fer intraveineux, transfusion de CGR et soutien nutritionnel.

La structuration d’un HDJ dédié permet d’assurer ces prises en charge dans un cadre sécurisé, standardisé et pluridisciplinaire, favorisant la stabilisation, la prévention des réhospitalisations et la préparation d’éventuelles procédures (TIPS).

\end{spacing}

\clearpage

% ============================================================
% OBJECTIFS
% ============================================================

\subsection{Objectifs}
\needspace{6\baselineskip}

\begin{center}
\fcolorbox{APHPdark}{APHPsoft}{
\begin{minipage}{0.95\textwidth}
\vspace{0.9em}

\begin{itemize}[leftmargin=1.1cm]
  \item réduire les hospitalisations évitables liées aux complications de la cirrh\-ose avancée ;
  \item prévenir les décompensations sévères (AKI, HRS, ACLF) par un accès rapide aux actes nécessaires ;
  \item proposer un dispositif de stabilisation pré--TIPS ;
  \item regrouper en ambulatoire les actes complexes : LVP + albumine, albumine seule, fer IV, transfusion de CGR ;
  \item renforcer la continuité entre ville, urgences, hépatologie et MCO.
\end{itemize}

\vspace{0.9em}
\end{minipage}}
\end{center}

\bigskip

% ============================================================
% POPULATION ÉLIGIBLE
% ============================================================

\subsection{Population éligible}

\begin{itemize}[leftmargin=1.1cm]
  \item cirr\-hose avancée (Child B–C) avec ascite récurrente ou réfractaire ;
  \item risque d’AKI ou HRS (hyponatrémie, insuffisance rénale fonctionnelle) ;
  \item anémie d’hypertension portale nécessitant fer IV ou transfusion programmée ;
  \item encéphalopathie hépatique fluctuante nécessitant surveillance rapprochée ;
  \item dénutrition sévère ou sarcopénie ;
  \item candidats au TIPS ou non éligibles nécessitant prévention des réhospitalisations.
\end{itemize}

\clearpage

% ============================================================
% PARCOURS DE SOINS
% ============================================================

\subsection{Parcours de soins}

\begin{figure}[!ht]
\centering
\caption{Parcours patient — HDJ cirrhose avancée\footnotemark}
\vspace{0.8cm}

\begin{tikzpicture}[
    node distance=1.6cm,
    box/.style={
        rectangle,
        rounded corners=3pt,
        draw=APHPdark,
        thick,
        text width=9.0cm,
        minimum height=1.5cm,
        align=center,
        fill=APHPsoft
    }
]

\node[box] (tri) {Orientation vers l’HDJ \\[3pt] (hépatologie / urgences / MCO / ville)};
\node[box, below=1.4cm of tri] (etape1) {Évaluation clinique initiale, biologie récente, imagerie ciblée si besoin ; vérification du statut vaccinal (pneumocoque, grippe, hépatites~A et~B)};
\node[box, below=1.4cm of etape1] (etape2) {Acte programmé : \\[3pt] LVP + albumine, albumine seule, fer IV, transfusion de CGR};
\node[box, below=1.4cm of etape2] (etape3) {Surveillance spécialisée : \\[3pt] constantes, tolérance, complications};
\node[box, below=1.4cm of etape3] (synth) {Synthèse médicale et programmation de la séance suivante};

\draw[->, thick, APHPdark] (tri) -- (etape1);
\draw[->, thick, APHPdark] (etape1) -- (etape2);
\draw[->, thick, APHPdark] (etape2) -- (etape3);
\draw[->, thick, APHPdark] (etape3) -- (synth);

\end{tikzpicture}
\end{figure}

\footnotetext{
\textbf{HDJ} : hôpital de jour ; 
\textbf{MCO} : médecine–chirurgie–obstétrique ; 
\textbf{LVP} : paracentèse évacuatrice (large volume paracentesis) ; 
\textbf{CGR} : concentrés de globules rouges.
}

\clearpage

% ============================================================
% CODAGE, TARIFS ET VOLUMÉTRIE 2024
% ============================================================

\subsection{Codage, tarifs et volumétrie de référence}
\needspace{12\baselineskip}

\begin{sidewaystable}[p]
\centering
\renewcommand{\arraystretch}{1.25}
\rowcolors{2}{APHPsoft}{white}

\begin{tabularx}{\textwidth}{
p{4.8cm}
X
>{\centering\arraybackslash}p{1.7cm}
>{\centering\arraybackslash}p{1.6cm}
>{\centering\arraybackslash}p{1.9cm}
>{\centering\arraybackslash}p{1.8cm}
>{\centering\arraybackslash}p{2.4cm}
}
\toprule
\rowcolor{APHPsoft}
\textbf{Type de séance} &
\textbf{DP / DR / DAS} &
\textbf{GHM} &
\textbf{GHS} &
\textbf{Tarif 2025} &
\textbf{Volume 2024} &
\textbf{Recette 2024} \\
\midrule

Ponction d’ascite + albumine &
DP : R18 (ascite)\newline
DAS : K74.6, K76.6, ±N17.x, ±E87.1 &
07M14T & 2559 & 603~€ &
160 &
96\,480~€ \\

Séance d’albumine seule &
DP : Z512 (réserve)\newline
DR : R18 ou K76.6 &
28Z17Z & 9616 & 440~€ &
18 &
7\,920~€ \\

Fer injectable (anémie HTP) &
DP : Z512\newline
DR : D50.8 ou D64.9\newline
DAS : K74.6, K76.6 &
28Z17Z & 9616 & 440~€ &
125 &
55\,000~€ \\

Transfusion de CGR &
DP : Z5130\newline
DR : D50.8 ou D62\newline
DAS : K74.6, K76.6 &
28Z14Z & 9613 & 791~€ &
15 &
11\,865~€ \\
\midrule

\textbf{Total annuel} & -- & -- & -- & -- &
\textbf{319} &
\textbf{171\,265~€} \\

\bottomrule
\end{tabularx}

\caption{Codage, tarifs et volumétrie — HDJ cirrhose avancée (2024)}
\end{sidewaystable}

\clearpage

% ============================================================
% TRACABILITÉ
% ============================================================

\subsection{Traçabilité minimale}

\begin{table}[h!]
\centering
\renewcommand{\arraystretch}{1.25}
\rowcolors{2}{APHPsoft}{white}

\begin{tabular}{p{5cm} p{9cm}}
\toprule
\rowcolor{APHPsoft}
\textbf{Intervention} & \textbf{Éléments requis} \\
\midrule

Ponction d’ascite &
Volume évacué ; repérage écho ; surveillance 4~h ; douleur ; hypotension ; biologie pré-acte \\

Perfusion d’albumine &
Prescription ; indication ; traçabilité du lot ; volume perfusé ; surveillance hémodynamique \\

Fer injectable &
Indication (anémie HTP) ; traçabilité lot ; protocole perfusion ; surveillance immédiate et retardée \\

Transfusion de CGR &
Traçabilité PSL ; groupage ; concordance ; surveillance renforcée ; incidents transfusionnels \\

Entretien médical &
Justification ; risque HRS/AKI ; bilan clinique ; adaptation thérapeutique \\

Surveillance spécialisée &
Constantes ; EVA douleur ; hémodynamique ; reins ; drainage post-ponction \\

Coordination / éducation &
Fiche de liaison ville–hôpital ; conseils HTP ; éducation sur signes d’alerte \\
\bottomrule
\end{tabular}

\caption{Traçabilité — HDJ cirrhose avancée}
\end{table}

\clearpage

% ============================================================
% ORGANISATION
% ============================================================

\subsection{Organisation}
\needspace{5\baselineskip}

\begin{itemize}[leftmargin=1.1cm]
    \item Direction : \textbf{Dr Valérie D’Halluin-Venier}
    \item Durée : 4--6~heures
    \item Lieu : Secteur HDJ — Service des maladies du foie
    \item Ressources : médecin sénior, infirmier expert/IPA, diététicien(ne), psychologue/addictologue
\end{itemize}

\bigskip

% ============================================================
% PROJECTIONS D’ACTIVITÉ
% ============================================================

\subsection{Projections d’activité et recettes prévisionnelles}

\noindent Basé sur un tarif moyen pondéré : \textbf{\textasciitilde540~€ / séance}.  
Hypothèse : \textbf{+50 patients par pallier} à partir de la volumétrie 2024 (277 actes).

\begin{table}[h!]
\centering
\renewcommand{\arraystretch}{1.18}
\rowcolors{2}{APHPsoft}{white}

\begin{tabular}{lccc}
\toprule
\rowcolor{APHPsoft}
\textbf{Année} & \textbf{Volume estimé} & \textbf{Tarif moyen} & \textbf{Recette brute estimée} \\
\midrule
Amorce    & 327 & 540~€ & 176\,580~€ \\
Montée    & 377 & 540~€ & 203\,580~€ \\
Croisière & 427 & 540~€ & 230\,580~€ \\
\bottomrule
\end{tabular}

\caption{Prévisions d’activité et recettes prévisionnelles — HDJ cirrhose avancée}
\end{table}

\clearpage

% ============================================================
% CONCLUSION
% ============================================================

\subsection{Conclusion}
\needspace{6\baselineskip}

Les HDJ dédiés à la cirrh\-ose avancée offrent un cadre structuré pour les interventions indispensables à la stabilisation des patients les plus fragiles. En rassemblant ponctions d’ascite, albumine, fer injectable et transfusions dans un parcours sécurisé et standardisé, ils contribuent à réduire les hospitalisations évitables, à prévenir les décompensations sévères et à optimiser la continuité des soins entre ville, urgences et hépatologie.
Pour les modalités techniques détaillées (ponction d’ascite, perfusion d’albumine, fer IV, transfusion),
voir les annexes~\ref{sec:annexe_ascite}, \ref{sec:annexe_fer} et \ref{sec:annexe_cgr}.


% ============================================================
% VALIDATION
% ============================================================

\begin{center}
\begin{tabular}{p{4cm} p{7cm} p{4cm}}
\toprule
\rowcolor{APHPsoft}
\textbf{Date d’envoi} & \textbf{Nom du relecteur} & \textbf{Date de validation} \\
\midrule
03/12/2025 & Pr V.\,Mallet & 08/12/2025 \\
03/12/2025 & Dr S.\,Bouam & NA \\
03/12/2025 & Dr V.\,D’Halluin-Venier & NA \\
NA & Pr R.\,Coriat & NA \\
\bottomrule
\end{tabular}
\end{center}

\clearpage

\printbibliography[heading=subbibliography,title={Références}]
\end{refsection}

% ============================
% CHAPITRE 4: Hépatométabolique
% ============================
\clearpage
\section{HDJ Hépatométabolique}
\begin{refsection}
% ============================================================
% HÔPITAL DE JOUR HÉPATOMÉTABOLIQUE
% ============================================================

\subsection{Rationnel médical}
\needspace{8\baselineskip}

\begin{spacing}{1.30}

L’augmentation rapide de l’obésité, du diabète de type 2 (DT2) et des maladies hépatiques métaboliques (MASLD/MASH) constitue aujourd’hui un défi majeur pour le système de santé. Dans cette population, la maladie hépatique est hautement prévalente : plus de 60\,\% des patients DT2 présentent une atteinte hépatique métabolique et 15--20\,\% une fibrose significative (F$\geq$2), les exposant à un risque accru de complications évolutives \cite{RN565}.

\medskip

En France, 3{,}5 à 4 millions de personnes vivent avec un DT2 selon les données récentes de Santé publique France et de l’Assurance Maladie (SNDS) \cite{SPF2021Diabete}. Cette population représente un réservoir important de patients susceptibles d’évoluer vers une maladie hépatique avancée, avec un impact croissant sur les parcours de soins et les capacités hospitalières.

% \medskip

% L’identification précoce des formes avancées reste difficile : les complications sévères sont rares à l’échelle populationnelle, ce qui nécessite un repérage ciblé et structuré. Dans la cohorte DT2 de l’Entrepôt de Données de Santé (EDS) de l’AP-HP (77\,368 patients), l’incidence annuelle des événements hépatiques graves n’est que de 1{,}31 pour 1\,000 patients-années.

\medskip

Pourtant, des interventions simples et peu coûteuses — réduction de la consommation d’alcool, perte pondérale modérée, amélioration de la qualité alimentaire — démontrent un impact tangible sur l’évolution de la maladie \cite{RN597}. Leur efficacité dépend cependant d’un repérage précoce et d’une organisation lisible du parcours.

\medskip

L’arrivée de nouvelles thérapeutiques (agonistes du GLP-1, resmétirom et autres agents en développement) renforce la nécessité d’un dispositif capable d’identifier, évaluer et suivre précocement les patients éligibles, tout en garantissant l’appropriation des recommandations sur le territoire.

\medskip

Dans ce contexte, des outils simples et robustes de stratification sont indispensables. Les biomarqueurs non invasifs constituent désormais la base du tri diagnostique dans la population DT2. Le score FIB-4, largement disponible dans les logiciels médicaux et recommandé par les sociétés savantes, permet d’exclure efficacement les formes avancées et d’orienter les patients présentant un FIB-4 $\geq$ 1{,}3 vers une évaluation spécialisée (élastométrie, imagerie) \cite{RN597,EASL2024MASLD} .

\medskip

Sur le territoire de Cochin, un flux régulier de patients à FIB-4 élevé est déjà identifié via les consultations de diabétologie, de cardiologie, les CPTS et les acteurs de premier recours. Les actions d’information menées localement et régionalement renforcent ce repérage et traduisent une dynamique territoriale structurée autour de la MASLD.

\medskip

Dans ce cadre, la création d’un Hôpital de Jour hépatométabolique répond à un besoin clairement identifié. Ce dispositif offre une évaluation intégrée, standardisée et rapide, combinant imagerie, exploration comportementale, évaluation nutritionnelle, activité physique adaptée et prise en charge psychologique. Il permet :

\begin{itemize}
    \item d’optimiser le triage des patients à risque ;
    \item de réduire les retards diagnostiques ;
    \item d’améliorer la pertinence des orientations (consultation, suivi, recherche) ;
    \item de proposer des interventions à fort impact populationnel ;
    \item d’inscrire le parcours dans une logique territoriale en lien avec les acteurs de premier recours.
\end{itemize}

\end{spacing}

\clearpage

% ============================================================
\subsection{Objectifs}
\needspace{6\baselineskip}

\begin{center}
\fcolorbox{APHPdark}{APHPsoft}{
\begin{minipage}{0.92\textwidth}
\begin{itemize}[leftmargin=1.1cm]
    \item Dépister précocement la fibrose hépatique significative ou avancée.
    \item Structurer une évaluation intégrée : biomarqueurs, imagerie, diététique, psychologie.
    \item Initier une prise en charge hygiéno-diététique et comportementale.
    \item Identifier les patients éligibles aux thérapeutiques MASLD/MASH et aux protocoles de recherche.
\end{itemize}
\end{minipage}}
\end{center}

\bigskip
\bigskip

% ============================================================
\subsection{Population éligible}
\needspace{5\baselineskip}

\begin{itemize}[leftmargin=1.1cm]
    \item Diabète de type 2 ou syndrome métabolique.
    \item FIB-4 $\geq$ 1{,}3.
    \item Suspicion clinique ou échographique de MASLD/MASH.
\end{itemize}

\clearpage

% ============================================================
% PARCOURS PATIENT
% ============================================================

\subsection{Parcours de soins (3--4 heures)}
\needspace{8\baselineskip}

\begin{figure}[!ht]
\centering
\caption{Parcours patient — HDJ Hépatométabolique}
\vspace{0.8cm}

\begin{tikzpicture}[
    node distance=1.6cm,
    box/.style={
        rectangle,
        rounded corners=3pt,
        draw=APHPdark,
        thick,
        text width=9.0cm,
        minimum height=1.2cm,
        align=center,
        fill=APHPsoft
    }
]
\node[box] (tri) {Tri initial \\ FIB-4 $\geq$ 1{,}3};
\node[box, below=1.4cm of tri] (entree) {Entrée en HDJ hépatométabolique};
\node[box, below=1.4cm of entree] (echo) {Échographie abdominale + Doppler};
\node[box, below=1.4cm of echo] (fibro) {FibroScan / élastographie};
\node[box, below=1.4cm of fibro] (diet) {Consultation diététique};
\node[box, below=1.4cm of diet] (psy) {Évaluation psychologique \\ (alcool / TCA)};
\node[box, below=1.4cm of psy] (synth) {Synthèse médicale \\ Plan thérapeutique};

\draw[->, thick, APHPdark] (tri) -- (entree);
\draw[->, thick, APHPdark] (entree) -- (echo);
\draw[->, thick, APHPdark] (echo) -- (fibro);
\draw[->, thick, APHPdark] (fibro) -- (diet);
\draw[->, thick, APHPdark] (diet) -- (psy);
\draw[->, thick, APHPdark] (psy) -- (synth);

\end{tikzpicture}
\end{figure}

\clearpage

% ============================================================
% Organisation et ressources nécessaires
% ============================================================

\subsection{Organisation}
\needspace{5\baselineskip}

\begin{itemize}[leftmargin=1.1cm]
    \item Direction: \textbf{Docteur Lucia Parlati}
    \item Durée : 3--4 heures.
    \item Lieu : Secteur HDJ — Service des maladies du foie.
    \item Ressources : médecin sénior, infirmier expert/IPA, diététicien(ne), psychologue/addictologue.

\end{itemize}

\bigskip
% ============================================================
% CODAGE ET GHS ASSOCIÉS
% ============================================================

\subsection{Codage et GHS associés}
\needspace{6\baselineskip}

\noindent\textbf{Cadre général.}  
Suspicion de MASLD/MASH avec FIB-4 ≥1,3 requérant un HDJ intégrant : élastographie, échographie, consultation médicale, évaluation diététique ou psychologique.  
Codage conforme aux avis DIM 2024–2025 (DP R945 ; DAS E11.98).

\medskip

\begin{table}[h!]
\centering
\renewcommand{\arraystretch}{1.25}
\rowcolors{2}{APHPsoft}{white}

\begin{tabularx}{\textwidth}{
>{\raggedright\arraybackslash}p{4.8cm}
>{\raggedright\arraybackslash}X
>{\centering\arraybackslash}p{2cm}
>{\centering\arraybackslash}p{1.8cm}
>{\centering\arraybackslash}p{2.2cm}}
\toprule
\rowcolor{APHPsoft}
\textbf{Type de séance} &
\textbf{DP / DR / DAS} &
\textbf{GHM} &
\textbf{GHS} &
\textbf{Tarif 2025} \\
\midrule

Suspicion MASLD/MASH (≥4 interventions) &
DP : R945 \newline
DAS : E11.98 &
07M14T &
2559 &
603~€ \\

Suspicion MASLD/MASH (=3 interventions) &
DP : R945 \newline
DAS : E11.98 &
07M14T &
2584 &
298~€ \\
\bottomrule
\end{tabularx}

\caption{Codage et GHS associés — HDJ Maladies du Foie Métaboliques}
\end{table}

\medskip

\noindent Actes CCAM typiques : HLQM002 (élastographie), ZCQM004 (échographie–Doppler), consultations médicale/diététique/psychologique.

\clearpage

% --------------------------------------------------------
% TABLEAU 2 — Traçabilité (VERSION SPÉCIFIQUE MASLD/MASH)
% --------------------------------------------------------

\begin{table}[h!]
\centering
\renewcommand{\arraystretch}{1.25}
\rowcolors{2}{APHPsoft}{white}

\begin{tabular}{p{5cm} p{9cm}}
\toprule
\rowcolor{APHPsoft}
\textbf{Intervention} & \textbf{Traçabilité requise} \\
\midrule

Consultation médicale de synthèse &
Motif, anamnèse ciblée (métabolique, hépatique, cardiovasculaire), facteurs de risque, résultats clés (FIB-4, LSM, biologie), diagnostic de probabilité MASLD/MASH, décision thérapeutique, critères d'orientation, messages ville–hôpital. \\

Évaluation diététique &
Analyse des apports, structure des repas, comportements alimentaires, objectifs pondéraux, sarcopénie/dénutrition éventuelle, recommandations personnalisées, plan nutritionnel réaliste, critères de suivi. \\

Évaluation psychologique / addictologique (alcool, TCA) &
Repérage des consommations (auditif/alcohol), screening TCA, analyse motivationnelle, facteurs émotionnels contextuels, ressources disponibles, recommandations ciblées, orientation éventuelle vers un suivi spécialisé. \\

Consultations spécialisées (cardiovasculaire, diabétologie...) &
Paramètres cardio-métaboliques (TA, PA, IMC, tour de taille), dépistage du risque cardiovasculaire, évaluation diabète/pré-diabète, recommandations et coordination des prises en charge. \\

Actes techniques (élastographie, échographie) &
Compte rendu normalisé en annexe \\

Courrier de sortie (obligatoire) &
Synthèse intégrée : diagnostic MASLD/MASH probable ou confirmé, niveau de risque cardio-métabolique, interventions réalisées, décisions thérapeutiques, calendrier de suivi, messages clefs pour le médecin traitant. \\

\end{tabular}

\caption{Traçabilité des interventions — HDJ MASLD/MASH}
\end{table}

\clearpage

% ============================================================
% VOLUMÉTRIE (RÉFÉRENCE)
% ============================================================

\subsection{Volumétrie de référence}

\noindent HDJ MASLD/MASH — estimations selon montée en charge prévue.

\begin{center}
\begin{tabular}{lccc}
\toprule
\textbf{Type de séance} & \textbf{Volume annuel} & \textbf{Tarif unitaire} & \textbf{Recette estimée} \\
\midrule
≥4 interventions (GHS 2559) & 120 & 603~€ & 72\,360~€ \\
3 interventions (GHS 2584)  & 30  & 298~€ & 8\,940~€ \\
\midrule
\textbf{Total} & \textbf{150} & -- & \textbf{\textasciitilde81\,300~€} \\
\bottomrule
\end{tabular}
\end{center}

\clearpage


% ============================================================
% PROJECTIONS D’ACTIVITÉ
% ============================================================

\subsection{Projections d’activité et recettes prévisionnelles}

\noindent Basé sur un tarif moyen pondéré : \textbf{\textasciitilde540~€ / séance}.

\begin{center}
\begin{tabular}{lccc}
\toprule
\textbf{Année} & \textbf{Volume estimé} & \textbf{Tarif moyen} & \textbf{Recette brute estimée} \\
\midrule
Amorce   & 150 & 540~€ & 81\,000~€ \\
Montée   & 300 & 540~€ & 162\,000~€ \\
Croisière & 500 & 540~€ & 270\,000~€ \\
\bottomrule
\end{tabular}
\end{center}

\clearpage

% ============================================================
% Conclusions
% ============================================================
\subsection{Conclusion}
\needspace{6\baselineskip}

L’HDJ hépatométabolique constitue un dispositif pertinent, simple à organiser et autosoutenable. Il optimise le dépistage, l’accès aux thérapeutiques et la prise en charge multidisciplinaire des patients MASLD/MASH et DT2.

\medskip

Pour les supports opérationnels (échographie, évaluation diététique et psychologique), 
voir les annexes~\ref{sec:annexe_echo}, \ref{sec:annexe_diete} et \ref{sec:annexe_psy}.


% ============================================================
% VALIDATION
% ============================================================

\begin{center}
\begin{tabular}{p{4cm} p{7cm} p{4cm}}
\toprule
\rowcolor{APHPsoft}
\textbf{Date d'envoie} & \textbf{Nom du relecteur} & \textbf{Date de validation} \\
\midrule

03/12/2025 & Pr V.\,Mallet & 03/12/2025 \\
05/12/2025 & Dr L.\,Parlati & NA \\
03/12/2025 & Dr S.\,Bouam & NA \\
NA & Pr R.\,Coriat & NA \\
\bottomrule
\end{tabular}
\end{center}

\clearpage
\printbibliography[heading=subbibliography,title={Références}]
\end{refsection}

% ============================
% CHAPITRE 5: Addictologie
% ============================
\clearpage
\section{HDJ Addictologie}
\begin{refsection}
% ============================================================
% HÔPITAL DE JOUR — ADDICTOLOGIE (HDJA)
% ============================================================
\subsection{Rationnel médical}
\begin{spacing}{1.30}
L’alcool constitue en France un déterminant majeur de morbi-mortalité évitable, responsable d’environ 41\,000 décès annuels\cite{Guerin2013AlcoolMortalite} — incluant 16\,000 cancers, 9\,900 maladies cardiovasculaires, 6\,800 maladies digestives et 5\,400 causes externes — pour un coût social estimé à 118 milliards d’euros par an.\cite{Kopp2015CoutSocial} Il représente la première cause identifiable de démence précoce (<65 ans)\cite{Schwarzinger2018AlcoholDementia}, et les troubles d’usage d’alcool réduisent l’espérance de vie de 10 à 13 ans dans leurs formes les plus sévères.\cite{Schwarzinger2017ChronicDisease}

Dans la population générale, 23,6\,\% des adultes dépassent les repères de consommation à faible risque\cite{Richard2019AlcoolSPF}, avec une prévalence particulièrement élevée dans les filières hospitalières exposées (hépatologie, oncologie, diabétologie, psychiatrie). L’alcool potentialise la progression de toutes les maladies chroniques du foie — y compris à faible niveau de consommation — et constitue le principal déterminant évolutif des complications sévères chez les patients atteints de diabète de type~2.\cite{Mallet2022T2DLiverBurden}

L’abstinence est le levier pronostique le plus puissant dans les maladies hépatiques liées à l’alcool : elle améliore la survie après hépatite alcoolique sévère\cite{Louvet2008AHA, Parlati2025RehabAH} et diminue le risque de décompensation chez les patients cirrhotiques.\cite{Addolorato_Baclofen_2007, Loomba_AlcoholAbstinence_2020} Les prises en charge intensives du sevrage réduisent également l’incidence des cancers attribuables à l’alcool et améliorent la survie globale.\cite{Schwarzinger2024RehabCancer}

Dans ce contexte, l’AP-HP a un rôle structurant à jouer dans l’organisation des parcours dédiés. Un HDJ addictologique intégré au sein du DMU DIGEST répondrait à un besoin clairement identifié : filière institutionnelle lisible, interventions spécialisées, continuité des soins et réduction des hospitalisations évitables. Ainsi, la structuration d’un HDJ addictologie s’inscrit pleinement dans les missions de santé publique de l’AP-HP.

\end{spacing}
% Réinitialisation des compteurs (doit être placée ici)

\clearpage
% ============================================================
% OBJECTIFS
% ============================================================

\subsection{Objectifs}

\begin{center}
\fcolorbox{APHPdark}{APHPsoft}{
\begin{minipage}{0.95\textwidth}
\vspace{0.7em}
\begin{itemize}[leftmargin=1.1cm]
\item Réaliser une évaluation somatique et addictologique complète ;
\item Réduire les risques et dommages liés à l’alcool ;
\item Stabiliser la trajectoire hépatologique et prévenir les décompensations ;
\item Accompagner la réduction ou l’arrêt des consommations ;
\item Renforcer l’autonomie et la continuité des soins ville–hôpital.
\end{itemize}
\vspace{0.7em}
\end{minipage}}
\end{center}

\clearpage

% ============================================================
% POPULATION ÉLIGIBLE
% ============================================================

\subsection{Population éligible}

\begin{itemize}[leftmargin=1.1cm]
\item Trouble de l’usage d’alcool (usage nocif ou dépendance) ;
\item Pathologies hépatiques alcool-attribuables (HAA, cirrhoses, MetALD) ;
\item Indication d’une évaluation somatique–addictologique conjointe ;
\item Objectif d’abstinence, de réduction des risques ou de stabilisation ;
\item Fragilités psychosociales et/ou cognitives nécessitant un suivi structuré.
\end{itemize}

\footnotetext{
\textbf{HAA} : hépatite alcoolique aiguë ; 
\textbf{MetALD} : metabolic dysfunction–associated alcohol-related liver disease.
}

\clearpage


% ============================================================
% PARCOURS DE SOINS
% ============================================================

\subsection{Parcours de soins}

% --- PAGE DES ACRONYMES (nouvelle page blanche dédiée) ---
\textbf{Acronymes utilisés dans les parcours HDJ Addictologie}
\vspace{1cm}

\begin{itemize}[leftmargin=1.2cm]
    \item \textbf{ELSA} : Équipe de liaison et de soins en addictologie
    \item \textbf{CSAPA} : Centre de soins, d’accompagnement et de prévention en addictologie
    \item \textbf{IDE} : Infirmier diplômé d’État
    \item \textbf{MOCA} : Montreal Cognitive Assessment
    \item \textbf{BEARNI} : Brief Evaluation of Alcohol-Related Neuropsychological Impairments
    \item \textbf{ECG} : Électrocardiogramme
    \item \textbf{EFR} : Explorations fonctionnelles respiratoires
    \item \textbf{CI} : Contre-indication
    \item \textbf{CIWA} : Clinical Institute Withdrawal Assessment
    \item \textbf{APA} : Activité physique adaptée
    \item \textbf{SSR} : Soins de suite et de réadaptation
\end{itemize}

\clearpage
% --- FIN PAGE ACRONYMES / DÉBUT DES FIGURES ---


% === FIGURE 1 — Évaluation somatique et addictologique ========================

\begin{figure}[h!]
\centering
\caption{Parcours évaluation addictologique et somatique}
\vspace{0.8cm}

\begin{tikzpicture}[
    node distance=1.8cm,
    box/.style={
        rectangle, rounded corners=3pt,
        draw=APHPdark, thick,
        text width=10cm, minimum height=1.7cm,
        align=center, fill=APHPsoft
    }
]

\node[box] (ori) {Orientation vers l’HDJ \\
\small médecin traitant, ELSA, CSAPA, psychiatrie, hépatologie, consultations d’addictologie};

\node[box, below=1.5cm of ori] (eval) {Évaluation médicale et infirmière \\
\small addictologie, hépatologie, entretien IDE};

\node[box, below=1.5cm of eval] (neuro) {Évaluations psychologique et neurocognitive \\
\small psychologue ; MOCA / BEARNI};

\node[box, below=1.5cm of neuro] (som) {Bilan somatique \\
\small ECG, EFR, scanner thoracique, biologie, échographie, Fibroscan};

\node[box, below=1.5cm of som] (synth) {Synthèse pluridisciplinaire \\
\small orientation : HDJ réduction / HDJ sevrage / hospitalisation complète};

\draw[->, thick, APHPdark] (ori) -- (eval);
\draw[->, thick, APHPdark] (eval) -- (neuro);
\draw[->, thick, APHPdark] (neuro) -- (som);
\draw[->, thick, APHPdark] (som) -- (synth);

\end{tikzpicture}
\end{figure}

% === FIGURE 2 — Réduction des risques ========================================

\begin{figure}[h!]
\centering
\caption{Parcours réduction des risques et dommages}
\vspace{0.8cm}

\begin{tikzpicture}[
    node distance=1.8cm,
    box/.style={rectangle, rounded corners=3pt, draw=APHPdark, thick,
    text width=10cm, minimum height=1.7cm, align=center, fill=APHPsoft}
]

\node[box] (ori) {Orientation vers l’HDJ \\
\small patients souhaitant réduire leur consommation sans objectif de sevrage complet};

\node[box, below=1.5cm of ori] (med)
{Suivi médical et paramédical renforcé \\ \small addictologie, IDE, psychologue};

\node[box, below=1.5cm of med] (medias)
{Thérapies par médiation \\ \small APA, socio-esthétique, art-thérapie, revue de presse, écriture, groupes};

\node[box, below=1.5cm of medias] (social)
{Accompagnement social \\ \small démarches administratives, insertion, gestion de crise sociale};

\node[box, below=1.5cm of social] (synth)
{Synthèse pluridisciplinaire \\ \small adaptation du programme, continuité des soins};

\draw[->, thick, APHPdark] (ori) -- (med);
\draw[->, thick, APHPdark] (med) -- (medias);
\draw[->, thick, APHPdark] (medias) -- (social);
\draw[->, thick, APHPdark] (social) -- (synth);

\end{tikzpicture}
\end{figure}

% === FIGURE 3 — Sevrage ambulatoire ========================================

\begin{figure}[h!]
\centering
\caption{Parcours Sevrage ambulatoire}
\vspace{0.8cm}

\begin{tikzpicture}[
    node distance=1.8cm,
    box/.style={rectangle, rounded corners=3pt, draw=APHPdark, thick,
    text width=10cm, minimum height=1.7cm, align=center, fill=APHPsoft}
]

\node[box] (indi)
{Indication de sevrage ambulatoire \\ \small critères de sécurité, absence de CI, environnement compatible};

\node[box, below=1.5cm of indi] (med)
{Évaluation et suivi médical \\ \small addictologie, IDE, monitorage CIWA quotidien};

\node[box, below=1.5cm of med] (psy)
{Suivi psychologique \\ \small soutien, renforcement motivationnel};

\node[box, below=1.5cm of psy] (medias)
{Thérapies par médiation \\ \small APA, art-thérapie, socio-esthétique, revue de presse, groupes};

\node[box, below=1.5cm of medias] (soc)
{Accompagnement social \\ \small démarches, stabilisation du cadre de vie};

\node[box, below=1.5cm of soc] (synth)
{Synthèse pluridisciplinaire \\ \small plan de continuité, prévention des rechutes};

\draw[->, thick, APHPdark] (indi) -- (med);
\draw[->, thick, APHPdark] (med) -- (psy);
\draw[->, thick, APHPdark] (psy) -- (medias);
\draw[->, thick, APHPdark] (medias) -- (soc);
\draw[->, thick, APHPdark] (soc) -- (synth);

\end{tikzpicture}
\end{figure}


% === FIGURE 4 — Consolidation de sevrage ====================================

\begin{figure}[h!]
\centering
\caption{Parcours Consolidation de sevrage}
\vspace{0.8cm}

\begin{tikzpicture}[
    node distance=1.8cm,
    box/.style={rectangle, rounded corners=3pt, draw=APHPdark, thick,
    text width=10cm, minimum height=1.7cm, align=center, fill=APHPsoft}
]

\node[box] (ori)
{Public concerné \\ \small post-sevrage ambulatoire ou résidentiel ; attente SSR ; retours de SSR};

\node[box, below=1.5cm of ori] (med)
{Suivi médical et paramédical structuré \\ \small addictologie ; IDE ; psychologue};

\node[box, below=1.5cm of med] (medias)
{Thérapies par médiation \\ \small APA, socio-esthétique, art-thérapie, revue de presse, écriture, groupes};

\node[box, below=1.5cm of medias] (reinser)
{Accompagnement social et réinsertion \\ \small démarches ; retour au domicile ; activités};

\node[box, below=1.5cm of reinser] (synth)
{Synthèse pluridisciplinaire \\ \small consolidation du sevrage ; prévention des rechutes};

\draw[->, thick, APHPdark] (ori) -- (med);
\draw[->, thick, APHPdark] (med) -- (medias);
\draw[->, thick, APHPdark] (medias) -- (reinser);
\draw[->, thick, APHPdark] (reinser) -- (synth);

\end{tikzpicture}
\end{figure}



% ============================================================
% TRAÇABILITÉ
% ============================================================
\clearpage
\subsection{Traçabilité des interventions}

\begin{table}[h!]
\centering
\renewcommand{\arraystretch}{1.25}
\rowcolors{2}{APHPsoft}{white}

\begin{tabular}{p{5cm} p{9.2cm}}
\toprule
\rowcolor{APHPsoft}
\textbf{Intervention} & \textbf{Traçabilité requise} \\
\midrule

Évaluation somatique et addictologique &
Anamnèse ; scores AUDIT/CIWA ; comorbidités ; bilans pré-thérapeutiques ; évaluations IDE, psychologue, neurocognition ; consultations spécialisées ; synthèse médicale. \\

Réduction des risques &
Objectifs personnalisés ; prévention ; entretiens motivationnels ; ateliers ; activités (Annexe~C.2) ; réévaluation hebdomadaire. \\

Sevrage ambulatoire &
Scores CIWA répétés ; adaptation thérapeutique ; surveillance ; activités psychologiques/sociales/médiations (Annexe~C.2) ; incidents ; synthèse médicale de fin de cure. \\

Consultations spécialisées &
Synthèse écrite obligatoire : objectifs, évolution, recommandations. \\

Synthèse médico-psycho-sociale &
Plan intégré ; orientation ; coordination ville–hôpital. \\
\bottomrule
\end{tabular}

\caption{Traçabilité des interventions — HDJ Addictologie}
\end{table}

\clearpage

% ============================================================
% ORGANISATION
% ============================================================

\subsection{Organisation}

\begin{itemize}[leftmargin=1.1cm]
    \item Direction : \textbf{Docteur Marion Corouge}
    \item Durée : 6--8~heures
    \item Lieu : Secteur HDJ — Service des maladies du foie
    \item Ressources : médecin senior, infirmier expert/IPA, psychologue/addictologue
\end{itemize}

\clearpage

% ============================================================
% SYNTHÈSE — CODAGE, VOLUMÉTRIE ET RECETTES
% ============================================================

\subsection{Synthèse codage–volumétrie–recettes}

\begin{sidewaystable}[p]
\centering
\renewcommand{\arraystretch}{1.22}
\rowcolors{2}{APHPsoft}{white}

\begin{tabularx}{\textwidth}{
p{4.8cm}
X
>{\centering\arraybackslash}p{1.6cm}
>{\centering\arraybackslash}p{1.6cm}
>{\centering\arraybackslash}p{1.7cm}
>{\centering\arraybackslash}p{2cm}
>{\centering\arraybackslash}p{2.7cm}
}
\toprule
\rowcolor{APHPsoft}
\textbf{Type de séance} &
\textbf{DP / DAS attendus} &
\textbf{GHM} &
\textbf{GHS} &
\textbf{Tarif 2025} &
\textbf{Volume annuel projeté} &
\textbf{Recette annuelle projetée} \\
\midrule

HDJ Évaluation somatique + addictologique &
DP : F101\newline
DAS : comorbidités majeures &
20Z051 & 7200 & 774~€ &
47 & 36\,378~€ \\

HDJ Réduction des risques et dommages &
DP : Z714\newline
DAS : comorbidités majeures &
23M06T & 7272 & 701~€ &
47 & 32\,947~€ \\

HDJ Sevrage ambulatoire / consolidation &
DP : F102 ou Z502\newline 
DAS : comorbidités majeures &
20Z04T & 7271 & 541~€ &
47 & 25\,478~€ \\
\midrule

\textbf{Total annuel — Phase d’amorce (1/j)} & -- & -- & -- & -- &
\textbf{141} &
\textbf{96\,703~€} \\
\bottomrule
\end{tabularx}   % ← OBLIGATOIRE : fermeture de tabularx

\caption{Synthèse pivotée — Codage, volumétrie et recettes projetées (HDJ Addictologie)}
\end{sidewaystable}

\clearpage
% ============================================================
% PROJECTIONS D’ACTIVITÉ ET RECETTES PRÉVISIONNELLES
% ============================================================

\subsection{Projections d’activité et recettes prévisionnelles}

\textbf{Hypothèses volumétriques et tarifaires (2025).}
\begin{itemize}
    \item 1, puis 3, puis 5 patients par jour ;
    \item 3 jours d’activité par semaine ;
    \item 47 semaines par an ;
    \item Tarifs GHS 2025 : 774~€ (évaluation), 701~€ (réduction des risques), 541~€ (sevrage) ;
    \item Recettes annuelles = somme pondérée des trois types d’actes selon leur fréquence moyenne observée.
\end{itemize}

\begin{table}[h!]
\centering
\renewcommand{\arraystretch}{1.22}
\rowcolors{2}{APHPsoft}{white}

\begin{tabular}{
p{5.5cm}
>{\centering\arraybackslash}p{2.5cm}
>{\centering\arraybackslash}p{2.5cm}
>{\centering\arraybackslash}p{2.5cm}
}
\toprule
\rowcolor{APHPsoft}
\textbf{Phase d’activité} &
\textbf{Patients/an} &
\textbf{Séances/an} &
\textbf{Recettes/an} \\
\midrule

Amorce (1/jour) &
141 &
141 &
96\,703~€ \\

Montée en charge (3/jour) &
423 &
423 &
289\,936~€ \\

Croisière (5/jour) &
705 &
705 &
483\,269~€ \\

\bottomrule
\end{tabular}

\caption{Projections d’activité et recettes prévisionnelles — HDJ Addictologie}
\end{table}

\clearpage


% ============================================================
% CONCLUSION
% ============================================================

\subsection{Conclusion}

L’HDJ addictologie constitue un dispositif structurant permettant une intervention précoce, intensive et coordonnée pour les patients présentant un trouble de l’usage d’alcool. Il améliore la sécurité du sevrage, la réduction des risques, la stabilité hépatique et limite les hospitalisations non programmées. Son positionnement transversal dans le GHU en fait un outil central de prévention secondaire à fort impact de santé publique.

% ============================================================
% APPELS D’ANNEXES ADDICTOLOGIE
% ============================================================

\subsection*{Annexes associées}
\begin{itemize}
  \item Annexe~\ref{annexe:addicto_activites} — Grille d’évaluation somatique–addictologique
  \item Annexe~\ref{annexe:addicto_tracabilite} — Grille des programmes de médiation (addictologie)
\end{itemize}

\clearpage


% ============================================================
% VALIDATION
% ============================================================

\begin{center}
\begin{tabular}{p{4cm} p{7cm} p{4cm}}
\toprule
\rowcolor{APHPsoft}
\textbf{Date d’envoi} & \textbf{Nom du relecteur} & \textbf{Date de validation} \\
\midrule
03/12/2025 & Pr V.\,Mallet & 08/12/2025 \\
03/12/2025 & Dr S.\,Bouam & 08/12/2025 \\
08/12/2025 & Dr M.\,Corouge & NA \\
08/12/2025 & Dr J.\,Nabarro & NA \\
08/12/2025 & Dr D.\,Karinthi & NA \\
NA          & Pr R.\,Coriat & NA \\
\bottomrule
\end{tabular}
\end{center}

\clearpage


\printbibliography[heading=subbibliography,title={Références}]
\end{refsection}

% ============================
% CHAPITRE 6: MICI
% ============================
\clearpage
\section{HDJ MICI}
\begin{refsection}
% ============================================================
% HÔPITAL DE JOUR — MICI (Maladies Inflammatoires Chroniques de l’Intestin)
% ============================================================

\setcounter{table}{0}
\setcounter{figure}{0}

\subsection{Rationnel médical}
\needspace{8\baselineskip}

\begin{spacing}{1.30}

Les maladies inflammatoires chroniques de l’intestin (MICI) — maladie de Crohn et rectocolite hémorragique — sont des affections évoluant par poussées inflammatoires, responsables d’un retentissement fonctionnel important, d’une altération de la qualité de vie et d’un risque de complications sévères (abcès, sténoses, fistules, colites aiguës graves), avec risques de sepsis et de recours urgent à la chirurgie.

En France, les MICI concernent près de 300\,000 personnes, avec une incidence et une prévalence en croissance régulière de 3 à 4~\% par an.\cite{Ng_LancetGlobalIBD_2017,Richard_EpidemioPresseMed_2025} L’Europe compte parmi les régions les plus touchées, avec une prévalence avoisinant 0.5~\% de la population adulte. L’incidence augmente de façon marquée dans la population pédiatrique.

Un traitement précoce et adapté — notamment par biothérapie — réduit significativement le risque de complications, d’hospitalisations et de recours à la chirurgie, tout en améliorant la qualité de vie. Les biothérapies (anti-TNF, vedolizumab, ustekinumab, risankizumab) et traitements ciblés ont transformé le pronostic, mais nécessitent une organisation rigoureuse : perfusions intraveineuses, initiations sous-cutanées encadrées, bilans pré-thérapeutiques, éducation thérapeutique, surveillance de tolérance et prise en charge coordonnée des comorbidités (anémie, nutrition, santé mentale).\cite{HAS_AntiTNF_2019}

Dans le cadre du socle organisationnel des HDJ digestifs, le HDJ MICI permet de concentrer en une séquence unique: perfusion de biothérapie, surveillance infirmière MICI, évaluation clinique inter-cure, éducation thérapeutique, vaccinations, coordination ville–hôpital et planification thérapeutique individualisée. Ce format constitue l’organisation recommandée pour sécuriser et optimiser l’utilisation des biothérapies dans les MICI. Le dispositif offre également un cadre privilégié pour identifier les patients éligibles aux essais cliniques et aux cohortes observationnelles, en cohérence avec les missions universitaires du GHU.

À Cochin, le recrutement provient des correspondants médicaux, des services d’urgences, de la pédiatrie et de la filière de transition organisée avec Necker. 

\end{spacing}

\clearpage

% ============================================================
% OBJECTIFS
% ============================================================

\subsection{Objectifs}
\needspace{6\baselineskip}

\vspace{0.8em}
\begin{center}
\fcolorbox{APHPdark}{APHPsoft}{
\begin{minipage}{0.95\textwidth}
\vspace{0.9em}

\begin{itemize}[leftmargin=1.1cm]
  \item assurer un accès sécurisé aux biothérapies intraveineuses ;
  \item organiser les évaluations inter-cures ;
  \item proposer une prise en charge pluridisciplinaire : IDE MICI, diététique, psychologie, ETP, vaccinations ;
  \item réduire les hospitalisations complètes non programmées et optimiser le suivi treat-to-target ;
  \item structurer la coordination ville–hôpital et la traçabilité des décisions (consultation, RCP MICI) ;
  \item participer au repérage des patients éligibles à la recherche clinique (cohortes, essais thérapeutiques).
\end{itemize}

\vspace{0.9em}
\end{minipage}}
\end{center}

\bigskip

% ============================================================
% POPULATION ÉLIGIBLE
% ============================================================

\subsection{Population éligible}
\needspace{6\baselineskip}

\begin{itemize}[leftmargin=1.1cm]
  \item maladie de Crohn ou rectocolite hémorragique relevant d’une biothérapie IV ;
  \item première injection SC (infliximab, anti-TNF) nécessitant apprentissage, supervision et surveillance dédiée ;
  \item perfusion de fer injectable dans le cadre d’une anémie ferriprive associée aux MICI ;
  \item réévaluation inter-cure incluant ≥3 interventions (gastroentérologue, infectiologue, psychologue, IDE MICI, diététique, ETP) ;
  \item patients adressés en RCP MICI pour initiation, switch ou optimisation de biothérapie ;
  \item parcours de transition pédiatrie (Necker) → adulte (Cochin).
\end{itemize}

\clearpage

% ============================================================
% PARCOURS DE SOINS
% ============================================================

\subsection{Parcours de soins}
\needspace{8\baselineskip}

\begin{figure}[!ht]
\centering
\caption{Parcours patient — HDJ MICI}
\vspace{0.8cm}

\begin{tikzpicture}[
    node distance=1.6cm,
    box/.style={
        rectangle,
        rounded corners=3pt,
        draw=APHPdark,
        thick,
        text width=9.0cm,
        minimum height=1.55cm,
        align=center,
        fill=APHPsoft
    }
]

\node[box] (tri) {Orientation vers l’HDJ \\ 
\small Consultations MICI, RCP, médecine de ville, urgences, transition NCK→CCH};

\node[box, below=1.4cm of tri] (etape1) {Évaluation initiale IDE MICI \\ 
\small Bilans pré-thérapeutiques, critères de \\sécurité, calprotectine, dépistage comorbidités};

\node[box, below=1.4cm of etape1] (etape2) {Acte thérapeutique programmé \\ 
\small perfusion biothérapie IV/initiation SC encadrée/fer injectable};

\node[box, below=1.4cm of etape2] (etape3) {Surveillance clinique et éducative \\ 
\small tolérance, observance, statut\\vaccinal, conseils pratiques, ETP MICI};

\node[box, below=1.4cm of etape3] (synth) {Synthèse médicale et coordination \\ 
\small évaluation inter-cure, ajustement thérapeutique, lien ville–hôpital};

\draw[->, thick, APHPdark] (tri) -- (etape1);
\draw[->, thick, APHPdark] (etape1) -- (etape2);
\draw[->, thick, APHPdark] (etape2) -- (etape3);
\draw[->, thick, APHPdark] (etape3) -- (synth);

\end{tikzpicture}
\end{figure}

\clearpage

% ============================================================
% Organisation et ressources nécessaires
% ============================================================

\subsection{Organisation}
\needspace{5\baselineskip}

\begin{itemize}[leftmargin=1.1cm]
    \item Direction : \textbf{Docteur V.\,Abitbol}
    \item Durée : 6--8~heures.
    \item Lieu : Secteur HDJ — Service de gastroentérologie et d’oncologie digestive.
    \item Ressources : médecin senior, infirmier expert/IPA, psychologue/addictologue.
\end{itemize}
    
\clearpage

% ============================================================
% CODAGE, VOLUMÉTRIE ET RECETTES — HDJ MICI
% ============================================================

\subsection{Codage, volumétrie et recettes — synthèse opérationnelle}
\needspace{8\baselineskip}

\noindent
Activité 2024 : \textbf{1\,346 séances}, majoritairement dédiées aux biothérapies IV.  
Capacité actuelle : \textbf{8 fauteuils} — \textbf{2 patients / fauteuil / jour} — 
\textbf{2 jours / semaine} (≈ \textbf{32 patients / semaine}).

\begin{table}[h!]
\centering
\renewcommand{\arraystretch}{1.22}
\rowcolors{2}{APHPsoft}{white}

\begin{tabularx}{\textwidth}{
p{5.2cm}
X
>{\centering\arraybackslash}p{2.1cm}
>{\centering\arraybackslash}p{1.9cm}
>{\centering\arraybackslash}p{3.1cm}
}
\toprule
\rowcolor{APHPsoft}
\textbf{Séance HDJ MICI} &
\textbf{Actes / codage principal} &
\textbf{GHM / GHS} &
\textbf{Tarif} &
\textbf{Volume / Recette 2024} \\
\midrule

\textbf{Biothérapie IV (entretien)} &
Perfusion IV \newline
DP : Z51.2 ; DR : K50.x / K51.x &
28Z17Z / 9616 &
440~€ &
\textbf{1\,333} \newline
\textbf{586\,520~€} \\

\textbf{Initiation SC (ETP)} &
Injection SC, ETP \newline
DP : K50.x / K51.x ; DAS : Z71.9 &
06M07T / 2152 &
655~€ &
0 \newline
0~€ \\

\textbf{Évaluation inter-cure ≥4 actes} &
Consultation + nutrition + psy + biologie \newline
DP : Z09.2 ; DR : MICI &
06M16Z / 2186 &
1\,071~€ &
12 \newline
12\,852~€ \\

\textbf{Évaluation inter-cure =3 actes} &
Consultation + évaluations ciblées \newline
DP : Z09.2 ; DR : MICI &
06M16Z / 5059 &
360~€ &
1 \newline
360~€ \\

\midrule
\textbf{Total HDJ MICI} &
— &
— &
— &
\textbf{1\,346 séances} \newline
\textbf{\textasciitilde 600\,000~€} \\
\bottomrule
\end{tabularx}

\caption{Synthèse opérationnelle — HDJ MICI : structure d’activité, codage PMSI, volumétrie et recettes (2024).}
\end{table}

\clearpage

% --------------------------------------------------------
% TABLEAU 2 — Traçabilité
% --------------------------------------------------------

\subsection{Traçabilité minimale}

\begin{table}[h!]
\centering
\renewcommand{\arraystretch}{1.25}
\rowcolors{2}{APHPsoft}{white}

\begin{tabular}{p{5cm} p{9cm}}
\toprule
\rowcolor{APHPsoft}
\textbf{Intervention} & \textbf{Traçabilité requise} \\
\midrule

Biothérapie IV 
& protocole et dose ; voie d’administration ; surveillance per-cure ; effets indésirables ; décision de poursuite \\

1\textsuperscript{re} injection SC (infliximab/anti-TNF) 
& éducation à l’auto-injection ; vérification des pré-requis de sécurité ; tolérance immédiate ; planification des injections suivantes \\

Évaluation infirmière MICI 
& paramètres cliniques ; bilans réalisés ; statut vaccinal ; observance déclarée ; comorbidités associées \\

Biologie de suivi 
& indication ; résultats clés ; interprétation clinique ; conduite à tenir (adaptation thérapeutique) \\

Consultation diététique / psychologique 
& synthèse des évaluations ; recommandations ; objectifs fixés ; suivi prévu \\

Synthèse médicale inter-cure 
& évaluation réponse ; décisions thérapeutiques ; prochaine cure ; coordination ville–hôpital \\

Vaccinologie 
& statut vaccinal documenté ; rappels effectués (grippe, pneumocoque, covid, hépatites) ; recommandations données \\

Éligibilité recherche clinique 
& critères repérés ; information patient ; orientation vers essais / cohortes ; contact recherche \\

\bottomrule
\end{tabular}

\caption{Traçabilité des interventions}
\end{table}

\clearpage

% ============================================================
% PROJECTIONS D’ACTIVITÉ
% ============================================================

\subsection{Projections d’activité et recettes prévisionnelles}
\needspace{6\baselineskip}

\noindent\textbf{Hypothèses.}  
Base d’activité 2024 : \textbf{1\,333 HDJ}.  
Croissance progressive estimée : \textbf{+100 HDJ / an}.  
Tarif moyen pondéré calculé : \textbf{446~€ / séance}.

\medskip

\begin{table}[h!]
\centering

\renewcommand{\arraystretch}{1.22}
\rowcolors{2}{APHPsoft}{white}

\begin{tabular}{
l
>{\centering\arraybackslash}p{3cm}
>{\centering\arraybackslash}p{3cm}
>{\centering\arraybackslash}p{4cm}
}
\toprule
\rowcolor{APHPsoft}
\textbf{Phase} &
\textbf{Volume projeté} &
\textbf{Tarif moyen} &
\textbf{Recette brute estimée} \\
\midrule

Amorce &
1\,333 &
446~€ &
594\,518~€ \\

Montée en charge &
1\,433 &
446~€ &
639\,118~€ \\

Croisière &
1\,533 &
446~€ &
683\,718~€ \\
\bottomrule
\end{tabular}

\caption{Projections d’activité et recettes prévisionnelles — HDJ MICI}

\footnotetext{
\textbf{Calcul du tarif moyen pondéré (446~€).}  
Biothérapie IV : 1\,469 séances × 440~€ ;  
Évaluations ≥4 interventions : 14 séances × 1\,071~€ ;  
Évaluations =3 interventions : 2 séances × 360~€.  
Recette totale 2024 : 662\,074~€ pour 1\,485 HDJ  
→ Tarif moyen pondéré = \(662\,074 / 1\,485 = 445{,}9 \approx 446~€\).
}

\end{table}

\clearpage

% ============================================================
% CONCLUSION
% ============================================================

\subsection{Conclusion}
\needspace{6\baselineskip}

L’HDJ MICI constitue un dispositif structurant, assurant une administration sécurisée des biothérapies et une surveillance « treat-to-target » conforme aux recommandations internationales. Il permet une prise en charge intégrée — médicale, infirmière, éducative et psychosociale — adaptée à la complexité croissante des parcours MICI.
Dans un contexte d’augmentation continue de la prévalence des MICI, l’HDJ garantit efficience organisationnelle, continuité des soins et réduction des hospitalisations complètes non programmées. L’activité est en progression régulière depuis la création du HDJ MICI en 2019.

% ============================================================
% VALIDATION DE LA FICHE
% ============================================================

\begin{center}
\begin{tabular}{p{4cm} p{7cm} p{4cm}}
\toprule
\rowcolor{APHPsoft}
\textbf{Date de relecture} & \textbf{Nom du relecteur} & \textbf{Date de validation} \\
\midrule
NA & Pr R.\,Coriat & NA \\
01/12/2025 & Pr V.\,Abitbol & 01/12/2025 \\
03/12/2025 & Dr S.\,Bouam & NA \\
01/12/2025 & Pr V.\,Mallet & 03/12/2026 \\
\bottomrule
\end{tabular}
\end{center}

\clearpage

\printbibliography[heading=subbibliography,title={Références}]
\end{refsection}

% ========================================
% CHAPITRE 7: Chimiothérapies des Cancers
% ========================================
\clearpage
\section{HDJ Chimiothérapie des Cancers Digestifs}
\begin{refsection}
% ============================================================
% HÔPITAL DE JOUR — CHIMIOTHÉRAPIE
% ============================================================

\setcounter{table}{0} % Réinitialisation des tableaux
\setcounter{figure}{0} % Réinitialisation des tableaux

% ============================================================
% RATIONNEL MÉDICAL
% ============================================================

\subsection{Rationnel médical}
\needspace{8\baselineskip}
\begin{spacing}{1.30}

La chimiothérapie intraveineuse des cancers digestifs est aujourd’hui majoritairement administrée en HDJ, conformément aux référentiels organisationnels nationaux. Le document de l’AP–HP dédié à la sécurisation des chimiothérapies injectables et le référentiel organisationnel de l’INCa définissent l’HDJ comme la modalité privilégiée d’administration des anticancéreux en raison de la structuration, de la coordination interprofessionnelle et de la maîtrise du risque iatrogène \cite{APHP2020, ONC2021, INCa2025_SECUMED}. 

Ce modèle permet la mise en œuvre sécurisée des principaux protocoles utilisés en oncologie digestive (FOLFOX, FOLFIRI, FOLFIRINOX, CAPOX, gemcitabine–platine, immunothérapies, thérapies ciblées injectables) en intégrant une évaluation pré-cycle systématique, le contrôle des toxicités précédentes, la conciliation médicamenteuse et la traçabilité complète du processus de préparation, délivrance et administration.

La sélection des patients repose sur la stabilité clinique (ECOG~0–2), l’absence de complication aiguë (sepsis, déshydratation, occlusion), la faisabilité thérapeutique en ambulatoire et des conditions sociales compatibles. Les pompes externes pour les perfusions prolongées et la prophylaxie antiémétique/hématologique renforcent la sécurité du parcours.

Les données internationales confirment que l’administration ambulatoire des chimiothérapies digestives est sûre lorsque les filières sont structurées. Les taux de recours non programmés (urgences ou réhospitalisations) varient entre 4 et 20\,\%, selon les protocoles, tout en restant maîtrisés dans les équipes spécialisées \cite{Prince2019}. 

L’HDJ de chimiothérapie digestive assure ainsi une continuité thérapeutique optimale, réduit les hospitalisations conventionnelles, améliore l’expérience patient et garantit un haut niveau de qualité et de sécurité grâce à l’expertise pluridisciplinaire et aux circuits d’escalade en cas de complication.

\end{spacing}

\clearpage

% ============================================================
% OBJECTIFS
% ============================================================

\subsection{Objectifs}
\needspace{6\baselineskip}

\begin{center}
\fcolorbox{APHPdark}{APHPsoft}{
\begin{minipage}{0.95\textwidth}
\vspace{0.9em}
\begin{itemize}[leftmargin=1.1cm]
\item sécuriser l’administration ambulatoire des chimiothérapies injectables ;
\item standardiser l’évaluation pré-cycle et la surveillance immédiate ;
\item réduire les hospitalisations non programmées liées aux toxicités aiguës ;
\item optimiser l’organisation du parcours et la coordination ville–hôpital ;
\item améliorer l’expérience patient et la continuité des soins.
\end{itemize}
\vspace{0.9em}
\end{minipage}}
\end{center}

\clearpage

% ============================================================
% POPULATION ÉLIGIBLE
% ============================================================

\subsection{Population éligible}
\needspace{6\baselineskip}

\begin{itemize}[leftmargin=1.1cm]
\item état général compatible (ECOG~0–2), stabilité clinique ;
\item absence de complication aiguë : fièvre, déshydratation, syndrome tumoral évolutif ;
\item biologie conforme aux seuils décisionnels du protocole ;
\item protocole réalisable en ambulatoire (durée compatible, absence de surveillance prolongée) ;
\item autonomie suffisante et présence d’un accompagnant pour le retour.
\end{itemize}

\clearpage

% ============================================================
% PARCOURS DE SOINS
% ============================================================

\subsection{Parcours de soins}
\needspace{8\baselineskip}

\begin{figure}[!ht]
\centering
\caption{Parcours patient — HDJ Chimiothérapie}
\vspace{0.8cm}

\begin{tikzpicture}[
node distance=1.6cm,
box/.style={
    rectangle, rounded corners=3pt,
    draw=APHPdark, thick,
    text width=9cm,
    minimum height=1.5cm,
    align=center,
    fill=APHPsoft
}]

\node[box] (tri) {Orientation (oncologie médicale / hématologie / ville)};
\node[box, below=1.4cm of tri] (e1) {Évaluation pré-cycle : examen clinique, biologie, consentement, toxicités antérieures};
\node[box, below=1.4cm of e1] (e2) {Administration : prémédication, préparation pharmaceutique, perfusion supervisée};
\node[box, below=1.4cm of e2] (e3) {Surveillance immédiate : constantes, hypersensibilité, gestion des NV};
\node[box, below=1.4cm of e3] (syn) {Synthèse médicale, éducation thérapeutique, planification du cycle suivant};

\draw[->, thick, APHPdark] (tri) -- (e1);
\draw[->, thick, APHPdark] (e1) -- (e2);
\draw[->, thick, APHPdark] (e2) -- (e3);
\draw[->, thick, APHPdark] (e3) -- (syn);

\end{tikzpicture}
\end{figure}

\clearpage

% ============================================================
% ORGANISATION
% ============================================================

\subsection{Organisation}
\needspace{5\baselineskip}

\begin{itemize}[leftmargin=1.1cm]
\item Direction : \textbf{Dr Anna Pellat}
\item Durée : 6--8~heures
\item Lieu : HDJ — Gastroentérologie et oncologie digestive
\item Ressources : médecin sénior, infirmier expert/IPA, psychologue/nutritionniste
\end{itemize}

\clearpage

% ============================================================
% CODAGE, TARIFS ET VOLUMÉTRIE
% ============================================================

\subsection{Codage, tarifs et volumétrie de référence}
\needspace{10\baselineskip}

\begin{sidewaystable}[p]

\centering
\renewcommand{\arraystretch}{1.25}
\rowcolors{2}{APHPsoft}{white}

\begin{tabularx}{\textwidth}{
p{4.8cm}
>{\raggedright\arraybackslash}X
>{\centering\arraybackslash}p{1.7cm}
>{\centering\arraybackslash}p{1.6cm}
>{\centering\arraybackslash}p{1.9cm}
>{\centering\arraybackslash}p{1.8cm}
>{\centering\arraybackslash}p{2.4cm}
}
\toprule
\rowcolor{APHPsoft}
\textbf{Type de séance} &
\textbf{DP / DAS (cancers digestifs)} &
\textbf{GHM} &
\textbf{GHS} &
\textbf{Tarif 2025} &
\textbf{Volume 2024} &
\textbf{Recette 2024} \\
\midrule

Chimiothérapie cytotoxique IV &
DP : C22.0 (CHC), C22.1 (cholangioK), C25.x (pancréas), C17–C21 (intestin), C16.x (œsogastre)\newline
DAS : comorbidités, dénutrition, douleur, thrombose, ascite &
28Z07Z & 9616 & 495~€ &
XXX & ZZZZ~€ \\

Immunothérapie (anti-PD1/PD-L1) &
DP : C22.x, C25.x, C17–C21\newline
DAS : toxicités immunes (colite, hépatite, endocrinopathies) &
28Z17Z & 9616 & 440~€ &
XXX & ZZZZ~€ \\

Thérapies ciblées IV / Anticorps monoclonaux &
DP : C17–C21, C22.x\newline
DAS : mutation RAS/BRAF, métastases, complications locales &
28Z17Z & 9616 & 440~€ &
XXX & ZZZZ~€ \\

\midrule
\textbf{Total annuel} & -- & -- & -- & -- &
\textbf{568} & \textbf{322\,400~€} \\
\bottomrule
\end{tabularx}

\caption{Codage, tarifs et volumétrie — HDJ Chimiothérapie digestive (2024)}
\end{sidewaystable}

\clearpage

% ============================================================
% TRACABILITÉ
% ============================================================

\subsection{Traçabilité minimale}
\needspace{8\baselineskip}

\begin{table}[h!]
\centering
\renewcommand{\arraystretch}{1.25}
\rowcolors{2}{APHPsoft}{white}

\begin{tabular}{p{4.3cm} p{8.2cm}}
\toprule
\rowcolor{APHPsoft}
\textbf{Intervention} & \textbf{Éléments requis} \\
\midrule

Évaluation pré-cycle &
Examen clinique structuré, biologie validée, note des toxicités CTCAE, décision de poursuite/modification, conciliation médicamenteuse. \\

Préparation et administration &
Vérification des identités, protocole, doses, lots, double contrôle IDE/pharmacie, prémédications, durée d’infusion, incidents éventuels. \\

Surveillance immédiate &
Constantes avant/pendant/après perfusion, dépistage des réactions d’hypersensibilité, gestion des nausées/vomissements, tolérance clinique. \\

Voie veineuse / PAC &
État du site, perméabilité, incidents locaux, hémostase, retrait/flush selon protocole. \\

Synthèse médicale &
Tolérance du cycle, points de vigilance, adaptations posologiques, recommandations personnalisées, planification du cycle suivant, messages ville–hôpital. \\
\bottomrule
\end{tabular}

\caption{Traçabilité — HDJ Chimiothérapie digestive}
\end{table}

\clearpage


% ============================================================
% PROJECTIONS D’ACTIVITÉ — CHIMIOTHÉRAPIE DIGESTIVE
% ============================================================

\subsection{Projections d’activité et recettes prévisionnelles}

\noindent Référence 2024 : \textbf{568 séances}, soit \textbf{352\,160~€} (tarif moyen \textbf{620~€}). \\
Hypothèse de croissance : \textbf{+50 séances / an}. \\
Tarif moyen stabilisé : \textbf{620~€ / séance}.

\begin{table}[h!]
\centering
\renewcommand{\arraystretch}{1.20}
\rowcolors{2}{APHPsoft}{white}
\begin{tabular}{
p{4.3cm}
>{\centering\arraybackslash}p{2.2cm}
>{\centering\arraybackslash}p{2.3cm}
>{\centering\arraybackslash}p{3.0cm}
}
\toprule
\rowcolor{APHPsoft}
\textbf{Phase} & \textbf{Volume estimé} & \textbf{Tarif moyen} & \textbf{Recette brute} \\
\midrule
Amorce        & 568 & 620~€ & 352\,160~€ \\
Montée        & 618 & 620~€ & 383\,160~€ \\
Croisière     & 668 & 620~€ & 413\,160~€ \\
\bottomrule
\end{tabular}
\caption{Projections d’activité et recettes prévisionnelles — HDJ Chimiothérapie digestive}
\end{table}

\clearpage




% ============================================================
% CONCLUSION
% ============================================================

\subsection{Conclusion}

L’HDJ de chimiothérapie constitue un pilier de l’organisation oncologique moderne. Il permet une administration sécurisée et standardisée des traitements anticancéreux, réduit l’hospitalisation complète et optimise l’expérience patient. La structuration du parcours — évaluation pré-cycle, gestion des toxicités, surveillance immédiate et coordination ville–hôpital — garantit un haut niveau de qualité et d’efficience.

\clearpage

\printbibliography[heading=subbibliography,title={Références}]
\end{refsection}

% ==========================================
% CHAPITRE 8: Radiologie Interventionnelles
% ==========================================
\clearpage
\section{HDJ Radiologie Interventionnelle}
\begin{refsection}
% ============================================================
% HÔPITAL DE JOUR — RADIOLOGIE INTERVENTIONNELLE
% ============================================================

\subsection{Rationnel médical}
\setcounter{table}{0}
\needspace{8\baselineskip}

\begin{spacing}{1.30}

L’essor des techniques mini-invasives permet la réalisation d’un nombre croissant de procédures interventionnelles en ambulatoire, tout en maintenant des standards élevés de sécurité. Les recommandations de la SFR, de la SFICV et du CIRSE soulignent que l’organisation en hôpital de jour optimise l’efficience, réduit les durées d’hospitalisation et améliore l’expérience patient \cite{Lakshminarayan2023, cirse2021}.

La sélection pré-procédure repose sur la stabilité clinique, la maîtrise du risque hémorragique et l’existence de conditions sociales compatibles avec un retour sécurisé à domicile. Les actes les plus courants incluent les biopsies percutanées, drainages, thermoablations, gestes vasculaires non artériels et interventions sur cathéters.

Une surveillance immédiate (1–3~h) permet d’identifier les complications précoces, principalement douleur et saignement. Les données disponibles montrent que cette stratégie ne majore ni les réadmissions ni les recours non programmés \cite{Chen2017}. La sortie est conditionnée à la stabilité clinique, à l’absence de saignement, et à la bonne compréhension des consignes.

\end{spacing}

\clearpage

% ============================================================
% OBJECTIFS
% ============================================================

\subsection{Objectifs}
\needspace{6\baselineskip}

\begin{center}
\fcolorbox{APHPdark}{APHPsoft}{
\begin{minipage}{0.95\textwidth}
\vspace{0.9em}
\begin{itemize}[leftmargin=1.1cm]
\item sécuriser la réalisation ambulatoire des actes interventionnels mini-invasifs ;
\item structurer une filière courte, standardisée et coordonnée ;
\item réduire les séjours conventionnels et optimiser le plateau technique ;
\item harmoniser la sélection des patients, la préparation et la surveillance post-geste ;
\item améliorer l’expérience patient grâce à un parcours fluide et une information claire.
\end{itemize}
\vspace{0.9em}
\end{minipage}}
\end{center}

\clearpage

% ============================================================
% POPULATION ÉLIGIBLE
% ============================================================

\subsection{Population éligible}
\needspace{6\baselineskip}

\begin{itemize}[leftmargin=1.1cm]
\item patients ASA~I–II (ASA~III après avis spécialisé) ;
\item absence de comorbidité décompensée (cardiaque, respiratoire, rénale) ;
\item bilan d’hémostase compatible avec le geste ;
\item accompagnant obligatoire pour le retour et la première nuit ;
\item compréhension et adhésion aux consignes post-procédure.
\end{itemize}

\clearpage

% ============================================================
% PARCOURS DE SOINS
% ============================================================

\subsection{Parcours de soins}
\needspace{8\baselineskip}

\begin{figure}[!ht]
\centering
\caption{Parcours patient — HDJ Radiologie interventionnelle}
\vspace{0.8cm}

\begin{tikzpicture}[
node distance=1.6cm,
box/.style={
    rectangle, rounded corners=3pt,
    draw=APHPdark, thick,
    text width=9cm,
    minimum height=1.5cm,
    align=center,
    fill=APHPsoft
}]
\node[box] (tri) {Orientation (consultation spécialisée, demande ciblée, ville)};
\node[box, below=1.4cm of tri] (e1) {Évaluation pré-procédure : anamnèse, traitements, hémostase, consentement};
\node[box, below=1.4cm of e1] (e2) {Acte interventionnel : biopsie, drainage, thermoablation, geste vasculaire};
\node[box, below=1.4cm of e2] (e3) {Surveillance 1–3~h : constantes, douleur, point de ponction};
\node[box, below=1.4cm of e3] (syn) {Synthèse médicale, consignes écrites, sortie sécurisée};

\draw[->, thick, APHPdark] (tri) -- (e1);
\draw[->, thick, APHPdark] (e1) -- (e2);
\draw[->, thick, APHPdark] (e2) -- (e3);
\draw[->, thick, APHPdark] (e3) -- (syn);

\end{tikzpicture}
\end{figure}

\clearpage

% ============================================================
% ORGANISATION
% ============================================================

\subsection{Organisation}
\needspace{6\baselineskip}

\begin{itemize}[leftmargin=1.1cm]
\item Direction : \textbf{Pr Anthony Dohan}
\item Durée moyenne : 3~heures
\item Lieu : secteur HDJ — hépato-gastroentérologie / oncologie digestive
\item Ressources : médecin sénior, infirmier expert
\end{itemize}

\clearpage

% ============================================================
% CODAGE ET GHS ASSOCIÉS
% ============================================================

\subsection{Codage et GHS associés}
\needspace{8\baselineskip}

\noindent\textbf{Cadre général.}  
Codage basé sur le DP correspondant à la lésion explorée ou traitée, complété par les DAS pertinents.  
Actes relevant de la CCAM d’imagerie interventionnelle.

\medskip

\begin{table}[h!]
\centering
\renewcommand{\arraystretch}{1.20}
\rowcolors{2}{APHPsoft}{white}

\begin{tabularx}{\textwidth}{
>{\raggedright\arraybackslash}p{4.5cm}
X
>{\centering\arraybackslash}p{1.6cm}
>{\centering\arraybackslash}p{1.6cm}
>{\centering\arraybackslash}p{2cm}}
\toprule
\rowcolor{APHPsoft}
\textbf{Type de séance} &
\textbf{DP / DAS} &
\textbf{GHM} &
\textbf{GHS} &
\textbf{Tarif 2025} \\
\midrule

Biopsie percutanée &
DP : selon organe (Cxx / Rxx) \newline DAS : selon contexte &
07M06T &
2528 &
1\,061~€ \\

Drainage percutané &
DP : K65, K83, N13 \newline DAS : infection, douleur &
07M02T &
2518 &
891~€ \\

\bottomrule
\end{tabularx}

\caption{Codage et GHS associés — HDJ Radiologie interventionnelle}
\end{table}

\clearpage

% ============================================================
% TRACABILITÉ
% ============================================================

\subsection{Traçabilité minimale}

\begin{table}[h!]
\centering
\renewcommand{\arraystretch}{1.20}
\rowcolors{2}{APHPsoft}{white}

\begin{tabular}{p{5cm} p{9cm}}
\toprule
\rowcolor{APHPsoft}
\textbf{Intervention} & \textbf{Éléments requis} \\
\midrule

Biopsie &
Indication ; imagerie préalable ; type d’aiguille ; nombre de prélèvements ; incidents ; contrôle post-geste. \\

Drainage &
Site ; guidage ; volume, aspect ; prélèvements ; mise en tension ; contrôle immédiat. \\

Actes vasculaires &
Voie d’abord ; matériel ; perméabilité post-geste ; incidents. \\

Synthèse de sortie &
Compte-rendu structuré ; mesures de sécurité ; consignes écrites ; suivi programmé. \\
\bottomrule
\end{tabular}

\caption{Traçabilité — HDJ Radiologie interventionnelle}
\end{table}

\clearpage

% ============================================================
% VOLUMÉTRIE DE RÉFÉRENCE
% ============================================================

\subsection{Volumétrie de référence (année N)}

\noindent Montée en charge progressive : 6 à 10 patients/jour.

\begin{center}
\begin{tabular}{lccc}
\toprule
\textbf{Type de séance} & \textbf{Volume N} & \textbf{Tarif unitaire} & \textbf{Recette estimée} \\
\midrule
Biopsies & XXX & 650~€ & ZZZZ~€ \\
Drainages & XXX & 780~€ & ZZZZ~€ \\
\midrule
\textbf{Total} & \textbf{XXX} & -- & \textbf{ZZZZ~€} \\
\bottomrule
\end{tabular}
\end{center}

\clearpage

% ============================================================
% PROJECTIONS
% ============================================================

\subsection{Projections d’activité et recettes prévisionnelles}

\noindent Hypothèse : transfert progressif des gestes vers l’ambulatoire.

\begin{center}
\begin{tabular}{lccc}
\toprule
\textbf{Année} & \textbf{Volume} & \textbf{Tarif moyen} & \textbf{Recette brute} \\
\midrule
Amorce & XXX & 850~€ & ZZZZ~€ \\
Montée & XXX & 850~€ & ZZZZ~€ \\
Croisière & XXX & 850~€ & ZZZZ~€ \\
\bottomrule
\end{tabular}
\end{center}

\clearpage

% ============================================================
% VALIDATION
% ============================================================

\begin{center}
\begin{tabular}{p{4cm} p{7cm} p{4cm}}
\toprule
\rowcolor{APHPsoft}
\textbf{Date d'envoi} & \textbf{Relecteur} & \textbf{Validation} \\
\midrule
01/12/2025 & Pr V.\,Mallet & 04/12/2026 \\
04/12/2025 & Pr A.\,Dohan  & NA \\
NA          & Dr S.\,Bouam & NA \\
NA          & Pr R.\,Coriat & NA \\
\bottomrule
\end{tabular}
\end{center}

\clearpage

\printbibliography[heading=subbibliography,title={Références}]
\end{refsection}

% ============================================================
% ANNEXES — PAGE DE GARDE
% ============================================================
\clearpage
\phantomsection
\section*{Procédures et documents associés}
\addcontentsline{toc}{section}{Procédures et documents associés}
\thispagestyle{empty}

\vspace*{4cm}
\begin{center}
{\Huge\bfseries\color{APHPdark} Procédures et documents associés}\\[1.2cm]
\color{APHPblue}\rule{0.55\textwidth}{0.6pt}\\[1.2cm]
{\Large Annexes opérationnelles des HDJ}\\[0.6cm]
{\large Protocoles, grilles d’évaluation et fiches de traçabilité}
\end{center}

\clearpage

% ============================================================
% PASSAGE EN MODE ANNEXES (A, B, C…)
% ============================================================
\appendix

% --- Début bloc Annexes : reset + ancres hyperref uniques
\renewcommand{\HsecPrefix}{app}
\renewcommand{\thesection}{\Alph{section}}
\makeatletter
\renewcommand*{\theHsection}{\HsecPrefix.\Alph{section}}
\renewcommand*{\theHsubsection}{\theHsection.\arabic{subsection}}
\renewcommand*{\theHsubsubsection}{\theHsubsection.\arabic{subsubsection}}
\makeatother
\setcounter{section}{0}


% ============================================================
% ANNEXE A — BIOPSIE HÉPATIQUE
% ============================================================
\section{Procédure standardisée de biopsie hépatique percutanée en HDJ}
\label{annexe:pbh}
% ============================================================
% ANNEXE — PROCÉDURE STANDARDISÉE
% BIOPSIE HÉPATIQUE PERCUTANÉE EN HDJ
% ============================================================

% \section*{Annexe — Procédure standardisée de biopsie hépatique percutanée en HDJ}
% \addcontentsline{toc}{section}{Annexe — Procédure PBH HDJ}

\subsection*{1. Objectif de la procédure}

La biopsie hépatique percutanée (PBH) vise à obtenir un prélèvement tissulaire hépatique
de qualité diagnostique suffisante, dans des conditions optimales de sécurité, de traçabilité
et de reproductibilité, en hospitalisation de jour.

\subsection*{2. Pré-requis avant programmation}

\textbf{Critères cliniques}
\begin{itemize}[leftmargin=1.1cm]
  \item patient adulte, cliniquement stable ;
  \item indication validée par un hépatologue senior ;
  \item absence de décompensation hépatique aiguë ;
  \item absence d’infection non contrôlée ;
  \item consentement éclairé signé.
\end{itemize}

\textbf{Bilan biologique (datant de moins de 7 jours)}
\begin{itemize}[leftmargin=1.1cm]
  \item plaquettes \(\geq 50\,000/mm^3\) ;
  \item INR \(\leq 1{,}5\) ;
  \item TCA \(\leq 1{,}5 \times témoin\) ;
  \item groupe sanguin et RAI disponibles.
\end{itemize}

\textbf{Traitements}
\begin{itemize}[leftmargin=1.1cm]
  \item arrêt ou adaptation des anticoagulants et antiagrégants selon protocole validé ;
  \item absence de contre-indication anesthésique locale.
\end{itemize}

\subsection*{3. Échographie pré-procédure obligatoire}

Une échographie hépatique est réalisée systématiquement avant la PBH afin de :
\begin{itemize}[leftmargin=1.1cm]
  \item confirmer l’indication et le site de ponction ;
  \item éliminer une lésion focale hépatique du lobe droit sur le trajet prévu ;
  \item évaluer l’anatomie vasculaire et biliaire ;
  \item repérer les structures à risque (vésicule biliaire, colon, poumon) ;
  \item vérifier l’absence d’ascite significative non contrôlée.
\end{itemize}

\textbf{La biopsie est réalisée préférentiellement sur le foie droit, en l’absence de lésion focale sur le trajet de ponction.}

\subsection*{4. Réalisation de la biopsie}

\begin{itemize}[leftmargin=1.1cm]
  \item installation du patient en décubitus dorsal ou latéral gauche ;
  \item antisepsie cutanée rigoureuse et champ stérile ;
  \item anesthésie locale à la lidocaïne ;
  \item guidage échographique en temps réel ;
  \item utilisation d’un dispositif de biopsie automatique (calibre 16G ou 18G) ;
  \item obtention d’au moins une carotte de longueur \(\geq 15\,mm\) si possible ;
  \item traçabilité du nombre de passages et des prélèvements obtenus.
\end{itemize}

\subsection*{5. Hémostase post-biopsie}

\begin{itemize}[leftmargin=1.1cm]
  \item compression manuelle immédiate du point de ponction ;
  \item contrôle échographique ciblé si nécessaire ;
  \item pansement compressif maintenu ;
  \item absence de saignement actif avant sortie de salle.
\end{itemize}

\subsection*{6. Surveillance post-procédure}

Surveillance spécialisée en HDJ pendant \textbf{6 heures} :
\begin{itemize}[leftmargin=1.1cm]
  \item constantes hémodynamiques régulières (TA, FC) ;
  \item évaluation de la douleur ;
  \item surveillance du point de ponction ;
  \item recherche de signes hémorragiques ou de complications.
\end{itemize}

Un hémogramme de contrôle est réalisé en cas de symptôme ou de doute clinique.

\subsection*{7. Critères de sortie}

\begin{itemize}[leftmargin=1.1cm]
  \item stabilité hémodynamique ;
  \item douleur contrôlée ;
  \item absence de signe hémorragique ;
  \item pansement sec ;
  \item remise de consignes écrites et contact médical.
\end{itemize}

\subsection*{8. Traçabilité}

La procédure fait l’objet d’une traçabilité complète incluant :
\begin{itemize}[leftmargin=1.1cm]
  \item indication médicale ;
  \item compte-rendu de l’échographie pré-biopsie ;
  \item modalités techniques de la PBH ;
  \item nombre et qualité des prélèvements ;
  \item surveillance post-procédure ;
  \item compte-rendu anatomopathologique.
\end{itemize}


\clearpage

% ============================================================
% ANNEXE B — PONCTION D’ASCITE
% ============================================================
\section{Ponction d’ascite avec perfusion d’albumine}
\label{annexe:ponction_ascite}
\input{annexes/annexe_ponction_ascite_perfusion_albumine}

\clearpage

% ============================================================
% ANNEXE C — FER INJECTABLE
% ============================================================
\section{Séance de fer injectable}
\label{annexe:fer_injectable}
\input{annexes/annexe_fer_injectable}

\clearpage

% ============================================================
% ANNEXE D — TRANSFUSION DE CGR
% ============================================================
\section{Transfusion de concentrés de globules rouges}
\label{annexe:transfusion_cgr}
% \section*{Annexe B.3 — Transfusion de CGR chez cirrhose avancée}
% \addcontentsline{toc}{subsection}{Annexe B.3 — Transfusion de CGR}
\needspace{10\baselineskip}

\begin{itemize}[leftmargin=1.1cm]
  \item \textbf{Transfusion de CGR (FELF\%).}
  Adaptée à l’hypervolémie possible et à l’HTP. Suspension lente.
  \item \textbf{Entretien médical haut risque.}
  Risque de surcharge, réaction transfusionnelle, aggravation EH ou ascite.
  \item \textbf{Surveillance renforcée.}
  Avant / pendant / après : constantes, surcharge, allergies, tableau clinique.
  \item \textbf{Acte infirmier.}
  Pose de voie veineuse en contexte de coagulopathie, surveillance locale.
  \item \textbf{Traçabilité.}
  Fiche transfusionnelle, lots, grille surveillance pluri-horaire.
\end{itemize}



\clearpage

% ============================================================
% ANNEXE E — ÉCHOGRAPHIE MASLD
% ============================================================
\section{Grille échographique MASLD}
\label{annexe:echo_masld}
\input{annexes/annexe_grille_echo_masld}

\clearpage

% ============================================================
% ANNEXE F — DIÉTÉTIQUE MASLD
% ============================================================
\section{Grille diététique MASLD}
\label{annexe:dietetique_masld}


% ============================================================
% Annexe B — Grille diététique MASLD (EASL–AFEF 2024)
% ============================================================

\small
\renewcommand{\arraystretch}{0.92}
\setlength{\tabcolsep}{3pt}

\begin{center}
\begin{minipage}{0.92\textwidth}
\setlength{\arrayrulewidth}{0.25pt}

\rowcolors{1}{}{APHPsoft!40}
\begin{tabular}{p{5.3cm} p{7.5cm}}

\multicolumn{2}{l}{\textbf{1. Paramètres anthropométriques}} \\
\hline
Poids & \rule{1.8cm}{0.4pt} kg \\
IMC & \rule{1.8cm}{0.4pt} \\
Tour de taille & \rule{2.1cm}{0.4pt} cm \\
\\[-2mm]

\multicolumn{2}{l}{\textbf{2. Structure alimentaire}} \\
\hline
Repas structurés & \ch Oui \quad \ch Non \\
Grignotages & \ch Oui \quad \ch Non \\
Collations sucrées & \ch Oui \quad \ch Non \\
Excès graisses saturées & \ch Oui \quad \ch Non \\
Aliments ultra-transformés & \ch Quot. \quad \ch Occas. \quad \ch Jamais \\
\\[-2mm]

\multicolumn{2}{l}{\textbf{3. Qualité nutritionnelle globale}} \\
\hline
Fruits/légumes (<2/j) & \ch Oui \quad \ch Non \\
Faible apport en fibres & \ch Oui \quad \ch Non \\
Apport prot. insuffisant & \ch Oui \quad \ch Non \\
Hydratation <1.5 L/j & \ch Oui \quad \ch Non \\
\\[-2mm]

\multicolumn{2}{l}{\textbf{4. Consommations sucrées}} \\
\hline
Boissons sucrées & \ch Quot. \quad \ch Occas. \quad \ch Jamais \\
Produits sucrés & \ch Quot. \quad \ch Occas. \\
Sucres cachés & \ch Oui \quad \ch Non \\
\\[-2mm]

\multicolumn{2}{l}{\textbf{5. Alcool}} \\
\hline
Type principal & \ch Vin \quad \ch Bière \quad \ch Spirit. \\
Quantité hebdo & \rule{2.6cm}{0.4pt} g/sem \\
Seuil AFEF & \ch 0 g \quad \ch Occas. \quad \ch > seuil \\
\\[-2mm]

\multicolumn{2}{l}{\textbf{6. Activité physique}} \\
\hline
& \ch <150 min/sem \\
& \ch 150–300 min/sem (objectif) \\
& \ch >300 min/sem \\
\\[-2mm]

\multicolumn{2}{l}{\textbf{7. Facteurs comportementaux}} \\
\hline
Alimentation émotionnelle & \ch Oui \quad \ch Non \\
Compulsions sucrées & \ch Oui \quad \ch Non \\
TCA suspecté & \ch Oui \quad \ch Non \\
\\[-2mm]

\multicolumn{2}{l}{\textbf{8. Régime méditerranéen (score simplifié)}} \\
\hline
Huile d’olive quotidienne & \ch Oui \quad \ch Non \\
≥2 portions/j fruits + légumes & \ch Oui \quad \ch Non \\
Poisson ≥2/sem & \ch Oui \quad \ch Non \\
\\[-2mm]

\multicolumn{2}{l}{\textbf{9. Objectifs diététiques (commentaires)}} \\
\hline
\multicolumn{2}{l}{
\rule{0.95\textwidth}{1.8cm}
} \\

\end{tabular}

\vspace{1mm}
{\footnotesize\itshape
Réf. : EASL 2024 ; AASLD 2023 ; AFEF 2024 ; EASO Obesity 2023.
}

\end{minipage}
\end{center}

\normalsize


\clearpage

% ============================================================
% ANNEXE G — PSY / ADDICTOLOGIE MASLD
% ============================================================
\section{Grille psychologique et addictologique MASLD}
\label{annexe:psy_masld}
% ============================================================
% Annexe C — Grille psychologique / addictologique MASLD
% ============================================================

\small                                      % police plus compacte
\renewcommand{\arraystretch}{0.92}          % lignes resserrées
\setlength{\tabcolsep}{3pt}                 % colonnes serrées

\begin{center}
\begin{minipage}{0.92\textwidth}

\rowcolors{1}{}{APHPsoft!40}
\setlength{\arrayrulewidth}{0.25pt}

\begin{tabular}{p{5.7cm} p{7.8cm}}

\multicolumn{2}{l}{\textbf{1. Alcool (AUDIT-C)}} \\
\hline
Score & \rule{1.3cm}{0.4pt} / 12 \\
& \ch Faible \quad \ch Modéré \quad \ch Élevé \\
\\[-2mm]

\multicolumn{2}{l}{\textbf{2. Troubles alimentaires / comportements}} \\
\hline
& \ch Hyperphagie \\
& \ch Grignotage émotionnel \\
& \ch Compulsions sucrées \\
& \ch Restriction cognitive \\
& \ch TCA suspecté → orientation \\
\\[-2mm]

\multicolumn{2}{l}{\textbf{3. Sommeil / stress / anxiété}} \\
\hline
& \ch Troubles du sommeil \\
& \ch Stress chronique \\
& \ch Anxiété marquée / coping limité \\
& \ch Symptômes dépressifs \\
\\[-2mm]

\multicolumn{2}{l}{\textbf{4. Motivation au changement (1–10)}} \\
\hline
Score & \rule{1.8cm}{0.4pt} / 10 \\
& \ch Ambivalence \quad \ch Motivation claire \\
\\[-2mm]

\multicolumn{2}{l}{\textbf{5. Facteurs psychosociaux influençant MASLD}} \\
\hline
& \ch Isolement social \\
& \ch Précarité / charge mentale \\
& \ch Perte de repères alimentaires \\
& \ch Trajectoires addictives antérieures \\
\\[-2mm]

\multicolumn{2}{l}{\textbf{6. Objectifs psychocomportementaux}} \\
\hline
\multicolumn{2}{l}{
\rule{0.94\textwidth}{1.8cm}
} \\

\end{tabular}

\vspace{1mm}

{\footnotesize\itshape
Réf. : EASL MASLD 2024 ; AASLD 2023 ; AFEF 2024 ; NICE AUDIT-C ; EASO 2023.
}

\end{minipage}
\end{center}

\normalsize


\clearpage

% ============================================================
% ANNEXE H — ACTIVITÉS HDJ ADDICTOLOGIE
% ============================================================
\section{Liste des activités réalisables en HDJ addictologie}
\label{annexe:addicto_activites}
% ============================================================
% ANNEXE 1 — Activités réalisables en HDJ addictologie
% Source principale : Instruction DGOS/R4/R1/2016/350
% Version enrichie pour HDJA : art-thérapie, socio-esthétique,
% APA, revue de presse, écriture thérapeutique
% ============================================================

% \section*{Annexe 1 — Liste des activités réalisables en HDJ addictologie}

\begin{table}[h!]
\centering
\renewcommand{\arraystretch}{1.25}
\rowcolors{2}{APHPsoft}{white}

\begin{tabular}{p{1.2cm} p{13.5cm}}
\toprule
\rowcolor{APHPsoft}
\textbf{N°} & \textbf{Activité} \\
\midrule

1 & Entretien motivationnel individuel \\
2 & Entretien motivationnel collectif \\
3 & Activité individuelle de médiation thérapeutique (artistique, corporelle, expressive) : art-thérapie, écriture thérapeutique, revue de presse \\
4 & Activité collective de médiation thérapeutique : art-thérapie, écriture collective, groupes de parole thématiques \\
5 & Activité individuelle de réadaptation / maintien des fonctions psychosociales \\
6 & Activité collective de réadaptation / maintien des fonctions psychosociales \\
7 & Groupe de parole (addictologie, retour d’expérience, gestion des émotions) \\
8 & Consultation nutritionnelle individuelle \\
9 & Conseils nutritionnels individuels / prescription diététique \\
10 & Atelier cuisine thérapeutique ou atelier alimentaire collectif \\
11 & Thérapie cognitive et comportementale (individuelle) \\
12 & Thérapie cognitive et comportementale (collective) \\
13 & Prise en charge psychomotrice individuelle \\
14 & Prise en charge psychomotrice collective \\
15 & Évaluation individuelle cognitive / fonctions exécutives \\
16 & Prise en charge cognitive individuelle thérapeutique \\
17 & Prise en charge cognitive collective thérapeutique \\
18 & Activité collective de prise en charge des altérations psychomotrices \\
19 & Éducation thérapeutique du patient (individuelle) \\
20 & Éducation thérapeutique du patient (collective) \\
21 & Information individuelle (alcool, santé, prévention, traitements) \\
22 & Information collective (prévention, risques, santé publique) \\
23 & Activités d’éducation ou d’information thérapeutique \\
24 & Évaluation socio-thérapeutique individuelle \\
25 & Activité individuelle relative aux activités de vie quotidienne \\
26 & Activité collective relative aux activités de vie quotidienne \\
27 & Assistance éducative individuelle \\
28 & Assistance éducative collective \\
29 & Développement ou restauration des compétences sociales (individuel) \\
30 & Développement ou restauration des compétences sociales (collectif) \\
31 & Activité d’aide à l’emploi / insertion \\
32 & Entretien individuel de relation d’aide sociale \\
33 & Entretien collectif de relation d’aide sociale \\
34 & Préparation au retour à domicile / maintien du lien social \\
35 & Activité systémique (thérapie familiale, de couple) \\
36 & Activité collective avec l’entourage / proches \\
37 & Activité systémique structurée \\
38 & Consultation médicale spécialisée (addictologie, hépatologie, psychiatrie) \\
39 & Consultation médicale de synthèse (au moins hebdomadaire) \\
40 & Entretien infirmier évaluatif ou thérapeutique \\
41 & Entretien psychologique individuel \\
42 & Entretien individuel avec éducateur spécialisé \\
43 & Entretien individuel de rééducation (kiné / ergo / psychomotricien) \\
44 & Activité pluriprofessionnelle d’évaluation \\
45 & Activité pluriprofessionnelle de synthèse \\
46 & Participation à un staff pluridisciplinaire \\
47 & Activité d’évaluation ou de synthèse interprofessionnelle (fin de cure) \\

\bottomrule
\end{tabular}

\caption{Liste consolidée des activités réalisables en HDJ addictologie (DGOS 2016 + médiations utilisées au HDJA)}
\end{table}


\clearpage

% ============================================================
% ANNEXE I — TRAÇABILITÉ HDJ ADDICTOLOGIE
% ============================================================
\section{Fiche de traçabilité HDJ addictologie}
\label{annexe:addicto_tracabilite}
\input{annexes/annexe_addicto_traçabilite}



\end{document}

