\documentclass[12pt,a4paper]{article}

% =======================
% PACKAGES DE BASE
% =======================
\usepackage{url}
\usepackage{xurl} % permet la coupure automatique des longues URL
\usepackage[french]{babel}
\usepackage{setspace}
\usepackage{geometry}
\usepackage{graphicx}
\usepackage{xcolor}
\usepackage{hyperref}
\usepackage{titlesec}
\usepackage{enumitem}
\usepackage{tikz}
\usetikzlibrary{positioning}
\usepackage{tocloft}
\usepackage{csquotes}
\usepackage{booktabs}
\usepackage{fontspec}
\usepackage{needspace}
\usepackage[table]{xcolor}
\usepackage{amssymb} % pour \square
\usepackage{tabularx}
\usepackage{pdflscape}
\usepackage{rotating}




% =======================
% FONT (XeLaTeX)
% =======================
\setmainfont{TeX Gyre Termes}

% =======================
% COULEURS APHP
% =======================
\definecolor{APHPblue}{RGB}{0,56,147}
\definecolor{APHPdark}{RGB}{0,36,95}
\definecolor{APHPgrey}{RGB}{90,90,90}
\definecolor{APHPsoft}{RGB}{225,235,250}

% =======================
% CASES À COCHER
% =======================
\newcommand{\ch}{\(\square\)\hspace{0.25em}}

% =======================
% LISIBILITÉ / SPACINGS
% =======================
\setlength{\parskip}{6pt}
\setlength{\parindent}{0pt}
\titlespacing*{\section}{0pt}{18pt}{12pt}
\titlespacing*{\subsection}{0pt}{14pt}{10pt}
\setlist[itemize]{itemsep=6pt, topsep=6pt}

% =======================
% TITRES
% =======================
\titleformat{\section}{\color{APHPdark}\Large\bfseries}{\thesection}{1em}{}
\titleformat{\subsection}{\color{APHPblue}\normalsize\bfseries}{\thesubsection}{1em}{}

% TABLE DES MATIÈRES
\renewcommand{\cftsecleader}{\cftdotfill{\cftdotsep}}

% =======================
% BIBLIOGRAPHIE
% =======================
\usepackage[
    backend=biber,
    style=vancouver,
    sorting=none,
    useprefix=false
]{biblatex}

% Coupure agressive des URL longues dans la bibliographie
\setcounter{biburlnumpenalty}{1}
\setcounter{biburlucpenalty}{1}
\setcounter{biburllcpenalty}{1}

% Style des URL plus compact
\renewcommand*{\UrlFont}{\small\ttfamily}

% Forcer les URL à se mettre sur une nouvelle ligne si nécessaire
\DeclareFieldFormat{url}{\newline\url{#1}}

% Empêche les débordements dans les références longues
\AtBeginBibliography{\sloppy}

\DeclareNameAlias{default}{family-given}
\DeclareNameAlias{sortname}{family-given}
\DeclareFieldFormat{author}{#1}
\DeclareFieldFormat{labelname}{#1}
\DeclareFieldFormat{journaltitle}{%
  \iffieldundef{shortjournal}{#1}{\printfield{shortjournal}}%
}

\AtBeginBibliography{%
  \renewcommand{\mkbibnamefamily}[1]{#1}%
  \renewcommand{\mkbibnamegiven}[1]{#1}%
  \renewcommand{\mkbibnameprefix}[1]{#1}%
  \renewcommand{\mkbibnamesuffix}[1]{#1}%
}

\addbibresource{references.bib}

% =======================
% GEOMETRIE
% =======================
\geometry{margin=2.4cm}
\setstretch{1.25}

% =======================
% DOCUMENT
% =======================
\begin{document}

% ============================
% PAGE DE GARDE
% ============================
\begin{titlepage}
\centering

\vspace*{0.5cm}

\begin{minipage}{0.49\textwidth}
    \raggedright
    \includegraphics[height=2.8cm]{logo/logo_cochin_portroyal.png}
\end{minipage}
\begin{minipage}{0.49\textwidth}
    \raggedleft
    \includegraphics[height=2.6cm]{logo/LogoUPC.jpg}
\end{minipage}

\vspace{2.5cm}

{\fontsize{26pt}{30pt}\selectfont\bfseries\color{APHPdark}
Programme des Hôpitaux de Jour\\[2mm]
Digestifs et Addictologiques}

\vspace{5mm}

\color{APHPblue}\rule{0.45\textwidth}{0.6pt}

\vspace{4mm}

{\fontsize{14pt}{18pt}\selectfont\bfseries\color{APHPgrey}
Document de Travail — Version 1}

\vfill

{\Large\bfseries AP-HP — Hôpital Cochin \\[2mm]}
{\large\bfseries DMU DIGEST}\\[6mm]
{\large Responsable : Professeur Vincent Mallet}

\end{titlepage}

% ============================
% TABLE DES MATIÈRES
% ============================
\tableofcontents
\newpage


% ==========================================
% CHAPITRE X: Conclusion Générale
% ==========================================
\clearpage
\section{Executive Summary}

\begin{refsection}
% ============================================================
% 1. E-SUMMARY — SYNTHÈSE EXÉCUTIVE
% ============================================================

\titleformat{\subsection}[runin]
  {\bfseries\color{APHPblue}}
  {}
  {0pt}
  {}

\begin{spacing}{1.20}

% ------------------------------------------------------------
% 1.1 Contexte et opportunité institutionnelle
% ------------------------------------------------------------

\subsection*{Contexte et opportunité institutionnelle}\ignorespaces
La structuration d’un plateau unique d’Hôpitaux de Jour (HDJ) regroupant les activités digestives et interventionnelles s’inscrit pleinement dans les objectifs nationaux du virage ambulatoire. Elle repose sur une opportunité organisationnelle majeure — la libération de locaux dédiés — permettant de regrouper sur un site unique des activités ambulatoires aujourd’hui dispersées, tout en améliorant leur lisibilité et leur efficience.

Ce regroupement vise à constituer un dispositif évolutif, aligné sur les recommandations nationales et internationales de référence (EASL, AFEF, SNFGE, INCa, SFR/SFICV), et adapté à la prise en charge de patients complexes nécessitant des parcours de soins programmés, sécurisés et coordonnés.

% ------------------------------------------------------------
% 1.2 Objectifs stratégiques du projet
% ------------------------------------------------------------

\subsection*{Objectifs stratégiques du projet}

Le projet poursuit cinq objectifs stratégiques structurants :
\begin{itemize}[leftmargin=1.1cm]
    \item améliorer l’accès rapide, sécurisé et lisible aux explorations et aux traitements spécialisés ;
    \item réduire les hospitalisations conventionnelles évitables par un recours structuré et coordonné à l’ambulatoire ;
    \item optimiser l’utilisation des ressources humaines expertes par une organisation mutualisée, transversale et efficiente ;
    \item renforcer l’attractivité institutionnelle et académique du site dans un contexte durable de tension sur les compétences spécialisées ;
    \item développer un parcours patient hospitalo-universitaire favorisant l’inclusion des patients dans des essais thérapeutiques.
\end{itemize}


% ------------------------------------------------------------
% 1.3 Population cible globale
% ------------------------------------------------------------

\subsection*{Population cible globale}
Le plateau HDJ s’adresse à une population adulte et présentant :
\begin{itemize}[leftmargin=1.1cm]
    \item des pathologies digestives et hépato-biliaires chroniques (MASLD/MASH, cirrhoses compensées et décompensées) ;
    \item des troubles addictologiques avec retentissement somatique ;
    \item des maladies inflammatoires chroniques de l’intestin (MICI) ;
    \item des cancers digestifs nécessitant des traitements systémiques ambulatoires ;
    \item des patients pris en charge en radiologie interventionnelle pour surveillance post examen;
    \item des patients en cours d'évaluation thérapeutique ou du suivi de traitements oraux.
\end{itemize}

Ces patients présentent un besoin élevé d’évaluations programmables, répétées et sécurisées, compatible avec une prise en charge ambulatoire structurée.

% ------------------------------------------------------------
% 1.4 Volumétrie et trajectoire d’activité
% ------------------------------------------------------------

\subsection*{Volumétrie et trajectoire d’activité}

Le plateau d’Hôpital de Jour est dimensionné pour une \textbf{capacité annuelle maximale théorique estimée entre 4\,700 et 5\,000 séances}, correspondant à un fonctionnement à pleine charge, sans marge organisationnelle.

L’objectif opérationnel à maturité médico-économique repose toutefois sur une \textbf{trajectoire d’activité cible comprise entre 2\,000 et 2\,500 séances annuelles}, compatible avec une montée en charge progressive, une sélectivité médicale raisonnée et un taux d’occupation volontairement inférieur au plafond théorique. Cette trajectoire est anticipée sur une période de \textbf{24 à 36 mois}, sur la base des données historiques locales, des besoins territoriaux identifiés et d’hypothèses prudentes de croissance.


% ------------------------------------------------------------
% 1.5 Recettes consolidées
% ------------------------------------------------------------

\subsection*{Recettes consolidées}
Sur la base des tarifs moyens pondérés par filière et des volumes projetés, les recettes consolidées du plateau HDJ sont estimées à \textbf{environ 3~M€ par an à maturité}, avec une fourchette comprise entre \textbf{2,5 et 3,4~M€}.

Le dispositif présente ainsi un profil \textbf{autosoutenable}, générateur de valeur médico-économique pour l’institution.

% ------------------------------------------------------------
% 1.6 Bénéfices institutionnels
% ------------------------------------------------------------

\subsection*{Bénéfices institutionnels}
La structuration du plateau HDJ permet :
\begin{itemize}[leftmargin=1.1cm]
    \item une redistribution ciblée des lits MCO mobilisés pour des prises en charge programmables ;
    \item une fluidification des parcours entre la ville, les urgences et l’hospitalisation complète ;
    \item une mutualisation efficiente des compétences spécialisées (IDE expertes, IPA, psychologie, diététique) ;
    \item une amélioration de l’attractivité des postes paramédicaux et médicaux spécialisés ;
    \item une lisibilité renforcée de l’offre ambulatoire digestive et interventionnelle à l’échelle du territoire.
\end{itemize}

% ------------------------------------------------------------
% 1.7 Gouvernance (niveau exécutif)
% ------------------------------------------------------------

\subsection*{Gouvernance}
Le plateau des HDJ mutualisés repose sur une gouvernance claire et intégrée, associant :
\begin{itemize}[leftmargin=1.1cm]
    \item une responsabilité médicale unique, garante de la cohérence des parcours et de la sécurité des prises en charge ;
    \item une articulation structurée entre le DMU digestif, les plateaux médico-techniques et la radiologie interventionnelle ;
    \item un pilotage médico-économique consolidé associant responsables médicaux, encadrement soignant, DIM et direction financière.
\end{itemize}

Les modalités organisationnelles détaillées, le schéma fonctionnel de référence et les principes de dimensionnement sont développés dans les sections suivantes.

\end{spacing}

\clearpage

\printbibliography[heading=subbibliography,title={Références}]
\end{refsection}


% ===============================================
% CHAPITRE 0: Introduction
% ===============================================
\clearpage
\section*{Introduction Générale}

\begin{refsection}
% ===============================================
% CHAPITRE 0: Introduction générale 
% ===============================================

\begin{spacing}{1.30}

\subsection*{L’AP-HP et le virage ambulatoire}

L’Assistance Publique–Hôpitaux de Paris (AP-HP), premier centre hospitalo-universitaire d’Europe, regroupe 39 hôpitaux organisés en GHU et assure soin, enseignement et recherche pour près de 12 millions d’habitants. Le plan stratégique médical 2023–2028 place le virage ambulatoire au cœur de la transformation, en visant une amélioration de l’accessibilité, des délais et de la fluidité des parcours \cite{APHP2023Plan}.  
L’hôpital de jour (HDJ) constitue un outil majeur de cette stratégie, permettant la réalisation coordonnée, en temps court, d’évaluations diagnostiques et thérapeutiques complexes dans un cadre sécurisé et universitaire.

\subsection*{Le GHU AP–HP.Centre et le site Cochin}

L’Hôpital Cochin–Port-Royal (GHU AP–HP.Centre–Université Paris Cité) constitue un pôle hospitalo-universitaire de référence. Il couvre un large champ d’expertises : hépato-gastroentérologie et oncologie digestive, chirurgie digestive et hépatobiliaire, endocrinologie et maladies métaboliques, rhumatologie, obstétrique, néonatologie, réanimation et soins intensifs spécialisés.  

Il accueille deux unités cliniques d’HGE — le service des maladies du foie et le service d’hépato-gastroentérologie et d’oncologie digestive — constituant des plateformes de recours régionales et nationales \cite{CollegialeHGE2023}. L’ensemble du plateau technique (endoscopie interventionnelle, radiologie interventionnelle, imagerie spécialisée, traitement loco-régional et systémique des cancers digestifs, préparation à la greffe de foie) permet une prise en charge intégrée des pathologies digestives complexes.

\subsection*{Poids épidémiologique des maladies digestives}

Les maladies digestives représentent la première cause d’hospitalisation en médecine–chirurgie–obstétrique en France \cite{DREES2021}. Les évolutions récentes incluent :

\begin{itemize}
\item une augmentation marquée des cancers pancréatiques et hépatiques, désormais parmi les principales causes de mortalité oncologique \cite{UEG2022DigestiveHealth} ;
\item une stabilité de l’incidence du cancer colorectal, mais un besoin croissant en endoscopies ;
\item une progression soutenue des maladies du foie : prévalence de la cirrhose autour de 0,3~\%, 150--200 nouveaux cas/million/an, et près de 15 000 décès annuels \cite{FrenchHepaticFailure_2020} ;
\item une forte croissance des maladies métaboliques du foie (MASLD/MASH), portée par l’épidémie d’obésité et de diabète, conformément aux dernières recommandations européennes \cite{EASL2024MASLD,RN597} ;
\item une hausse continue de la prévalence des MICI, avec environ 250~000 patients en France et plus de 20~000 suivis à l’AP-HP \cite{Ng2017,SNDS_MICI2022}. 
\end{itemize}

Les troubles liés à l’alcool restent un déterminant majeur de morbi-mortalité digestive, première cause de cirrhose et de carcinome hépatocellulaire, avec un poids important sur les hospitalisations \cite{FrenchHepaticFailure_2020}.

\subsection*{Contraintes structurelles et nécessité du développement ambulatoire}

Les services d’HGE font face à une augmentation de la complexité des patients (vieillissement, précarité) alors que le capacitaire se réduit (fermetures de lits, tensions en personnel). De nombreuses indications relevant autrefois de l’hospitalisation conventionnelle peuvent désormais être conduites en HDJ : initiation de biothérapies MICI, immunothérapies, bilans spécialisés, évaluations multidisciplinaires, actes interventionnels simples.  
Le projet médical HGE 2023–2028 désigne l’HDJ comme un levier prioritaire pour réduire les DMS, absorber la croissance épidémiologique et renforcer l’attractivité régionale \cite{CollegialeHGE2023}.

\subsection*{Rôle structurant des Hôpitaux de Jour HGE}

Les HDJ thématiques du pôle répondent à plusieurs missions stratégiques :

\begin{itemize}
\item standardisation des filières diagnostiques et thérapeutiques ;
\item concentration des évaluations spécialisées sur une même journée ;
\item réduction des délais d’accès aux biothérapies, à l’imagerie ou à l’endoscopie ;
\item intégration de compétences pluridisciplinaires (diététique, psychologie, addictologie, IPA, ETP) ;
\item optimisation de la coordination ville–hôpital et de la traçabilité universitaire.
\end{itemize}

Ils constituent aujourd’hui une architecture cohérente, alignée sur les enjeux épidémiologiques et capacitaires du CHU.

\subsection*{Logique du présent document}

Ce document rassemble les fiches techniques standardisées des HDJ du pôle d’HGE et prépare la construction d’un HDJ mutualisé à horizon 2027.  
Il décrit l’activité actuelle (hépatologie, MICI, addictologie, hépatométabolique, radiologie interventionnelle, oncologie digestive) et fournit la base opérationnelle du futur HDJ commun.  
La libération des locaux occupés aujourd’hui par l’hématologie rendra possible la création d’un espace unique regroupant des parcours spécialisés, standardisés et pluridisciplinaires, améliorant l’accessibilité, la qualité et la fluidité des prises en charge ambulatoires au sein du GHU AP–HP.Centre.

\end{spacing}

\printbibliography[heading=subbibliography,title={Références}]
\end{refsection}


% ===============================================
% CHAPITRE 1: PBH
% ===============================================
\clearpage
\section{HDJ PBH}

\begin{refsection}
% ============================================================
% HÔPITAL DE JOUR — BIOPSIE HÉPATIQUE (PBH)
% ============================================================

\subsection{Rationnel médical}
\needspace{8\baselineskip}

\begin{spacing}{1.30}

La biopsie hépatique (PBH) demeure l’examen de référence pour le diagnostic, la classification et le suivi histologique de nombreuses maladies du foie. Malgré les progrès des outils non invasifs, elle reste indispensable dans plusieurs situations cliniques complexes où seule l’analyse tissulaire permet une caractérisation fine et reproductible des lésions.

Elle est essentielle dans les maladies hépatiques rares (hépatites auto-immunes, cholangites, granulomatoses), en particulier lorsque les présentations cliniques ou immunologiques sont atypiques, discordantes ou insuffisamment discriminantes. Dans les toxicités médicamenteuses, notamment celles induites par les immunothérapies (anti–PD-1/PD-L1, anti–CTLA-4) ou certaines thérapies ciblées, la PBH est un examen clé permettant de distinguer des lésions auto-immunes, granulomateuses, cholestatiques ou toxiques. Elle occupe une place centrale dans les algorithmes diagnostiques et décisionnels  \cite{Peeraphatdit_ImmuneHepatitis_2020, DeMartin_IOHepatitis_2020}.

La PBH est également incontournable pour évaluer la réponse aux traitements dans les maladies chroniques du foie. La régression de la fibrose et de la cirrhose constitue un critère pronostique majeur et un objectif essentiel des essais thérapeutiques contemporains. Aucune méthode non invasive ne permet aujourd’hui d’attester la réorganisation architecturale du foie : seule la PBH peut documenter de manière standardisée une réversion histologique, ce qui en fait l’examen de référence dans les études sur la MASLD/NASH et dans les essais visant à démontrer une amélioration de la fibrose \cite{Friedman_FibrosisRegression_2023}.

La PBH conserve une utilité en oncologie hépatique pour caractériser certaines lésions bénignes ou malignes lorsque la confirmation histologique conditionne la prise en charge. En maladies infectieuses, elle peut être indispensable pour rechercher des agents pathogènes difficiles à isoler. En hématologie (SMD, aplasie, LAM), la présence d’une fibrose constitue un facteur pronostique majeur avant allogreffe.

L’Hôpital de Jour fournit un cadre sécurisé : protocole décisionnel, guidage échographique (HLHJ003/HLHJ006), surveillance spécialisée, traçabilité, continuité ville–hôpital.

\end{spacing}

\clearpage

% ============================================================
% OBJECTIFS
% ============================================================

\subsection{Objectifs}
\needspace{6\baselineskip}

\vspace{0.8em}
\begin{center}
\fcolorbox{APHPdark}{APHPsoft}{
\begin{minipage}{0.95\textwidth}
\vspace{0.9em}

\begin{itemize}[leftmargin=1.1cm]
  \item obtenir un diagnostic histologique précis dans les maladies du foie complexes ou atypiques ;
  \item documenter les toxicités médicamenteuses, notamment sous immunothérapie ;
  \item évaluer la fibrose et la réversion structurale dans les protocoles thérapeutiques ;
  \item caractériser des lésions tumorales bénignes ou malignes lorsque cela conditionne la stratégie ;
  \item explorer des atteintes infectieuses hépatiques rares (hépatocultures) ;
  \item réaliser l’examen dans un cadre sécurisé, standardisé et traçable en HDJ.
\end{itemize}

\vspace{0.9em}
\end{minipage}}
\end{center}

\clearpage

% ============================================================
% POPULATION ÉLIGIBLE
% ============================================================

\subsection{Population éligible}
\needspace{6\baselineskip}

\begin{itemize}[leftmargin=1.1cm]
  \item maladie chronique du foie ;
  \item suspicion de maladie auto-immune ou cholestatique avec bilan discordant ;
  \item toxicité médicamenteuse sévère ou atypique, notamment sous immunothérapie ;
  \item protocoles de recherche nécessitant une évaluation histologique ;
  \item exploration des lésions tumorales primitives ou secondaires ;
  \item suspicion d’hépatite infectieuse atypique ou d’infection profonde ;
  \item évaluation pré-allogreffe en hématologie (SMD, aplasie, LAM).
\end{itemize}

\clearpage

% ============================================================
% PARCOURS DE SOINS
% ============================================================

\subsection{Parcours de soins}
\needspace{8\baselineskip}

\begin{figure}[!ht]
\centering
\caption{Parcours patient — PBH en Hôpital de Jour}
\vspace{0.8cm}

\begin{tikzpicture}[
    node distance=1.6cm,
    box/.style={
        rectangle,
        rounded corners=3pt,
        draw=APHPdark,
        thick,
        text width=8.6cm,
        minimum height=1.5cm,
        align=center,
        fill=APHPsoft
    }
]
\node[box] (tri) {Orientation vers l’HDJ (hépatologie / MCO / ville)};
\node[box, below=1.4cm of tri] (e1) {Évaluation initiale : coagulation, imagerie, consentement};
\node[box, below=1.4cm of e1] (e2) {Biopsie sous guidage échographique (HLHJ003/HLHJ006)};
\node[box, below=1.4cm of e2] (e3) {Surveillance spécialisée 6 heures : constantes, douleur, saignement};
\node[box, below=1.4cm of e3] (syn) {Compte-rendu, consignes et coordination ville–hôpital};

\draw[->, thick, APHPdark] (tri) -- (e1);
\draw[->, thick, APHPdark] (e1) -- (e2);
\draw[->, thick, APHPdark] (e2) -- (e3);
\draw[->, thick, APHPdark] (e3) -- (syn);

\end{tikzpicture}
\end{figure}

\clearpage

% ============================================================
% PANORAMA CODAGE + VOLUMÉTRIE 2024
% ============================================================

\begin{sidewaystable}[h!]
\centering
\renewcommand{\arraystretch}{1.25}
\rowcolors{2}{APHPsoft}{white}

\begin{tabular}{
p{5.2cm}
p{2.6cm}
p{1.8cm}
p{1.8cm}
>{\centering\arraybackslash}p{2.0cm}
>{\centering\arraybackslash}p{2.2cm}
>{\centering\arraybackslash}p{3.0cm}
}
\toprule
\rowcolor{APHPsoft}
\textbf{Type de séance} &
\textbf{DP / DR / DAS} &
\textbf{GHM} &
\textbf{GHS} &
\textbf{Tarif 2025} &
\textbf{Volume 2024} &
\textbf{Recette 2024} \\
\midrule

Fibrose / MASLD &
K74.0 / K76.0 \newline DAS comorbidités majeures &
07M08T & 2538 & 1\,238~€ & 41 & 50\,758~€ \\

CHC / tumeur maligne &
C22.0 \newline DAS comorbidités &
07M06T & 2528 & 1\,061~€ & 23 & 24\,403~€ \\

Cirrhose alcoolique &
K70.2 &
07M07T & 2533 & 840~€ & 5 & 4\,200~€ \\

HAI / tumeur bénigne &
K75.4 / D13.4 &
07M04T & 2523 & 919~€ & 35 & 32\,165~€ \\

Normale ou histologie\\non disponible &
R93.2 &
07M14T & 2559 & 603~€ & 144 & 86\,832~€ \\

\midrule
\textbf{TOTAL} & -- & -- & -- & -- & \textbf{248} & \textbf{198\,358~€} \\
\bottomrule
\end{tabular}

\caption{Panorama PBH HDJ — Codage, volumétrie et recettes (2024)}
\end{sidewaystable}

\clearpage

% ============================================================
% TRACABILITÉ
% ============================================================

\subsection{Traçabilité minimale}

\begin{table}[h!]
\centering
\renewcommand{\arraystretch}{1.25}
\rowcolors{2}{APHPsoft}{white}

\begin{tabular}{p{5cm} p{9cm}}
\toprule
\rowcolor{APHPsoft}
\textbf{Intervention} & \textbf{Traçabilité requise} \\
\midrule
Biopsie hépatique &
Fiche d’acte, repérage échographique, nombre de carottes, longueur, calibre \\
Surveillance spécialisée &
Feuille pluri-horaire (6h), douleur, tension, saignement local \\
Entretien médical &
Indication, consentement, comorbidités, risque hémorragique \\
Examens complémentaires &
Imageries et bilans pré-biopsie \\
Coordination &
Compte-rendu structuré, consignes, liaison ville–hôpital \\
\bottomrule
\end{tabular}

\caption{Traçabilité — PBH en HDJ}
\end{table}

\clearpage

% ============================================================
% PROJECTIONS D’ACTIVITÉ
% ============================================================

\subsection{Projections d’activité et recettes prévisionnelles}

\noindent Référence 2024 : \textbf{248 PBH}, soit \textbf{198\,400~€} (tarif moyen 800~€). \\
Hypothèse de croissance : \textbf{+25 actes / an}. \\
Tarif moyen stable : \textbf{800~€ / séance}.

\begin{table}[h!]
\centering
\renewcommand{\arraystretch}{1.20}
\rowcolors{2}{APHPsoft}{white}
\begin{tabular}{
p{4.3cm}
>{\centering\arraybackslash}p{2.2cm}
>{\centering\arraybackslash}p{2.3cm}
>{\centering\arraybackslash}p{3.0cm}
}
\toprule
\rowcolor{APHPsoft}
\textbf{Phase} & \textbf{Volume estimé} & \textbf{Tarif moyen} & \textbf{Recette brute} \\
\midrule
Amorce        & 273 & 800~€ & 218\,400~€ \\
Montée        & 298 & 800~€ & 238\,400~€ \\
Croisière     & 323 & 800~€ & 258\,400~€ \\
\bottomrule
\end{tabular}
\caption{Projections d’activité et recettes prévisionnelles — PBH HDJ (à partir de 2024)}
\end{table}


% ============================================================
% CONCLUSION
% ============================================================

\subsection{Conclusion}

La biopsie hépatique en Hôpital de Jour s’intègre dans un parcours sécurisé et standardisé, essentiel pour la prise en charge des maladies hépatiques rares, des toxicités médicamenteuses complexes et des programmes thérapeutiques nécessitant une évaluation histologique. Elle constitue un outil majeur pour la décision clinique et la stratification pronostique.

\clearpage

% ============================================================
% VALIDATION
% ============================================================

\begin{center}
\begin{tabular}{p{4cm} p{7cm} p{4cm}}
\toprule
\rowcolor{APHPsoft}
\textbf{Date d'envoi} & \textbf{Relecteur} & \textbf{Validation} \\
\midrule
03/12/2025 & Pr V.\,Mallet & 07/12/2025 \\
03/12/2025 & Dr S.\,Bouam & 07/12/2025 \\
NA & Dr V.\,D'Halluin & NA \\
NA & Pr R.\,Coriat & NA \\


\bottomrule
\end{tabular}
\end{center}

\clearpage

\printbibliography[heading=subbibliography,title={Références}]
\end{refsection}

% ======================================
% CHAPITRE 2: Evaluation des Cirrhoses
% ======================================
\clearpage
\section{HDJ Evaluation des Cirrhoses}

\begin{refsection}
% ============================================================
% HÔPITAL DE JOUR — ÉVALUATION DES CIRRHOSES ET HTP CLINIQUE
% ============================================================

\subsection{Rationnel médical}
\needspace{8\baselineskip}

\begin{spacing}{1.28}

L’incidence des cirrhoses continue de progresser en Europe, portée par l’augmentation des hépatopathies métaboliques (MASLD), du diabète de type 2, de l’obésité et de la consommation d’alcool. La prévalence des formes compensées croît en moyenne de 3 à 5\,\% par an \cite{EASL_DecompCirrhosis_2018}. En France, les données PMSI et SNDS confirment également une hausse régulière des hospitalisations liées aux maladies hépatiques chroniques \cite{RN597}.

L’évolution de la fibrose vers la cirrhose s’accompagne d’une élévation du gradient porto-systémique (HVPG). Une valeur $>$\,10\,mmHg définit l’hypertension portale cliniquement significative (CSPH), étape pivot à partir de laquelle survient la première décompensation \cite{EASL_DecompCirrhosis_2018}. Les recommandations de Baveno\,VII permettent d’identifier la CSPH par des critères non invasifs :
\begin{itemize}
  \item élasticité hépatique (LSM) $\geq 25$\,kPa, ou
  \item LSM de 20--25\,kPa avec plaquettes $<$\,150\,G/L \cite{RN597}.
\end{itemize}

Chez ces patients, l’introduction précoce d’un bêtabloquant non sélectif (carvédilol) réduit l’HVPG, prévient la première décompensation et améliore la survie, y compris en l’absence de varices œsophagiennes.

Plusieurs facteurs aggravants doivent être systématiquement évalués :
\begin{itemize}
  \item consommation d’alcool ; l’abstinence améliore la survie \cite{Loomba_AlcoholAbstinence_2020,Addolorato_AlcoholCirrhosis_2016} ;
  \item sarcopénie (30--70\,\%) \cite{Tantai_Sarcopenia_2022} ;
  \item dénutrition, risque infectieux et complications extra-hépatiques.
\end{itemize}

L’évaluation annuelle d’une cirrhose compensée avancée requiert une approche multidimensionnelle combinant LSM hépato-splénique, imagerie, évaluation nutritionnelle et psychologique/addictologique, bilan vaccinal et décision endoscopique selon Baveno\,VII.

Un HDJ dédié permet de regrouper ces évaluations en une séance unique, standardisée et conforme aux recommandations tout en garantissant la réalisation d’au moins trois interventions valorisantes.

\end{spacing}

\clearpage

% ============================================================
% OBJECTIFS
% ============================================================

\subsection{Objectifs}
\needspace{6\baselineskip}

\begin{center}
\fcolorbox{APHPdark}{APHPsoft}{
\begin{minipage}{0.95\textwidth}
\vspace{0.9em}

\begin{itemize}[leftmargin=1cm]
  \item identifier la CSPH par critères non invasifs (Baveno\,VII) ;
  \item concentrer en une séance l’ensemble des évaluations clés : LSM, échographie, nutrition, psychologue/addictologue ;
  \item initier ou ajuster le carvédilol en conditions sécurisées ;
  \item structurer un parcours annuel de prévention (HTP, alcool, sarcopénie) ;
  \item actualiser le statut vaccinal ;
  \item produire une synthèse médicale facilitant la coordination ville–hôpital.
\end{itemize}

\vspace{0.9em}
\end{minipage}}
\end{center}

\clearpage

% ============================================================
% POPULATION ÉLIGIBLE
% ============================================================

\subsection{Population éligible}
\needspace{6\baselineskip}

\begin{itemize}[leftmargin=1cm]
  \item cirrhose compensée (Child\,A) ou stabilité post-décompensation ;
  \item LSM $\geq$\,15\,kPa, thrombopénie $<$\,150\,G/L ou facteurs de risque de CSPH ;
  \item consommation d’alcool active ou récente ;
  \item sarcopénie ou risque nutritionnel ;
  \item besoin d’une requalification annuelle selon Baveno\,VII.
\end{itemize}

\clearpage

% ============================================================
% PARCOURS DE SOINS
% ============================================================

\subsection{Parcours de soins}
\needspace{8\baselineskip}

\begin{figure}[!ht]
\centering
\caption{Parcours patient — HDJ cirrhose / HTP clinique}
\vspace{0.7cm}

\begin{tikzpicture}[
    node distance=1.5cm,
    box/.style={
        rectangle,
        rounded corners=3pt,
        draw=APHPdark,
        thick,
        text width=9cm,
        minimum height=1.4cm,
        align=center,
        fill=APHPsoft
    }
]

\node[box] (tri) {Orientation vers l’HDJ : hépatologie, ville, MCO};
\node[box, below=1.3cm of tri] (e1) {Évaluation initiale : LSM hépato-splénique, biologie, plaquettes};
\node[box, below=1.3cm of e1] (e2) {Échographie hépatique, évaluation nutritionnelle, psychologue/addictologue, vaccinations};
\node[box, below=1.3cm of e2] (e3) {Décisions : indication FOGD (Baveno\,VII), initiation du carvédilol si LSM $\geq 25$\,kPa};
\node[box, below=1.3cm of e3] (syn) {Synthèse médicale structurée et plan annuel};

\draw[->, thick, APHPdark] (tri) -- (e1);
\draw[->, thick, APHPdark] (e1) -- (e2);
\draw[->, thick, APHPdark] (e2) -- (e3);
\draw[->, thick, APHPdark] (e3) -- (syn);

\end{tikzpicture}
\end{figure}

\clearpage

% ============================================================
% CODAGE ET GHS ASSOCIÉS
% ============================================================

\subsection{Codage et GHS associés}
\begin{table}[h!]
\centering
\renewcommand{\arraystretch}{1.20}
\rowcolors{2}{APHPsoft}{white}

\begin{tabular}{p{4.6cm} p{6.8cm} c c c}
\toprule
\rowcolor{APHPsoft}
\textbf{Type de séance} & \textbf{DP / DR / DAS} & \textbf{GHM} & \textbf{GHS} & \textbf{Tarif} \\
\midrule
Évaluation complète (≥4) & 
DP: Z098\newline DR: K74.6\newline DAS: R18, K76.6, E44.x, F10.x &
07M13Z & 9616 & 941~€ \\

Évaluation complète (=3) & 
DP: Z098\newline DR: K74.6\newline DAS: R18, K76.6, E44.x, F10.x &
07M13Z & 9616 & 420~€ \\

LSM + échographie (≥4) &
DP: Z098\newline DR: K74.6\newline DAS: R16.1, K76.6, R18 &
07M13Z & 9616 & 941~€ \\

LSM + échographie (=3) &
DP: Z098\newline DR: K74.6\newline DAS: R16.1, K76.6, R18 &
07M13Z & 9616 & 420~€ \\

Nutrition + psychologue (≥4) &
DP: Z098\newline DR: K74.6\newline DAS: E43--E46, F10.x, R18 &
07M13Z & 9613 & 941~€ \\

Nutrition + psychologue (=3) &
DP: Z098\newline DR: K74.6\newline DAS: E43--E46, F10.x, R18 &
07M13Z & 9613 & 420~€ \\
\bottomrule
\end{tabular}

\caption{Codage et GHS associés — HDJ cirrhose}
\end{table}

\clearpage


% ============================================================
% TRACABILITÉ
% ============================================================
\needspace{8\baselineskip}   
\subsection{Traçabilité minimale}
\begin{table}[h!]
\centering
\renewcommand{\arraystretch}{1.20}
\rowcolors{2}{APHPsoft}{white}

\begin{tabular}{p{4.5cm} p{8.1cm}}
\toprule
\rowcolor{APHPsoft}
\textbf{Intervention} & \textbf{Éléments requis} \\
\midrule
LSM hépato-splénique & Critères qualité, IQR/med, validation, seuils Baveno~VII, message décisionnel \\
Échographie hépatique & CHC, HTP, flux portal, signes indirects, mesure splénique \\
Nutrition / sarcopénie & IMC, perte pondérale, dynamométrie, plan nutritionnel \\
Psychologie / alcool & Évaluation motivationnelle, repérage, orientation \\
FOGD & Indications Baveno~VII, résultats, calendrier \\
Décision carvédilol & Dose initiale, titration, objectifs tensionnels, suivi IDE \\
Vaccinations & Pneumocoque, grippe, COVID, VHA/VHB \\
Synthèse médicale & Classification Baveno, plan thérapeutique, coordination \\
\bottomrule
\end{tabular}

\caption{Traçabilité — HDJ cirrhose}
\end{table}

\clearpage

% ============================================================
% Organisation
% ============================================================

\subsection{Organisation}
\needspace{5\baselineskip}

\begin{itemize}[leftmargin=1.1cm]
  \item Direction : \textbf{Dr Lucia Parlati}
  \item Durée : 3--4~heures
  \item Lieu : Secteur HDJ — Service des maladies du foie
  \item Ressources : médecin sénior, infirmier expert/IPA, diététicien(ne), psychologue/addictologue
\end{itemize}

\clearpage

% ============================================================
% VOLUMÉTRIE DE RÉFÉRENCE
% ============================================================

\subsection{Volumétrie de référence}
\needspace{6\baselineskip}

\noindent File active annuelle : 7\,500 patients.  
Prévalence estimée de la cirrhose : 21\,\% → \textbf{1\,575 patients}.  
Taux de recours HDJ cible : \textbf{40\,\%} → environ \textbf{630 séances/an}.

\begin{center}
\begin{tabular}{lccc}
\toprule
\textbf{Séance} & \textbf{Volume} & \textbf{Tarif moyen} & \textbf{Recette annuelle} \\
\midrule
Évaluation complète (≥3–4 interv.) & 190 & 690~€ & 131\,000~€ \\
LSM + échographie                    & 285 & 690~€ & 197\,000~€ \\
Nutrition + psychologue              & 155 & 690~€ & 107\,000~€ \\
\midrule
\textbf{Total}                       & \textbf{630} & -- & \textbf{435\,000~€} \\
\bottomrule
\end{tabular}
\end{center}

\clearpage

% ============================================================
% PROJECTIONS
% ============================================================

\subsection{Projections d’activité}
\needspace{6\baselineskip}

\noindent Hypothèse : progression du recours HDJ de 40\,\% à 60\,\% de la file active.

\begin{center}
\begin{tabular}{lccc}
\toprule
\textbf{Année} & \textbf{Volume estimé} & \textbf{Tarif moyen} & \textbf{Recette brute} \\
\midrule
Amorce   & 630 & 690~€ & 435\,000~€ \\
Montée   & 790 & 690~€ & 545\,000~€ \\
Croisière & 945 & 690~€ & 652\,000~€ \\
\bottomrule
\end{tabular}
\end{center}

\clearpage

% ============================================================
% Conclusion
% ============================================================

\subsection{Conclusion}

L’HDJ dédié à l’évaluation des cirrhoses offre une organisation intégrée, conforme à Baveno\,VII, regroupant LSM, imagerie, évaluation nutritionnelle, repérage addictologique et initiation du carvédilol lorsque indiqué. Cette approche coordonnée renforce la prévention des décompensations, améliore la qualité du dépistage du CHC et fluidifie les parcours entre l’hépatologie et la médecine de ville.

\clearpage

% ============================================================
% VALIDATION
% ============================================================

\begin{center}
\begin{tabular}{p{4cm} p{7cm} p{4cm}}
\toprule
\rowcolor{APHPsoft}
\textbf{Date} & \textbf{Relecteur} & \textbf{Validation} \\
\midrule

03/12/2025   & Pr V.\,Mallet  & 03/12/2025 \\
NA           & Dr L.\,Parlati  & NA \\
03/12/2025   & Dr S.\,Bouam   & NA \\
NA           & Pr R.\,Coriat  & NA \\

\bottomrule
\end{tabular}
\end{center}

\clearpage

\printbibliography[heading=subbibliography,title={Références}]
\end{refsection}

% ======================================================
% CHAPITRE 3: Prise en Charge des Cirrhoses Terminales
% ======================================================
\clearpage
\section{HDJ Traitement des Cirrhoses Terminales}

\begin{refsection}
% ============================================================
% HÔPITAL DE JOUR — CIRRHOSE AVANCÉE / COMPLICATIONS DE LA CIRRHOS...
% ============================================================

\subsection{Rationnel médical}
\needspace{8\baselineskip}
\setcounter{table}{0}
\setcounter{figure}{0}

\begin{spacing}{1.30}

L’ensemble des maladies chroniques du foie évolue progressivement vers la cirrh\-ose et ses complications. La phase décompensée — ascite, encéphalopathie hépatique, hyponatrémie, hémorragie digestive, insuffisance rénale aiguë (AKI) ou syndrome hépato\-rénal (HRS) — correspond au stade terminal de la maladie, avec une mortalité annuelle souvent supérieure à 20--30\,\% \cite{EASL_DecompCirrhosis_2018,AASLD_PalliativeCirrhosis_2022}. 

Malgré les progrès réalisés dans le traitement des hépatites virales, l’incidence des décompensations ne diminue pas, en lien avec l’augmentation de l’obésité, du diabète de type~2, des maladies hépatiques métaboliques (MASLD), ainsi que la persistance de la consommation d’alcool \cite{Gines_LancetCirrhosis_2021}. En France, les données PMSI/AP--HP confirment une progression des séjours liés à l’ascite, aux infections et à l’AKI, avec une mortalité hospitalière de 11--15\,\% \cite{FrenchHepaticFailure_2020}.

Les complications de la cirrh\-ose avancée requièrent des interventions répétées et accessibles rapidement : ponction d’ascite avec perfusion d’albumine, albumine au long cours (type ANSWER) \cite{Caraceni_ANSWER_2018}, fer intraveineux, transfusion de CGR et soutien nutritionnel.

La structuration d’un HDJ dédié permet d’assurer ces prises en charge dans un cadre sécurisé, standardisé et pluridisciplinaire, favorisant la stabilisation, la prévention des réhospitalisations et la préparation d’éventuelles procédures (TIPS).

\end{spacing}

\clearpage

% ============================================================
% OBJECTIFS
% ============================================================

\subsection{Objectifs}
\needspace{6\baselineskip}

\begin{center}
\fcolorbox{APHPdark}{APHPsoft}{
\begin{minipage}{0.95\textwidth}
\vspace{0.9em}

\begin{itemize}[leftmargin=1.1cm]
  \item réduire les hospitalisations évitables liées aux complications de la cirrh\-ose avancée ;
  \item prévenir les décompensations sévères (AKI, HRS, ACLF) par un accès rapide aux actes nécessaires ;
  \item proposer un dispositif de stabilisation pré--TIPS ;
  \item regrouper en ambulatoire les actes complexes : LVP + albumine, albumine seule, fer IV, transfusion de CGR ;
  \item renforcer la continuité entre ville, urgences, hépatologie et MCO.
\end{itemize}

\vspace{0.9em}
\end{minipage}}
\end{center}

\bigskip

% ============================================================
% POPULATION ÉLIGIBLE
% ============================================================

\subsection{Population éligible}

\begin{itemize}[leftmargin=1.1cm]
  \item cirr\-hose avancée (Child B–C) avec ascite récurrente ou réfractaire ;
  \item risque d’AKI ou HRS (hyponatrémie, insuffisance rénale fonctionnelle) ;
  \item anémie d’hypertension portale nécessitant fer IV ou transfusion programmée ;
  \item encéphalopathie hépatique fluctuante nécessitant surveillance rapprochée ;
  \item dénutrition sévère ou sarcopénie ;
  \item candidats au TIPS ou non éligibles nécessitant prévention des réhospitalisations.
\end{itemize}

\clearpage

% ============================================================
% PARCOURS DE SOINS
% ============================================================

\subsection{Parcours de soins}

\begin{figure}[!ht]
\centering
\caption{Parcours patient — HDJ cirrhose avancée\footnotemark}
\vspace{0.8cm}

\begin{tikzpicture}[
    node distance=1.6cm,
    box/.style={
        rectangle,
        rounded corners=3pt,
        draw=APHPdark,
        thick,
        text width=9.0cm,
        minimum height=1.5cm,
        align=center,
        fill=APHPsoft
    }
]

\node[box] (tri) {Orientation vers l’HDJ \\[3pt] (hépatologie / urgences / MCO / ville)};
\node[box, below=1.4cm of tri] (etape1) {Évaluation clinique initiale, biologie récente, imagerie ciblée si besoin ; vérification du statut vaccinal (pneumocoque, grippe, hépatites~A et~B)};
\node[box, below=1.4cm of etape1] (etape2) {Acte programmé : \\[3pt] LVP + albumine, albumine seule, fer IV, transfusion de CGR};
\node[box, below=1.4cm of etape2] (etape3) {Surveillance spécialisée : \\[3pt] constantes, tolérance, complications};
\node[box, below=1.4cm of etape3] (synth) {Synthèse médicale et programmation de la séance suivante};

\draw[->, thick, APHPdark] (tri) -- (etape1);
\draw[->, thick, APHPdark] (etape1) -- (etape2);
\draw[->, thick, APHPdark] (etape2) -- (etape3);
\draw[->, thick, APHPdark] (etape3) -- (synth);

\end{tikzpicture}
\end{figure}

\footnotetext{
\textbf{HDJ} : hôpital de jour ; 
\textbf{MCO} : médecine–chirurgie–obstétrique ; 
\textbf{LVP} : paracentèse évacuatrice (large volume paracentesis) ; 
\textbf{CGR} : concentrés de globules rouges.
}

\clearpage

% ============================================================
% CODAGE, TARIFS ET VOLUMÉTRIE 2024
% ============================================================

\subsection{Codage, tarifs et volumétrie de référence}
\needspace{12\baselineskip}

\begin{sidewaystable}[p]
\centering
\renewcommand{\arraystretch}{1.25}
\rowcolors{2}{APHPsoft}{white}

\begin{tabularx}{\textwidth}{
p{4.8cm}
X
>{\centering\arraybackslash}p{1.7cm}
>{\centering\arraybackslash}p{1.6cm}
>{\centering\arraybackslash}p{1.9cm}
>{\centering\arraybackslash}p{1.8cm}
>{\centering\arraybackslash}p{2.4cm}
}
\toprule
\rowcolor{APHPsoft}
\textbf{Type de séance} &
\textbf{DP / DR / DAS} &
\textbf{GHM} &
\textbf{GHS} &
\textbf{Tarif 2025} &
\textbf{Volume 2024} &
\textbf{Recette 2024} \\
\midrule

Ponction d’ascite + albumine &
DP : R18 (ascite)\newline
DAS : K74.6, K76.6, ±N17.x, ±E87.1 &
07M14T & 2559 & 603~€ &
160 &
96\,480~€ \\

Séance d’albumine seule &
DP : Z512 (réserve)\newline
DR : R18 ou K76.6 &
28Z17Z & 9616 & 440~€ &
18 &
7\,920~€ \\

Fer injectable (anémie HTP) &
DP : Z512\newline
DR : D50.8 ou D64.9\newline
DAS : K74.6, K76.6 &
28Z17Z & 9616 & 440~€ &
125 &
55\,000~€ \\

Transfusion de CGR &
DP : Z5130\newline
DR : D50.8 ou D62\newline
DAS : K74.6, K76.6 &
28Z14Z & 9613 & 791~€ &
15 &
11\,865~€ \\
\midrule

\textbf{Total annuel} & -- & -- & -- & -- &
\textbf{319} &
\textbf{171\,265~€} \\

\bottomrule
\end{tabularx}

\caption{Codage, tarifs et volumétrie — HDJ cirrhose avancée (2024)}
\end{sidewaystable}

\clearpage

% ============================================================
% TRACABILITÉ
% ============================================================

\subsection{Traçabilité minimale}

\begin{table}[h!]
\centering
\renewcommand{\arraystretch}{1.25}
\rowcolors{2}{APHPsoft}{white}

\begin{tabular}{p{5cm} p{9cm}}
\toprule
\rowcolor{APHPsoft}
\textbf{Intervention} & \textbf{Éléments requis} \\
\midrule

Ponction d’ascite &
Volume évacué ; repérage écho ; surveillance 4~h ; douleur ; hypotension ; biologie pré-acte \\

Perfusion d’albumine &
Prescription ; indication ; traçabilité du lot ; volume perfusé ; surveillance hémodynamique \\

Fer injectable &
Indication (anémie HTP) ; traçabilité lot ; protocole perfusion ; surveillance immédiate et retardée \\

Transfusion de CGR &
Traçabilité PSL ; groupage ; concordance ; surveillance renforcée ; incidents transfusionnels \\

Entretien médical &
Justification ; risque HRS/AKI ; bilan clinique ; adaptation thérapeutique \\

Surveillance spécialisée &
Constantes ; EVA douleur ; hémodynamique ; reins ; drainage post-ponction \\

Coordination / éducation &
Fiche de liaison ville–hôpital ; conseils HTP ; éducation sur signes d’alerte \\
\bottomrule
\end{tabular}

\caption{Traçabilité — HDJ cirrhose avancée}
\end{table}

\clearpage

% ============================================================
% ORGANISATION
% ============================================================

\subsection{Organisation}
\needspace{5\baselineskip}

\begin{itemize}[leftmargin=1.1cm]
    \item Direction : \textbf{Dr Valérie D’Halluin-Venier}
    \item Durée : 4--6~heures
    \item Lieu : Secteur HDJ — Service des maladies du foie
    \item Ressources : médecin sénior, infirmier expert/IPA, diététicien(ne), psychologue/addictologue
\end{itemize}

\bigskip

% ============================================================
% PROJECTIONS D’ACTIVITÉ
% ============================================================

\subsection{Projections d’activité et recettes prévisionnelles}

\noindent Basé sur un tarif moyen pondéré : \textbf{\textasciitilde540~€ / séance}.  
Hypothèse : \textbf{+50 patients par pallier} à partir de la volumétrie 2024 (277 actes).

\begin{table}[h!]
\centering
\renewcommand{\arraystretch}{1.18}
\rowcolors{2}{APHPsoft}{white}

\begin{tabular}{lccc}
\toprule
\rowcolor{APHPsoft}
\textbf{Année} & \textbf{Volume estimé} & \textbf{Tarif moyen} & \textbf{Recette brute estimée} \\
\midrule
Amorce    & 327 & 540~€ & 176\,580~€ \\
Montée    & 377 & 540~€ & 203\,580~€ \\
Croisière & 427 & 540~€ & 230\,580~€ \\
\bottomrule
\end{tabular}

\caption{Prévisions d’activité et recettes prévisionnelles — HDJ cirrhose avancée}
\end{table}

\clearpage

% ============================================================
% CONCLUSION
% ============================================================

\subsection{Conclusion}
\needspace{6\baselineskip}

Les HDJ dédiés à la cirrh\-ose avancée offrent un cadre structuré pour les interventions indispensables à la stabilisation des patients les plus fragiles. En rassemblant ponctions d’ascite, albumine, fer injectable et transfusions dans un parcours sécurisé et standardisé, ils contribuent à réduire les hospitalisations évitables, à prévenir les décompensations sévères et à optimiser la continuité des soins entre ville, urgences et hépatologie.
Pour les modalités techniques détaillées (ponction d’ascite, perfusion d’albumine, fer IV, transfusion),
voir les annexes~\ref{sec:annexe_ascite}, \ref{sec:annexe_fer} et \ref{sec:annexe_cgr}.


% ============================================================
% VALIDATION
% ============================================================

\begin{center}
\begin{tabular}{p{4cm} p{7cm} p{4cm}}
\toprule
\rowcolor{APHPsoft}
\textbf{Date d’envoi} & \textbf{Nom du relecteur} & \textbf{Date de validation} \\
\midrule
03/12/2025 & Pr V.\,Mallet & 08/12/2025 \\
03/12/2025 & Dr S.\,Bouam & NA \\
03/12/2025 & Dr V.\,D’Halluin-Venier & NA \\
NA & Pr R.\,Coriat & NA \\
\bottomrule
\end{tabular}
\end{center}

\clearpage

\printbibliography[heading=subbibliography,title={Références}]
\end{refsection}

% ============================
% CHAPITRE 4: Hépatométabolique
% ============================
\clearpage
\section{HDJ Hépatométabolique}

\begin{refsection}
% ============================================================
% HÔPITAL DE JOUR HÉPATOMÉTABOLIQUE
% ============================================================

\subsection{Rationnel médical}
\needspace{8\baselineskip}

\begin{spacing}{1.30}

L’augmentation rapide de l’obésité, du diabète de type 2 (DT2) et des maladies hépatiques métaboliques (MASLD/MASH) constitue aujourd’hui un défi majeur pour le système de santé. Dans cette population, la maladie hépatique est hautement prévalente : plus de 60\,\% des patients DT2 présentent une atteinte hépatique métabolique et 15--20\,\% une fibrose significative (F$\geq$2), les exposant à un risque accru de complications évolutives \cite{RN565}.

\medskip

En France, 3{,}5 à 4 millions de personnes vivent avec un DT2 selon les données récentes de Santé publique France et de l’Assurance Maladie (SNDS) \cite{SPF2021Diabete}. Cette population représente un réservoir important de patients susceptibles d’évoluer vers une maladie hépatique avancée, avec un impact croissant sur les parcours de soins et les capacités hospitalières.

% \medskip

% L’identification précoce des formes avancées reste difficile : les complications sévères sont rares à l’échelle populationnelle, ce qui nécessite un repérage ciblé et structuré. Dans la cohorte DT2 de l’Entrepôt de Données de Santé (EDS) de l’AP-HP (77\,368 patients), l’incidence annuelle des événements hépatiques graves n’est que de 1{,}31 pour 1\,000 patients-années.

\medskip

Pourtant, des interventions simples et peu coûteuses — réduction de la consommation d’alcool, perte pondérale modérée, amélioration de la qualité alimentaire — démontrent un impact tangible sur l’évolution de la maladie \cite{RN597}. Leur efficacité dépend cependant d’un repérage précoce et d’une organisation lisible du parcours.

\medskip

L’arrivée de nouvelles thérapeutiques (agonistes du GLP-1, resmétirom et autres agents en développement) renforce la nécessité d’un dispositif capable d’identifier, évaluer et suivre précocement les patients éligibles, tout en garantissant l’appropriation des recommandations sur le territoire.

\medskip

Dans ce contexte, des outils simples et robustes de stratification sont indispensables. Les biomarqueurs non invasifs constituent désormais la base du tri diagnostique dans la population DT2. Le score FIB-4, largement disponible dans les logiciels médicaux et recommandé par les sociétés savantes, permet d’exclure efficacement les formes avancées et d’orienter les patients présentant un FIB-4 $\geq$ 1{,}3 vers une évaluation spécialisée (élastométrie, imagerie) \cite{RN597,EASL2024MASLD} .

\medskip

Sur le territoire de Cochin, un flux régulier de patients à FIB-4 élevé est déjà identifié via les consultations de diabétologie, de cardiologie, les CPTS et les acteurs de premier recours. Les actions d’information menées localement et régionalement renforcent ce repérage et traduisent une dynamique territoriale structurée autour de la MASLD.

\medskip

Dans ce cadre, la création d’un Hôpital de Jour hépatométabolique répond à un besoin clairement identifié. Ce dispositif offre une évaluation intégrée, standardisée et rapide, combinant imagerie, exploration comportementale, évaluation nutritionnelle, activité physique adaptée et prise en charge psychologique. Il permet :

\begin{itemize}
    \item d’optimiser le triage des patients à risque ;
    \item de réduire les retards diagnostiques ;
    \item d’améliorer la pertinence des orientations (consultation, suivi, recherche) ;
    \item de proposer des interventions à fort impact populationnel ;
    \item d’inscrire le parcours dans une logique territoriale en lien avec les acteurs de premier recours.
\end{itemize}

\end{spacing}

\clearpage

% ============================================================
\subsection{Objectifs}
\needspace{6\baselineskip}

\begin{center}
\fcolorbox{APHPdark}{APHPsoft}{
\begin{minipage}{0.92\textwidth}
\begin{itemize}[leftmargin=1.1cm]
    \item Dépister précocement la fibrose hépatique significative ou avancée.
    \item Structurer une évaluation intégrée : biomarqueurs, imagerie, diététique, psychologie.
    \item Initier une prise en charge hygiéno-diététique et comportementale.
    \item Identifier les patients éligibles aux thérapeutiques MASLD/MASH et aux protocoles de recherche.
\end{itemize}
\end{minipage}}
\end{center}

\bigskip
\bigskip

% ============================================================
\subsection{Population éligible}
\needspace{5\baselineskip}

\begin{itemize}[leftmargin=1.1cm]
    \item Diabète de type 2 ou syndrome métabolique.
    \item FIB-4 $\geq$ 1{,}3.
    \item Suspicion clinique ou échographique de MASLD/MASH.
\end{itemize}

\clearpage

% ============================================================
% PARCOURS PATIENT
% ============================================================

\subsection{Parcours de soins (3--4 heures)}
\needspace{8\baselineskip}

\begin{figure}[!ht]
\centering
\caption{Parcours patient — HDJ Hépatométabolique}
\vspace{0.8cm}

\begin{tikzpicture}[
    node distance=1.6cm,
    box/.style={
        rectangle,
        rounded corners=3pt,
        draw=APHPdark,
        thick,
        text width=9.0cm,
        minimum height=1.2cm,
        align=center,
        fill=APHPsoft
    }
]
\node[box] (tri) {Tri initial \\ FIB-4 $\geq$ 1{,}3};
\node[box, below=1.4cm of tri] (entree) {Entrée en HDJ hépatométabolique};
\node[box, below=1.4cm of entree] (echo) {Échographie abdominale + Doppler};
\node[box, below=1.4cm of echo] (fibro) {FibroScan / élastographie};
\node[box, below=1.4cm of fibro] (diet) {Consultation diététique};
\node[box, below=1.4cm of diet] (psy) {Évaluation psychologique \\ (alcool / TCA)};
\node[box, below=1.4cm of psy] (synth) {Synthèse médicale \\ Plan thérapeutique};

\draw[->, thick, APHPdark] (tri) -- (entree);
\draw[->, thick, APHPdark] (entree) -- (echo);
\draw[->, thick, APHPdark] (echo) -- (fibro);
\draw[->, thick, APHPdark] (fibro) -- (diet);
\draw[->, thick, APHPdark] (diet) -- (psy);
\draw[->, thick, APHPdark] (psy) -- (synth);

\end{tikzpicture}
\end{figure}

\clearpage

% ============================================================
% Organisation et ressources nécessaires
% ============================================================

\subsection{Organisation}
\needspace{5\baselineskip}

\begin{itemize}[leftmargin=1.1cm]
    \item Direction: \textbf{Docteur Lucia Parlati}
    \item Durée : 3--4 heures.
    \item Lieu : Secteur HDJ — Service des maladies du foie.
    \item Ressources : médecin sénior, infirmier expert/IPA, diététicien(ne), psychologue/addictologue.

\end{itemize}

\bigskip
% ============================================================
% CODAGE ET GHS ASSOCIÉS
% ============================================================

\subsection{Codage et GHS associés}
\needspace{6\baselineskip}

\noindent\textbf{Cadre général.}  
Suspicion de MASLD/MASH avec FIB-4 ≥1,3 requérant un HDJ intégrant : élastographie, échographie, consultation médicale, évaluation diététique ou psychologique.  
Codage conforme aux avis DIM 2024–2025 (DP R945 ; DAS E11.98).

\medskip

\begin{table}[h!]
\centering
\renewcommand{\arraystretch}{1.25}
\rowcolors{2}{APHPsoft}{white}

\begin{tabularx}{\textwidth}{
>{\raggedright\arraybackslash}p{4.8cm}
>{\raggedright\arraybackslash}X
>{\centering\arraybackslash}p{2cm}
>{\centering\arraybackslash}p{1.8cm}
>{\centering\arraybackslash}p{2.2cm}}
\toprule
\rowcolor{APHPsoft}
\textbf{Type de séance} &
\textbf{DP / DR / DAS} &
\textbf{GHM} &
\textbf{GHS} &
\textbf{Tarif 2025} \\
\midrule

Suspicion MASLD/MASH (≥4 interventions) &
DP : R945 \newline
DAS : E11.98 &
07M14T &
2559 &
603~€ \\

Suspicion MASLD/MASH (=3 interventions) &
DP : R945 \newline
DAS : E11.98 &
07M14T &
2584 &
298~€ \\
\bottomrule
\end{tabularx}

\caption{Codage et GHS associés — HDJ Maladies du Foie Métaboliques}
\end{table}

\medskip

\noindent Actes CCAM typiques : HLQM002 (élastographie), ZCQM004 (échographie–Doppler), consultations médicale/diététique/psychologique.

\clearpage

% --------------------------------------------------------
% TABLEAU 2 — Traçabilité (VERSION SPÉCIFIQUE MASLD/MASH)
% --------------------------------------------------------

\begin{table}[h!]
\centering
\renewcommand{\arraystretch}{1.25}
\rowcolors{2}{APHPsoft}{white}

\begin{tabular}{p{5cm} p{9cm}}
\toprule
\rowcolor{APHPsoft}
\textbf{Intervention} & \textbf{Traçabilité requise} \\
\midrule

Consultation médicale de synthèse &
Motif, anamnèse ciblée (métabolique, hépatique, cardiovasculaire), facteurs de risque, résultats clés (FIB-4, LSM, biologie), diagnostic de probabilité MASLD/MASH, décision thérapeutique, critères d'orientation, messages ville–hôpital. \\

Évaluation diététique &
Analyse des apports, structure des repas, comportements alimentaires, objectifs pondéraux, sarcopénie/dénutrition éventuelle, recommandations personnalisées, plan nutritionnel réaliste, critères de suivi. \\

Évaluation psychologique / addictologique (alcool, TCA) &
Repérage des consommations (auditif/alcohol), screening TCA, analyse motivationnelle, facteurs émotionnels contextuels, ressources disponibles, recommandations ciblées, orientation éventuelle vers un suivi spécialisé. \\

Consultations spécialisées (cardiovasculaire, diabétologie...) &
Paramètres cardio-métaboliques (TA, PA, IMC, tour de taille), dépistage du risque cardiovasculaire, évaluation diabète/pré-diabète, recommandations et coordination des prises en charge. \\

Actes techniques (élastographie, échographie) &
Compte rendu normalisé en annexe \\

Courrier de sortie (obligatoire) &
Synthèse intégrée : diagnostic MASLD/MASH probable ou confirmé, niveau de risque cardio-métabolique, interventions réalisées, décisions thérapeutiques, calendrier de suivi, messages clefs pour le médecin traitant. \\

\end{tabular}

\caption{Traçabilité des interventions — HDJ MASLD/MASH}
\end{table}

\clearpage

% ============================================================
% VOLUMÉTRIE (RÉFÉRENCE)
% ============================================================

\subsection{Volumétrie de référence}

\noindent HDJ MASLD/MASH — estimations selon montée en charge prévue.

\begin{center}
\begin{tabular}{lccc}
\toprule
\textbf{Type de séance} & \textbf{Volume annuel} & \textbf{Tarif unitaire} & \textbf{Recette estimée} \\
\midrule
≥4 interventions (GHS 2559) & 120 & 603~€ & 72\,360~€ \\
3 interventions (GHS 2584)  & 30  & 298~€ & 8\,940~€ \\
\midrule
\textbf{Total} & \textbf{150} & -- & \textbf{\textasciitilde81\,300~€} \\
\bottomrule
\end{tabular}
\end{center}

\clearpage


% ============================================================
% PROJECTIONS D’ACTIVITÉ
% ============================================================

\subsection{Projections d’activité et recettes prévisionnelles}

\noindent Basé sur un tarif moyen pondéré : \textbf{\textasciitilde540~€ / séance}.

\begin{center}
\begin{tabular}{lccc}
\toprule
\textbf{Année} & \textbf{Volume estimé} & \textbf{Tarif moyen} & \textbf{Recette brute estimée} \\
\midrule
Amorce   & 150 & 540~€ & 81\,000~€ \\
Montée   & 300 & 540~€ & 162\,000~€ \\
Croisière & 500 & 540~€ & 270\,000~€ \\
\bottomrule
\end{tabular}
\end{center}

\clearpage

% ============================================================
% Conclusions
% ============================================================
\subsection{Conclusion}
\needspace{6\baselineskip}

L’HDJ hépatométabolique constitue un dispositif pertinent, simple à organiser et autosoutenable. Il optimise le dépistage, l’accès aux thérapeutiques et la prise en charge multidisciplinaire des patients MASLD/MASH et DT2.

\medskip

Pour les supports opérationnels (échographie, évaluation diététique et psychologique), 
voir les annexes~\ref{sec:annexe_echo}, \ref{sec:annexe_diete} et \ref{sec:annexe_psy}.


% ============================================================
% VALIDATION
% ============================================================

\begin{center}
\begin{tabular}{p{4cm} p{7cm} p{4cm}}
\toprule
\rowcolor{APHPsoft}
\textbf{Date d'envoie} & \textbf{Nom du relecteur} & \textbf{Date de validation} \\
\midrule

03/12/2025 & Pr V.\,Mallet & 03/12/2025 \\
05/12/2025 & Dr L.\,Parlati & NA \\
03/12/2025 & Dr S.\,Bouam & NA \\
NA & Pr R.\,Coriat & NA \\
\bottomrule
\end{tabular}
\end{center}

\clearpage
\printbibliography[heading=subbibliography,title={Références}]
\end{refsection}

% ============================
% CHAPITRE 5: Addictologie
% ============================
\clearpage
\section{HDJ Addictologie}

\begin{refsection}
% ============================================================
% HÔPITAL DE JOUR — ADDICTOLOGIE (HDJA)
% ============================================================
\subsection{Rationnel médical}
\begin{spacing}{1.30}
L’alcool constitue en France un déterminant majeur de morbi-mortalité évitable, responsable d’environ 41\,000 décès annuels\cite{Guerin2013AlcoolMortalite} — incluant 16\,000 cancers, 9\,900 maladies cardiovasculaires, 6\,800 maladies digestives et 5\,400 causes externes — pour un coût social estimé à 118 milliards d’euros par an.\cite{Kopp2015CoutSocial} Il représente la première cause identifiable de démence précoce (<65 ans)\cite{Schwarzinger2018AlcoholDementia}, et les troubles d’usage d’alcool réduisent l’espérance de vie de 10 à 13 ans dans leurs formes les plus sévères.\cite{Schwarzinger2017ChronicDisease}

Dans la population générale, 23,6\,\% des adultes dépassent les repères de consommation à faible risque\cite{Richard2019AlcoolSPF}, avec une prévalence particulièrement élevée dans les filières hospitalières exposées (hépatologie, oncologie, diabétologie, psychiatrie). L’alcool potentialise la progression de toutes les maladies chroniques du foie — y compris à faible niveau de consommation — et constitue le principal déterminant évolutif des complications sévères chez les patients atteints de diabète de type~2.\cite{Mallet2022T2DLiverBurden}

L’abstinence est le levier pronostique le plus puissant dans les maladies hépatiques liées à l’alcool : elle améliore la survie après hépatite alcoolique sévère\cite{Louvet2008AHA, Parlati2025RehabAH} et diminue le risque de décompensation chez les patients cirrhotiques.\cite{Addolorato_Baclofen_2007, Loomba_AlcoholAbstinence_2020} Les prises en charge intensives du sevrage réduisent également l’incidence des cancers attribuables à l’alcool et améliorent la survie globale.\cite{Schwarzinger2024RehabCancer}

Dans ce contexte, l’AP-HP a un rôle structurant à jouer dans l’organisation des parcours dédiés. Un HDJ addictologique intégré au sein du DMU DIGEST répondrait à un besoin clairement identifié : filière institutionnelle lisible, interventions spécialisées, continuité des soins et réduction des hospitalisations évitables. Ainsi, la structuration d’un HDJ addictologie s’inscrit pleinement dans les missions de santé publique de l’AP-HP.

\end{spacing}
% Réinitialisation des compteurs (doit être placée ici)

\clearpage
% ============================================================
% OBJECTIFS
% ============================================================

\subsection{Objectifs}

\begin{center}
\fcolorbox{APHPdark}{APHPsoft}{
\begin{minipage}{0.95\textwidth}
\vspace{0.7em}
\begin{itemize}[leftmargin=1.1cm]
\item Réaliser une évaluation somatique et addictologique complète ;
\item Réduire les risques et dommages liés à l’alcool ;
\item Stabiliser la trajectoire hépatologique et prévenir les décompensations ;
\item Accompagner la réduction ou l’arrêt des consommations ;
\item Renforcer l’autonomie et la continuité des soins ville–hôpital.
\end{itemize}
\vspace{0.7em}
\end{minipage}}
\end{center}

\clearpage

% ============================================================
% POPULATION ÉLIGIBLE
% ============================================================

\subsection{Population éligible}

\begin{itemize}[leftmargin=1.1cm]
\item Trouble de l’usage d’alcool (usage nocif ou dépendance) ;
\item Pathologies hépatiques alcool-attribuables (HAA, cirrhoses, MetALD) ;
\item Indication d’une évaluation somatique–addictologique conjointe ;
\item Objectif d’abstinence, de réduction des risques ou de stabilisation ;
\item Fragilités psychosociales et/ou cognitives nécessitant un suivi structuré.
\end{itemize}

\footnotetext{
\textbf{HAA} : hépatite alcoolique aiguë ; 
\textbf{MetALD} : metabolic dysfunction–associated alcohol-related liver disease.
}

\clearpage


% ============================================================
% PARCOURS DE SOINS
% ============================================================

\subsection{Parcours de soins}

% --- PAGE DES ACRONYMES (nouvelle page blanche dédiée) ---
\textbf{Acronymes utilisés dans les parcours HDJ Addictologie}
\vspace{1cm}

\begin{itemize}[leftmargin=1.2cm]
    \item \textbf{ELSA} : Équipe de liaison et de soins en addictologie
    \item \textbf{CSAPA} : Centre de soins, d’accompagnement et de prévention en addictologie
    \item \textbf{IDE} : Infirmier diplômé d’État
    \item \textbf{MOCA} : Montreal Cognitive Assessment
    \item \textbf{BEARNI} : Brief Evaluation of Alcohol-Related Neuropsychological Impairments
    \item \textbf{ECG} : Électrocardiogramme
    \item \textbf{EFR} : Explorations fonctionnelles respiratoires
    \item \textbf{CI} : Contre-indication
    \item \textbf{CIWA} : Clinical Institute Withdrawal Assessment
    \item \textbf{APA} : Activité physique adaptée
    \item \textbf{SSR} : Soins de suite et de réadaptation
\end{itemize}

\clearpage
% --- FIN PAGE ACRONYMES / DÉBUT DES FIGURES ---


% === FIGURE 1 — Évaluation somatique et addictologique ========================

\begin{figure}[h!]
\centering
\caption{Parcours évaluation addictologique et somatique}
\vspace{0.8cm}

\begin{tikzpicture}[
    node distance=1.8cm,
    box/.style={
        rectangle, rounded corners=3pt,
        draw=APHPdark, thick,
        text width=10cm, minimum height=1.7cm,
        align=center, fill=APHPsoft
    }
]

\node[box] (ori) {Orientation vers l’HDJ \\
\small médecin traitant, ELSA, CSAPA, psychiatrie, hépatologie, consultations d’addictologie};

\node[box, below=1.5cm of ori] (eval) {Évaluation médicale et infirmière \\
\small addictologie, hépatologie, entretien IDE};

\node[box, below=1.5cm of eval] (neuro) {Évaluations psychologique et neurocognitive \\
\small psychologue ; MOCA / BEARNI};

\node[box, below=1.5cm of neuro] (som) {Bilan somatique \\
\small ECG, EFR, scanner thoracique, biologie, échographie, Fibroscan};

\node[box, below=1.5cm of som] (synth) {Synthèse pluridisciplinaire \\
\small orientation : HDJ réduction / HDJ sevrage / hospitalisation complète};

\draw[->, thick, APHPdark] (ori) -- (eval);
\draw[->, thick, APHPdark] (eval) -- (neuro);
\draw[->, thick, APHPdark] (neuro) -- (som);
\draw[->, thick, APHPdark] (som) -- (synth);

\end{tikzpicture}
\end{figure}

% === FIGURE 2 — Réduction des risques ========================================

\begin{figure}[h!]
\centering
\caption{Parcours réduction des risques et dommages}
\vspace{0.8cm}

\begin{tikzpicture}[
    node distance=1.8cm,
    box/.style={rectangle, rounded corners=3pt, draw=APHPdark, thick,
    text width=10cm, minimum height=1.7cm, align=center, fill=APHPsoft}
]

\node[box] (ori) {Orientation vers l’HDJ \\
\small patients souhaitant réduire leur consommation sans objectif de sevrage complet};

\node[box, below=1.5cm of ori] (med)
{Suivi médical et paramédical renforcé \\ \small addictologie, IDE, psychologue};

\node[box, below=1.5cm of med] (medias)
{Thérapies par médiation \\ \small APA, socio-esthétique, art-thérapie, revue de presse, écriture, groupes};

\node[box, below=1.5cm of medias] (social)
{Accompagnement social \\ \small démarches administratives, insertion, gestion de crise sociale};

\node[box, below=1.5cm of social] (synth)
{Synthèse pluridisciplinaire \\ \small adaptation du programme, continuité des soins};

\draw[->, thick, APHPdark] (ori) -- (med);
\draw[->, thick, APHPdark] (med) -- (medias);
\draw[->, thick, APHPdark] (medias) -- (social);
\draw[->, thick, APHPdark] (social) -- (synth);

\end{tikzpicture}
\end{figure}

% === FIGURE 3 — Sevrage ambulatoire ========================================

\begin{figure}[h!]
\centering
\caption{Parcours Sevrage ambulatoire}
\vspace{0.8cm}

\begin{tikzpicture}[
    node distance=1.8cm,
    box/.style={rectangle, rounded corners=3pt, draw=APHPdark, thick,
    text width=10cm, minimum height=1.7cm, align=center, fill=APHPsoft}
]

\node[box] (indi)
{Indication de sevrage ambulatoire \\ \small critères de sécurité, absence de CI, environnement compatible};

\node[box, below=1.5cm of indi] (med)
{Évaluation et suivi médical \\ \small addictologie, IDE, monitorage CIWA quotidien};

\node[box, below=1.5cm of med] (psy)
{Suivi psychologique \\ \small soutien, renforcement motivationnel};

\node[box, below=1.5cm of psy] (medias)
{Thérapies par médiation \\ \small APA, art-thérapie, socio-esthétique, revue de presse, groupes};

\node[box, below=1.5cm of medias] (soc)
{Accompagnement social \\ \small démarches, stabilisation du cadre de vie};

\node[box, below=1.5cm of soc] (synth)
{Synthèse pluridisciplinaire \\ \small plan de continuité, prévention des rechutes};

\draw[->, thick, APHPdark] (indi) -- (med);
\draw[->, thick, APHPdark] (med) -- (psy);
\draw[->, thick, APHPdark] (psy) -- (medias);
\draw[->, thick, APHPdark] (medias) -- (soc);
\draw[->, thick, APHPdark] (soc) -- (synth);

\end{tikzpicture}
\end{figure}


% === FIGURE 4 — Consolidation de sevrage ====================================

\begin{figure}[h!]
\centering
\caption{Parcours Consolidation de sevrage}
\vspace{0.8cm}

\begin{tikzpicture}[
    node distance=1.8cm,
    box/.style={rectangle, rounded corners=3pt, draw=APHPdark, thick,
    text width=10cm, minimum height=1.7cm, align=center, fill=APHPsoft}
]

\node[box] (ori)
{Public concerné \\ \small post-sevrage ambulatoire ou résidentiel ; attente SSR ; retours de SSR};

\node[box, below=1.5cm of ori] (med)
{Suivi médical et paramédical structuré \\ \small addictologie ; IDE ; psychologue};

\node[box, below=1.5cm of med] (medias)
{Thérapies par médiation \\ \small APA, socio-esthétique, art-thérapie, revue de presse, écriture, groupes};

\node[box, below=1.5cm of medias] (reinser)
{Accompagnement social et réinsertion \\ \small démarches ; retour au domicile ; activités};

\node[box, below=1.5cm of reinser] (synth)
{Synthèse pluridisciplinaire \\ \small consolidation du sevrage ; prévention des rechutes};

\draw[->, thick, APHPdark] (ori) -- (med);
\draw[->, thick, APHPdark] (med) -- (medias);
\draw[->, thick, APHPdark] (medias) -- (reinser);
\draw[->, thick, APHPdark] (reinser) -- (synth);

\end{tikzpicture}
\end{figure}



% ============================================================
% TRAÇABILITÉ
% ============================================================
\clearpage
\subsection{Traçabilité des interventions}

\begin{table}[h!]
\centering
\renewcommand{\arraystretch}{1.25}
\rowcolors{2}{APHPsoft}{white}

\begin{tabular}{p{5cm} p{9.2cm}}
\toprule
\rowcolor{APHPsoft}
\textbf{Intervention} & \textbf{Traçabilité requise} \\
\midrule

Évaluation somatique et addictologique &
Anamnèse ; scores AUDIT/CIWA ; comorbidités ; bilans pré-thérapeutiques ; évaluations IDE, psychologue, neurocognition ; consultations spécialisées ; synthèse médicale. \\

Réduction des risques &
Objectifs personnalisés ; prévention ; entretiens motivationnels ; ateliers ; activités (Annexe~C.2) ; réévaluation hebdomadaire. \\

Sevrage ambulatoire &
Scores CIWA répétés ; adaptation thérapeutique ; surveillance ; activités psychologiques/sociales/médiations (Annexe~C.2) ; incidents ; synthèse médicale de fin de cure. \\

Consultations spécialisées &
Synthèse écrite obligatoire : objectifs, évolution, recommandations. \\

Synthèse médico-psycho-sociale &
Plan intégré ; orientation ; coordination ville–hôpital. \\
\bottomrule
\end{tabular}

\caption{Traçabilité des interventions — HDJ Addictologie}
\end{table}

\clearpage

% ============================================================
% ORGANISATION
% ============================================================

\subsection{Organisation}

\begin{itemize}[leftmargin=1.1cm]
    \item Direction : \textbf{Docteur Marion Corouge}
    \item Durée : 6--8~heures
    \item Lieu : Secteur HDJ — Service des maladies du foie
    \item Ressources : médecin senior, infirmier expert/IPA, psychologue/addictologue
\end{itemize}

\clearpage

% ============================================================
% SYNTHÈSE — CODAGE, VOLUMÉTRIE ET RECETTES
% ============================================================

\subsection{Synthèse codage–volumétrie–recettes}

\begin{sidewaystable}[p]
\centering
\renewcommand{\arraystretch}{1.22}
\rowcolors{2}{APHPsoft}{white}

\begin{tabularx}{\textwidth}{
p{4.8cm}
X
>{\centering\arraybackslash}p{1.6cm}
>{\centering\arraybackslash}p{1.6cm}
>{\centering\arraybackslash}p{1.7cm}
>{\centering\arraybackslash}p{2cm}
>{\centering\arraybackslash}p{2.7cm}
}
\toprule
\rowcolor{APHPsoft}
\textbf{Type de séance} &
\textbf{DP / DAS attendus} &
\textbf{GHM} &
\textbf{GHS} &
\textbf{Tarif 2025} &
\textbf{Volume annuel projeté} &
\textbf{Recette annuelle projetée} \\
\midrule

HDJ Évaluation somatique + addictologique &
DP : F101\newline
DAS : comorbidités majeures &
20Z051 & 7200 & 774~€ &
47 & 36\,378~€ \\

HDJ Réduction des risques et dommages &
DP : Z714\newline
DAS : comorbidités majeures &
23M06T & 7272 & 701~€ &
47 & 32\,947~€ \\

HDJ Sevrage ambulatoire / consolidation &
DP : F102 ou Z502\newline 
DAS : comorbidités majeures &
20Z04T & 7271 & 541~€ &
47 & 25\,478~€ \\
\midrule

\textbf{Total annuel — Phase d’amorce (1/j)} & -- & -- & -- & -- &
\textbf{141} &
\textbf{96\,703~€} \\
\bottomrule
\end{tabularx}   % ← OBLIGATOIRE : fermeture de tabularx

\caption{Synthèse pivotée — Codage, volumétrie et recettes projetées (HDJ Addictologie)}
\end{sidewaystable}

\clearpage
% ============================================================
% PROJECTIONS D’ACTIVITÉ ET RECETTES PRÉVISIONNELLES
% ============================================================

\subsection{Projections d’activité et recettes prévisionnelles}

\textbf{Hypothèses volumétriques et tarifaires (2025).}
\begin{itemize}
    \item 1, puis 3, puis 5 patients par jour ;
    \item 3 jours d’activité par semaine ;
    \item 47 semaines par an ;
    \item Tarifs GHS 2025 : 774~€ (évaluation), 701~€ (réduction des risques), 541~€ (sevrage) ;
    \item Recettes annuelles = somme pondérée des trois types d’actes selon leur fréquence moyenne observée.
\end{itemize}

\begin{table}[h!]
\centering
\renewcommand{\arraystretch}{1.22}
\rowcolors{2}{APHPsoft}{white}

\begin{tabular}{
p{5.5cm}
>{\centering\arraybackslash}p{2.5cm}
>{\centering\arraybackslash}p{2.5cm}
>{\centering\arraybackslash}p{2.5cm}
}
\toprule
\rowcolor{APHPsoft}
\textbf{Phase d’activité} &
\textbf{Patients/an} &
\textbf{Séances/an} &
\textbf{Recettes/an} \\
\midrule

Amorce (1/jour) &
141 &
141 &
96\,703~€ \\

Montée en charge (3/jour) &
423 &
423 &
289\,936~€ \\

Croisière (5/jour) &
705 &
705 &
483\,269~€ \\

\bottomrule
\end{tabular}

\caption{Projections d’activité et recettes prévisionnelles — HDJ Addictologie}
\end{table}

\clearpage


% ============================================================
% CONCLUSION
% ============================================================

\subsection{Conclusion}

L’HDJ addictologie constitue un dispositif structurant permettant une intervention précoce, intensive et coordonnée pour les patients présentant un trouble de l’usage d’alcool. Il améliore la sécurité du sevrage, la réduction des risques, la stabilité hépatique et limite les hospitalisations non programmées. Son positionnement transversal dans le GHU en fait un outil central de prévention secondaire à fort impact de santé publique.

% ============================================================
% APPELS D’ANNEXES ADDICTOLOGIE
% ============================================================

\subsection*{Annexes associées}
\begin{itemize}
  \item Annexe~\ref{annexe:addicto_activites} — Grille d’évaluation somatique–addictologique
  \item Annexe~\ref{annexe:addicto_tracabilite} — Grille des programmes de médiation (addictologie)
\end{itemize}

\clearpage


% ============================================================
% VALIDATION
% ============================================================

\begin{center}
\begin{tabular}{p{4cm} p{7cm} p{4cm}}
\toprule
\rowcolor{APHPsoft}
\textbf{Date d’envoi} & \textbf{Nom du relecteur} & \textbf{Date de validation} \\
\midrule
03/12/2025 & Pr V.\,Mallet & 08/12/2025 \\
03/12/2025 & Dr S.\,Bouam & 08/12/2025 \\
08/12/2025 & Dr M.\,Corouge & NA \\
08/12/2025 & Dr J.\,Nabarro & NA \\
08/12/2025 & Dr D.\,Karinthi & NA \\
NA          & Pr R.\,Coriat & NA \\
\bottomrule
\end{tabular}
\end{center}

\clearpage


\printbibliography[heading=subbibliography,title={Références}]
\end{refsection}

% ============================
% CHAPITRE 6: MICI
% ============================
\clearpage
\section{HDJ MICI}

\begin{refsection}
% ============================================================
% HÔPITAL DE JOUR — MICI (Maladies Inflammatoires Chroniques de l’Intestin)
% ============================================================

\setcounter{table}{0}
\setcounter{figure}{0}

\subsection{Rationnel médical}
\needspace{8\baselineskip}

\begin{spacing}{1.30}

Les maladies inflammatoires chroniques de l’intestin (MICI) — maladie de Crohn et rectocolite hémorragique — sont des affections évoluant par poussées inflammatoires, responsables d’un retentissement fonctionnel important, d’une altération de la qualité de vie et d’un risque de complications sévères (abcès, sténoses, fistules, colites aiguës graves), avec risques de sepsis et de recours urgent à la chirurgie.

En France, les MICI concernent près de 300\,000 personnes, avec une incidence et une prévalence en croissance régulière de 3 à 4~\% par an.\cite{Ng_LancetGlobalIBD_2017,Richard_EpidemioPresseMed_2025} L’Europe compte parmi les régions les plus touchées, avec une prévalence avoisinant 0.5~\% de la population adulte. L’incidence augmente de façon marquée dans la population pédiatrique.

Un traitement précoce et adapté — notamment par biothérapie — réduit significativement le risque de complications, d’hospitalisations et de recours à la chirurgie, tout en améliorant la qualité de vie. Les biothérapies (anti-TNF, vedolizumab, ustekinumab, risankizumab) et traitements ciblés ont transformé le pronostic, mais nécessitent une organisation rigoureuse : perfusions intraveineuses, initiations sous-cutanées encadrées, bilans pré-thérapeutiques, éducation thérapeutique, surveillance de tolérance et prise en charge coordonnée des comorbidités (anémie, nutrition, santé mentale).\cite{HAS_AntiTNF_2019}

Dans le cadre du socle organisationnel des HDJ digestifs, le HDJ MICI permet de concentrer en une séquence unique: perfusion de biothérapie, surveillance infirmière MICI, évaluation clinique inter-cure, éducation thérapeutique, vaccinations, coordination ville–hôpital et planification thérapeutique individualisée. Ce format constitue l’organisation recommandée pour sécuriser et optimiser l’utilisation des biothérapies dans les MICI. Le dispositif offre également un cadre privilégié pour identifier les patients éligibles aux essais cliniques et aux cohortes observationnelles, en cohérence avec les missions universitaires du GHU.

À Cochin, le recrutement provient des correspondants médicaux, des services d’urgences, de la pédiatrie et de la filière de transition organisée avec Necker. 

\end{spacing}

\clearpage

% ============================================================
% OBJECTIFS
% ============================================================

\subsection{Objectifs}
\needspace{6\baselineskip}

\vspace{0.8em}
\begin{center}
\fcolorbox{APHPdark}{APHPsoft}{
\begin{minipage}{0.95\textwidth}
\vspace{0.9em}

\begin{itemize}[leftmargin=1.1cm]
  \item assurer un accès sécurisé aux biothérapies intraveineuses ;
  \item organiser les évaluations inter-cures ;
  \item proposer une prise en charge pluridisciplinaire : IDE MICI, diététique, psychologie, ETP, vaccinations ;
  \item réduire les hospitalisations complètes non programmées et optimiser le suivi treat-to-target ;
  \item structurer la coordination ville–hôpital et la traçabilité des décisions (consultation, RCP MICI) ;
  \item participer au repérage des patients éligibles à la recherche clinique (cohortes, essais thérapeutiques).
\end{itemize}

\vspace{0.9em}
\end{minipage}}
\end{center}

\bigskip

% ============================================================
% POPULATION ÉLIGIBLE
% ============================================================

\subsection{Population éligible}
\needspace{6\baselineskip}

\begin{itemize}[leftmargin=1.1cm]
  \item maladie de Crohn ou rectocolite hémorragique relevant d’une biothérapie IV ;
  \item première injection SC (infliximab, anti-TNF) nécessitant apprentissage, supervision et surveillance dédiée ;
  \item perfusion de fer injectable dans le cadre d’une anémie ferriprive associée aux MICI ;
  \item réévaluation inter-cure incluant ≥3 interventions (gastroentérologue, infectiologue, psychologue, IDE MICI, diététique, ETP) ;
  \item patients adressés en RCP MICI pour initiation, switch ou optimisation de biothérapie ;
  \item parcours de transition pédiatrie (Necker) → adulte (Cochin).
\end{itemize}

\clearpage

% ============================================================
% PARCOURS DE SOINS
% ============================================================

\subsection{Parcours de soins}
\needspace{8\baselineskip}

\begin{figure}[!ht]
\centering
\caption{Parcours patient — HDJ MICI}
\vspace{0.8cm}

\begin{tikzpicture}[
    node distance=1.6cm,
    box/.style={
        rectangle,
        rounded corners=3pt,
        draw=APHPdark,
        thick,
        text width=9.0cm,
        minimum height=1.55cm,
        align=center,
        fill=APHPsoft
    }
]

\node[box] (tri) {Orientation vers l’HDJ \\ 
\small Consultations MICI, RCP, médecine de ville, urgences, transition NCK→CCH};

\node[box, below=1.4cm of tri] (etape1) {Évaluation initiale IDE MICI \\ 
\small Bilans pré-thérapeutiques, critères de \\sécurité, calprotectine, dépistage comorbidités};

\node[box, below=1.4cm of etape1] (etape2) {Acte thérapeutique programmé \\ 
\small perfusion biothérapie IV/initiation SC encadrée/fer injectable};

\node[box, below=1.4cm of etape2] (etape3) {Surveillance clinique et éducative \\ 
\small tolérance, observance, statut\\vaccinal, conseils pratiques, ETP MICI};

\node[box, below=1.4cm of etape3] (synth) {Synthèse médicale et coordination \\ 
\small évaluation inter-cure, ajustement thérapeutique, lien ville–hôpital};

\draw[->, thick, APHPdark] (tri) -- (etape1);
\draw[->, thick, APHPdark] (etape1) -- (etape2);
\draw[->, thick, APHPdark] (etape2) -- (etape3);
\draw[->, thick, APHPdark] (etape3) -- (synth);

\end{tikzpicture}
\end{figure}

\clearpage

% ============================================================
% Organisation et ressources nécessaires
% ============================================================

\subsection{Organisation}
\needspace{5\baselineskip}

\begin{itemize}[leftmargin=1.1cm]
    \item Direction : \textbf{Docteur V.\,Abitbol}
    \item Durée : 6--8~heures.
    \item Lieu : Secteur HDJ — Service de gastroentérologie et d’oncologie digestive.
    \item Ressources : médecin senior, infirmier expert/IPA, psychologue/addictologue.
\end{itemize}
    
\clearpage

% ============================================================
% CODAGE, VOLUMÉTRIE ET RECETTES — HDJ MICI
% ============================================================

\subsection{Codage, volumétrie et recettes — synthèse opérationnelle}
\needspace{8\baselineskip}

\noindent
Activité 2024 : \textbf{1\,346 séances}, majoritairement dédiées aux biothérapies IV.  
Capacité actuelle : \textbf{8 fauteuils} — \textbf{2 patients / fauteuil / jour} — 
\textbf{2 jours / semaine} (≈ \textbf{32 patients / semaine}).

\begin{table}[h!]
\centering
\renewcommand{\arraystretch}{1.22}
\rowcolors{2}{APHPsoft}{white}

\begin{tabularx}{\textwidth}{
p{5.2cm}
X
>{\centering\arraybackslash}p{2.1cm}
>{\centering\arraybackslash}p{1.9cm}
>{\centering\arraybackslash}p{3.1cm}
}
\toprule
\rowcolor{APHPsoft}
\textbf{Séance HDJ MICI} &
\textbf{Actes / codage principal} &
\textbf{GHM / GHS} &
\textbf{Tarif} &
\textbf{Volume / Recette 2024} \\
\midrule

\textbf{Biothérapie IV (entretien)} &
Perfusion IV \newline
DP : Z51.2 ; DR : K50.x / K51.x &
28Z17Z / 9616 &
440~€ &
\textbf{1\,333} \newline
\textbf{586\,520~€} \\

\textbf{Initiation SC (ETP)} &
Injection SC, ETP \newline
DP : K50.x / K51.x ; DAS : Z71.9 &
06M07T / 2152 &
655~€ &
0 \newline
0~€ \\

\textbf{Évaluation inter-cure ≥4 actes} &
Consultation + nutrition + psy + biologie \newline
DP : Z09.2 ; DR : MICI &
06M16Z / 2186 &
1\,071~€ &
12 \newline
12\,852~€ \\

\textbf{Évaluation inter-cure =3 actes} &
Consultation + évaluations ciblées \newline
DP : Z09.2 ; DR : MICI &
06M16Z / 5059 &
360~€ &
1 \newline
360~€ \\

\midrule
\textbf{Total HDJ MICI} &
— &
— &
— &
\textbf{1\,346 séances} \newline
\textbf{\textasciitilde 600\,000~€} \\
\bottomrule
\end{tabularx}

\caption{Synthèse opérationnelle — HDJ MICI : structure d’activité, codage PMSI, volumétrie et recettes (2024).}
\end{table}

\clearpage

% --------------------------------------------------------
% TABLEAU 2 — Traçabilité
% --------------------------------------------------------

\subsection{Traçabilité minimale}

\begin{table}[h!]
\centering
\renewcommand{\arraystretch}{1.25}
\rowcolors{2}{APHPsoft}{white}

\begin{tabular}{p{5cm} p{9cm}}
\toprule
\rowcolor{APHPsoft}
\textbf{Intervention} & \textbf{Traçabilité requise} \\
\midrule

Biothérapie IV 
& protocole et dose ; voie d’administration ; surveillance per-cure ; effets indésirables ; décision de poursuite \\

1\textsuperscript{re} injection SC (infliximab/anti-TNF) 
& éducation à l’auto-injection ; vérification des pré-requis de sécurité ; tolérance immédiate ; planification des injections suivantes \\

Évaluation infirmière MICI 
& paramètres cliniques ; bilans réalisés ; statut vaccinal ; observance déclarée ; comorbidités associées \\

Biologie de suivi 
& indication ; résultats clés ; interprétation clinique ; conduite à tenir (adaptation thérapeutique) \\

Consultation diététique / psychologique 
& synthèse des évaluations ; recommandations ; objectifs fixés ; suivi prévu \\

Synthèse médicale inter-cure 
& évaluation réponse ; décisions thérapeutiques ; prochaine cure ; coordination ville–hôpital \\

Vaccinologie 
& statut vaccinal documenté ; rappels effectués (grippe, pneumocoque, covid, hépatites) ; recommandations données \\

Éligibilité recherche clinique 
& critères repérés ; information patient ; orientation vers essais / cohortes ; contact recherche \\

\bottomrule
\end{tabular}

\caption{Traçabilité des interventions}
\end{table}

\clearpage

% ============================================================
% PROJECTIONS D’ACTIVITÉ
% ============================================================

\subsection{Projections d’activité et recettes prévisionnelles}
\needspace{6\baselineskip}

\noindent\textbf{Hypothèses.}  
Base d’activité 2024 : \textbf{1\,333 HDJ}.  
Croissance progressive estimée : \textbf{+100 HDJ / an}.  
Tarif moyen pondéré calculé : \textbf{446~€ / séance}.

\medskip

\begin{table}[h!]
\centering

\renewcommand{\arraystretch}{1.22}
\rowcolors{2}{APHPsoft}{white}

\begin{tabular}{
l
>{\centering\arraybackslash}p{3cm}
>{\centering\arraybackslash}p{3cm}
>{\centering\arraybackslash}p{4cm}
}
\toprule
\rowcolor{APHPsoft}
\textbf{Phase} &
\textbf{Volume projeté} &
\textbf{Tarif moyen} &
\textbf{Recette brute estimée} \\
\midrule

Amorce &
1\,333 &
446~€ &
594\,518~€ \\

Montée en charge &
1\,433 &
446~€ &
639\,118~€ \\

Croisière &
1\,533 &
446~€ &
683\,718~€ \\
\bottomrule
\end{tabular}

\caption{Projections d’activité et recettes prévisionnelles — HDJ MICI}

\footnotetext{
\textbf{Calcul du tarif moyen pondéré (446~€).}  
Biothérapie IV : 1\,469 séances × 440~€ ;  
Évaluations ≥4 interventions : 14 séances × 1\,071~€ ;  
Évaluations =3 interventions : 2 séances × 360~€.  
Recette totale 2024 : 662\,074~€ pour 1\,485 HDJ  
→ Tarif moyen pondéré = \(662\,074 / 1\,485 = 445{,}9 \approx 446~€\).
}

\end{table}

\clearpage

% ============================================================
% CONCLUSION
% ============================================================

\subsection{Conclusion}
\needspace{6\baselineskip}

L’HDJ MICI constitue un dispositif structurant, assurant une administration sécurisée des biothérapies et une surveillance « treat-to-target » conforme aux recommandations internationales. Il permet une prise en charge intégrée — médicale, infirmière, éducative et psychosociale — adaptée à la complexité croissante des parcours MICI.
Dans un contexte d’augmentation continue de la prévalence des MICI, l’HDJ garantit efficience organisationnelle, continuité des soins et réduction des hospitalisations complètes non programmées. L’activité est en progression régulière depuis la création du HDJ MICI en 2019.

% ============================================================
% VALIDATION DE LA FICHE
% ============================================================

\begin{center}
\begin{tabular}{p{4cm} p{7cm} p{4cm}}
\toprule
\rowcolor{APHPsoft}
\textbf{Date de relecture} & \textbf{Nom du relecteur} & \textbf{Date de validation} \\
\midrule
NA & Pr R.\,Coriat & NA \\
01/12/2025 & Pr V.\,Abitbol & 01/12/2025 \\
03/12/2025 & Dr S.\,Bouam & NA \\
01/12/2025 & Pr V.\,Mallet & 03/12/2026 \\
\bottomrule
\end{tabular}
\end{center}

\clearpage

\printbibliography[heading=subbibliography,title={Références}]
\end{refsection}


% ========================================
% CHAPITRE 7: Chimiothérapies des Cancers
% ========================================
\clearpage
\section{HDJ Chimiothérapie des Cancers Digestifs}

\begin{refsection}
% ============================================================
% HÔPITAL DE JOUR — CHIMIOTHÉRAPIE
% ============================================================

\setcounter{table}{0} % Réinitialisation des tableaux
\setcounter{figure}{0} % Réinitialisation des tableaux

% ============================================================
% RATIONNEL MÉDICAL
% ============================================================

\subsection{Rationnel médical}
\needspace{8\baselineskip}
\begin{spacing}{1.30}

La chimiothérapie intraveineuse des cancers digestifs est aujourd’hui majoritairement administrée en HDJ, conformément aux référentiels organisationnels nationaux. Le document de l’AP–HP dédié à la sécurisation des chimiothérapies injectables et le référentiel organisationnel de l’INCa définissent l’HDJ comme la modalité privilégiée d’administration des anticancéreux en raison de la structuration, de la coordination interprofessionnelle et de la maîtrise du risque iatrogène \cite{APHP2020, ONC2021, INCa2025_SECUMED}. 

Ce modèle permet la mise en œuvre sécurisée des principaux protocoles utilisés en oncologie digestive (FOLFOX, FOLFIRI, FOLFIRINOX, CAPOX, gemcitabine–platine, immunothérapies, thérapies ciblées injectables) en intégrant une évaluation pré-cycle systématique, le contrôle des toxicités précédentes, la conciliation médicamenteuse et la traçabilité complète du processus de préparation, délivrance et administration.

La sélection des patients repose sur la stabilité clinique (ECOG~0–2), l’absence de complication aiguë (sepsis, déshydratation, occlusion), la faisabilité thérapeutique en ambulatoire et des conditions sociales compatibles. Les pompes externes pour les perfusions prolongées et la prophylaxie antiémétique/hématologique renforcent la sécurité du parcours.

Les données internationales confirment que l’administration ambulatoire des chimiothérapies digestives est sûre lorsque les filières sont structurées. Les taux de recours non programmés (urgences ou réhospitalisations) varient entre 4 et 20\,\%, selon les protocoles, tout en restant maîtrisés dans les équipes spécialisées \cite{Prince2019}. 

L’HDJ de chimiothérapie digestive assure ainsi une continuité thérapeutique optimale, réduit les hospitalisations conventionnelles, améliore l’expérience patient et garantit un haut niveau de qualité et de sécurité grâce à l’expertise pluridisciplinaire et aux circuits d’escalade en cas de complication.

\end{spacing}

\clearpage

% ============================================================
% OBJECTIFS
% ============================================================

\subsection{Objectifs}
\needspace{6\baselineskip}

\begin{center}
\fcolorbox{APHPdark}{APHPsoft}{
\begin{minipage}{0.95\textwidth}
\vspace{0.9em}
\begin{itemize}[leftmargin=1.1cm]
\item sécuriser l’administration ambulatoire des chimiothérapies injectables ;
\item standardiser l’évaluation pré-cycle et la surveillance immédiate ;
\item réduire les hospitalisations non programmées liées aux toxicités aiguës ;
\item optimiser l’organisation du parcours et la coordination ville–hôpital ;
\item améliorer l’expérience patient et la continuité des soins.
\end{itemize}
\vspace{0.9em}
\end{minipage}}
\end{center}

\clearpage

% ============================================================
% POPULATION ÉLIGIBLE
% ============================================================

\subsection{Population éligible}
\needspace{6\baselineskip}

\begin{itemize}[leftmargin=1.1cm]
\item état général compatible (ECOG~0–2), stabilité clinique ;
\item absence de complication aiguë : fièvre, déshydratation, syndrome tumoral évolutif ;
\item biologie conforme aux seuils décisionnels du protocole ;
\item protocole réalisable en ambulatoire (durée compatible, absence de surveillance prolongée) ;
\item autonomie suffisante et présence d’un accompagnant pour le retour.
\end{itemize}

\clearpage

% ============================================================
% PARCOURS DE SOINS
% ============================================================

\subsection{Parcours de soins}
\needspace{8\baselineskip}

\begin{figure}[!ht]
\centering
\caption{Parcours patient — HDJ Chimiothérapie}
\vspace{0.8cm}

\begin{tikzpicture}[
node distance=1.6cm,
box/.style={
    rectangle, rounded corners=3pt,
    draw=APHPdark, thick,
    text width=9cm,
    minimum height=1.5cm,
    align=center,
    fill=APHPsoft
}]

\node[box] (tri) {Orientation (oncologie médicale / hématologie / ville)};
\node[box, below=1.4cm of tri] (e1) {Évaluation pré-cycle : examen clinique, biologie, consentement, toxicités antérieures};
\node[box, below=1.4cm of e1] (e2) {Administration : prémédication, préparation pharmaceutique, perfusion supervisée};
\node[box, below=1.4cm of e2] (e3) {Surveillance immédiate : constantes, hypersensibilité, gestion des NV};
\node[box, below=1.4cm of e3] (syn) {Synthèse médicale, éducation thérapeutique, planification du cycle suivant};

\draw[->, thick, APHPdark] (tri) -- (e1);
\draw[->, thick, APHPdark] (e1) -- (e2);
\draw[->, thick, APHPdark] (e2) -- (e3);
\draw[->, thick, APHPdark] (e3) -- (syn);

\end{tikzpicture}
\end{figure}

\clearpage

% ============================================================
% ORGANISATION
% ============================================================

\subsection{Organisation}
\needspace{5\baselineskip}

\begin{itemize}[leftmargin=1.1cm]
\item Direction : \textbf{Dr Anna Pellat}
\item Durée : 6--8~heures
\item Lieu : HDJ — Gastroentérologie et oncologie digestive
\item Ressources : médecin sénior, infirmier expert/IPA, psychologue/nutritionniste
\end{itemize}

\clearpage

% ============================================================
% CODAGE, TARIFS ET VOLUMÉTRIE
% ============================================================

\subsection{Codage, tarifs et volumétrie de référence}
\needspace{10\baselineskip}

\begin{sidewaystable}[p]

\centering
\renewcommand{\arraystretch}{1.25}
\rowcolors{2}{APHPsoft}{white}

\begin{tabularx}{\textwidth}{
p{4.8cm}
>{\raggedright\arraybackslash}X
>{\centering\arraybackslash}p{1.7cm}
>{\centering\arraybackslash}p{1.6cm}
>{\centering\arraybackslash}p{1.9cm}
>{\centering\arraybackslash}p{1.8cm}
>{\centering\arraybackslash}p{2.4cm}
}
\toprule
\rowcolor{APHPsoft}
\textbf{Type de séance} &
\textbf{DP / DAS (cancers digestifs)} &
\textbf{GHM} &
\textbf{GHS} &
\textbf{Tarif 2025} &
\textbf{Volume 2024} &
\textbf{Recette 2024} \\
\midrule

Chimiothérapie cytotoxique IV &
DP : C22.0 (CHC), C22.1 (cholangioK), C25.x (pancréas), C17–C21 (intestin), C16.x (œsogastre)\newline
DAS : comorbidités, dénutrition, douleur, thrombose, ascite &
28Z07Z & 9616 & 495~€ &
XXX & ZZZZ~€ \\

Immunothérapie (anti-PD1/PD-L1) &
DP : C22.x, C25.x, C17–C21\newline
DAS : toxicités immunes (colite, hépatite, endocrinopathies) &
28Z17Z & 9616 & 440~€ &
XXX & ZZZZ~€ \\

Thérapies ciblées IV / Anticorps monoclonaux &
DP : C17–C21, C22.x\newline
DAS : mutation RAS/BRAF, métastases, complications locales &
28Z17Z & 9616 & 440~€ &
XXX & ZZZZ~€ \\

\midrule
\textbf{Total annuel} & -- & -- & -- & -- &
\textbf{568} & \textbf{322\,400~€} \\
\bottomrule
\end{tabularx}

\caption{Codage, tarifs et volumétrie — HDJ Chimiothérapie digestive (2024)}
\end{sidewaystable}

\clearpage

% ============================================================
% TRACABILITÉ
% ============================================================

\subsection{Traçabilité minimale}
\needspace{8\baselineskip}

\begin{table}[h!]
\centering
\renewcommand{\arraystretch}{1.25}
\rowcolors{2}{APHPsoft}{white}

\begin{tabular}{p{4.3cm} p{8.2cm}}
\toprule
\rowcolor{APHPsoft}
\textbf{Intervention} & \textbf{Éléments requis} \\
\midrule

Évaluation pré-cycle &
Examen clinique structuré, biologie validée, note des toxicités CTCAE, décision de poursuite/modification, conciliation médicamenteuse. \\

Préparation et administration &
Vérification des identités, protocole, doses, lots, double contrôle IDE/pharmacie, prémédications, durée d’infusion, incidents éventuels. \\

Surveillance immédiate &
Constantes avant/pendant/après perfusion, dépistage des réactions d’hypersensibilité, gestion des nausées/vomissements, tolérance clinique. \\

Voie veineuse / PAC &
État du site, perméabilité, incidents locaux, hémostase, retrait/flush selon protocole. \\

Synthèse médicale &
Tolérance du cycle, points de vigilance, adaptations posologiques, recommandations personnalisées, planification du cycle suivant, messages ville–hôpital. \\
\bottomrule
\end{tabular}

\caption{Traçabilité — HDJ Chimiothérapie digestive}
\end{table}

\clearpage


% ============================================================
% PROJECTIONS D’ACTIVITÉ — CHIMIOTHÉRAPIE DIGESTIVE
% ============================================================

\subsection{Projections d’activité et recettes prévisionnelles}

\noindent Référence 2024 : \textbf{568 séances}, soit \textbf{352\,160~€} (tarif moyen \textbf{620~€}). \\
Hypothèse de croissance : \textbf{+50 séances / an}. \\
Tarif moyen stabilisé : \textbf{620~€ / séance}.

\begin{table}[h!]
\centering
\renewcommand{\arraystretch}{1.20}
\rowcolors{2}{APHPsoft}{white}
\begin{tabular}{
p{4.3cm}
>{\centering\arraybackslash}p{2.2cm}
>{\centering\arraybackslash}p{2.3cm}
>{\centering\arraybackslash}p{3.0cm}
}
\toprule
\rowcolor{APHPsoft}
\textbf{Phase} & \textbf{Volume estimé} & \textbf{Tarif moyen} & \textbf{Recette brute} \\
\midrule
Amorce        & 568 & 620~€ & 352\,160~€ \\
Montée        & 618 & 620~€ & 383\,160~€ \\
Croisière     & 668 & 620~€ & 413\,160~€ \\
\bottomrule
\end{tabular}
\caption{Projections d’activité et recettes prévisionnelles — HDJ Chimiothérapie digestive}
\end{table}

\clearpage




% ============================================================
% CONCLUSION
% ============================================================

\subsection{Conclusion}

L’HDJ de chimiothérapie constitue un pilier de l’organisation oncologique moderne. Il permet une administration sécurisée et standardisée des traitements anticancéreux, réduit l’hospitalisation complète et optimise l’expérience patient. La structuration du parcours — évaluation pré-cycle, gestion des toxicités, surveillance immédiate et coordination ville–hôpital — garantit un haut niveau de qualité et d’efficience.

\clearpage

\printbibliography[heading=subbibliography,title={Références}]
\end{refsection}

% ==========================================
% CHAPITRE 8: Radiologie Interventionnelles
% ==========================================
\clearpage
\section{HDJ Radiologie Interventionnelle}

\begin{refsection}
% ============================================================
% HÔPITAL DE JOUR — RADIOLOGIE INTERVENTIONNELLE
% ============================================================

\subsection{Rationnel médical}
\setcounter{table}{0}
\needspace{8\baselineskip}

\begin{spacing}{1.30}

L’essor des techniques mini-invasives permet la réalisation d’un nombre croissant de procédures interventionnelles en ambulatoire, tout en maintenant des standards élevés de sécurité. Les recommandations de la SFR, de la SFICV et du CIRSE soulignent que l’organisation en hôpital de jour optimise l’efficience, réduit les durées d’hospitalisation et améliore l’expérience patient \cite{Lakshminarayan2023, cirse2021}.

La sélection pré-procédure repose sur la stabilité clinique, la maîtrise du risque hémorragique et l’existence de conditions sociales compatibles avec un retour sécurisé à domicile. Les actes les plus courants incluent les biopsies percutanées, drainages, thermoablations, gestes vasculaires non artériels et interventions sur cathéters.

Une surveillance immédiate (1–3~h) permet d’identifier les complications précoces, principalement douleur et saignement. Les données disponibles montrent que cette stratégie ne majore ni les réadmissions ni les recours non programmés \cite{Chen2017}. La sortie est conditionnée à la stabilité clinique, à l’absence de saignement, et à la bonne compréhension des consignes.

\end{spacing}

\clearpage

% ============================================================
% OBJECTIFS
% ============================================================

\subsection{Objectifs}
\needspace{6\baselineskip}

\begin{center}
\fcolorbox{APHPdark}{APHPsoft}{
\begin{minipage}{0.95\textwidth}
\vspace{0.9em}
\begin{itemize}[leftmargin=1.1cm]
\item sécuriser la réalisation ambulatoire des actes interventionnels mini-invasifs ;
\item structurer une filière courte, standardisée et coordonnée ;
\item réduire les séjours conventionnels et optimiser le plateau technique ;
\item harmoniser la sélection des patients, la préparation et la surveillance post-geste ;
\item améliorer l’expérience patient grâce à un parcours fluide et une information claire.
\end{itemize}
\vspace{0.9em}
\end{minipage}}
\end{center}

\clearpage

% ============================================================
% POPULATION ÉLIGIBLE
% ============================================================

\subsection{Population éligible}
\needspace{6\baselineskip}

\begin{itemize}[leftmargin=1.1cm]
\item patients ASA~I–II (ASA~III après avis spécialisé) ;
\item absence de comorbidité décompensée (cardiaque, respiratoire, rénale) ;
\item bilan d’hémostase compatible avec le geste ;
\item accompagnant obligatoire pour le retour et la première nuit ;
\item compréhension et adhésion aux consignes post-procédure.
\end{itemize}

\clearpage

% ============================================================
% PARCOURS DE SOINS
% ============================================================

\subsection{Parcours de soins}
\needspace{8\baselineskip}

\begin{figure}[!ht]
\centering
\caption{Parcours patient — HDJ Radiologie interventionnelle}
\vspace{0.8cm}

\begin{tikzpicture}[
node distance=1.6cm,
box/.style={
    rectangle, rounded corners=3pt,
    draw=APHPdark, thick,
    text width=9cm,
    minimum height=1.5cm,
    align=center,
    fill=APHPsoft
}]
\node[box] (tri) {Orientation (consultation spécialisée, demande ciblée, ville)};
\node[box, below=1.4cm of tri] (e1) {Évaluation pré-procédure : anamnèse, traitements, hémostase, consentement};
\node[box, below=1.4cm of e1] (e2) {Acte interventionnel : biopsie, drainage, thermoablation, geste vasculaire};
\node[box, below=1.4cm of e2] (e3) {Surveillance 1–3~h : constantes, douleur, point de ponction};
\node[box, below=1.4cm of e3] (syn) {Synthèse médicale, consignes écrites, sortie sécurisée};

\draw[->, thick, APHPdark] (tri) -- (e1);
\draw[->, thick, APHPdark] (e1) -- (e2);
\draw[->, thick, APHPdark] (e2) -- (e3);
\draw[->, thick, APHPdark] (e3) -- (syn);

\end{tikzpicture}
\end{figure}

\clearpage

% ============================================================
% ORGANISATION
% ============================================================

\subsection{Organisation}
\needspace{6\baselineskip}

\begin{itemize}[leftmargin=1.1cm]
\item Direction : \textbf{Pr Anthony Dohan}
\item Durée moyenne : 3~heures
\item Lieu : secteur HDJ — hépato-gastroentérologie / oncologie digestive
\item Ressources : médecin sénior, infirmier expert
\end{itemize}

\clearpage

% ============================================================
% CODAGE ET GHS ASSOCIÉS
% ============================================================

\subsection{Codage et GHS associés}
\needspace{8\baselineskip}

\noindent\textbf{Cadre général.}  
Codage basé sur le DP correspondant à la lésion explorée ou traitée, complété par les DAS pertinents.  
Actes relevant de la CCAM d’imagerie interventionnelle.

\medskip

\begin{table}[h!]
\centering
\renewcommand{\arraystretch}{1.20}
\rowcolors{2}{APHPsoft}{white}

\begin{tabularx}{\textwidth}{
>{\raggedright\arraybackslash}p{4.5cm}
X
>{\centering\arraybackslash}p{1.6cm}
>{\centering\arraybackslash}p{1.6cm}
>{\centering\arraybackslash}p{2cm}}
\toprule
\rowcolor{APHPsoft}
\textbf{Type de séance} &
\textbf{DP / DAS} &
\textbf{GHM} &
\textbf{GHS} &
\textbf{Tarif 2025} \\
\midrule

Biopsie percutanée &
DP : selon organe (Cxx / Rxx) \newline DAS : selon contexte &
07M06T &
2528 &
1\,061~€ \\

Drainage percutané &
DP : K65, K83, N13 \newline DAS : infection, douleur &
07M02T &
2518 &
891~€ \\

\bottomrule
\end{tabularx}

\caption{Codage et GHS associés — HDJ Radiologie interventionnelle}
\end{table}

\clearpage

% ============================================================
% TRACABILITÉ
% ============================================================

\subsection{Traçabilité minimale}

\begin{table}[h!]
\centering
\renewcommand{\arraystretch}{1.20}
\rowcolors{2}{APHPsoft}{white}

\begin{tabular}{p{5cm} p{9cm}}
\toprule
\rowcolor{APHPsoft}
\textbf{Intervention} & \textbf{Éléments requis} \\
\midrule

Biopsie &
Indication ; imagerie préalable ; type d’aiguille ; nombre de prélèvements ; incidents ; contrôle post-geste. \\

Drainage &
Site ; guidage ; volume, aspect ; prélèvements ; mise en tension ; contrôle immédiat. \\

Actes vasculaires &
Voie d’abord ; matériel ; perméabilité post-geste ; incidents. \\

Synthèse de sortie &
Compte-rendu structuré ; mesures de sécurité ; consignes écrites ; suivi programmé. \\
\bottomrule
\end{tabular}

\caption{Traçabilité — HDJ Radiologie interventionnelle}
\end{table}

\clearpage

% ============================================================
% VOLUMÉTRIE DE RÉFÉRENCE
% ============================================================

\subsection{Volumétrie de référence (année N)}

\noindent Montée en charge progressive : 6 à 10 patients/jour.

\begin{center}
\begin{tabular}{lccc}
\toprule
\textbf{Type de séance} & \textbf{Volume N} & \textbf{Tarif unitaire} & \textbf{Recette estimée} \\
\midrule
Biopsies & XXX & 650~€ & ZZZZ~€ \\
Drainages & XXX & 780~€ & ZZZZ~€ \\
\midrule
\textbf{Total} & \textbf{XXX} & -- & \textbf{ZZZZ~€} \\
\bottomrule
\end{tabular}
\end{center}

\clearpage

% ============================================================
% PROJECTIONS
% ============================================================

\subsection{Projections d’activité et recettes prévisionnelles}

\noindent Hypothèse : transfert progressif des gestes vers l’ambulatoire.

\begin{center}
\begin{tabular}{lccc}
\toprule
\textbf{Année} & \textbf{Volume} & \textbf{Tarif moyen} & \textbf{Recette brute} \\
\midrule
Amorce & XXX & 850~€ & ZZZZ~€ \\
Montée & XXX & 850~€ & ZZZZ~€ \\
Croisière & XXX & 850~€ & ZZZZ~€ \\
\bottomrule
\end{tabular}
\end{center}

\clearpage

% ============================================================
% VALIDATION
% ============================================================

\begin{center}
\begin{tabular}{p{4cm} p{7cm} p{4cm}}
\toprule
\rowcolor{APHPsoft}
\textbf{Date d'envoi} & \textbf{Relecteur} & \textbf{Validation} \\
\midrule
01/12/2025 & Pr V.\,Mallet & 04/12/2026 \\
04/12/2025 & Pr A.\,Dohan  & NA \\
NA          & Dr S.\,Bouam & NA \\
NA          & Pr R.\,Coriat & NA \\
\bottomrule
\end{tabular}
\end{center}

\clearpage

\printbibliography[heading=subbibliography,title={Références}]
\end{refsection}

% ==========================================
% CHAPITRE X: Conclusion Générale
% ==========================================
\clearpage
\section{Conclusion Générale}

\begin{refsection}
% ============================================================
% CONCLUSION GÉNÉRALE
% ============================================================

\section*{Conclusion générale}
\addcontentsline{toc}{section}{Conclusion générale}

\begin{spacing}{1.25}

La consolidation des activités d’hôpital de jour digestif, interventionnel et addictologique au sein
d’un plateau unique constitue un levier structurant d’efficience pour le GHU AP--HP.Centre.
L’unification organisationnelle permet une augmentation substantielle de la capacité de prise en
charge, une amélioration mesurable de l’accessibilité aux parcours spécialisés et une réduction
documentée des hospitalisations conventionnelles évitables, dans un contexte durable de tension
capacitaire et de ressources humaines contraintes.

La spécificité du projet repose sur une organisation graduée des prises en charge ambulatoires,
distinguant trois niveaux de complexité et d’intensité de surveillance. Cette segmentation permet
d’allouer les ressources médicales et paramédicales au plus près des besoins médicaux réels,
tout en évitant un surdimensionnement en lits hospitaliers. Les prises en charge nécessitant une
surveillance continue sont concentrées sur des lits médicalisés, tandis que les traitements
intermédiaires sont réalisés sur des fauteuils dédiés.

Les parcours d’évaluation multidisciplinaire et de coordination — notamment en hépatologie
métabolique, cirrhose compensée, addictologie et certaines prises en charge MICI — sont organisés
au sein d’un espace non allongé de type \emph{lounge}. Inspiré des standards de confort et de
fluidité des salons premium, cet espace accueille des patients autonomes ne nécessitant ni
alitement ni surveillance continue, mais requérant un temps médical et paramédical structuré,
séquentiel et à forte valeur ajoutée. Cette organisation optimise les flux, améliore l’expérience
patient et constitue un levier majeur d’efficience capacitaire.

À régime de croisière, l’activité consolidée du plateau est estimée entre \textbf{4\,700 et 5\,000
séances d’hôpital de jour par an}, pour des \textbf{recettes annuelles proches de 3\,M€}. Ce modèle
positionne le site Cochin comme un centre expert régional de l’ambulatoire digestif, en cohérence
avec les orientations stratégiques de l’AP--HP, les recommandations des sociétés savantes et les
standards internationaux d’organisation des parcours complexes.

\end{spacing}
% ============================================================
% PRINCIPES DE CAPACITÉ ET NIVEAUX DE PRISE EN CHARGE
% ============================================================

\subsection*{Principe général de dimensionnement}

Le dimensionnement du plateau repose sur une distinction fonctionnelle entre trois niveaux
d’accueil ambulatoire, définis selon l’intensité de surveillance requise et la durée des prises en
charge. Les hypothèses retenues correspondent volontairement à une \textbf{phase de démarrage
prudente}, permettant une montée en charge ultérieure sans modification structurelle.

\begin{itemize}[leftmargin=1.1cm]
    \item \textbf{Niveau~1~: lits médicalisés} — surveillance continue, patient allongé.
    \item \textbf{Niveau~2~: fauteuils de soins} — surveillance infirmière directe, durée intermédiaire.
    \item \textbf{Niveau~3~: espace non allongé de type lounge} — patients autonomes, coordination
    multidisciplinaire sans besoin de lit.
\end{itemize}
% ============================================================
% TRADUCTION EN CAPACITÉ PHYSIQUE — HYPOTHÈSES CONSERVATRICES
% ============================================================

\subsection*{Traduction en capacité physique (hypothèses conservatrices)}

La traduction des hypothèses de flux en capacité physique repose sur des hypothèses
volontairement conservatrices, intégrant les aléas organisationnels (annulations, no-show),
les contraintes de ressources humaines, la variabilité des durées de prise en charge et les
indisponibilités ponctuelles des plateaux techniques.

Le plateau est supposé fonctionner sur une base de \textbf{44 semaines réellement opérées par an},
à raison de \textbf{5 jours par semaine}, soit \textbf{220 jours ouvrés/an}.

Les hypothèses de réalisation retenues sont :
\begin{itemize}[leftmargin=1.1cm]
    \item \textbf{Lits médicalisés} : 1 patient/lit/jour avec un coefficient de réalisation de \textbf{0.85}.
    \item \textbf{Fauteuils de soins} : 2 patients/fauteuil/jour avec un coefficient de réalisation de \textbf{0.80}.
    \item \textbf{Fauteuils lounge} : 2 patients/fauteuil/jour avec un coefficient de réalisation de \textbf{0.75}.
\end{itemize}

La capacité annuelle attendue est calculée selon la formule :
\[
HDJ/an = 220 \times \big(
N_L \times 1.0 \times 0.85
+ N_F \times 2.0 \times 0.80
+ N_{Lg} \times 2.0 \times 0.75
\big).
\]

Sur cette base, un dimensionnement conservateur compatible avec une cible de
\textbf{4\,700 à 5\,000 HDJ/an} est le suivant :
\begin{itemize}[leftmargin=1.1cm]
    \item \textbf{6 lits médicalisés},
    \item \textbf{6 fauteuils de soins},
    \item \textbf{5 fauteuils lounge} (espace non allongé de type \emph{salon premium}).
\end{itemize}

Ce calibrage correspond à une capacité annuelle théorique de :
\[
220 \times (6\times1.0\times0.85
+ 6\times2.0\times0.80
+ 5\times2.0\times0.75)
= \textbf{4\,884 HDJ/an}.
\]
% ============================================================
% TABLEAU SYNTHÉTIQUE — CAPACITÉ PHYSIQUE, ACTIVITÉ ET RECETTES
% ============================================================

\begin{sidewaystable}[p]
\centering
\renewcommand{\arraystretch}{1.25}
\rowcolors{2}{APHPsoft}{white}

\begin{tabular}{
p{4.6cm}
>{\centering\arraybackslash}p{1.6cm}
>{\centering\arraybackslash}p{3.0cm}
>{\centering\arraybackslash}p{3.0cm}
>{\centering\arraybackslash}p{3.4cm}
>{\centering\arraybackslash}p{3.4cm}
}
\toprule
\textbf{Type de capacité} &
\textbf{N (unités)} &
\textbf{Hypothèse de flux (conservatrice)} &
\textbf{Capacité annuelle estimée} &
\textbf{Activités principales concernées} &
\textbf{Recettes annuelles estimées \newline (ensemble du sous-plateau)} \\
\midrule
Lits médicalisés &
\textbf{6} &
1 patient/lit/jour $\times$ 0.85 &
$\approx$ 1\,122 HDJ/an &
PBH, cirrhoses avancées, chimiothérapies digestives prolongées,
radiologie interventionnelle post-acte &
$\approx$ 1.5--1.7 M€ \\
Fauteuils de soins &
\textbf{6} &
2 patients/fauteuil/jour $\times$ 0.80 &
$\approx$ 2\,112 HDJ/an &
Fer IV, albumine seule, biothérapies courtes,
chimiothérapies simples &
$\approx$ 0.6--0.7 M€ \\
Fauteuils lounge (non allongé) &
\textbf{5} &
2 patients/fauteuil/jour $\times$ 0.75 &
$\approx$ 1\,650 HDJ/an &
Hépatométabolique, évaluation des cirrhoses,
addictologie, parcours multidisciplinaires &
$\approx$ 0.6--0.8 M€ \\
\midrule
\textbf{Total plateau HDJ} &
— &
— &
\textbf{$\approx$ 4\,884 HDJ/an} &
— &
\textbf{$\approx$ 3 M€} \\
\bottomrule
\end{tabular}

\caption{Dimensionnement physique du plateau d’hôpital de jour selon des hypothèses
volontairement conservatrices. Le \emph{lounge} constitue une capacité assise premium,
non assimilée à des lits, permettant l’optimisation des flux sans surdimensionnement
hospitalier.}
\end{sidewaystable}

\printbibliography[heading=subbibliography,title={Références}]
\end{refsection}


% ================================
% ANNEXES A — Hépatométabolique
% ================================
\clearpage
\appendix
\renewcommand{\thesection}{A.\arabic{section}}
\setcounter{section}{0}

\section{Grille échographique MASLD}
\label{sec:annexe_echo}
\addcontentsline{toc}{section}{Annexe A.1 — Grille échographique MASLD}
% --- MODE COMPACT POUR TOUT FAIRE TENIR SUR UNE PAGE ---
\small                                      % ↓ Réduction de la police
\renewcommand{\arraystretch}{0.92}          % ↓ Lignes plus compactes
\setlength{\tabcolsep}{3pt}                 % ↓ Colonnes plus serrées

\begin{center}
\begin{minipage}{0.95\textwidth}

\rowcolors{1}{}{APHPsoft!40}
\setlength{\arrayrulewidth}{0.25pt}

\begin{tabular}{p{5.7cm} p{7.6cm}}

\multicolumn{2}{l}{\textbf{1. Morphologie hépatique}} \\
\hline
& \ch Taille normale \\
& \ch Hépatomégalie \\
& \ch Contours irréguliers \\
& \ch Nodularité \\
\\[-2mm]

\multicolumn{2}{l}{\textbf{2. Texture parenchymateuse}} \\
\hline
& \ch Homogène \\
& \ch Hétérogène \\
& \ch Réflectivité augmentée \\
\\[-2mm]

\multicolumn{2}{l}{\textbf{3. Stéatose (échogénicité)}} \\
\hline
& \ch Grade 0 \\
& \ch Grade 1 \\
& \ch Grade 2 \\
& \ch Grade 3 \\
\\[-2mm]

\multicolumn{2}{l}{\textbf{4. Atténuation du faisceau}} \\
\hline
& \ch Normale \\
& \ch Accrue \\
\\[-2mm]

\multicolumn{2}{l}{\textbf{5. Visualisation vasculaire}} \\
\hline
& \ch Normale \\
& \ch Dégradée \\
\\[-2mm]

\multicolumn{2}{l}{\textbf{6. Doppler portale}} \\
\hline
& \ch Hépatopète \\
& \ch Diminué \\
& \ch Hépatofuge \\
& Vitesse : \rule{2.1cm}{0.4pt} cm/s \\
\\[-2mm]

\multicolumn{2}{l}{\textbf{7. Veines hépatiques}} \\
\hline
& \ch Triphasiques \\
& \ch Aplatis \\
& \ch Monophasique \\
\\[-2mm]

\multicolumn{2}{l}{\textbf{8. Signes HTP}} \\
\hline
& \ch Splénomégalie \\
& \ch Ascite \\
& \ch Collatérales \\
\\[-2mm]

\multicolumn{2}{l}{\textbf{9. Diamètre porte}} \\
\hline
& \rule{2.1cm}{0.4pt} mm (N ≤ 12) \\
\\[-2mm]

\multicolumn{2}{l}{\textbf{10. Commentaires}} \\
\hline
\multicolumn{2}{l}{
\rule{0.94\textwidth}{1.6cm}
} \\

\end{tabular}

\vspace{1mm}
{\footnotesize\itshape
Réf. : EASL 2024 ; AFEF 2024 ; WFUMB/EFSUMB 2023 ; AASLD 2023.
}

\end{minipage}
\end{center}

\normalsize



\section{Grille diététique MASLD}
\label{sec:annexe_diete}
\addcontentsline{toc}{section}{Annexe A.2 — Grille diététique MASLD}
\section*{Annexe B — Grille diététique MASLD}

\begin{table}[h!]
\centering
\rowcolors{2}{white}{APHPsoft}
\begin{tabular}{|p{6cm}|p{8cm}|}
\hline
\rowcolor{APHPdark}\color{white}\textbf{Item} &
\color{white}\textbf{Évaluation} \\ \hline

\textbf{Paramètres anthropométriques} &
Poids : \rule{1.5cm}{0.4pt} kg \quad
IMC : \rule{1.5cm}{0.4pt} \\
& Tour de taille : \rule{2cm}{0.4pt} cm \\ \hline

\textbf{Apports alimentaires} &
\ch Repas structurés \\
& \ch Grignotages fréquents \\
& \ch Excès sucres rapides \\
& \ch Excès graisses saturées \\ \hline

\textbf{Consommations sucrées} &
\ch Sodas / jus \\
& \ch Produits sucrés quotidiens \\
& \ch Sucres cachés \\ \hline

\textbf{Alcool} &
\ch 0 g \quad
\ch Occasionnel \quad
\ch > seuil AFEF \\ \hline

\textbf{Activité physique} &
\ch <150 min/sem \\
& \ch 150–300 min/sem (objectif) \\ \hline

\textbf{Objectifs diététiques} &
\rule{8cm}{3cm} \\ \hline

\end{tabular}
\end{table}



\section{Grille psychologique / addictologique MASLD}
\label{sec:annexe_psy}
\addcontentsline{toc}{section}{Annexe A.3 — Grille psy / addictologie MASLD}
\section*{Annexe C — Grille psychologique / addictologique MASLD}

\begin{table}[h!]
\centering
\rowcolors{2}{white}{APHPsoft}
\begin{tabular}{|p{6cm}|p{8cm}|}
\hline
\rowcolor{APHPdark}\color{white}\textbf{Item} &
\color{white}\textbf{Évaluation} \\ \hline

\textbf{Alcool (AUDIT-C)} &
Score : \rule{1.5cm}{0.4pt} \\
& \ch Faible \quad \ch Modéré \quad \ch Élevé \\ \hline

\textbf{Troubles alimentaires} &
\ch Hyperphagie \\
& \ch Grignotage émotionnel \\
& \ch Restriction cognitive \\
& \ch TCA suspecté → orientation \\ \hline

\textbf{Sommeil / stress} &
\ch Troubles du sommeil \\
& \ch Stress chronique \\
& \ch Anxiété / coping limité \\ \hline

\textbf{Motivation au changement} &
Score 1–10 : \rule{2cm}{0.4pt} \\ \hline

\textbf{Objectifs psychocomportementaux} &
\rule{8cm}{3cm} \\ \hline

\end{tabular}
\end{table}



% ============================================
% ANNEXES B — Cirrhose avancée
% ============================================
\renewcommand{\thesection}{B.\arabic{section}}
\setcounter{section}{0}

% --- B.1 Ponction d’ascite + albumine ---
\section{Fiche HDJ — Ponction d’ascite + perfusion d’albumine}
\label{sec:annexe_ascite}
\addcontentsline{toc}{section}{Annexe B.1 — Ponction d’ascite + albumine}
% -------------------------------
% ANNEXE B.1 — Ponction d’ascite + albumine
% -------------------------------
\section*{Annexe B.1 — Ponction d’ascite avec perfusion d’albumine}
\addcontentsline{toc}{subsection}{Annexe B.1 — Ponction d’ascite avec perfusion d’albumine}
\needspace{10\baselineskip}

\begin{itemize}[leftmargin=1.1cm]
  \item \textbf{Ponction d’ascite (HPHB003).}  
  Acte sous asepsie stricte, évacuation adaptée à la fragilité du patient. Volume évacué, difficultés techniques et paramètres hémodynamiques.
  \item \textbf{Entretien médical individualisé.}  
  État clinique, Child-Pugh, MELD, critères AKI/HRS, indication d’albumine, note dédiée.
  \item \textbf{Surveillance particulière.}  
  Constantes rapprochées, monitorage neurologique, dépistage précoce des complications.
  \item \textbf{Acte infirmier.}  
  Pose de voie difficile, compression prolongée, surveillance locale.
  \item \textbf{Traçabilité.}  
  Fiche HPHB003, CCAM, note médicale, feuille de surveillance.
\end{itemize}

\clearpage


% --- B.2 Fer injectable ---
\section{Fiche HDJ — Fer injectable}
\label{sec:annexe_fer}
\addcontentsline{toc}{section}{Annexe B.2 — Fer injectable}
\section*{Annexe B.2 — Séance de fer injectable chez cirrhose avancée}
\addcontentsline{toc}{subsection}{Annexe B.2 — Séance de fer injectable}
\needspace{10\baselineskip}

\begin{itemize}[leftmargin=1.1cm]
  \item \textbf{Administration IV (type Ferinject\textsuperscript{\textregistered}).}
  Posologie adaptée, traçabilité du lot, protocole dédié.
  \item \textbf{Entretien médical.}
  Recherche de décompensation, bilan hémodynamique, adaptation thérapeutique.
  \item \textbf{Surveillance post-perfusion.}
  30–60 minutes, constantes, allergie, aggravation potentielle.
  \item \textbf{Acte infirmier technique.}
  Pose de voie, prévention extravasation, gestion des pansements.
  \item \textbf{Traçabilité.}
  Fiche perfusion, note médicale, grille post-perfusion.
\end{itemize}

\clearpage


% --- B.3 Transfusion de CGR ---
\section{Fiche HDJ — Transfusion de CGR}
\label{sec:annexe_cgr}
\addcontentsline{toc}{section}{Annexe B.3 — Transfusion de CGR}
\input{annexes/transfusion_sanguine}

% ================================
% ANNEXES C — Addictologie
% ================================
\clearpage
\renewcommand{\thesection}{C.\arabic{section}}
\setcounter{section}{0}

\section{Grille d’évaluation somatique–addictologique}
\addcontentsline{toc}{section}{Annexe C.1 — Grille d’évaluation somatique–addictologique}
\input{annexes/annexe_activites_addictologie}

\section{Grille des programmes de médiation}
\addcontentsline{toc}{section}{Annexe C.2 — Grille des médiations addictologie}
% ============================================================
% ANNEXE 2 — Fiche de traçabilité HDJ addictologie
% ============================================================

\section*{Annexe 2 — Fiche de traçabilité HDJ addictologie}

\begin{table}[h!]
\centering
\renewcommand{\arraystretch}{1.30}
\rowcolors{2}{white}{APHPsoft}

\begin{tabular}{p{5cm} p{10cm}}
\toprule
\rowcolor{APHPsoft}
\textbf{Élément} & \textbf{Contenu à renseigner} \\
\midrule

Identité du patient &
Nom, prénom, date de naissance, IPP \\

Date de la séance &
JJ/MM/AAAA \\

Motif de venue &
Évaluation / Réduction des risques / Sevrage ambulatoire / Consolidation / Synthèse \\

Intervenants présents &
Médecin (addicto/hépato), IDE, psychologue, diététicien, AS, éducateur spécialisé, psychomotricien, socio-esthéticienne, art-thérapeute, APA \\

Activités réalisées &
Lister les activités avec \textbf{numéros de l’annexe 1 (1–47)} + court descriptif :  
• médiations (3–4) : socio-esthétique, art-thérapie, écriture, revue de presse  
• APA (activité physique adaptée)  
• ateliers cuisine (10)  
• groupes (7, 12, 17, 20, 22, 30, 33…) \\

Évaluation clinique &
Anamnèse, CIWA, AUDIT, paramètres vitaux, consommation récente, risques, événements intercurrents \\

Évaluation psycho-sociale &
Logement, emploi, isolement, précarité, vulnérabilités, violences, ressources mobilisables \\

Suivi somatique &
Examens réalisés : biologie, ECG, EFR, imagerie, Fibroscan ; résultats pertinents pour la PEC \\

Décision médicale &
Plan thérapeutique actualisé, sevrage, adaptation des traitements, réduction des risques, orientation (CSAPA, ville, HDJ, SSR, hospitalisation) \\

Événements indésirables &
Incidents, aggravations, EI médicamenteux, rupture de contact, consommation lors du sevrage \\

Continuité des soins &
RDV programmés, coordination médecin traitant / CSAPA / psychologue / ville / SSR ; documents remis \\

Signature &
Intervenant(s) responsable(s) \\
\bottomrule
\end{tabular}

\caption{Fiche opérationnelle de traçabilité — HDJ addictologie (version enrichie)}
\end{table}


% ================================
% ANNEXES D — Addictologie
% ================================
\clearpage
\renewcommand{\thesection}{C.\arabic{section}}
\setcounter{section}{0}

\section{Fiche évaluation des cirrhoses}
\addcontentsline{toc}{section}{Annexe D.1 — Fiche évaluation des cirrhoses}
\input{annexes/evaluation_des_cirrhoses}


\end{document}

