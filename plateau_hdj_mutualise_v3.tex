% ============================================================
% 3. PLATEAU HDJ DIGESTIF MUTUALISÉ — ORGANISATION TRANSVERSALE
% ============================================================
% VERSION ÉDITÉE (PRÉSENTATION + LISIBILITÉ)
%
% Recommandations préambule (si non déjà présent) :
% \usepackage{booktabs}
% \usepackage{colortbl}
% \usepackage[table]{xcolor}
% \usepackage{array}
% \usepackage{enumitem}
% \usepackage{setspace}
% \usepackage{amsmath}
% \usepackage{rotating}
% \usepackage{caption}
% \usepackage{tikz}
% \usepackage{placeins} % \FloatBarrier
%
% Suggestion typographie (éviter run-in sur \subsection) :
% \titleformat{\subsection}{\Large\bfseries\color{APHPblue}}{\thesubsection}{0.6em}{}
% \titleformat{\subsubsection}{\bfseries\color{APHPdark}}{\thesubsubsection}{0.6em}{}
% \titleformat{\paragraph}[runin]{\bfseries\color{APHPdark}}{}{0pt}{}[.\;]
% ============================================================

% \section{Plateau HDJ digestif mutualisé — Organisation transversale}
\label{sec:plateau_hdj_mutualise}

\titleformat{\subsection}[runin]
  {\bfseries\color{APHPblue}}
  {}
  {0pt}
  {}

\begin{spacing}{1.30}


\noindent
Le plateau HDJ digestif mutualisé est un dispositif ambulatoire intégré regroupant les filières digestives et interventionnelles du service. Il est dimensionné selon des hypothèses conservatrices, avec une \textbf{capacité physique de 18 places} réparties en trois niveaux, pour un \textbf{plafond organisationnel} de \textbf{5\,214 séances/an} (base 220 jours et coefficients différenciés), et un \textbf{potentiel de recettes} estimé à \textbf{≈3,2 M€}. Le présent chapitre décrit l’organisation fonctionnelle, le dimensionnement capacitaire, la validation sur données historiques et les modalités de pilotage.

% ============================================================
% 3.1 PRINCIPES D'ORGANISATION
% ============================================================
\subsection*{Principes d’organisation —} 

L’organisation du plateau vise une allocation graduée des moyens, une standardisation des parcours et une optimisation de l’occupation, afin de maximiser l’efficience médico-économique tout en garantissant la qualité et la sécurité des soins.

\begin{itemize}[leftmargin=1.1cm]
  \item \textbf{Graduation des prises en charge} : adaptation des moyens humains, techniques et hôteliers à l’intensité réelle des besoins cliniques, selon trois niveaux formalisés.
  \item \textbf{Mutualisation des ressources} : partage des espaces, des équipes soignantes et des circuits logistiques entre filières contributrices, réduisant les coûts fixes unitaires.
  \item \textbf{Flexibilité capacitaire} : vases communicants inter-filières permettant une réallocation dynamique des places non consommées et une optimisation du taux d’occupation global.
  \item \textbf{Qualité hôtelière} : environnement de soins garantissant confort, confidentialité et dignité, contribuant à l’attractivité au sein du GHU.
\end{itemize}

% ============================================================
% 3.2 QUALITÉ HÔTELIÈRE ET INTIMITÉ DES PATIENTS
% ============================================================
\subsection*{Qualité hôtelière et intimité des patients —}

\noindent
Le plateau HDJ est conçu pour offrir un environnement de soins de haut niveau. La confidentialité et l’intimité des patients constituent un axe structurant, décliné par niveau d’intensité de soins.

\begin{itemize}[leftmargin=1.1cm]
  \item \textbf{Niveau 1 (Haute technicité)} : sectorisation en \textbf{chambres individuelles} pour l’ensemble des lits médicalisés, assurant confidentialité et sécurité des soins.
  \item \textbf{Niveau 2 (Ambulatoire)} : \textbf{boxes individuels ou semi-cloisonnés} pour les fauteuils, conciliant ergonomie de travail et respect de l’intimité.
  \item \textbf{Niveau 3 (Flux et orientation)} : \textbf{6 espaces lounge} avec dispositifs de rupture visuelle et acoustique, complétés par des \textbf{box de consultation privés}. La différenciation des flux (entrants/sortants/en cours) optimise la fluidité. L’environnement est valorisé par la lumière naturelle, un mobilier contemporain et une signalétique intuitive.
\end{itemize}

\noindent
Cette configuration répond à une double exigence : (i) préserver la dignité de patients souvent fragilisés par des pathologies chroniques ; (ii) positionner le plateau comme une référence en qualité perçue.

% ============================================================
% 3.3 PARCOURS ET FLUX PATIENTS
% ============================================================
\subsection*{Parcours patient —}

\noindent
Les patients suivent un parcours standardisé, modulé selon la filière et la complexité clinique.

\begin{enumerate}[leftmargin=1.1cm]
  \item \textbf{Orientation} : consultation spécialisée, RCP, urgences, ville ou hospitalisation complète.
  \item \textbf{Classification} : attribution d’un niveau de prise en charge (1, 2 ou 3) selon critères médicaux.
  \item \textbf{Programmation} : affectation à une place et un créneau horaire.
  \item \textbf{Accueil centralisé} : admission administrative et paramédicale unique.
  \item \textbf{Prise en charge} : actes médicaux, techniques ou multidisciplinaires.
  \item \textbf{Surveillance} : proportionnée au niveau de risque.
  \item \textbf{Sortie sécurisée} : synthèse médicale, prescriptions et organisation du suivi.
\end{enumerate}

% ============================================================
% 3.4 PRINCIPES DE DIMENSIONNEMENT CAPACITAIRE
% ============================================================
\subsection*{Principes de dimensionnement capacitaire —}

\noindent
Le dimensionnement repose sur une approche capacitaire pragmatique, fondée sur l’intensité de surveillance requise, la durée moyenne de prise en charge et les contraintes opérationnelles. Les hypothèses correspondent volontairement à une \textbf{phase de démarrage prudente}, permettant une montée en charge progressive sans modification structurelle des locaux ni des effectifs.

\paragraph{Segmentation de la capacité physique}
La capacité est structurée selon trois modalités d’accueil, correspondant à des niveaux croissants d’intensité de soins :
\begin{itemize}[leftmargin=1.1cm]
  \item \textbf{Lits médicalisés} : surveillance continue, patient allongé, actes ou situations à risque.
  \item \textbf{Fauteuils de soins} : surveillance infirmière directe pour prises en charge intermédiaires.
  \item \textbf{Espace lounge} : patients autonomes relevant de parcours programmés coordonnés, sans besoin de surveillance continue.
\end{itemize}

\paragraph{Hypothèses opérationnelles conservatrices}
La conversion des besoins en capacité intègre les aléas (annulations, variabilité des durées, contraintes RH spécialisées, indisponibilités ponctuelles). Le fonctionnement est estimé sur \textbf{44 semaines réellement opérées/an}, \textbf{5 jours/semaine}, soit \textbf{220 jours ouvrés/an}.

\noindent Hypothèses de productivité :
\begin{itemize}[leftmargin=1.1cm]
  \item \textbf{Lits médicalisés} : 1 patient/lit/jour, coefficient de réalisation \textbf{0,85}.
  \item \textbf{Fauteuils de soins} : 2 patients/fauteuil/jour, coefficient de réalisation \textbf{0,80}.
  \item \textbf{Fauteuils lounge} : productivité retenue \textbf{2 patients/fauteuil/jour} avec coefficient \textbf{0,75} (variabilité des durées, temps non productifs). Capacité théorique maximale : 3 patients/fauteuil/jour.
\end{itemize}

\paragraph{Formule de capacité annuelle}
\[
\mbox{HDJ/an} = 220 \times \Big(
N_L \times 1{,}0 \times 0{,}85
\;+\; N_F \times 2{,}0 \times 0{,}80
\;+\; N_{Lg} \times 2{,}0 \times 0{,}75
\Big)
\]

\noindent Application numérique avec $N_L=6$, $N_F=6$, $N_{Lg}=6$ :
\[
220 \times \big(6 \times 0{,}85 + 6 \times 1{,}60 + 6 \times 1{,}50\big)
= 220 \times \big(5{,}10 + 9{,}60 + 9{,}00\big)
= 220 \times 23{,}70
= \textbf{5\,214 \mbox{ séances/an}}.
\]

\noindent
Cette valeur constitue un \textbf{plafond organisationnel} (réserve d’absorption des fluctuations) et non un objectif de production immédiat.

% ============================================================
% 3.4bis VALIDATION PMSI
% ============================================================
\subsection*{Adéquation charge/capacité — Validation sur données historiques —}
\noindent
Le dimensionnement a été confronté aux données PMSI 2023–2024 des services contributeurs. L'analyse rétrospective porte sur les séjours de 0 jour (CMD 07, 25, 28) et les actes externes transférables en HDJ, selon un périmètre défini par la faisabilité clinique et organisationnelle.
\begin{table}[!htbp]
\centering
\caption{Adéquation charge/capacité — Confrontation PMSI 2024 vs cible HDJ}
\label{tab:adequation_pmsi}
\small
\rowcolors{2}{APHPsoft}{white}
\renewcommand{\arraystretch}{1.35}
\begin{tabular}{@{}p{4cm}p{2.6cm}p{2.6cm}p{3cm}@{}}
\toprule
\textbf{Filière} & \textbf{Réf. 2024} & \textbf{Cible HDJ} & \textbf{Évolution} \\
\midrule
Hépatologie (existant)      & 374 & 569 & +52\% \\
Hépatologie (création)      & —   & 950 & Création \\
MICI + Oncologie            & 1 901 & 2 201 & +16\% \\
Addictologie                & —   & 705   & Création \\
Radiologie interv.          & 150 & 250   & +67\% \\
\midrule
\textbf{Total}              & \textbf{2 425} & \textbf{4 670} & — \\
\bottomrule
\end{tabular}
\end{table}

\noindent
La capacité plafond (\textbf{5\,214 séances/an}) couvre la demande projetée (\textbf{4\,670 séances/an}), laissant une marge d’environ \textbf{30\%} pour la montée en charge et les aléas. La filière « Maladies du Foie » (≈32\% des flux) dispose d’une capacité compatible avec une cible de 1\,519 séances/an, via la combinaison des niveaux 1 et 3 selon les indications.

\FloatBarrier

% ============================================================
% 3.5 NIVEAUX DE PRISE EN CHARGE — VUE D'ENSEMBLE
% ============================================================
\subsection*{Trois niveaux de prise en charge —}

\noindent
Le plateau HDJ est structuré en trois niveaux fonctionnels, définis selon l’intensité de surveillance et la durée de séjour. Cette graduation permet une allocation optimale des ressources et une rotation adaptée aux prises en charge.

\begin{table}[!htbp]
\centering
\caption{Synthèse des niveaux de prise en charge (base 220 jours/an, coefficients différenciés)}
\label{tab:niveaux_synthese}
\small
\rowcolors{2}{APHPsoft}{white}
\renewcommand{\arraystretch}{1.35}
\begin{tabular}{@{}p{2.5cm}p{2cm}p{1.8cm}p{2cm}p{1.6cm}p{2.5cm}@{}}
\toprule
\textbf{Niveau} & \textbf{Places} & \textbf{Durée} & \textbf{Rotation} & \textbf{Coeff.} & \textbf{Capacité/an} \\
\midrule
1 — Lits      & 6 chambres  & 6–8 h & 1 pat./jour & 0,85 & $\sim$1\,122 \\
2 — Fauteuils & 6 fauteuils & 3–4 h & 2 pat./jour & 0,80 & $\sim$2\,112 \\
3 — Lounge    & 6 fauteuils & 2–3 h & 2 pat./jour & 0,75 & $\sim$1\,980 \\
\midrule
\textbf{Total} & \textbf{18 places} & — & — & — & \textbf{$\sim$5\,214} \\
\bottomrule
\end{tabular}
\end{table}

% ============================================================
% 3.5.1 NIVEAU 1 — LITS MÉDICALISÉS
% ============================================================
\subsubsection*{Niveau 1 — Lits médicalisés en chambre individuelle —}

\paragraph{Principe}
Le niveau 1 accueille les patients nécessitant une surveillance continue et/ou des actes invasifs à risque. Chaque patient occupe une \textbf{chambre individuelle} pour la journée, garantissant confidentialité, repos et conditions optimales de surveillance.

\paragraph{Indications}
\begin{itemize}[leftmargin=1.1cm]
  \item biopsies hépatiques transpariétales (repos strict post-procédure) ;
  \item ponctions d’ascite grand volume ($>$5 litres, compensation albumine) ;
  \item chimiothérapies prolongées ou à risque (surveillance) ;
  \item surveillance post-radiologie interventionnelle (embolisation, drainage).
\end{itemize}

\paragraph{Organisation}
\begin{itemize}[leftmargin=1.1cm]
  \item \textbf{Capacité} : 6 chambres individuelles équipées (scope, oxygène, aspiration).
  \item \textbf{Durée} : 6 à 8 heures (arrivée 8h00, sortie 16h00–18h00).
  \item \textbf{Rotation} : 1 patient/lit/jour (pas de double rotation).
  \item \textbf{Ratio IDE} : 1 IDE / 3 patients.
  \item \textbf{Présence médicale} : médecin senior disponible sur le plateau.
\end{itemize}

\paragraph{Capacité annuelle}
6 lits $\times$ 220 jours $\times$ 1 patient/jour $\times$ 0,85 = \textbf{1\,122 séances/an}.

% ============================================================
% 3.5.2 NIVEAU 2 — FAUTEUILS DE SOINS
% ============================================================
\subsubsection*{Niveau 2 — Fauteuils de soins en box individuel —}

\paragraph{Principe}
Le niveau 2 correspond à des soins techniques nécessitant une surveillance infirmière directe, sans décubitus prolongé. Les fauteuils sont positionnés en \textbf{boxes individuels ou semi-cloisonnés}.

\paragraph{Indications}
\begin{itemize}[leftmargin=1.1cm]
  \item perfusions d’albumine ;
  \item fer injectable IV ;
  \item transfusions programmées ;
  \item biothérapies IV (infliximab, vedolizumab, ustekinumab) ;
  \item chimiothérapies digestives simples (durée $<$4h) ;
  \item traitements IV divers (antibiotiques, immunoglobulines).
\end{itemize}

\paragraph{Organisation}
\begin{itemize}[leftmargin=1.1cm]
  \item \textbf{Capacité} : 6 fauteuils inclinables en boxes.
  \item \textbf{Durée} : 3 à 4 heures.
  \item \textbf{Rotation} : 2 patients/fauteuil/jour (matin + après-midi).
  \item \textbf{Ratio IDE} : 1 IDE / 4 patients.
  \item \textbf{Présence médicale} : médecin joignable ; passage selon besoin clinique.
\end{itemize}

\paragraph{Capacité annuelle}
6 fauteuils $\times$ 220 jours $\times$ 2 patients/jour $\times$ 0,80 = \textbf{2\,112 séances/an}.

% ============================================================
% 3.5.3 NIVEAU 3 — LOUNGE
% ============================================================
\subsubsection*{Niveau 3 — Plateau ambulatoire léger et multidisciplinaire —}

\paragraph{Principe}
Le niveau 3 est dédié aux parcours à \textbf{faible intensité de surveillance} et à \textbf{forte valeur évaluative}. Le patient est installé en lounge ; les professionnels gravitent autour de lui (modèle \emph{One-Stop Shop} diagnostique et thérapeutique).

\paragraph{Indications et files actives}
\begin{itemize}[leftmargin=1.1cm, label=\textbullet]
  \item \textbf{Bilan hépatique non-invasif complet} : échographie morphologique, élastographies (hépatique et splénique), quantification stéatose (CAP/LISA).
  \item \textbf{Addictologie intégrée} : sevrage ambulatoire, suivi post-sevrage complexe, réduction des risques.
  \item \textbf{MICI stables} : administration de biothérapies SC et ETP.
  \item \textbf{Parcours pré-thérapeutiques} : bilans pré-inclusion, initiation de traitements oraux complexes.
\end{itemize}

\paragraph{Paramètres opérationnels}
\begin{itemize}[leftmargin=1.1cm, label=\textbullet]
  \item \textbf{Capacité} : 6 modules lounge avec isolement visuel et acoustique.
  \item \textbf{Rotation retenue} : 2 patients/fauteuil/jour (théorique max : 3).
  \item \textbf{Durée} : 2 à 3 heures.
  \item \textbf{Coefficient de réalisation} : 0,75.
  \item \textbf{Ratio IDE} : 1 IDE / 6 patients (coordination, faible technicité), avec renfort ponctuel selon charge.
\end{itemize}

\paragraph{Capacité annuelle prévisionnelle —}
\begin{itemize}[leftmargin=1.1cm]
  \item \textbf{Hypothèse retenue (2 rotations/j)} : 6 $\times$ 220 $\times$ 2 $\times$ 0,75 = \textbf{1\,980 séances/an}.
  \item \textbf{Cible de performance (3 rotations/j)} : 6 $\times$ 220 $\times$ 3 $\times$ 0,75 = \textbf{2\,970 séances/an}.
\end{itemize}

\FloatBarrier

% ============================================================
% 3.6 SYNTHÈSE CAPACITAIRE ET IMPACT MÉDICO-ÉCONOMIQUE
% ============================================================
\subsection*{Synthèse capacitaire et impact médico-économique —}
\label{sec:synthese_medico_eco}

\noindent
La répartition capacitaire permet une allocation différenciée des activités selon leur intensité de soins et leur rendement médico-économique. Les lits médicalisés concentrent les prises en charge complexes à forte valeur ajoutée clinique ; les fauteuils et le lounge absorbent des volumes élevés d’actes programmés à forte efficience.

\begin{sidewaystable}[p]
\centering
\caption{Synthèse capacitaire et médico-économique du plateau HDJ (hypothèses conservatrices, 220 jours/an)}
\label{tab:synthese_medico_eco}
\renewcommand{\arraystretch}{1.15}
\rowcolors{2}{APHPsoft}{white}
\begin{tabular}{
  p{4.2cm}
  >{\centering\arraybackslash}p{1.4cm}
  >{\centering\arraybackslash}p{3.2cm}
  >{\centering\arraybackslash}p{2.8cm}
  >{\centering\arraybackslash}p{3.6cm}
  >{\centering\arraybackslash}p{3.2cm}
}
\toprule
\textbf{Type de capacité} &
\textbf{N} &
\textbf{Hypothèse de flux} &
\textbf{Capacité annuelle} &
\textbf{Activités principales} &
\textbf{Recettes estimées} \\
\midrule
Lits médicalisés &
6 &
1 pat./lit/j $\times$ 0,85 &
$\approx$ 1\,122 &
PBH, cirrhoses avancées, chimiothérapies prolongées, RI post-acte &
$\approx$ 1,5–1,7 M€ \\
Fauteuils de soins &
6 &
2 pat./faut./j $\times$ 0,80 &
$\approx$ 2\,112 &
Fer IV, albumine, biothérapies IV, chimiothérapies simples &
$\approx$ 0,6–0,7 M€ \\
Fauteuils lounge &
6 &
2 pat./faut./j $\times$ 0,75 &
$\approx$ 1\,980 &
Hépatométabolique, cirrhoses compensées, addictologie, parcours coordonnés &
$\approx$ 0,7–1,0 M€ \\
\midrule
\textbf{Total plateau} &
\textbf{18} &
— &
\textbf{$\approx$ 5\,214} &
— &
\textbf{$\approx$ 3,2 M€} \\
\bottomrule
\end{tabular}

\vspace{0.25cm}
\footnotesize
\emph{Note :} la capacité présentée correspond à un plafond organisationnel. La trajectoire d’activité cible est volontairement inférieure, afin de préserver une réserve d’absorption des fluctuations et d’autoriser la montée en charge. L’espace lounge constitue une capacité assise optimisant les flux sans création de lits supplémentaires.
\end{sidewaystable}

\FloatBarrier

\noindent\textbf{Écart entre plafond et projections opérationnelles.}
À titre de comparaison, la somme des projections « croisière » des filières opérationnelles atteint environ 4\,700 séances et 2,9~M€, soit un taux d’occupation implicite de 90\%. La marge résiduelle (≈10\%) constitue une réserve opérationnelle (variations saisonnières, aléas RH, indisponibilités techniques, montée en charge future).

% ============================================================
% 3.7 FLEXIBILITÉ — VASES COMMUNICANTS
% ============================================================
\subsection*{Flexibilité capacitaire — Système de vases communicants —}
\label{sec:vases_communicants}

\noindent
La flexibilité inter-filières repose sur un mécanisme de réallocation des places non consommées, garantissant l’optimisation de l’occupation et la réduction des délais de programmation.

\paragraph{Principe}
Chaque filière dispose de créneaux réservés ; les places non programmées ou libérées (annulation, sortie anticipée) basculent vers un \textbf{pool mutualisé} accessible à l’ensemble des filières contributrices.

\paragraph{Mécanisme d’allocation}
\begin{enumerate}[leftmargin=1.1cm]
  \item \textbf{J-7 à J-3} : confirmation des patients programmés par les référents médicaux.
  \item \textbf{J-2 à J-1} : bascule des créneaux non confirmés dans le \textbf{pool mutualisé} ; proposition aux filières via liste d’attente partagée.
  \item \textbf{Jour J} : réattribution en temps réel des places libérées, sous coordination IDE/IPA.
\end{enumerate}

\paragraph{Critères de priorisation du pool}
\begin{enumerate}[leftmargin=1.1cm]
  \item urgences médicales relatives (décompensation, infection, anémie symptomatique) ;
  \item ancienneté de la demande ($>$7 jours) ;
  \item optimisation des créneaux (matin/après-midi) ;
  \item équité inter-filières sur période glissante (suivi mensuel).
\end{enumerate}

\paragraph{Outils de gestion}
\begin{itemize}[leftmargin=1.1cm]
  \item tableau de programmation partagé (support institutionnel type Teams/SharePoint) accessible aux référents ;
  \item liste d’attente mutualisée (date de demande, niveau de priorité) ;
  \item signalement immédiat des places libérées (IDE/IPA coordinatrices) ;
  \item reporting hebdomadaire du taux d’occupation par filière et par niveau.
\end{itemize}

\paragraph{Bénéfices attendus}
\begin{itemize}[leftmargin=1.1cm]
  \item optimisation du taux d’occupation global (cible $>$85\%) ;
  \item réduction des délais de programmation ;
  \item équité d’accès entre filières contributrices ;
  \item absorption des variations saisonnières.
\end{itemize}

% ============================================================
% 3.8 PLANNING HEBDOMADAIRE TYPE
% ============================================================
\subsection*{Planning hebdomadaire type —}

\noindent
La programmation repose sur des créneaux réservés, avec bascule dans le pool mutualisé à J-2 pour les places non confirmées.

\begin{table}[!htbp]
\centering
\caption{Planning hebdomadaire — Lits médicalisés (niveau 1) : créneaux réservés et pool mutualisé}
\label{tab:planning_hebdo_lits}
\small
\renewcommand{\arraystretch}{1.35}
\rowcolors{2}{APHPsoft}{white}
\begin{tabular}{@{}p{1.8cm}p{5.9cm}p{1.8cm}p{1.8cm}@{}}
\toprule
\textbf{Jour} & \textbf{Activités programmées} & \textbf{Réservé} & \textbf{Pool} \\
\midrule
Lundi    & PBH (2) + Ascite (3)                                      & 5 lits & 1 lit \\
Mardi    & Radiologie interventionnelle (3–4) + Chimio lourde (1)     & 4 lits & 2 lits \\
Mercredi & MICI immunothérapies IV (4–5) + Chimio lourde (1)         & 5 lits & 1 lit \\
Jeudi    & PBH (2) + Ascite (3)                                      & 5 lits & 1 lit \\
Vendredi & Radiologie interventionnelle (3–4) + Chimio lourde (1)     & 4 lits & 2 lits \\
\bottomrule
\end{tabular}
\end{table}

\begin{table}[!htbp]
\centering
\caption{Planning hebdomadaire — Fauteuils de soins (niveau 2)}
\label{tab:planning_hebdo_fauteuils}
\small
\renewcommand{\arraystretch}{1.35}
\rowcolors{2}{APHPsoft}{white}
\begin{tabular}{@{}p{1.8cm}p{5.9cm}p{5.9cm}@{}}
\toprule
\textbf{Jour} & \textbf{Matin (8h–12h)} & \textbf{Après-midi (13h–17h)} \\
\midrule
Lundi    & Fer IV, Albumine, Transfusions & Biothérapies \\
Mardi    & Biothérapies                   & Chimio simples \\
Mercredi & Fer IV, Albumine, Transfusions & Biothérapies \\
Jeudi    & Chimio simples                 & Biothérapies \\
Vendredi & Biothérapies                   & Rattrapage / Pool \\
\bottomrule
\end{tabular}
\end{table}

\noindent
L’espace lounge (niveau 3, 6 fauteuils) fonctionne quotidiennement, avec programmation adaptée aux disponibilités des intervenants (médecins, diététicien(ne)s, psychologues, IPA).

\FloatBarrier

% ============================================================
% 3.9 ZONES FONCTIONNELLES
% ============================================================
\subsection*{Zones fonctionnelles —}

\noindent
La traduction spatiale de l’organisation repose sur des zones identifiées :
\begin{itemize}[leftmargin=1.1cm]
  \item \textbf{Zone d’accueil} : admission centralisée, attente, orientation.
  \item \textbf{Zone niveau 1} : 6 chambres individuelles équipées.
  \item \textbf{Zone niveau 2} : 6 boxes individuels avec fauteuils.
  \item \textbf{Zone niveau 3} : espace lounge (6 fauteuils, séparations visuelles et acoustiques).
  \item \textbf{Zone consultations} : bureaux intégrés (diététique, psychologie, addictologie, IPA).
  \item \textbf{Zone technique} : échographie, élastographie, prélèvements, préparation traitements.
  \item \textbf{Zone logistique} : pharmacie, stockage, circuits propres/sales.
\end{itemize}

% ============================================================
% 3.10 RESSOURCES HUMAINES MUTUALISÉES
% ============================================================
\subsection*{Ressources humaines mutualisées —}

\noindent
Le fonctionnement repose sur une équipe dédiée, polyvalente et formée aux trois niveaux de prise en charge. Le dimensionnement intègre la couverture des absences et la montée en compétences progressive.

\begin{table}[!htbp]
\centering
\caption{Effectifs non médicaux (PNM) — Phase pilote et maturité}
\label{tab:effectifs_pnm}
\small
\rowcolors{2}{APHPsoft}{white}
\renewcommand{\arraystretch}{1.35}
\begin{tabular}{@{}p{4.6cm}p{2.6cm}p{2.6cm}p{3.6cm}@{}}
\toprule
\textbf{Catégorie} & \textbf{Phase pilote} & \textbf{Maturité} & \textbf{Missions principales} \\
\midrule
IDE expertes        & 4–5 ETP  & 6–7 ETP   & Soins, surveillance, coordination \\
IPA                 & 1 ETP    & 1,5 ETP   & Coordination, évaluation, ETP \\
Cadre de santé      & 0,5 ETP  & 1 ETP     & Management, organisation \\
Secrétariat médical & 0,5 ETP  & 1 ETP     & Programmation, accueil, DPI \\
Psychologue         & 0,3 ETP  & 0,5 ETP   & Soutien, addictologie \\
Diététicien(ne)     & 0,3 ETP  & 0,5 ETP   & Bilans nutritionnels, ETP \\
\midrule
\textbf{Total PNM}  & \textbf{$\sim$7 ETP} & \textbf{$\sim$11,5 ETP} & — \\
\bottomrule
\end{tabular}
\end{table}

\begin{table}[!htbp]
\centering
\caption{Vacations médicales hebdomadaires (PM)}
\label{tab:effectifs_pm}
\small
\rowcolors{2}{APHPsoft}{white}
\renewcommand{\arraystretch}{1.35}
\begin{tabular}{@{}p{6.6cm}p{3.0cm}p{3.0cm}@{}}
\toprule
\textbf{Spécialité médicale} & \textbf{Phase pilote} & \textbf{Maturité} \\
\midrule
Hépatologie (PBH, cirrhose)        & 2 demi-j/sem & 4 demi-j/sem \\
MICI (biothérapies)                 & 2 demi-j/sem & 3 demi-j/sem \\
Addictologie                        & 1 demi-j/sem & 2 demi-j/sem \\
Oncologie digestive                  & 1 demi-j/sem & 2 demi-j/sem \\
Radiologie interventionnelle        & —            & 1 demi-j/sem \\
\midrule
\textbf{Total vacations/semaine}    & \textbf{6 demi-j} & \textbf{12 demi-j} \\
\bottomrule
\end{tabular}
\end{table}

\paragraph{Plan de formation à la polyvalence}
La mutualisation des IDE requiert une montée en compétences croisée ; un plan sur 12 mois est prévu avant ouverture.

\begin{table}[!htbp]
\centering
\caption{Plan de formation croisée — Polyvalence IDE du plateau HDJ}
\label{tab:formation_ide}
\small
\rowcolors{2}{APHPsoft}{white}
\renewcommand{\arraystretch}{1.35}
\begin{tabular}{@{}p{4.6cm}p{3.1cm}p{2.5cm}p{2.5cm}@{}}
\toprule
\textbf{Compétence} & \textbf{Public cible} & \textbf{Durée} & \textbf{Validation} \\
\midrule
Surveillance post-RI (embolisation, drainage) & IDE addictologie, MICI & 3 jours & Attestation RI \\
Chimiothérapies digestives                    & IDE non-oncologie      & 5 jours & Habilitation URC \\
Gestion des sevrages (CIWA, anxiolyse)        & IDE non-addictologie   & 2 jours & Attestation addicto \\
Biothérapies IV (perfusion, EI)               & Toutes IDE             & 2 jours & Attestation MICI \\
Surveillance post-PBH                         & IDE non-hépatologie    & 1 jour  & Compagnonnage \\
\bottomrule
\end{tabular}
\end{table}

\noindent\textbf{Principe :} chaque IDE maîtrise son cœur de métier et \textbf{au moins deux compétences transversales}. L’affectation quotidienne tient compte des habilitations validées.

\paragraph{Ratios de sécurité}
Les ratios IDE/patients sont différenciés par niveau :
\begin{itemize}[leftmargin=1.1cm]
  \item \textbf{Niveau 1} : 1 IDE / 3 patients (surveillance renforcée post-acte).
  \item \textbf{Niveau 2} : 1 IDE / 4 patients (soins techniques standards).
  \item \textbf{Niveau 3} : 1 IDE / 6 patients (coordination, faible technicité), ajustable selon charge et complexité.
\end{itemize}

\noindent
En configuration maximale (18 patients simultanés), l’effectif requis est compatible avec le dimensionnement (6–7 ETP IDE) permettant la couverture des absences.

\FloatBarrier

% ============================================================
% 3.11 SCHÉMA FONCTIONNEL
% ============================================================
\subsection*{Schéma fonctionnel du plateau —}

\noindent
Le schéma illustre le parcours patient depuis l’orientation jusqu’à la sortie, avec classification par niveau et pool mutualisé, ainsi que la gestion des complications.

\vspace{0.8cm}

\begin{center}
\begin{tikzpicture}[
  node distance=1.2cm,
  box/.style={
    rectangle, rounded corners=3pt,
    draw=APHPdark, thick,
    text width=5.2cm, minimum height=1.1cm,
    align=center, fill=APHPsoft
  },
  levelbox/.style={
    rectangle, rounded corners=3pt,
    draw=APHPblue, thick,
    text width=3.2cm, minimum height=0.9cm,
    align=center, fill=white
  },
  arrow/.style={->, thick, APHPdark}
]
\node[box] (orientation) {Orientation \\ Consultation / Ville / Urgences};
\node[box, below=of orientation] (classif) {Classification patient \\ Niveau 1, 2 ou 3};
\node[box, below=of classif] (accueil) {Accueil centralisé \\ Admission unique};

\node[levelbox, below left=1.5cm and -0.5cm of accueil] (n1) {Niveau 1 \\ 6 chambres};
\node[levelbox, below=1.5cm of accueil] (n2) {Niveau 2 \\ 6 fauteuils};
\node[levelbox, below right=1.5cm and -0.5cm of accueil] (n3) {Niveau 3 \\ 6 lounge};

\node[box, below=3.5cm of accueil] (sortie) {Sortie sécurisée \\ Synthèse + Suivi};
\node[box, right=2.5cm of classif] (pool) {Pool mutualisé \\ Vases communicants};

\node[box, below right=2cm and 3cm of sortie, fill=red!10] (hospit) {HC\\ ou SAU};

\draw[arrow] (orientation) -- (classif);
\draw[arrow] (classif) -- (accueil);
\draw[arrow] (accueil) -- (n1);
\draw[arrow] (accueil) -- (n2);
\draw[arrow] (accueil) -- (n3);
\draw[arrow] (n1) -- (sortie);
\draw[arrow] (n2) -- (sortie);
\draw[arrow] (n3) -- (sortie);
\draw[arrow, dashed] (pool) -- (accueil);

\draw[arrow, red, dashed] (n1) to[bend right=20] node[midway, right, font=\scriptsize] {Complication} (hospit);
\draw[arrow, red, dashed] (n2) to[bend right=10] (hospit);

\end{tikzpicture}
\end{center}

\begin{center}
\captionof{figure}{Schéma fonctionnel du plateau HDJ avec classification par niveau et pool mutualisé}
\label{fig:schema_fonctionnel}
\end{center}

\paragraph{Circuit de l’aval — Gestion des complications}
En cas de complication nécessitant une hospitalisation conventionnelle :
\begin{enumerate}[leftmargin=1.1cm]
  \item \textbf{Évaluation immédiate} par le médecin senior du plateau.
  \item \textbf{Contact du service d’aval} : Maladies du Foie, Gastroentérologie, SAU, Réanimation médicale si défaillance vitale.
  \item \textbf{Transfert sécurisé} avec dossier de liaison et transmissions IDE-IDE.
  \item \textbf{Traçabilité} : signalement EI si applicable, analyse en CODIR.
\end{enumerate}

\noindent
\textbf{Objectif :} taux de transfert non programmé vers HC/SAU $<$ 0,5\% des séances.

\FloatBarrier

% ============================================================
% 3.12 GOUVERNANCE
% ============================================================
\subsection*{Gouvernance —}

\noindent
Le plateau est placé sous une responsabilité médicale unique. La gouvernance s’appuie sur deux instances complémentaires.

\paragraph{Comité de pilotage stratégique (COPIL) — Trimestriel}
\begin{itemize}[leftmargin=1.1cm]
  \item \textbf{Composition} : responsable médical, référents filières, cadre supérieur, direction des soins, direction médico-économique.
  \item \textbf{Missions} : orientations stratégiques, arbitrages capacitaires, validation de la clé de redistribution des recettes.
\end{itemize}

\paragraph{Comité opérationnel (CODIR) — Mensuel}
\begin{itemize}[leftmargin=1.1cm]
  \item \textbf{Composition} : responsable médical, cadre de santé, IPA, secrétariat, pharmacie.
  \item \textbf{Missions} : suivi opérationnel, gestion des flux, résolution des dysfonctionnements, analyse EI.
\end{itemize}

% ============================================================
% 3.12bis TABLEAU DE BORD
% ============================================================
\subsection*{Pilotage et qualité — Tableau de bord —}

\begin{table}[H]
\centering
\caption{Indicateurs de pilotage}
\label{tab:kpi}
\small
\rowcolors{2}{APHPsoft}{white}
\renewcommand{\arraystretch}{1.35}
\begin{tabular}{@{}p{3cm}p{8cm}p{2.2cm}@{}}
\toprule
\textbf{Domaine} & \textbf{Indicateurs} & \textbf{Fréquence} \\
\midrule
Activité          & Séances/filière/niveau, taux d’occupation, délais de programmation, annulations & Mensuel \\
Médico-économique & Recettes/filière, écart au budget, coût moyen/séance                           & Mensuel \\
Qualité/sécurité  & Complications, transferts non programmés, réhospitalisations à 30 j, satisfaction, EI & Trimestriel \\
RH                & Absentéisme, vacances de postes, heures supplémentaires, formations réalisées   & Trimestriel \\
\bottomrule
\end{tabular}
\end{table}

\paragraph{Règlement intérieur de l’unité}
Un règlement intérieur, validé par le COPIL avant ouverture, définit les règles d’arbitrage en cas de conflit de programmation :
\begin{enumerate}[leftmargin=1.1cm]
  \item \textbf{Priorisation médicale} : application de la grille de priorisation (cf.\ §3.7). En cas d’égalité, le médecin senior présent tranche.
  \item \textbf{Escalade} : en cas de conflit inter-filières, contact du responsable médical de l’UF (ou délégué) sous 15 minutes.
  \item \textbf{Traçabilité} : arbitrage consigné dans le DPI (motif, décideur, solution). Reporting mensuel en CODIR.
  \item \textbf{Recours} : litiges récurrents portés au COPIL trimestriel pour ajustement des créneaux.
\end{enumerate}

\noindent
Le règlement intérieur précise également : horaires d’ouverture, procédure d’annulation tardive (<24h), gestion des urgences relatives, circuit de signalement des EI.

\FloatBarrier

% ============================================================
% 3.13 MODÈLE ÉCONOMIQUE UF
% ============================================================
\subsection*{Modèle économique de l’unité fonctionnelle —}
\label{sec:modele_economique_uf}

\noindent
Le plateau HDJ est constitué en \textbf{unité fonctionnelle (UF) distincte}, permettant une traçabilité des recettes et une redistribution vers les UF cliniques contributrices.

\paragraph{Centralisation}
L’ensemble des recettes T2A (GHS, suppléments, molécules onéreuses) est imputé à l’UF HDJ Digestif Mutualisé.

\paragraph{Redistribution}
Les recettes sont redistribuées selon un mécanisme mixte validé annuellement en COPIL :
\begin{itemize}[leftmargin=1.1cm]
  \item \textbf{Base forfaitaire (60\%)} : enveloppe fixe par UF clinique, garantissant une prévisibilité budgétaire ;
  \item \textbf{Part variable (40\%)} : au prorata de l’activité réellement réalisée, incitant à l’optimisation de la programmation.
\end{itemize}

\begin{table}[!htbp]
\centering
\caption{Redistribution indicative des recettes (total ajusté à 100\% annuellement en COPIL)}
\label{tab:redistribution}
\small
\rowcolors{2}{APHPsoft}{white}
\renewcommand{\arraystretch}{1.35}
\begin{tabular}{@{}p{5.2cm}p{4.7cm}p{2.6cm}@{}}
\toprule
\textbf{UF bénéficiaire} & \textbf{Filières} & \textbf{\% indicatif} \\
\midrule
Maladies du Foie             & PBH, cirrhoses, hépatométabolique & 30–35\% \\
Gastroentérologie            & MICI, oncologie digestive         & 35–40\% \\
Addictologie                 & HDJ addictologie                  & 15–20\% \\
Radiologie interventionnelle & Post-procédures RI                & 10–15\% \\
\bottomrule
\end{tabular}
\end{table}

\paragraph{Périmètre de la redistribution}
La clé s’applique aux \textbf{recettes T2A directes} (GHS, suppléments, molécules onéreuses en sus). Elle ne concerne pas :
\begin{itemize}[leftmargin=1.1cm]
  \item les dotations MERRI, affectées aux UF cliniques selon les règles institutionnelles ;
  \item les enveloppes MIGAC ;
  \item les recettes liées aux essais cliniques, imputées à l’UF investigatrice.
\end{itemize}

\noindent
La redistribution est calculée mensuellement par le DIM sur la base des séjours clôturés, puis validée trimestriellement en COPIL avec ajustement rétroactif si nécessaire.

\end{spacing}
