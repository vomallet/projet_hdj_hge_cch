% ============================================================
% 1. E-SUMMARY — SYNTHÈSE EXÉCUTIVE
% ============================================================

\titleformat{\subsection}[runin]
  {\bfseries\color{APHPblue}}
  {}
  {0pt}
  {}

\begin{spacing}{1.20}

% ------------------------------------------------------------
% 1.1 Contexte et opportunité institutionnelle
% ------------------------------------------------------------

\subsection*{Contexte et opportunité institutionnelle}\ignorespaces
La structuration d’un plateau unique d’Hôpitaux de Jour (HDJ) regroupant les activités digestives et interventionnelles s’inscrit pleinement dans les objectifs nationaux du virage ambulatoire. Elle repose sur une opportunité organisationnelle majeure — la libération de locaux dédiés — permettant de regrouper sur un site unique des activités ambulatoires aujourd’hui dispersées, tout en améliorant leur lisibilité et leur efficience.

Ce regroupement vise à constituer un dispositif évolutif, aligné sur les recommandations nationales et internationales de référence (EASL, AFEF, SNFGE, INCa, SFR/SFICV), et adapté à la prise en charge de patients complexes nécessitant des parcours de soins programmés, sécurisés et coordonnés.

% ------------------------------------------------------------
% 1.2 Objectifs stratégiques du projet
% ------------------------------------------------------------

\subsection*{Objectifs stratégiques du projet}

Le projet poursuit cinq objectifs stratégiques structurants :
\begin{itemize}[leftmargin=1.1cm]
    \item améliorer l’accès rapide, sécurisé et lisible aux explorations et aux traitements spécialisés ;
    \item réduire les hospitalisations conventionnelles évitables par un recours structuré et coordonné à l’ambulatoire ;
    \item optimiser l’utilisation des ressources humaines expertes par une organisation mutualisée, transversale et efficiente ;
    \item renforcer l’attractivité institutionnelle et académique du site dans un contexte durable de tension sur les compétences spécialisées ;
    \item développer un parcours patient hospitalo-universitaire favorisant l’inclusion des patients dans des essais thérapeutiques.
\end{itemize}


% ------------------------------------------------------------
% 1.3 Population cible globale
% ------------------------------------------------------------

\subsection*{Population cible globale}
Le plateau HDJ s’adresse à une population adulte et présentant :
\begin{itemize}[leftmargin=1.1cm]
    \item des pathologies digestives et hépato-biliaires chroniques (MASLD/MASH, cirrhoses compensées et décompensées) ;
    \item des troubles addictologiques avec retentissement somatique ;
    \item des maladies inflammatoires chroniques de l’intestin (MICI) ;
    \item des cancers digestifs nécessitant des traitements systémiques ambulatoires ;
    \item des patients pris en charge en radiologie interventionnelle pour surveillance post examen;
    \item des patients en cours d'évaluation thérapeutique ou du suivi de traitements oraux.
\end{itemize}

Ces patients présentent un besoin élevé d’évaluations programmables, répétées et sécurisées, compatible avec une prise en charge ambulatoire structurée.

% ------------------------------------------------------------
% 1.4 Volumétrie et trajectoire d’activité
% ------------------------------------------------------------

\subsection*{Volumétrie et trajectoire d’activité}

Le plateau d’Hôpital de Jour est dimensionné pour une \textbf{capacité annuelle maximale théorique estimée entre 4\,700 et 5\,000 séances}, correspondant à un fonctionnement à pleine charge, sans marge organisationnelle.

L’objectif opérationnel à maturité médico-économique repose toutefois sur une \textbf{trajectoire d’activité cible comprise entre 2\,000 et 2\,500 séances annuelles}, compatible avec une montée en charge progressive, une sélectivité médicale raisonnée et un taux d’occupation volontairement inférieur au plafond théorique. Cette trajectoire est anticipée sur une période de \textbf{24 à 36 mois}, sur la base des données historiques locales, des besoins territoriaux identifiés et d’hypothèses prudentes de croissance.


% ------------------------------------------------------------
% 1.5 Recettes consolidées
% ------------------------------------------------------------

\subsection*{Recettes consolidées}
Sur la base des tarifs moyens pondérés par filière et des volumes projetés, les recettes consolidées du plateau HDJ sont estimées à \textbf{environ 3~M€ par an à maturité}, avec une fourchette comprise entre \textbf{2,5 et 3,4~M€}.

Le dispositif présente ainsi un profil \textbf{autosoutenable}, générateur de valeur médico-économique pour l’institution.

% ------------------------------------------------------------
% 1.6 Bénéfices institutionnels
% ------------------------------------------------------------

\subsection*{Bénéfices institutionnels}
La structuration du plateau HDJ permet :
\begin{itemize}[leftmargin=1.1cm]
    \item une redistribution ciblée des lits MCO mobilisés pour des prises en charge programmables ;
    \item une fluidification des parcours entre la ville, les urgences et l’hospitalisation complète ;
    \item une mutualisation efficiente des compétences spécialisées (IDE expertes, IPA, psychologie, diététique) ;
    \item une amélioration de l’attractivité des postes paramédicaux et médicaux spécialisés ;
    \item une lisibilité renforcée de l’offre ambulatoire digestive et interventionnelle à l’échelle du territoire.
\end{itemize}

% ------------------------------------------------------------
% 1.7 Gouvernance (niveau exécutif)
% ------------------------------------------------------------

\subsection*{Gouvernance}
Le plateau des HDJ mutualisés repose sur une gouvernance claire et intégrée, associant :
\begin{itemize}[leftmargin=1.1cm]
    \item une responsabilité médicale unique, garante de la cohérence des parcours et de la sécurité des prises en charge ;
    \item une articulation structurée entre le DMU digestif, les plateaux médico-techniques et la radiologie interventionnelle ;
    \item un pilotage médico-économique consolidé associant responsables médicaux, encadrement soignant, DIM et direction financière.
\end{itemize}

Les modalités organisationnelles détaillées, le schéma fonctionnel de référence et les principes de dimensionnement sont développés dans les sections suivantes.

\end{spacing}

\clearpage
