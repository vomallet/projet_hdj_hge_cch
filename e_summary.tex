% ============================================================
% 1. E-SUMMARY — SYNTHÈSE EXÉCUTIVE
% ============================================================

% \addcontentsline{toc}{section}{E-summary}

\titleformat{\subsection}[runin]
  {\bfseries\color{APHPblue}}
  {}
  {0pt}
  {}

\begin{spacing}{1.20}

% ------------------------------------------------------------
% 1.1 Contexte et opportunité institutionnelle
% ------------------------------------------------------------

\subsection*{Contexte et opportunité institutionnelle}\ignorespaces
La structuration d’un plateau unique d’Hôpitaux de Jour (HDJ) regroupant les activités des Maladies de l’Appareil Digestif et de la Radiologie Interventionnelle s’inscrit pleinement dans les objectifs nationaux du virage ambulatoire. Elle s’appuie sur l’opportunité organisationnelle créée par la libération des locaux d’hématologie et vise à regrouper, sur un site unique, des activités ambulatoires spécialisées aujourd’hui dispersées.

Ce regroupement stratégique permet de constituer un plateau lisible, efficient et évolutif, aligné sur les recommandations nationales et internationales de référence (EASL, Baveno~VII, AFEF, DNFGE, INCa, SFR/SFICV), et adapté à la prise en charge de patients complexes nécessitant des parcours de soins fluides, programmés et sécurisés.

% ------------------------------------------------------------
% 1.2 Objectifs stratégiques du projet
% ------------------------------------------------------------

\subsection*{Objectifs stratégiques du projet}
Le projet poursuit quatre objectifs structurants :
\begin{itemize}[leftmargin=1.1cm]
    \item améliorer l’accès rapide, sécurisé et lisible aux explorations et traitements spécialisés ;
    \item réduire les hospitalisations conventionnelles évitables par un recours organisé à l’ambulatoire ;
    \item optimiser l’utilisation des ressources humaines expertes par une organisation mutualisée et transversale ;
    \item renforcer l’attractivité institutionnelle et académique du site dans un contexte de tension croissante sur les compétences spécialisées.
\end{itemize}

% ------------------------------------------------------------
% 1.3 Population cible globale
% ------------------------------------------------------------

\subsection*{Population cible globale}
Le plateau HDJ s’adresse à une population adulte large et en croissance relevant :
\begin{itemize}[leftmargin=1.1cm]
    \item des pathologies digestives et hépato-biliaires chroniques (MASLD/MASH, cirrhoses compensées et décompensées) ;
    \item des troubles addictologiques avec retentissement somatique ;
    \item des maladies inflammatoires chroniques de l’intestin ;
    \item des cancers digestifs nécessitant des traitements systémiques ambulatoires ;
    \item des patients pris en charge en radiologie interventionnelle.
\end{itemize}

Ces patients présentent un besoin élevé d’évaluations répétées, programmables et sécurisées en ambulatoire.

% ------------------------------------------------------------
% 1.4 Volumétrie et trajectoire d’activité
% ------------------------------------------------------------

\subsection*{Volumétrie et trajectoire d’activité}
Le plateau HDJ est dimensionné pour une \textbf{capacité annuelle maximale théorique de l’ordre de 4\,700 à 5\,000 séances}, toutes filières confondues.  

La trajectoire d’activité cible à maturité médico-économique est estimée entre \textbf{2\,000 et 2\,500 séances par an}, avec une montée en charge progressive sur une période de \textbf{24 à 36 mois}. Cette trajectoire repose sur les données historiques locales, les besoins identifiés sur le territoire et des hypothèses prudentes de croissance.

% ------------------------------------------------------------
% 1.5 Recettes consolidées
% ------------------------------------------------------------

\subsection*{Recettes consolidées}
Sur la base des tarifs moyens pondérés par filière et des volumes projetés, les recettes consolidées du plateau HDJ sont estimées à \textbf{environ 3~M€ par an à maturité}, avec une fourchette comprise entre \textbf{2,5 et 3,4~M€}.  

Le dispositif présente ainsi un profil \textbf{autosoutenable}, générateur de valeur médico-économique pour l’institution.

% ------------------------------------------------------------
% 1.6 Bénéfices institutionnels
% ------------------------------------------------------------

\subsection*{Bénéfices institutionnels}
La structuration du plateau HDJ permet :
\begin{itemize}[leftmargin=1.1cm]
    \item une redistribution des lits MCO mobilisés pour des prises en charge programmables ;
    \item une fluidification des parcours entre la ville, les urgences et l’hospitalisation complète ;
    \item une mutualisation efficiente des compétences spécialisées (IDE expertes, IPA, psychologues, diététicien(ne)s) ;
    \item une amélioration de l’attractivité des postes paramédicaux et médicaux spécialisés ;
    \item une lisibilité renforcée de l’offre ambulatoire digestive et interventionnelle sur le territoire.
\end{itemize}

% ------------------------------------------------------------
% 1.7 Gouvernance (niveau exécutif)
% ------------------------------------------------------------

\subsection*{Gouvernance}
Le plateau HDJ digestif mutualisé repose sur une gouvernance claire et intégrée, avec :
\begin{itemize}[leftmargin=1.1cm]
    \item une responsabilité médicale unique du plateau, garante de la cohérence des parcours et de la sécurité des prises en charge ;
    \item une articulation structurée entre le DMU digestif, le plateau médico-technique et la radiologie interventionnelle ;
    \item un pilotage médico-économique consolidé associant responsables médicaux, encadrement soignant, DIM et direction financière.
\end{itemize}

L’organisation opérationnelle du plateau, le schéma fonctionnel de référence et les modalités détaillées de gouvernance sont présentés dans les sections suivantes.

\end{spacing}

\clearpage
