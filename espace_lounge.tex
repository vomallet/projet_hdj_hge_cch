% ============================================================
% FICHE — ESPACE LOUNGE HDJ
% ============================================================

\section{Espace lounge ambulatoire HDJ — Organisation, périmètre et modalités de fonctionnement}

\begin{spacing}{1.25}

\subsection*{Rationnel organisationnel}

L’espace \emph{lounge} constitue un mode d’accueil ambulatoire spécifique, non allongé, destiné à des patients autonomes ne nécessitant ni lit médicalisé ni surveillance continue, mais requérant un temps médical et paramédical structuré, coordonné et à forte valeur ajoutée clinique.

Il répond à une double logique :
\begin{itemize}[leftmargin=1.1cm]
    \item \textbf{clinique}, en permettant la réalisation concentrée d’évaluations multidisciplinaires complexes sans alitement ;
    \item \textbf{capacitaire}, en optimisant l’utilisation des ressources hospitalières et en évitant un surdimensionnement en lits ou fauteuils de soins.
\end{itemize}

Ce modèle s’inscrit dans les recommandations actuelles du virage ambulatoire et dans les standards internationaux de prise en charge programmée de patients complexes à faible risque immédiat.

\subsection*{Définition et principes}

L’espace lounge est défini comme une \textbf{capacité assise premium}, distincte des lits et fauteuils de soins, reposant sur les principes suivants :
\begin{itemize}[leftmargin=1.1cm]
    \item absence d’alitement et de surveillance continue ;
    \item patients cliniquement stables, autonomes, évalués en amont ;
    \item séquences médicales et paramédicales programmées sur une même demi-journée ou journée ;
    \item coordination pluridisciplinaire centralisée par l’équipe HDJ.
\end{itemize}

Il ne s’agit pas d’une zone d’attente, mais d’un \textbf{espace de soins organisé}, avec des objectifs cliniques définis.

\subsection*{Population éligible}

Relèvent de l’espace lounge des patients adultes :
\begin{itemize}[leftmargin=1.1cm]
    \item hémodynamiquement stables, sans risque immédiat ;
    \item autonomes pour les déplacements et les soins courants ;
    \item ne nécessitant ni perfusion prolongée ni surveillance rapprochée post-acte.
\end{itemize}

Les principales indications incluent :
\begin{itemize}[leftmargin=1.1cm]
    \item parcours hépatométabolique (MASLD/MASH) ;
    \item évaluation de cirrhose compensée ;
    \item parcours addictologique avec retentissement somatique ;
    \item certaines évaluations MICI sélectionnées ;
    \item consultations multidisciplinaires programmées (médecin, IPA, diététique, psychologie).
\end{itemize}

\subsection*{Modalités de fonctionnement}

L’organisation repose sur une programmation séquentielle des interventions :
\begin{itemize}[leftmargin=1.1cm]
    \item accueil infirmier et vérification des critères d’éligibilité ;
    \item évaluations médicales et paramédicales successives ;
    \item examens fonctionnels ou échographiques si indiqués ;
    \item synthèse médicale finale et organisation du suivi.
\end{itemize}

La durée de présence est variable, généralement comprise entre \textbf{2 et 5 heures}, sans contrainte d’alitement.

\subsection*{Encadrement et ressources}

L’espace lounge mobilise :
\begin{itemize}[leftmargin=1.1cm]
    \item IDE formées à l’évaluation clinique ambulatoire ;
    \item IPA assurant coordination, traçabilité et continuité des parcours ;
    \item médecins seniors intervenant sur des plages dédiées ;
    \item professionnels associés (diététicien(ne), psychologue, addictologue).
\end{itemize}

Les locaux sont mutualisés et modulables, sans équipement lourd de surveillance.

\subsection*{Sécurité et critères d’exclusion}

Les patients sont exclus du lounge en cas :
\begin{itemize}[leftmargin=1.1cm]
    \item d’instabilité clinique ou hémodynamique ;
    \item de nécessité de surveillance continue ou de perfusion prolongée ;
    \item de risque immédiat post-acte ;
    \item de dépendance fonctionnelle incompatible avec une prise en charge assise.
\end{itemize}

Un basculement vers un fauteuil de soins ou un lit HDJ est possible à tout moment si l’état clinique le justifie.

\subsection*{Apport capacitaire et médico-économique}

L’espace lounge constitue un levier majeur d’efficience :
\begin{itemize}[leftmargin=1.1cm]
    \item augmentation du nombre de patients pris en charge sans création de lits ;
    \item réduction des hospitalisations conventionnelles évitables ;
    \item optimisation du temps médical spécialisé ;
    \item amélioration de l’expérience patient.
\end{itemize}

Il permet d’absorber une part significative de l’activité ambulatoire spécialisée à coût marginal limité, tout en maintenant des standards élevés de qualité et de sécurité.

\end{spacing}
