% ============================================================
% 1. E-SUMMARY — SYNTHÈSE EXÉCUTIVE
% ============================================================

\titleformat{\subsection}[runin]
  {\bfseries\color{APHPblue}}
  {}
  {0pt}
  {}

\begin{spacing}{1.30}

% ------------------------------------------------------------
% 1.1 Contexte et opportunité institutionnelle
% ------------------------------------------------------------

\subsection*{Contexte et opportunité institutionnelle —}\ignorespaces
La création d'une unité fonctionnelle (UF) dédiée aux Hôpitaux de Jour digestifs mutualisés s'inscrit pleinement dans les objectifs nationaux du virage ambulatoire. Elle repose sur une opportunité organisationnelle majeure — la libération de locaux dédiés — permettant de regrouper sur un site unique des activités ambulatoires aujourd'hui dispersées, tout en améliorant leur lisibilité et leur efficience.

Ce regroupement vise à constituer un dispositif évolutif, aligné sur les recommandations nationales et internationales de référence (EASL, AFEF, SNFGE, INCa, SFR/SFICV), et adapté à la prise en charge de patients complexes nécessitant des parcours de soins programmés, sécurisés et coordonnés.

% ------------------------------------------------------------
% 1.2 Objectifs stratégiques du projet
% ------------------------------------------------------------

\subsection*{Objectifs stratégiques du projet —}
Le projet poursuit cinq objectifs stratégiques structurants :
\begin{itemize}[leftmargin=1.1cm]
    \item améliorer l'accès rapide, sécurisé et lisible aux explorations et aux traitements spécialisés ;
    \item réduire les hospitalisations conventionnelles évitables par un recours structuré et coordonné à l'ambulatoire ;
    \item optimiser l'utilisation des ressources humaines expertes par une organisation mutualisée, transversale et efficiente ;
    \item renforcer l'attractivité institutionnelle et académique du site dans un contexte durable de tension sur les compétences spécialisées ;
    \item développer un parcours patient hospitalo-universitaire favorisant l'inclusion des patients dans des essais thérapeutiques.
\end{itemize}

% ------------------------------------------------------------
% 1.3 Population cible globale
% ------------------------------------------------------------

\subsection*{Population cible globale —}
Le plateau HDJ s'adresse à une population adulte présentant :
\begin{itemize}[leftmargin=1.1cm]
    \item des pathologies hépato-biliaires (Maladies chroniques du foie et des voies biliaires, cirrhoses compensées et décompensées) ;
    \item des troubles addictologiques avec retentissement somatique ;
    \item des maladies inflammatoires chroniques de l'intestin (MICI) ;
    \item des cancers digestifs nécessitant des traitements systémiques ambulatoires ;
    \item des patients pris en charge en radiologie interventionnelle pour surveillance post-examen ;
    \item des patients en cours d'évaluation thérapeutique ou de suivi de traitements oraux.
\end{itemize}

Ces patients présentent un besoin élevé d'évaluations programmables, répétées et sécurisées, compatible avec une prise en charge ambulatoire structurée.

% ------------------------------------------------------------
% 1.4 Volumétrie et trajectoire d'activité
% ------------------------------------------------------------

\subsection*{Volumétrie et trajectoire d'activité —}
Le plateau dispose de 18 places réparties sur trois niveaux (6 lits médicalisés, 6 fauteuils de soins, 6 fauteuils lounge). L'objectif opérationnel à maturité médico-économique repose toutefois sur une \textbf{trajectoire d'activité cible comprise entre 4\,500 et 5\,000 séances annuelles}, compatible avec une montée en charge progressive, une sélectivité médicale raisonnée et un taux d'occupation volontairement inférieur au plafond théorique. Cette trajectoire est anticipée sur une période de \textbf{24 à 36 mois}, sur la base des données historiques locales, des besoins territoriaux identifiés et d'hypothèses prudentes de croissance.

% ------------------------------------------------------------
% 1.5 Recettes consolidées
% ------------------------------------------------------------

\subsection*{Recettes consolidées —}
Sur la base des tarifs moyens pondérés par filière et des volumes projetés, les recettes consolidées du plateau HDJ sont estimées à \textbf{environ 2,9~M€ par an à maturité (plafond 3,2~M€)}, avec une fourchette comprise entre \textbf{2,8 et 3,6~M€}.

Le dispositif présente ainsi un profil \textbf{autosoutenable}, générateur de valeur médico-économique pour l'institution. Les recettes sont centralisées sur l'UF HDJ puis redistribuées vers les UF cliniques contributrices selon une clé de répartition validée en comité de pilotage.

% ------------------------------------------------------------
% 1.6 Bénéfices institutionnels
% ------------------------------------------------------------

\subsection*{Bénéfices institutionnels —}
La structuration du plateau HDJ permet :
\begin{itemize}[leftmargin=1.1cm]
    \item une redistribution ciblée des lits MCO mobilisés pour des prises en charge programmables ;
    \item une fluidification des parcours entre la ville, les urgences et l'hospitalisation complète ;
    \item une mutualisation efficiente des compétences spécialisées (IDE expertes, IPA, psychologie, diététique) ;
    \item une amélioration de l'attractivité des postes paramédicaux et médicaux spécialisés ;
    \item une lisibilité renforcée de l'offre ambulatoire digestive et interventionnelle à l'échelle du territoire.
\end{itemize}

% ------------------------------------------------------------
% 1.7 Gouvernance (niveau exécutif)
% ------------------------------------------------------------

\subsection*{Gouvernance —}
Le plateau des HDJ mutualisés repose sur une gouvernance claire et intégrée, associant :
\begin{itemize}[leftmargin=1.1cm]
    \item une responsabilité médicale unique, garante de la cohérence des parcours et de la sécurité des prises en charge ;
    \item une articulation structurée entre les services des pathologies digestives, les plateaux médico-techniques et la radiologie interventionnelle ;
    \item un pilotage médico-économique consolidé associant responsables médicaux, encadrement soignant, DIM et direction financière.
\end{itemize}

% ------------------------------------------------------------
% 1.8 Structure du document
% ------------------------------------------------------------

\subsection*{Structure du document —}
Le présent document est organisé en trois parties :

\begin{enumerate}[leftmargin=1.1cm]
    \item \textbf{Organisation transversale et dimensionnement} — Principes d'organisation du plateau mutualisé, niveaux de prise en charge, zones fonctionnelles, gouvernance et capacité cible.
    \item \textbf{Fiches opérationnelles par filière} — Pour chaque activité (biopsies hépatiques, cirrhoses, hépatométabolique, addictologie, MICI, chimiothérapie, radiologie interventionnelle) : rationnel médical, population éligible, parcours de soins, codage PMSI, projections d'activité et ressources nécessaires.
    \item \textbf{Livraison clé en main} — Organisation fonctionnelle opérationnelle, plan de déploiement, ressources humaines (ETP), modèle économique de l'UF et mécanisme de redistribution des recettes vers les UF cliniques contributrices.
\end{enumerate}

Les annexes fournissent les procédures standardisées et grilles d'évaluation directement utilisables.

\end{spacing}

\clearpage