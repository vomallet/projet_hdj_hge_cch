% ============================================================
% 1. E-SUMMARY — SYNTHÈSE EXÉCUTIVE
% ============================================================
% VERSION FINALE VALIDÉE
% - Suppression du plafond 5 214 (focus sur cible 4 675)
% - Renommage "lounge" → "boxes de consultations programmées"
% - Mise à jour ETP conformément au plan RH
% - Ajout mention phasage et plan RH
% ============================================================

\titleformat{\subsection}[runin]
  {\bfseries\color{APHPblue}}
  {}
  {0pt}
  {}

\begin{spacing}{1.30}

% ------------------------------------------------------------
% 1.0 ENCADRÉ SYNTHÈSE DIRECTION
% ------------------------------------------------------------

\begin{tcolorbox}[
  colback=APHPsoft,
  colframe=APHPblue,
  title=\textbf{Synthèse pour validation directionnelle},
  fonttitle=\bfseries\color{white},
  coltitle=white,
  boxrule=0.8pt,
  arc=2pt,
  left=6pt,
  right=6pt,
  top=4pt,
  bottom=4pt
]


\small
\renewcommand{\arraystretch}{1.25}
\begin{tabular}{@{}p{4.2cm}p{9cm}@{}}
\textbf{Capacité physique} & 18 places (6 lits + 6 fauteuils + 6 boxes consultations) \\
\textbf{Activité cible (croisière)} & 4 675 séances/an (taux d'occupation 90\%) \\
\textbf{Base PMSI 2024} & 2 425 séjours 0-jour transférables \\
\textbf{Recettes projetées} & 2,9 M€/an (croisière) \\
\textbf{ETP PNM} & 7 ETP (pilote) → 11,5 ETP (maturité) \\
\textbf{ETP PM} & 6 demi-j/sem (pilote) → 12 demi-j/sem (maturité) \\
\textbf{Phasage ouverture} & 6 places (M0) → 12 places (M+6) → 18 places (M+12) \\
\textbf{Montée en charge} & 12–24 mois (3 phases avec conditions de passage) \\
\end{tabular}
\end{tcolorbox}

\vspace{1em}

% ------------------------------------------------------------
% 1.1 Contexte et opportunité institutionnelle
% ------------------------------------------------------------

\subsection*{Contexte et opportunité institutionnelle —}\ignorespaces
La création d'une unité fonctionnelle (UF) dédiée aux Hôpitaux de Jour digestifs mutualisés s'inscrit pleinement dans les objectifs nationaux du virage ambulatoire. Elle repose sur une opportunité organisationnelle majeure — la libération de locaux dédiés — permettant de regrouper sur un site unique des activités ambulatoires aujourd'hui dispersées, tout en améliorant leur lisibilité et leur efficience.

Ce regroupement vise à constituer un dispositif évolutif, aligné sur les recommandations nationales et internationales de référence (EASL, AFEF, SNFGE, INCa, SFR/SFICV), et adapté à la prise en charge de patients complexes nécessitant des parcours de soins programmés, sécurisés et coordonnés.

% ------------------------------------------------------------
% 1.2 Objectifs stratégiques du projet
% ------------------------------------------------------------

\subsection*{Objectifs stratégiques du projet —}
Le projet poursuit cinq objectifs stratégiques structurants :
\begin{itemize}[leftmargin=1.1cm]
    \item améliorer l'accès rapide, sécurisé et lisible aux explorations et aux traitements spécialisés ;
    \item réduire les hospitalisations conventionnelles évitables par un recours structuré et coordonné à l'ambulatoire ;
    \item optimiser l'utilisation des ressources humaines expertes par une organisation mutualisée, transversale et efficiente ;
    \item renforcer l'attractivité institutionnelle et académique du site dans un contexte durable de tension sur les compétences spécialisées ;
    \item développer un parcours patient hospitalo-universitaire favorisant l'inclusion des patients dans des essais thérapeutiques.
\end{itemize}

% ------------------------------------------------------------
% 1.3 Population cible globale
% ------------------------------------------------------------

\subsection*{Population cible globale —}
Le plateau HDJ s'adresse à une population adulte présentant :
\begin{itemize}[leftmargin=1.1cm]
    \item des pathologies hépato-biliaires (Maladies chroniques du foie et des voies biliaires, cirrhoses compensées et décompensées) ;
    \item des troubles addictologiques avec retentissement somatique ;
    \item des maladies inflammatoires chroniques de l'intestin (MICI) ;
    \item des cancers digestifs nécessitant des traitements systémiques ambulatoires ;
    \item des patients pris en charge en radiologie interventionnelle pour surveillance post-examen ;
    \item des patients en cours d'évaluation thérapeutique ou de suivi de traitements oraux.
\end{itemize}

Ces patients présentent un besoin élevé d'évaluations programmables, répétées et sécurisées, compatible avec une prise en charge ambulatoire structurée.

% ------------------------------------------------------------
% 1.4 Preuve par les flux — Données PMSI 2024
% ------------------------------------------------------------

\subsection*{Preuve par les flux — Données PMSI 2024 —}
Le dimensionnement repose sur l'analyse rétrospective des séjours de durée 0 jour réalisés en 2024 dans les UF contributrices du site Cochin :

\vspace{0.5em}
\noindent
\begin{tabular}{@{}p{5.5cm}p{2.5cm}p{2.5cm}p{2.5cm}@{}}
\toprule
\textbf{Filière} & \textbf{Réf. 2024} & \textbf{Cible croisière} & \textbf{Évolution} \\
\midrule
Hépatologie (existant) & 374 & 569 & +52\% \\
Hépatologie (création) & — & 950 & Création \\
MICI + Oncologie digestive & 1 901 & 2 201 & +16\% \\
Addictologie & — & 705 & Création \\
Radiologie interventionnelle & 150 & 250 & +67\% \\
\midrule
\textbf{Total} & \textbf{2 425} & \textbf{4 675} & \textbf{+93\%} \\
\bottomrule
\end{tabular}

\vspace{0.5em}
\noindent
La capacité cible (4 675 séances/an à 90\% d'occupation) correspond exactement à la demande projetée, avec une réserve opérationnelle de 10\% pour l'absorption des aléas organisationnels.

% ------------------------------------------------------------
% 1.5 Volumétrie et trajectoire d'activité
% ------------------------------------------------------------

\subsection*{Volumétrie et trajectoire d'activité —}
Le plateau dispose de 18 places réparties sur trois niveaux (6 lits médicalisés, 6 fauteuils de soins, 6 boxes de consultations programmées). L'objectif opérationnel à maturité médico-économique repose sur une \textbf{trajectoire d'activité cible de 4\,675 séances annuelles}, correspondant à un taux d'occupation de 90\%. Cette trajectoire est anticipée sur une période de \textbf{12 à 24 mois}, selon un phasage progressif validé par la direction des soins.

% ------------------------------------------------------------
% 1.6 Recettes consolidées
% ------------------------------------------------------------

\subsection*{Recettes consolidées —}
Sur la base des tarifs moyens pondérés par filière et des volumes projetés, les recettes consolidées du plateau HDJ sont estimées à \textbf{environ 2,9~M€ par an à maturité}.

Le dispositif présente un profil \textbf{autosoutenable}, générateur de valeur médico-économique pour l'institution.

% ------------------------------------------------------------
% 1.7 Plan RH et phasage d'ouverture
% ------------------------------------------------------------

\subsection*{Plan RH et phasage d'ouverture —}
L'ouverture repose sur une stratégie RH sécurisée avec un phasage progressif :

\vspace{0.5em}
\noindent
\begin{tabular}{@{}p{3cm}p{3.5cm}p{2.5cm}p{3.5cm}@{}}
\toprule
\textbf{Phase} & \textbf{Capacité} & \textbf{ETP IDE} & \textbf{Échéance} \\
\midrule
Phase 1 (pilote) & 6 places & 4 ETP & M0 \\
Phase 2 & 12 places & 5,5 ETP & M+6 \\
Phase 3 (maturité) & 18 places & 6–7 ETP & M+12 \\
\bottomrule
\end{tabular}

\vspace{0.5em}
\noindent
Le sourcing des 4 IDE pilotes est sécurisé : 2 ETP par redéploiement/mobilité interne, 1 ETP par mobilité GHU, 1 ETP par recrutement externe (publication M-6). Les conditions de passage inter-phases (taux d'occupation $>$80\%, effectif complet, validation COPIL) garantissent une montée en charge maîtrisée.

% ------------------------------------------------------------
% 1.8 Bénéfices institutionnels
% ------------------------------------------------------------

\subsection*{Bénéfices institutionnels —}
La structuration du plateau HDJ permet :
\begin{itemize}[leftmargin=1.1cm]
    \item une redistribution ciblée des lits MCO mobilisés pour des prises en charge programmables ;
    \item une fluidification des parcours entre la ville, les urgences et l'hospitalisation complète ;
    \item une mutualisation efficiente des compétences spécialisées (IDE expertes, IPA, psychologie, diététique) ;
    \item une amélioration de l'attractivité des postes paramédicaux et médicaux spécialisés ;
    \item une lisibilité renforcée de l'offre ambulatoire digestive et interventionnelle à l'échelle du territoire.
\end{itemize}

% ------------------------------------------------------------
% 1.9 Gouvernance (niveau exécutif)
% ------------------------------------------------------------

\subsection*{Gouvernance —}
Le plateau des HDJ mutualisés repose sur une gouvernance claire et intégrée, associant :
\begin{itemize}[leftmargin=1.1cm]
    \item une responsabilité médicale unique, garante de la cohérence des parcours et de la sécurité des prises en charge ;
    \item une articulation structurée entre les services des pathologies digestives, les plateaux médico-techniques et la radiologie interventionnelle ;
    \item un pilotage médico-économique consolidé associant responsables médicaux, encadrement soignant, DIM et direction financière ;
    \item un COPIL trimestriel validant les passages de phase et les ajustements capacitaires.
\end{itemize}

% ------------------------------------------------------------
% 1.10 Structure du document
% ------------------------------------------------------------

\subsection*{Structure du document —}
Le présent document est organisé en trois parties :

\begin{enumerate}[leftmargin=1.1cm]
    \item \textbf{Organisation transversale et dimensionnement} — Principes d'organisation du plateau mutualisé, niveaux de prise en charge, zones fonctionnelles, gouvernance, capacité cible et plan RH d'ouverture.
    \item \textbf{Fiches opérationnelles par filière} — Pour chaque activité (biopsies hépatiques, cirrhoses, hépatométabolique, addictologie, MICI, chimiothérapie, radiologie interventionnelle) : rationnel médical, population éligible, parcours de soins, codage PMSI, projections d'activité et ressources nécessaires.
    \item \textbf{Livraison clé en main} — Organisation fonctionnelle opérationnelle, plan de déploiement phasé, ressources humaines (ETP), modèle économique de l'UF.
\end{enumerate}

Les annexes fournissent les procédures standardisées et grilles d'évaluation directement utilisables.

\end{spacing}

\clearpage