% ===============================================
% CHAPITRE 0: Introduction générale 
% ===============================================

\begin{spacing}{1.30}

\subsection*{L’AP-HP et le virage ambulatoire}

L’Assistance Publique–Hôpitaux de Paris (AP-HP), premier centre hospitalo-universitaire d’Europe, regroupe 39 hôpitaux organisés en GHU et assure soin, enseignement et recherche pour près de 12 millions d’habitants. Le plan stratégique médical 2023–2028 place le virage ambulatoire au cœur de la transformation, en visant une amélioration de l’accessibilité, des délais et de la fluidité des parcours \cite{APHP2023Plan}.  
L’hôpital de jour (HDJ) constitue un outil majeur de cette stratégie, permettant la réalisation coordonnée, en temps court, d’évaluations diagnostiques et thérapeutiques complexes dans un cadre sécurisé et universitaire.

\subsection*{Le GHU AP–HP.Centre et le site Cochin}

L’Hôpital Cochin–Port-Royal (GHU AP–HP.Centre–Université Paris Cité) constitue un pôle hospitalo-universitaire de référence. Il couvre un large champ d’expertises : hépato-gastroentérologie et oncologie digestive, chirurgie digestive et hépatobiliaire, endocrinologie et maladies métaboliques, rhumatologie, obstétrique, néonatologie, réanimation et soins intensifs spécialisés.  

Il accueille deux unités cliniques d’HGE — le service des maladies du foie et le service d’hépato-gastroentérologie et d’oncologie digestive — constituant des plateformes de recours régionales et nationales \cite{CollegialeHGE2023}. L’ensemble du plateau technique (endoscopie interventionnelle, radiologie interventionnelle, imagerie spécialisée, traitement loco-régional et systémique des cancers digestifs, préparation à la greffe de foie) permet une prise en charge intégrée des pathologies digestives complexes.

\subsection*{Poids épidémiologique des maladies digestives}

Les maladies digestives représentent la première cause d’hospitalisation en médecine–chirurgie–obstétrique en France \cite{DREES2021}. Les évolutions récentes incluent :

\begin{itemize}
\item une augmentation marquée des cancers pancréatiques et hépatiques, désormais parmi les principales causes de mortalité oncologique \cite{UEG2022DigestiveHealth} ;
\item une stabilité de l’incidence du cancer colorectal, mais un besoin croissant en endoscopies ;
\item une progression soutenue des maladies du foie : prévalence de la cirrhose autour de 0,3~\%, 150--200 nouveaux cas/million/an, et près de 15 000 décès annuels \cite{FrenchHepaticFailure_2020} ;
\item une forte croissance des maladies métaboliques du foie (MASLD/MASH), portée par l’épidémie d’obésité et de diabète, conformément aux dernières recommandations européennes \cite{EASL2024MASLD,RN597} ;
\item une hausse continue de la prévalence des MICI, avec environ 250~000 patients en France et plus de 20~000 suivis à l’AP-HP \cite{Ng2017,SNDS_MICI2022}. 
\end{itemize}

Les troubles liés à l’alcool restent un déterminant majeur de morbi-mortalité digestive, première cause de cirrhose et de carcinome hépatocellulaire, avec un poids important sur les hospitalisations \cite{FrenchHepaticFailure_2020}.

\subsection*{Contraintes structurelles et nécessité du développement ambulatoire}

Les services d’HGE font face à une augmentation de la complexité des patients (vieillissement, précarité) alors que le capacitaire se réduit (fermetures de lits, tensions en personnel). De nombreuses indications relevant autrefois de l’hospitalisation conventionnelle peuvent désormais être conduites en HDJ : initiation de biothérapies MICI, immunothérapies, bilans spécialisés, évaluations multidisciplinaires, actes interventionnels simples.  
Le projet médical HGE 2023–2028 désigne l’HDJ comme un levier prioritaire pour réduire les DMS, absorber la croissance épidémiologique et renforcer l’attractivité régionale \cite{CollegialeHGE2023}.

\subsection*{Rôle structurant des Hôpitaux de Jour HGE}

Les HDJ thématiques du pôle répondent à plusieurs missions stratégiques :

\begin{itemize}
\item standardisation des filières diagnostiques et thérapeutiques ;
\item concentration des évaluations spécialisées sur une même journée ;
\item réduction des délais d’accès aux biothérapies, à l’imagerie ou à l’endoscopie ;
\item intégration de compétences pluridisciplinaires (diététique, psychologie, addictologie, IPA, ETP) ;
\item optimisation de la coordination ville–hôpital et de la traçabilité universitaire.
\end{itemize}

Ils constituent aujourd’hui une architecture cohérente, alignée sur les enjeux épidémiologiques et capacitaires du CHU.

\subsection*{Logique du présent document}

Ce document rassemble les fiches techniques standardisées des HDJ du pôle d’HGE et prépare la construction d’un HDJ mutualisé à horizon 2027.  
Il décrit l’activité actuelle (hépatologie, MICI, addictologie, hépatométabolique, radiologie interventionnelle, oncologie digestive) et fournit la base opérationnelle du futur HDJ commun.  
La libération des locaux occupés aujourd’hui par l’hématologie rendra possible la création d’un espace unique regroupant des parcours spécialisés, standardisés et pluridisciplinaires, améliorant l’accessibilité, la qualité et la fluidité des prises en charge ambulatoires au sein du GHU AP–HP.Centre.

\end{spacing}
