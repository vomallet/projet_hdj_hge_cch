% ============================================================
% CHAPITRE 2 — INTRODUCTION GÉNÉRALE
% ============================================================

\titleformat{\subsection}[runin]
  {\bfseries\color{APHPblue}}
  {}
  {0pt}
  {}

\begin{spacing}{1.30}

\subsection*{L’AP-HP et le virage ambulatoire}

L’Assistance Publique–Hôpitaux de Paris (AP-HP), premier centre hospitalo-universitaire d’Europe, regroupe 39 hôpitaux organisés en GHU et assure des missions de soins, d’enseignement et de recherche pour près de 12 millions d’habitants.  
Le plan stratégique médical 2023–2028 place le virage ambulatoire au cœur de la transformation de l’institution, avec pour objectifs l’amélioration de l’accessibilité aux soins, la réduction des délais et la fluidification des parcours \cite{APHP2023Plan}.  
Dans ce contexte, l’HDJ constitue un levier structurant, permettant la réalisation coordonnée, sur un temps court, d’évaluations diagnostiques et thérapeutiques complexes dans un cadre sécurisé et universitaire.

\subsection*{Le GHU AP–HP.Centre et le site Cochin}

L’hôpital Cochin–Port-Royal (GHU AP–HP.Centre–Université Paris Cité) est un pôle hospitalo-universitaire de recours national. Il couvre un champ étendu d’expertises médico-chirurgicales — hépato-gastroentérologie et oncologie digestive, chirurgie digestive et hépatobiliaire, endocrinologie, obstétrique, néonatologie, réanimation spécialisée — et dispose d’un plateau technique complet incluant imagerie spécialisée, radiologie interventionnelle, endoscopie interventionnelle, anatomopathologie et traitements systémiques des cancers digestifs.  
Les deux unités cliniques d’hépato-gastroentérologie (maladies du foie ; hépato-gastroentérologie et oncologie digestive) constituent des plateformes de recours régionales et nationales \cite{CollegialeHGE2023}.

\subsection*{Poids épidémiologique des maladies digestives}

Les maladies digestives constituent la première cause d’hospitalisation en médecine–chirurgie–obstétrique en France \cite{DREES2021}. Les évolutions épidémiologiques récentes sont caractérisées par :
\begin{itemize}
\item une augmentation de l’incidence et de la mortalité des cancers hépatiques et pancréatiques \cite{UEG2022DigestiveHealth} ;
\item une progression continue des maladies chroniques du foie, avec une prévalence estimée de la cirrhose autour de 0,3~\% et près de 15\,000 décès annuels \cite{FrenchHepaticFailure_2020} ;
\item une croissance rapide des maladies métaboliques du foie (MASLD/MASH), portée par l’augmentation de l’obésité et du diabète \cite{EASL2024MASLD, RN597} ;
\item une prévalence élevée et durable des maladies inflammatoires chroniques de l’intestin (MICI), concernant environ 250\,000 patients en France \cite{Ng2017, SNDS_MICI2022}.
\end{itemize}

Les pathologies liées à l’alcool demeurent par ailleurs un déterminant majeur de la morbi-mortalité digestive, constituant la première cause de cirrhose et de carcinome hépatocellulaire.

\subsection*{Contraintes structurelles et nécessité du développement ambulatoire}

L’augmentation des besoins en soins spécialisés s’inscrit dans un contexte durable de contraintes capacitaires, marqué par les fermetures de lits et les tensions sur les ressources humaines. De nombreuses situations auparavant prises en charge en hospitalisation conventionnelle relèvent désormais de modalités ambulatoires : bilans spécialisés, évaluations multidisciplinaires, traitements systémiques de courte durée ou actes interventionnels ciblés.

Le projet médical HGE 2023–2028 identifie ainsi le développement structuré de l’hôpital de jour (HDJ) comme un levier essentiel pour réduire les durées moyennes de séjour, absorber la croissance épidémiologique et renforcer l’attractivité du pôle \cite{CollegialeHGE2023}.

\subsection*{Logique et objectifs du présent document}

Le développement d’HDJ mutualisés vise à garantir la dynamique propre à chacune des spécialités impliquées (activité clinique, essais thérapeutiques, parcours patients), à maintenir un haut niveau de qualité des prises en charge et à optimiser l’utilisation des moyens alloués, en particulier les capacités d’accueil du bâtiment Achard.

Le présent document décrit l’activité actuelle des principales filières concernées (hépatologie, MICI, addictologie, hépatométabolique, radiologie interventionnelle, oncologie digestive) et prépare leur évolution vers une organisation consolidée, rendue possible par la libération des locaux anciennement occupés par l’hématologie (6eme étage du batiment Achard).

Cette réorganisation a pour objectif d’améliorer l’accessibilité, la qualité et la fluidité des prises en charge ambulatoires au sein du GHU AP–HP.Centre.


\end{spacing}
