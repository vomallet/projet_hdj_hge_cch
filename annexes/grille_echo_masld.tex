% --- MODE COMPACT POUR TOUT FAIRE TENIR SUR UNE PAGE ---
\small                                      % ↓ Réduction de la police
\renewcommand{\arraystretch}{0.92}          % ↓ Lignes plus compactes
\setlength{\tabcolsep}{3pt}                 % ↓ Colonnes plus serrées

\begin{center}
\begin{minipage}{0.95\textwidth}

\rowcolors{1}{}{APHPsoft!40}
\setlength{\arrayrulewidth}{0.25pt}

\begin{tabular}{p{5.7cm} p{7.6cm}}

\multicolumn{2}{l}{\textbf{1. Morphologie hépatique}} \\
\hline
& \ch Taille normale \\
& \ch Hépatomégalie \\
& \ch Contours irréguliers \\
& \ch Nodularité \\
\\[-2mm]

\multicolumn{2}{l}{\textbf{2. Texture parenchymateuse}} \\
\hline
& \ch Homogène \\
& \ch Hétérogène \\
& \ch Réflectivité augmentée \\
\\[-2mm]

\multicolumn{2}{l}{\textbf{3. Stéatose (échogénicité)}} \\
\hline
& \ch Grade 0 \\
& \ch Grade 1 \\
& \ch Grade 2 \\
& \ch Grade 3 \\
\\[-2mm]

\multicolumn{2}{l}{\textbf{4. Atténuation du faisceau}} \\
\hline
& \ch Normale \\
& \ch Accrue \\
\\[-2mm]

\multicolumn{2}{l}{\textbf{5. Visualisation vasculaire}} \\
\hline
& \ch Normale \\
& \ch Dégradée \\
\\[-2mm]

\multicolumn{2}{l}{\textbf{6. Doppler portale}} \\
\hline
& \ch Hépatopète \\
& \ch Diminué \\
& \ch Hépatofuge \\
& Vitesse : \rule{2.1cm}{0.4pt} cm/s \\
\\[-2mm]

\multicolumn{2}{l}{\textbf{7. Veines hépatiques}} \\
\hline
& \ch Triphasiques \\
& \ch Aplatis \\
& \ch Monophasique \\
\\[-2mm]

\multicolumn{2}{l}{\textbf{8. Signes HTP}} \\
\hline
& \ch Splénomégalie \\
& \ch Ascite \\
& \ch Collatérales \\
\\[-2mm]

\multicolumn{2}{l}{\textbf{9. Diamètre porte}} \\
\hline
& \rule{2.1cm}{0.4pt} mm (N ≤ 12) \\
\\[-2mm]

\multicolumn{2}{l}{\textbf{10. Commentaires}} \\
\hline
\multicolumn{2}{l}{
\rule{0.94\textwidth}{1.6cm}
} \\

\end{tabular}

\vspace{1mm}
{\footnotesize\itshape
Réf. : EASL 2024 ; AFEF 2024 ; WFUMB/EFSUMB 2023 ; AASLD 2023.
}

\end{minipage}
\end{center}

\normalsize
