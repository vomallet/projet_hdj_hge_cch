% ============================================================
% ANNEXE 2 — Fiche de traçabilité HDJ addictologie
% ============================================================

\section*{Annexe 2 — Fiche de traçabilité HDJ addictologie}

\begin{table}[h!]
\centering
\renewcommand{\arraystretch}{1.30}
\rowcolors{2}{white}{APHPsoft}

\begin{tabular}{p{5cm} p{10cm}}
\toprule
\rowcolor{APHPsoft}
\textbf{Élément} & \textbf{Contenu à renseigner} \\
\midrule

Identité du patient &
Nom, prénom, date de naissance, IPP \\

Date de la séance &
JJ/MM/AAAA \\

Motif de venue &
Évaluation / Réduction des risques / Sevrage ambulatoire / Consolidation / Synthèse \\

Intervenants présents &
Médecin (addicto/hépato), IDE, psychologue, diététicien, AS, éducateur spécialisé, psychomotricien, socio-esthéticienne, art-thérapeute, APA \\

Activités réalisées &
Lister les activités avec \textbf{numéros de l’annexe 1 (1–47)} + court descriptif :  
• médiations (3–4) : socio-esthétique, art-thérapie, écriture, revue de presse  
• APA (activité physique adaptée)  
• ateliers cuisine (10)  
• groupes (7, 12, 17, 20, 22, 30, 33…) \\

Évaluation clinique &
Anamnèse, CIWA, AUDIT, paramètres vitaux, consommation récente, risques, événements intercurrents \\

Évaluation psycho-sociale &
Logement, emploi, isolement, précarité, vulnérabilités, violences, ressources mobilisables \\

Suivi somatique &
Examens réalisés : biologie, ECG, EFR, imagerie, Fibroscan ; résultats pertinents pour la PEC \\

Décision médicale &
Plan thérapeutique actualisé, sevrage, adaptation des traitements, réduction des risques, orientation (CSAPA, ville, HDJ, SSR, hospitalisation) \\

Événements indésirables &
Incidents, aggravations, EI médicamenteux, rupture de contact, consommation lors du sevrage \\

Continuité des soins &
RDV programmés, coordination médecin traitant / CSAPA / psychologue / ville / SSR ; documents remis \\

Signature &
Intervenant(s) responsable(s) \\
\bottomrule
\end{tabular}

\caption{Fiche opérationnelle de traçabilité — HDJ addictologie (version enrichie)}
\end{table}
