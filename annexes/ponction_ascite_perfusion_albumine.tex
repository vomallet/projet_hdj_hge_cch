% -------------------------------
% ANNEXE B.1 — Ponction d’ascite + albumine
% -------------------------------
\section*{Annexe B.1 — Ponction d’ascite avec perfusion d’albumine}
\addcontentsline{toc}{subsection}{Annexe B.1 — Ponction d’ascite avec perfusion d’albumine}
\needspace{10\baselineskip}

\begin{itemize}[leftmargin=1.1cm]
  \item \textbf{Ponction d’ascite (HPHB003).}  
  Acte sous asepsie stricte, évacuation adaptée à la fragilité du patient. Volume évacué, difficultés techniques et paramètres hémodynamiques.
  \item \textbf{Entretien médical individualisé.}  
  État clinique, Child-Pugh, MELD, critères AKI/HRS, indication d’albumine, note dédiée.
  \item \textbf{Surveillance particulière.}  
  Constantes rapprochées, monitorage neurologique, dépistage précoce des complications.
  \item \textbf{Acte infirmier.}  
  Pose de voie difficile, compression prolongée, surveillance locale.
  \item \textbf{Traçabilité.}  
  Fiche HPHB003, CCAM, note médicale, feuille de surveillance.
\end{itemize}

\clearpage
