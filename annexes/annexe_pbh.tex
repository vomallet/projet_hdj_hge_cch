% ============================================================
% ANNEXE — PROCÉDURE STANDARDISÉE
% BIOPSIE HÉPATIQUE PERCUTANÉE EN HDJ
% ============================================================

% \section*{Annexe — Procédure standardisée de biopsie hépatique percutanée en HDJ}
% \addcontentsline{toc}{section}{Annexe — Procédure PBH HDJ}

\subsection*{1. Objectif de la procédure}

La biopsie hépatique percutanée (PBH) vise à obtenir un prélèvement tissulaire hépatique
de qualité diagnostique suffisante, dans des conditions optimales de sécurité, de traçabilité
et de reproductibilité, en hospitalisation de jour.

\subsection*{2. Pré-requis avant programmation}

\textbf{Critères cliniques}
\begin{itemize}[leftmargin=1.1cm]
  \item patient adulte, cliniquement stable ;
  \item indication validée par un hépatologue senior ;
  \item absence de décompensation hépatique aiguë ;
  \item absence d’infection non contrôlée ;
  \item consentement éclairé signé.
\end{itemize}

\textbf{Bilan biologique (datant de moins de 7 jours)}
\begin{itemize}[leftmargin=1.1cm]
  \item plaquettes \(\geq 50\,000/mm^3\) ;
  \item INR \(\leq 1{,}5\) ;
  \item TCA \(\leq 1{,}5 \times témoin\) ;
  \item groupe sanguin et RAI disponibles.
\end{itemize}

\textbf{Traitements}
\begin{itemize}[leftmargin=1.1cm]
  \item arrêt ou adaptation des anticoagulants et antiagrégants selon protocole validé ;
  \item absence de contre-indication anesthésique locale.
\end{itemize}

\subsection*{3. Échographie pré-procédure obligatoire}

Une échographie hépatique est réalisée systématiquement avant la PBH afin de :
\begin{itemize}[leftmargin=1.1cm]
  \item confirmer l’indication et le site de ponction ;
  \item éliminer une lésion focale hépatique du lobe droit sur le trajet prévu ;
  \item évaluer l’anatomie vasculaire et biliaire ;
  \item repérer les structures à risque (vésicule biliaire, colon, poumon) ;
  \item vérifier l’absence d’ascite significative non contrôlée.
\end{itemize}

\textbf{La biopsie est réalisée préférentiellement sur le foie droit, en l’absence de lésion focale sur le trajet de ponction.}

\subsection*{4. Réalisation de la biopsie}

\begin{itemize}[leftmargin=1.1cm]
  \item installation du patient en décubitus dorsal ou latéral gauche ;
  \item antisepsie cutanée rigoureuse et champ stérile ;
  \item anesthésie locale à la lidocaïne ;
  \item guidage échographique en temps réel ;
  \item utilisation d’un dispositif de biopsie automatique (calibre 16G ou 18G) ;
  \item obtention d’au moins une carotte de longueur \(\geq 15\,mm\) si possible ;
  \item traçabilité du nombre de passages et des prélèvements obtenus.
\end{itemize}

\subsection*{5. Hémostase post-biopsie}

\begin{itemize}[leftmargin=1.1cm]
  \item compression manuelle immédiate du point de ponction ;
  \item contrôle échographique ciblé si nécessaire ;
  \item pansement compressif maintenu ;
  \item absence de saignement actif avant sortie de salle.
\end{itemize}

\subsection*{6. Surveillance post-procédure}

Surveillance spécialisée en HDJ pendant \textbf{6 heures} :
\begin{itemize}[leftmargin=1.1cm]
  \item constantes hémodynamiques régulières (TA, FC) ;
  \item évaluation de la douleur ;
  \item surveillance du point de ponction ;
  \item recherche de signes hémorragiques ou de complications.
\end{itemize}

Un hémogramme de contrôle est réalisé en cas de symptôme ou de doute clinique.

\subsection*{7. Critères de sortie}

\begin{itemize}[leftmargin=1.1cm]
  \item stabilité hémodynamique ;
  \item douleur contrôlée ;
  \item absence de signe hémorragique ;
  \item pansement sec ;
  \item remise de consignes écrites et contact médical.
\end{itemize}

\subsection*{8. Traçabilité}

La procédure fait l’objet d’une traçabilité complète incluant :
\begin{itemize}[leftmargin=1.1cm]
  \item indication médicale ;
  \item compte-rendu de l’échographie pré-biopsie ;
  \item modalités techniques de la PBH ;
  \item nombre et qualité des prélèvements ;
  \item surveillance post-procédure ;
  \item compte-rendu anatomopathologique.
\end{itemize}
