% ============================================================
% ANNEXE 1 — Activités réalisables en HDJ addictologie
% Source principale : Instruction DGOS/R4/R1/2016/350
% Version enrichie pour HDJA : art-thérapie, socio-esthétique,
% APA, revue de presse, écriture thérapeutique
% ============================================================

% \section*{Annexe 1 — Liste des activités réalisables en HDJ addictologie}

\begin{table}[h!]
\centering
\renewcommand{\arraystretch}{1.25}
\rowcolors{2}{APHPsoft}{white}

\begin{tabular}{p{1.2cm} p{13.5cm}}
\toprule
\rowcolor{APHPsoft}
\textbf{N°} & \textbf{Activité} \\
\midrule

1 & Entretien motivationnel individuel \\
2 & Entretien motivationnel collectif \\
3 & Activité individuelle de médiation thérapeutique (artistique, corporelle, expressive) : art-thérapie, écriture thérapeutique, revue de presse \\
4 & Activité collective de médiation thérapeutique : art-thérapie, écriture collective, groupes de parole thématiques \\
5 & Activité individuelle de réadaptation / maintien des fonctions psychosociales \\
6 & Activité collective de réadaptation / maintien des fonctions psychosociales \\
7 & Groupe de parole (addictologie, retour d’expérience, gestion des émotions) \\
8 & Consultation nutritionnelle individuelle \\
9 & Conseils nutritionnels individuels / prescription diététique \\
10 & Atelier cuisine thérapeutique ou atelier alimentaire collectif \\
11 & Thérapie cognitive et comportementale (individuelle) \\
12 & Thérapie cognitive et comportementale (collective) \\
13 & Prise en charge psychomotrice individuelle \\
14 & Prise en charge psychomotrice collective \\
15 & Évaluation individuelle cognitive / fonctions exécutives \\
16 & Prise en charge cognitive individuelle thérapeutique \\
17 & Prise en charge cognitive collective thérapeutique \\
18 & Activité collective de prise en charge des altérations psychomotrices \\
19 & Éducation thérapeutique du patient (individuelle) \\
20 & Éducation thérapeutique du patient (collective) \\
21 & Information individuelle (alcool, santé, prévention, traitements) \\
22 & Information collective (prévention, risques, santé publique) \\
23 & Activités d’éducation ou d’information thérapeutique \\
24 & Évaluation socio-thérapeutique individuelle \\
25 & Activité individuelle relative aux activités de vie quotidienne \\
26 & Activité collective relative aux activités de vie quotidienne \\
27 & Assistance éducative individuelle \\
28 & Assistance éducative collective \\
29 & Développement ou restauration des compétences sociales (individuel) \\
30 & Développement ou restauration des compétences sociales (collectif) \\
31 & Activité d’aide à l’emploi / insertion \\
32 & Entretien individuel de relation d’aide sociale \\
33 & Entretien collectif de relation d’aide sociale \\
34 & Préparation au retour à domicile / maintien du lien social \\
35 & Activité systémique (thérapie familiale, de couple) \\
36 & Activité collective avec l’entourage / proches \\
37 & Activité systémique structurée \\
38 & Consultation médicale spécialisée (addictologie, hépatologie, psychiatrie) \\
39 & Consultation médicale de synthèse (au moins hebdomadaire) \\
40 & Entretien infirmier évaluatif ou thérapeutique \\
41 & Entretien psychologique individuel \\
42 & Entretien individuel avec éducateur spécialisé \\
43 & Entretien individuel de rééducation (kiné / ergo / psychomotricien) \\
44 & Activité pluriprofessionnelle d’évaluation \\
45 & Activité pluriprofessionnelle de synthèse \\
46 & Participation à un staff pluridisciplinaire \\
47 & Activité d’évaluation ou de synthèse interprofessionnelle (fin de cure) \\

\bottomrule
\end{tabular}

\caption{Liste consolidée des activités réalisables en HDJ addictologie (DGOS 2016 + médiations utilisées au HDJA)}
\end{table}
