% ============================================================
% HÔPITAL DE JOUR — RADIOLOGIE INTERVENTIONNELLE
% ============================================================

\setcounter{table}{0}
\setcounter{figure}{0}

\subsection{Rationnel médical}
\needspace{8\baselineskip}
\begin{spacing}{1.30}

L’essor des techniques mini-invasives permet désormais d’envisager la réalisation d’un nombre croissant d’actes interventionnels en ambulatoire, tout en maintenant un haut niveau de sécurité. Les recommandations de la SFR, de la SFICV et du CIRSE confirment que l’organisation en hôpital de jour constitue le modèle de référence pour optimiser l’efficience, en réduisant les durées d’hospitalisation et en améliorant l’expérience patient \cite{Lakshminarayan2023, cirse2021}.

La sélection pré-procédure repose sur la stabilité clinique, l’évaluation du risque hémorragique et la garantie de conditions sociales compatibles avec un retour sécurisé à domicile. Les actes les plus fréquents en HDJ incluent les biopsies percutanées, les drainages, les thermoablations, les gestes vasculaires non artériels et les interventions sur dispositifs veineux.
La surveillance immédiate (1–6~h) permet de dépister les complications précoces, principalement la douleur et le saignement. Les données disponibles montrent que ce modèle n’augmente ni les réadmissions ni les recours non programmés \cite{Chen2017}. La sortie repose sur la stabilité clinique et la bonne compréhension des consignes.

Enfin, l’utilisation actuelle de lits d’hospitalisation conventionnelle — notamment en chirurgie — pour les biopsies profondes (foie, rein) nécessitant une surveillance allongée de 6~h pourrait être remplacée par un HDJ situé à proximité immédiate du plateau de radiologie interventionnelle (Achard). Cette évolution permettrait de libérer des lits d’hospitalisation complète, de réduire les besoins en brancardage et de diminuer les retards liés aux transferts intra-hospitaliers.

\end{spacing}

\clearpage

% ============================================================
% OBJECTIFS
% ============================================================

\subsection{Objectifs}
\needspace{6\baselineskip}

\begin{center}
\fcolorbox{APHPdark}{APHPsoft}{
\begin{minipage}{0.95\textwidth}
\vspace{0.9em}
\begin{itemize}[leftmargin=1.1cm]
\item sécuriser la réalisation ambulatoire des actes interventionnels mini-invasifs ;
\item structurer une filière courte, standardisée et coordonnée ;
\item réduire les séjours conventionnels et optimiser l’utilisation du plateau technique ;
\item harmoniser la sélection, la préparation et la surveillance post-geste ;
\item améliorer l’expérience patient grâce à un parcours fluide et une information claire.
\end{itemize}
\vspace{0.9em}
\end{minipage}}
\end{center}

\clearpage

% ============================================================
% POPULATION ÉLIGIBLE
% ============================================================

\subsection{Population éligible}
\needspace{6\baselineskip}

\begin{itemize}[leftmargin=1.1cm]
\item patients ASA~I–II (ASA~III après avis spécialisé) ;
\item absence de comorbidité décompensée (cardiaque, respiratoire, rénale) ;
\item hémostase compatible avec le geste ;
\item présence d’un accompagnant pour le retour et la première nuit ;
\item compréhension et adhésion aux consignes post-procédure.
\end{itemize}

\clearpage

% ============================================================
% PARCOURS DE SOINS
% ============================================================

\subsection{Parcours de soins}
\needspace{8\baselineskip}

\begin{figure}[!ht]
\centering
\caption{Parcours patient — HDJ Radiologie interventionnelle}
\vspace{0.8cm}

\begin{tikzpicture}[
node distance=1.6cm,
box/.style={
    rectangle, rounded corners=3pt,
    draw=APHPdark, thick,
    text width=9cm,
    minimum height=1.5cm,
    align=center, fill=APHPsoft
}]
\node[box] (tri) {Orientation (consultation spécialisée, demande ciblée, ville)};
\node[box, below=1.4cm of tri] (e1) {Évaluation pré-procédure : anamnèse, traitements, hémostase, consentement};
\node[box, below=1.4cm of e1] (e2) {Acte interventionnel : biopsie, drainage, thermoablation, geste vasculaire};
\node[box, below=1.4cm of e2] (e3) {Surveillance 1–3~h : constantes, douleur, point de ponction};
\node[box, below=1.4cm of e3] (syn) {Synthèse médicale, consignes écrites, sortie sécurisée};

\draw[->, thick, APHPdark] (tri) -- (e1);
\draw[->, thick, APHPdark] (e1) -- (e2);
\draw[->, thick, APHPdark] (e2) -- (e3);
\draw[->, thick, APHPdark] (e3) -- (syn);
\end{tikzpicture}
\end{figure}

\clearpage

% ============================================================
% ORGANISATION
% ============================================================

\subsection{Organisation}
\needspace{6\baselineskip}

\begin{itemize}[leftmargin=1.1cm]
\item Direction : \textbf{Pr Anthony Dohan}
\item Durée moyenne : 4-5~heures
\item Lieu : secteur HDJ — hépato-gastroentérologie / oncologie digestive
\item Ressources : médecin senior, infirmier expert
\end{itemize}

\clearpage

% ============================================================
% CODAGE, TARIFS ET VOLUMÉTRIE DE RÉFÉRENCE
% ============================================================

\subsection{Codage, tarifs et volumétrie de référence}
\needspace{12\baselineskip}

\begin{sidewaystable}[p]
\centering
\renewcommand{\arraystretch}{1.25}
\rowcolors{2}{APHPsoft}{white}

\begin{tabularx}{\textwidth}{
p{4.8cm}
X
>{\centering\arraybackslash}p{2cm}
>{\centering\arraybackslash}p{1.8cm}
>{\centering\arraybackslash}p{2cm}
>{\centering\arraybackslash}p{1.8cm}
>{\centering\arraybackslash}p{2.6cm}
}
\toprule
\rowcolor{APHPsoft}
\textbf{Type de séance} &
\textbf{DP / DR / DAS} &
\textbf{GHM} &
\textbf{GHS} &
\textbf{Tarif 2025} &
\textbf{Volume 2024} &
\textbf{Recette 2024} \\
\midrule

Biopsie percutanée &
DP : Rxxxx ou Cxxxx selon organe ; DAS pertinents\newline
CCAM : HLHB001, HLHJ003, HLHJ006 &
07M06T & 2528 & 1\,061~€ &
60 &
63\,660~€ \\

Drainage percutané &
DP : K65, K83, N13 ; DAS : infection, douleur\newline
CCAM : HMJH006 &
07M02T & 2518 & 891~€ &
51 &
45\,441~€ \\

Ablation thermo-induite &
DP : C22.x, K76.x, tumeurs diverses\newline
CCAM : HLNM001 &
07K061 & 2570 & 3\,370~€ &
39 &
131\,430~€ \\

\midrule
\textbf{Total annuel} & -- & -- & -- & -- &
\textbf{150} &
\textbf{240\,531~€} \\
\bottomrule
\end{tabularx}

\caption{Codage, tarifs et volumétrie de référence — HDJ Radiologie interventionnelle (Données 2024 Pr A. Dohan)}
\end{sidewaystable}

\clearpage


% ============================================================
% TRACABILITÉ
% ============================================================

\subsection{Traçabilité minimale}

\begin{table}[h!]
\centering
\renewcommand{\arraystretch}{1.20}
\rowcolors{2}{APHPsoft}{white}

\begin{tabular}{p{5cm} p{9cm}}
\toprule
\rowcolor{APHPsoft}
\textbf{Intervention} & \textbf{Éléments requis} \\
\midrule
Biopsie &
Indication ; imagerie préalable ; type d’aiguille ; nombre de prélèvements ; incidents ; contrôle post-geste. \\
Drainage &
Site ; guidage ; volume et aspect ; prélèvements ; mise en tension ; contrôle immédiat. \\
Actes vasculaires &
Voie d’abord ; matériel ; perméabilité post-geste ; incidents. \\
Synthèse de sortie &
Compte rendu structuré ; mesures de sécurité ; consignes écrites ; suivi programmé. \\
\bottomrule
\end{tabular}

\caption{Traçabilité — HDJ Radiologie interventionnelle}
\end{table}

\clearpage

% ============================================================
% PROJECTIONS D’ACTIVITÉ
% ============================================================

\subsection{Projections d’activité et recettes prévisionnelles}
\needspace{8\baselineskip}

\noindent\textbf{Hypothèses.}  
Volumétrie de référence 2024 : \textbf{150 actes}. \\
Tarif moyen pondéré : \textbf{1\,603~€ / séance}. \\
Croissance attendue liée à l’ouverture des locaux d’hématologie (Achard), 
au rapprochement du plateau interventionnel et à la diminution des 
hospitalisations conventionnelles : \textbf{+50 actes/an}.

\begin{table}[h!]
\centering
\renewcommand{\arraystretch}{1.20}
\rowcolors{2}{APHPsoft}{white}
\begin{tabular}{
p{4.3cm}
>{\centering\arraybackslash}p{2.2cm}
>{\centering\arraybackslash}p{2.3cm}
>{\centering\arraybackslash}p{3.0cm}
}
\toprule
\rowcolor{APHPsoft}
\textbf{Phase} & \textbf{Volume estimé} & \textbf{Tarif moyen} & \textbf{Recette brute} \\
\midrule
Amorce        & 150 & 1\,603~€ & 240\,450~€ \\
Montée        & 200 & 1\,603~€ & 320\,600~€ \\
Croisière     & 250 & 1\,603~€ & 400\,750~€ \\
\bottomrule
\end{tabular}
\caption{Projections d’activité — HDJ Radiologie interventionnelle}
\end{table}

\clearpage


% ============================================================
% VALIDATION
% ============================================================

\begin{center}
\begin{tabular}{p{4cm} p{7cm} p{4cm}}
\toprule
\rowcolor{APHPsoft}
\textbf{Date d'envoi} & \textbf{Relecteur} & \textbf{Validation} \\
\midrule
01/12/2025 & Pr V.\,Mallet & 08/12/2026 \\
04/12/2025 & Pr A.\,Dohan  & NA \\
NA         & Dr S.\,Bouam  & NA \\
NA         & Pr R.\,Coriat & NA \\
\bottomrule
\end{tabular}
\end{center}

\clearpage
