% ============================================================
% HÔPITAL DE JOUR — RADIOLOGIE INTERVENTIONNELLE
% ============================================================

\subsection{Rationnel médical}
\setcounter{table}{0}
\needspace{8\baselineskip}

\begin{spacing}{1.30}

L’essor des techniques mini-invasives permet la réalisation d’un nombre croissant de procédures interventionnelles en ambulatoire, tout en maintenant des standards élevés de sécurité. Les recommandations de la SFR, de la SFICV et du CIRSE soulignent que l’organisation en hôpital de jour optimise l’efficience, réduit les durées d’hospitalisation et améliore l’expérience patient \cite{Lakshminarayan2023, cirse2021}.

La sélection pré-procédure repose sur la stabilité clinique, la maîtrise du risque hémorragique et l’existence de conditions sociales compatibles avec un retour sécurisé à domicile. Les actes les plus courants incluent les biopsies percutanées, drainages, thermoablations, gestes vasculaires non artériels et interventions sur cathéters.

Une surveillance immédiate (1–3~h) permet d’identifier les complications précoces, principalement douleur et saignement. Les données disponibles montrent que cette stratégie ne majore ni les réadmissions ni les recours non programmés \cite{Chen2017}. La sortie est conditionnée à la stabilité clinique, à l’absence de saignement, et à la bonne compréhension des consignes.

\end{spacing}

\clearpage

% ============================================================
% OBJECTIFS
% ============================================================

\subsection{Objectifs}
\needspace{6\baselineskip}

\begin{center}
\fcolorbox{APHPdark}{APHPsoft}{
\begin{minipage}{0.95\textwidth}
\vspace{0.9em}
\begin{itemize}[leftmargin=1.1cm]
\item sécuriser la réalisation ambulatoire des actes interventionnels mini-invasifs ;
\item structurer une filière courte, standardisée et coordonnée ;
\item réduire les séjours conventionnels et optimiser le plateau technique ;
\item harmoniser la sélection des patients, la préparation et la surveillance post-geste ;
\item améliorer l’expérience patient grâce à un parcours fluide et une information claire.
\end{itemize}
\vspace{0.9em}
\end{minipage}}
\end{center}

\clearpage

% ============================================================
% POPULATION ÉLIGIBLE
% ============================================================

\subsection{Population éligible}
\needspace{6\baselineskip}

\begin{itemize}[leftmargin=1.1cm]
\item patients ASA~I–II (ASA~III après avis spécialisé) ;
\item absence de comorbidité décompensée (cardiaque, respiratoire, rénale) ;
\item bilan d’hémostase compatible avec le geste ;
\item accompagnant obligatoire pour le retour et la première nuit ;
\item compréhension et adhésion aux consignes post-procédure.
\end{itemize}

\clearpage

% ============================================================
% PARCOURS DE SOINS
% ============================================================

\subsection{Parcours de soins}
\needspace{8\baselineskip}

\begin{figure}[!ht]
\centering
\caption{Parcours patient — HDJ Radiologie interventionnelle}
\vspace{0.8cm}

\begin{tikzpicture}[
node distance=1.6cm,
box/.style={
    rectangle, rounded corners=3pt,
    draw=APHPdark, thick,
    text width=9cm,
    minimum height=1.5cm,
    align=center,
    fill=APHPsoft
}]
\node[box] (tri) {Orientation (consultation spécialisée, demande ciblée, ville)};
\node[box, below=1.4cm of tri] (e1) {Évaluation pré-procédure : anamnèse, traitements, hémostase, consentement};
\node[box, below=1.4cm of e1] (e2) {Acte interventionnel : biopsie, drainage, thermoablation, geste vasculaire};
\node[box, below=1.4cm of e2] (e3) {Surveillance 1–3~h : constantes, douleur, point de ponction};
\node[box, below=1.4cm of e3] (syn) {Synthèse médicale, consignes écrites, sortie sécurisée};

\draw[->, thick, APHPdark] (tri) -- (e1);
\draw[->, thick, APHPdark] (e1) -- (e2);
\draw[->, thick, APHPdark] (e2) -- (e3);
\draw[->, thick, APHPdark] (e3) -- (syn);

\end{tikzpicture}
\end{figure}

\clearpage

% ============================================================
% ORGANISATION
% ============================================================

\subsection{Organisation}
\needspace{6\baselineskip}

\begin{itemize}[leftmargin=1.1cm]
\item Direction : \textbf{Pr Anthony Dohan}
\item Durée moyenne : 3~heures
\item Lieu : secteur HDJ — hépato-gastroentérologie / oncologie digestive
\item Ressources : médecin sénior, infirmier expert
\end{itemize}

\clearpage

% ============================================================
% CODAGE ET GHS ASSOCIÉS
% ============================================================

\subsection{Codage et GHS associés}
\needspace{8\baselineskip}

\noindent\textbf{Cadre général.}  
Codage basé sur le DP correspondant à la lésion explorée ou traitée, complété par les DAS pertinents.  
Actes relevant de la CCAM d’imagerie interventionnelle.

\medskip

\begin{table}[h!]
\centering
\renewcommand{\arraystretch}{1.20}
\rowcolors{2}{APHPsoft}{white}

\begin{tabularx}{\textwidth}{
>{\raggedright\arraybackslash}p{4.5cm}
X
>{\centering\arraybackslash}p{1.6cm}
>{\centering\arraybackslash}p{1.6cm}
>{\centering\arraybackslash}p{2cm}}
\toprule
\rowcolor{APHPsoft}
\textbf{Type de séance} &
\textbf{DP / DAS} &
\textbf{GHM} &
\textbf{GHS} &
\textbf{Tarif 2025} \\
\midrule

Biopsie percutanée &
DP : selon organe (Cxx / Rxx) \newline DAS : selon contexte &
07M06T &
2528 &
1\,061~€ \\

Drainage percutané &
DP : K65, K83, N13 \newline DAS : infection, douleur &
07M02T &
2518 &
891~€ \\

\bottomrule
\end{tabularx}

\caption{Codage et GHS associés — HDJ Radiologie interventionnelle}
\end{table}

\clearpage

% ============================================================
% TRACABILITÉ
% ============================================================

\subsection{Traçabilité minimale}

\begin{table}[h!]
\centering
\renewcommand{\arraystretch}{1.20}
\rowcolors{2}{APHPsoft}{white}

\begin{tabular}{p{5cm} p{9cm}}
\toprule
\rowcolor{APHPsoft}
\textbf{Intervention} & \textbf{Éléments requis} \\
\midrule

Biopsie &
Indication ; imagerie préalable ; type d’aiguille ; nombre de prélèvements ; incidents ; contrôle post-geste. \\

Drainage &
Site ; guidage ; volume, aspect ; prélèvements ; mise en tension ; contrôle immédiat. \\

Actes vasculaires &
Voie d’abord ; matériel ; perméabilité post-geste ; incidents. \\

Synthèse de sortie &
Compte-rendu structuré ; mesures de sécurité ; consignes écrites ; suivi programmé. \\
\bottomrule
\end{tabular}

\caption{Traçabilité — HDJ Radiologie interventionnelle}
\end{table}

\clearpage

% ============================================================
% VOLUMÉTRIE DE RÉFÉRENCE
% ============================================================

\subsection{Volumétrie de référence (année N)}

\noindent Montée en charge progressive : 6 à 10 patients/jour.

\begin{center}
\begin{tabular}{lccc}
\toprule
\textbf{Type de séance} & \textbf{Volume N} & \textbf{Tarif unitaire} & \textbf{Recette estimée} \\
\midrule
Biopsies & XXX & 650~€ & ZZZZ~€ \\
Drainages & XXX & 780~€ & ZZZZ~€ \\
\midrule
\textbf{Total} & \textbf{XXX} & -- & \textbf{ZZZZ~€} \\
\bottomrule
\end{tabular}
\end{center}

\clearpage

% ============================================================
% PROJECTIONS
% ============================================================

\subsection{Projections d’activité et recettes prévisionnelles}

\noindent Hypothèse : transfert progressif des gestes vers l’ambulatoire.

\begin{center}
\begin{tabular}{lccc}
\toprule
\textbf{Année} & \textbf{Volume} & \textbf{Tarif moyen} & \textbf{Recette brute} \\
\midrule
Amorce & XXX & 850~€ & ZZZZ~€ \\
Montée & XXX & 850~€ & ZZZZ~€ \\
Croisière & XXX & 850~€ & ZZZZ~€ \\
\bottomrule
\end{tabular}
\end{center}

\clearpage

% ============================================================
% VALIDATION
% ============================================================

\begin{center}
\begin{tabular}{p{4cm} p{7cm} p{4cm}}
\toprule
\rowcolor{APHPsoft}
\textbf{Date d'envoi} & \textbf{Relecteur} & \textbf{Validation} \\
\midrule
01/12/2025 & Pr V.\,Mallet & 04/12/2026 \\
04/12/2025 & Pr A.\,Dohan  & NA \\
NA          & Dr S.\,Bouam & NA \\
NA          & Pr R.\,Coriat & NA \\
\bottomrule
\end{tabular}
\end{center}

\clearpage
