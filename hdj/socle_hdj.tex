% ============================================================
% SOCLE MÉDICAL TRANSVERSAL — HDJ DIGESTIFS
% ============================================================
\titleformat{\subsection}[runin]
  {\bfseries\color{APHPblue}}
  {}
  {0pt}
  {}

%\section{Socle médical transversal des HDJ digestifs}
\label{sec:socle_medical}

\subsection*{Virage ambulatoire et parcours complexes}

Le développement des hôpitaux de jour digestifs s’inscrit pleinement dans le cadre du virage
ambulatoire, qui constitue un axe stratégique majeur des politiques de santé hospitalières.
Il répond à l’augmentation continue des pathologies chroniques, évolutives et complexes,
nécessitant des évaluations répétées, programmables et coordonnées, sans justification
d’hospitalisation conventionnelle prolongée dès lors qu’une organisation structurée,
sécurisée et multidisciplinaire est disponible.

Les HDJ permettent de concentrer, sur un temps court, des actes diagnostiques, thérapeutiques
et éducatifs à forte valeur ajoutée médicale, tout en assurant la continuité des parcours
ville–hôpital, la traçabilité universitaire des prises en charge et la sécurisation des circuits
de soins pour des patients souvent fragiles ou à risque.

\subsection*{Maladies hépatiques chroniques et métaboliques}

Les maladies hépatiques chroniques constituent un déterminant majeur de morbi-mortalité
digestive et représentent une part croissante de l’activité spécialisée. Les pathologies
métaboliques hépatiques (MASLD/MASH), les cirrhoses compensées et décompensées, ainsi que
leurs complications (hypertension portale, dénutrition, sarcopénie, complications rénales
et infectieuses) génèrent des besoins importants d’évaluations itératives et coordonnées.

Les recommandations internationales (EASL, Baveno~VII, AFEF) promeuvent des parcours
structurés reposant sur des outils non invasifs, l’imagerie spécialisée, l’évaluation
nutritionnelle et comportementale, et une stratification fine du risque. Ces approches sont
particulièrement compatibles avec une organisation en hôpital de jour, permettant
d’optimiser le suivi tout en limitant le recours à l’hospitalisation complète.

\subsection*{Maladies inflammatoires chroniques de l’intestin}

Les maladies inflammatoires chroniques de l’intestin (MICI) représentent des pathologies
chroniques à forte intensité de suivi, marquées par l’essor des biothérapies, des stratégies
de monitorage thérapeutique et des évaluations multidisciplinaires régulières. De nombreuses
étapes du parcours — initiation ou optimisation thérapeutique, surveillance biologique et
clinique, éducation thérapeutique, coordination paramédicale — relèvent désormais d’une
prise en charge ambulatoire spécialisée.

Les HDJ dédiés aux MICI permettent de sécuriser ces parcours complexes, de réduire les
hospitalisations évitables liées aux poussées ou aux effets indésirables, et d’améliorer la
réactivité thérapeutique dans un cadre standardisé et expert.

\subsection*{Oncologie digestive et traitements systémiques}

La prise en charge des cancers digestifs connaît une transformation profonde avec le recours
croissant aux chimiothérapies, immunothérapies et traitements ciblés administrés en
ambulatoire. Ces prises en charge nécessitent une organisation rigoureuse intégrant
évaluation clinique pré-thérapeutique, surveillance des toxicités, coordination étroite avec
les équipes d’oncologie et accès rapide au plateau technique.

Les HDJ d’oncologie digestive constituent aujourd’hui le cadre organisationnel de référence
pour concilier sécurité des soins, efficience hospitalière et amélioration de la qualité de vie
des patients, tout en optimisant l’utilisation des lits d’hospitalisation conventionnelle.

\subsection*{Addictologie et pathologies associées}

Les troubles liés à l’usage de l’alcool et des substances psychoactives constituent un facteur
majeur de morbi-mortalité digestive, métabolique et systémique. Ils interfèrent de manière
significative avec l’évolution des maladies hépatiques, oncologiques et métaboliques, et
complexifient les parcours de soins.

Leur prise en charge efficace repose sur des parcours intégrés associant évaluation
somatique, accompagnement addictologique, approche motivationnelle, éducation
thérapeutique et coordination pluridisciplinaire. Les HDJ addictologiques offrent un cadre
structuré et lisible pour ces évaluations complexes, favorisant l’adhésion des patients, la
prévention des complications et la réduction des hospitalisations non programmées.

\subsection*{Radiologie interventionnelle et actes programmés}

L’essor des techniques de radiologie interventionnelle mini-invasive a profondément modifié
les parcours digestifs et oncologiques, permettant la réalisation d’un nombre croissant
d’actes en ambulatoire. Ces procédures nécessitent une sélection rigoureuse des patients,
une préparation standardisée, une surveillance post-acte adaptée et une coordination étroite
entre cliniciens, radiologues et équipes soignantes.

L’intégration de la radiologie interventionnelle au sein d’un plateau HDJ mutualisé garantit
la sécurité des parcours, réduit les durées d’hospitalisation et améliore l’efficience
organisationnelle à l’échelle de l’établissement.

\subsection*{Rationalité médico-économique et organisationnelle}

La concentration des prises en charge digestives complexes en hôpital de jour permet de
réduire les hospitalisations conventionnelles évitables, d’optimiser l’allocation des
ressources médicales et paramédicales spécialisées, et d’améliorer la lisibilité des parcours
sur le territoire.

Ce modèle favorise une montée en charge progressive, une adaptation dynamique des
capacités (lits médicalisés, fauteuils de soins, espaces non allongés), et un pilotage
médico-économique fin, tout en maintenant un haut niveau de qualité, de sécurité et de
traçabilité des soins.

Sur cette base médico-organisationnelle commune, les sections suivantes déclinent,
sous forme de fiches opérationnelles standardisées, les principales activités d’hôpital
de jour digestives du site. Chaque fiche décrit le rationnel médical, la population éligible,
le parcours patient, les actes réalisés, les ressources mobilisées ainsi que les éléments
de codage et de projection d’activité, afin d’assurer une lecture homogène, comparative
et directement exploitable sur le plan organisationnel et médico-économique.

\medskip

\noindent\textit{Note méthodologique — données financières.}
Les montants financiers présentés dans ce document sont exprimés en tarifs T2A en vigueur
au 1\textsuperscript{er} mars 2025. Ils intègrent la majoration régionale Île-de-France (7~\%)
ainsi que la majoration dite «~Ségur~» (3,5~\%), conformément à la réglementation en vigueur.

