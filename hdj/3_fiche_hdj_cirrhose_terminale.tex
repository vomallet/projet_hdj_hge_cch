% ============================================================
% HÔPITAL DE JOUR — CIRRHOSE AVANCÉE / COMPLICATIONS DE LA CIRRHOS...
% ============================================================

\subsection{Rationnel médical}
\needspace{8\baselineskip}
\setcounter{table}{0}
\setcounter{figure}{0}

\begin{spacing}{1.30}

L’ensemble des maladies chroniques du foie évolue progressivement vers la cirrh\-ose et ses complications. La phase décompensée — ascite, encéphalopathie hépatique, hyponatrémie, hémorragie digestive, insuffisance rénale aiguë (AKI) ou syndrome hépato\-rénal (HRS) — correspond au stade terminal de la maladie, avec une mortalité annuelle souvent supérieure à 20--30\,\% \cite{EASL_DecompCirrhosis_2018,AASLD_PalliativeCirrhosis_2022}. 

Malgré les progrès réalisés dans le traitement des hépatites virales, l’incidence des décompensations ne diminue pas, en lien avec l’augmentation de l’obésité, du diabète de type~2, des maladies hépatiques métaboliques (MASLD), ainsi que la persistance de la consommation d’alcool \cite{Gines_LancetCirrhosis_2021}. En France, les données PMSI/AP--HP confirment une progression des séjours liés à l’ascite, aux infections et à l’AKI, avec une mortalité hospitalière de 11--15\,\% \cite{FrenchHepaticFailure_2020}.

Les complications de la cirrh\-ose avancée requièrent des interventions répétées et accessibles rapidement : ponction d’ascite avec perfusion d’albumine, albumine au long cours (type ANSWER) \cite{Caraceni_ANSWER_2018}, fer intraveineux, transfusion de CGR et soutien nutritionnel.

La structuration d’un HDJ dédié permet d’assurer ces prises en charge dans un cadre sécurisé, standardisé et pluridisciplinaire, favorisant la stabilisation, la prévention des réhospitalisations et la préparation d’éventuelles procédures (TIPS).

\end{spacing}

\clearpage

% ============================================================
% OBJECTIFS
% ============================================================

\subsection{Objectifs}
\needspace{6\baselineskip}

\begin{center}
\fcolorbox{APHPdark}{APHPsoft}{
\begin{minipage}{0.95\textwidth}
\vspace{0.9em}

\begin{itemize}[leftmargin=1.1cm]
  \item réduire les hospitalisations évitables liées aux complications de la cirrh\-ose avancée ;
  \item prévenir les décompensations sévères (AKI, HRS, ACLF) par un accès rapide aux actes nécessaires ;
  \item proposer un dispositif de stabilisation pré--TIPS ;
  \item regrouper en ambulatoire les actes complexes : LVP + albumine, albumine seule, fer IV, transfusion de CGR ;
  \item renforcer la continuité entre ville, urgences, hépatologie et MCO.
\end{itemize}

\vspace{0.9em}
\end{minipage}}
\end{center}

\bigskip

% ============================================================
% POPULATION ÉLIGIBLE
% ============================================================

\subsection{Population éligible}

\begin{itemize}[leftmargin=1.1cm]
  \item cirr\-hose avancée (Child B–C) avec ascite récurrente ou réfractaire ;
  \item risque d’AKI ou HRS (hyponatrémie, insuffisance rénale fonctionnelle) ;
  \item anémie d’hypertension portale nécessitant fer IV ou transfusion programmée ;
  \item encéphalopathie hépatique fluctuante nécessitant surveillance rapprochée ;
  \item dénutrition sévère ou sarcopénie ;
  \item candidats au TIPS ou non éligibles nécessitant prévention des réhospitalisations.
\end{itemize}

\clearpage

% ============================================================
% PARCOURS DE SOINS
% ============================================================

\subsection{Parcours de soins}

\begin{figure}[!ht]
\centering
\caption{Parcours patient — HDJ cirrhose avancée\footnotemark}
\vspace{0.8cm}

\begin{tikzpicture}[
    node distance=1.6cm,
    box/.style={
        rectangle,
        rounded corners=3pt,
        draw=APHPdark,
        thick,
        text width=9.0cm,
        minimum height=1.5cm,
        align=center,
        fill=APHPsoft
    }
]

\node[box] (tri) {Orientation vers l’HDJ \\[3pt] (hépatologie / urgences / MCO / ville)};
\node[box, below=1.4cm of tri] (etape1) {Évaluation clinique initiale, biologie récente, imagerie ciblée si besoin ; vérification du statut vaccinal (pneumocoque, grippe, hépatites~A et~B)};
\node[box, below=1.4cm of etape1] (etape2) {Acte programmé : \\[3pt] LVP + albumine, albumine seule, fer IV, transfusion de CGR};
\node[box, below=1.4cm of etape2] (etape3) {Surveillance spécialisée : \\[3pt] constantes, tolérance, complications};
\node[box, below=1.4cm of etape3] (synth) {Synthèse médicale et programmation de la séance suivante};

\draw[->, thick, APHPdark] (tri) -- (etape1);
\draw[->, thick, APHPdark] (etape1) -- (etape2);
\draw[->, thick, APHPdark] (etape2) -- (etape3);
\draw[->, thick, APHPdark] (etape3) -- (synth);

\end{tikzpicture}
\end{figure}

\footnotetext{
\textbf{HDJ} : hôpital de jour ; 
\textbf{MCO} : médecine–chirurgie–obstétrique ; 
\textbf{LVP} : paracentèse évacuatrice (large volume paracentesis) ; 
\textbf{CGR} : concentrés de globules rouges.
}

\clearpage

% ============================================================
% CODAGE, TARIFS ET VOLUMÉTRIE 2024
% ============================================================

\subsection{Codage, tarifs et volumétrie de référence}
\needspace{12\baselineskip}

\begin{sidewaystable}[p]
\centering
\renewcommand{\arraystretch}{1.25}
\rowcolors{2}{APHPsoft}{white}

\begin{tabularx}{\textwidth}{
p{4.8cm}
X
>{\centering\arraybackslash}p{1.7cm}
>{\centering\arraybackslash}p{1.6cm}
>{\centering\arraybackslash}p{1.9cm}
>{\centering\arraybackslash}p{1.8cm}
>{\centering\arraybackslash}p{2.4cm}
}
\toprule
\rowcolor{APHPsoft}
\textbf{Type de séance} &
\textbf{DP / DR / DAS} &
\textbf{GHM} &
\textbf{GHS} &
\textbf{Tarif 2025} &
\textbf{Volume 2024} &
\textbf{Recette 2024} \\
\midrule

Ponction d’ascite + albumine &
DP : R18 (ascite)\newline
DAS : K74.6, K76.6, ±N17.x, ±E87.1 &
07M14T & 2559 & 603~€ &
160 &
96\,480~€ \\

Séance d’albumine seule &
DP : Z512 (réserve)\newline
DR : R18 ou K76.6 &
28Z17Z & 9616 & 440~€ &
18 &
7\,920~€ \\

Fer injectable (anémie HTP) &
DP : Z512\newline
DR : D50.8 ou D64.9\newline
DAS : K74.6, K76.6 &
28Z17Z & 9616 & 440~€ &
125 &
55\,000~€ \\

Transfusion de CGR &
DP : Z5130\newline
DR : D50.8 ou D62\newline
DAS : K74.6, K76.6 &
28Z14Z & 9613 & 791~€ &
15 &
11\,865~€ \\
\midrule

\textbf{Total annuel} & -- & -- & -- & -- &
\textbf{319} &
\textbf{171\,265~€} \\

\bottomrule
\end{tabularx}

\caption{Codage, tarifs et volumétrie — HDJ cirrhose avancée (2024)}
\end{sidewaystable}

\clearpage

% ============================================================
% TRACABILITÉ
% ============================================================

\subsection{Traçabilité minimale}

\begin{table}[h!]
\centering
\renewcommand{\arraystretch}{1.25}
\rowcolors{2}{APHPsoft}{white}

\begin{tabular}{p{5cm} p{9cm}}
\toprule
\rowcolor{APHPsoft}
\textbf{Intervention} & \textbf{Éléments requis} \\
\midrule

Ponction d’ascite &
Volume évacué ; repérage écho ; surveillance 4~h ; douleur ; hypotension ; biologie pré-acte \\

Perfusion d’albumine &
Prescription ; indication ; traçabilité du lot ; volume perfusé ; surveillance hémodynamique \\

Fer injectable &
Indication (anémie HTP) ; traçabilité lot ; protocole perfusion ; surveillance immédiate et retardée \\

Transfusion de CGR &
Traçabilité PSL ; groupage ; concordance ; surveillance renforcée ; incidents transfusionnels \\

Entretien médical &
Justification ; risque HRS/AKI ; bilan clinique ; adaptation thérapeutique \\

Surveillance spécialisée &
Constantes ; EVA douleur ; hémodynamique ; reins ; drainage post-ponction \\

Coordination / éducation &
Fiche de liaison ville–hôpital ; conseils HTP ; éducation sur signes d’alerte \\
\bottomrule
\end{tabular}

\caption{Traçabilité — HDJ cirrhose avancée}
\end{table}

\clearpage

% ============================================================
% ORGANISATION
% ============================================================

\subsection{Organisation}
\needspace{5\baselineskip}

\begin{itemize}[leftmargin=1.1cm]
    \item Direction : \textbf{Dr Valérie D’Halluin-Venier}
    \item Durée : 4--6~heures
    \item Lieu : Secteur HDJ — Service des maladies du foie
    \item Ressources : médecin sénior, infirmier expert/IPA, diététicien(ne), psychologue/addictologue
\end{itemize}

\bigskip

% ============================================================
% PROJECTIONS D’ACTIVITÉ
% ============================================================

\subsection{Projections d’activité et recettes prévisionnelles}

\noindent Basé sur un tarif moyen pondéré : \textbf{\textasciitilde540~€ / séance}.  
Hypothèse : \textbf{+50 patients par pallier} à partir de la volumétrie 2024 (277 actes).

\begin{table}[h!]
\centering
\renewcommand{\arraystretch}{1.18}
\rowcolors{2}{APHPsoft}{white}

\begin{tabular}{lccc}
\toprule
\rowcolor{APHPsoft}
\textbf{Année} & \textbf{Volume estimé} & \textbf{Tarif moyen} & \textbf{Recette brute estimée} \\
\midrule
Amorce    & 327 & 540~€ & 176\,580~€ \\
Montée    & 377 & 540~€ & 203\,580~€ \\
Croisière & 427 & 540~€ & 230\,580~€ \\
\bottomrule
\end{tabular}

\caption{Prévisions d’activité et recettes prévisionnelles — HDJ cirrhose avancée}
\end{table}

\clearpage

% ============================================================
% CONCLUSION
% ============================================================

\subsection{Conclusion}
\needspace{6\baselineskip}

Les HDJ dédiés à la cirrh\-ose avancée offrent un cadre structuré pour les interventions indispensables à la stabilisation des patients les plus fragiles. En rassemblant ponctions d’ascite, albumine, fer injectable et transfusions dans un parcours sécurisé et standardisé, ils contribuent à réduire les hospitalisations évitables, à prévenir les décompensations sévères et à optimiser la continuité des soins entre ville, urgences et hépatologie.
Pour les modalités techniques détaillées (ponction d’ascite, perfusion d’albumine, fer IV, transfusion),
voir les annexes~\ref{sec:annexe_ascite}, \ref{sec:annexe_fer} et \ref{sec:annexe_cgr}.


% ============================================================
% VALIDATION
% ============================================================

\begin{center}
\begin{tabular}{p{4cm} p{7cm} p{4cm}}
\toprule
\rowcolor{APHPsoft}
\textbf{Date d’envoi} & \textbf{Nom du relecteur} & \textbf{Date de validation} \\
\midrule
03/12/2025 & Pr V.\,Mallet & 08/12/2025 \\
03/12/2025 & Dr S.\,Bouam & NA \\
03/12/2025 & Dr V.\,D’Halluin-Venier & NA \\
NA & Pr R.\,Coriat & NA \\
\bottomrule
\end{tabular}
\end{center}

\clearpage
