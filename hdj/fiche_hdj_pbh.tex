% ============================================================
% HÔPITAL DE JOUR — BIOPSIE HÉPATIQUE (PBH)
% ============================================================

\subsection{Rationnel médical}
\needspace{8\baselineskip}

\begin{spacing}{1.30}

La biopsie hépatique (PBH) demeure l’examen de référence pour le diagnostic, la classification et le suivi histologique de nombreuses maladies du foie. Malgré les progrès des outils non invasifs, elle reste indispensable dans plusieurs situations cliniques complexes où seule l’analyse tissulaire permet une caractérisation fine et reproductible des lésions.

Elle est essentielle dans les maladies hépatiques rares (hépatite auto-immune, cholangites, granulomatoses), en particulier lorsque les présentations cliniques ou immunologiques sont atypiques, discordantes ou insuffisamment discriminantes. Dans les toxicités médicamenteuses, notamment celles induites par les immunothérapies (anti–PD-1/PD-L1, anti–CTLA-4) ou certaines thérapies ciblées, la PBH est un examen clé permettant de distinguer des lésions auto-immunes, granulomateuses, cholestatiques ou toxiques. Elle occupe une place centrale dans les algorithmes diagnostiques et décisionnels proposés récemment \cite{Peeraphatdit_ImmuneHepatitis_2020,DeMartin_IOHepatitis_2020}.

La PBH est également incontournable pour évaluer la réponse aux traitements dans les maladies chroniques du foie. La régression de la fibrose et de la cirrhose constitue un critère pronostique majeur et un objectif essentiel des essais thérapeutiques contemporains. Aucune méthode non invasive ne permet aujourd’hui d’attester la réorganisation architecturale du foie : seule la PBH peut documenter de manière standardisée une réversion histologique, ce qui en fait l’examen de référence dans les études sur la MASLD/NASH et dans les essais visant à démontrer une amélioration de la fibrose \cite{Friedman_FibrosisRegression_2023}.

La PBH conserve une utilité en oncologie hépatique pour caractériser certaines lésions bénignes ou malignes (carcinome hépatocellulaire, tumeurs primitives rares), lorsque la confirmation histologique conditionne la prise en charge. En maladies infectieuses, elle peut être indispensable pour rechercher des agents pathogènes difficiles à isoler (hépatocultures, tuberculose hépatique, mycoses profondes, infections virales atypiques).

En hématologie, notamment dans les syndromes myélodysplasiques (SMD) ou avant allogreffe de moelle, la présence d’une fibrose ou d’une cirrhose constitue un facteur de risque indépendant de complications sévères post-greffe, justifiant une évaluation histologique. Cette indication deviendra stratégique avec l’arrivée des services de maladies infectieuses et d’hématologie de l’hôpital Cochin sur le site en 2027.

L’Hôpital de Jour offre un environnement rigoureux pour cet examen : protocole décisionnel formalisé, réalisation sous guidage échographique (HLHJ003 ou HLHJ006), surveillance renforcée, traçabilité complète et coordination ville–hôpital.

\end{spacing}

\clearpage

% ============================================================
% OBJECTIFS
% ============================================================

\subsection{Objectifs}
\needspace{6\baselineskip}

\vspace{0.8em}
\begin{center}
\fcolorbox{APHPdark}{APHPsoft}{
\begin{minipage}{0.95\textwidth}
\vspace{0.9em}

\begin{itemize}[leftmargin=1.1cm]
  \item obtenir un diagnostic histologique précis dans les maladies du foie complexes ou atypiques ;
  \item documenter les toxicités médicamenteuses, en particulier celles liées aux immunothérapies ;
  \item évaluer la fibrose et détecter une éventuelle régression structurale dans les protocoles thérapeutiques (MASLD/NASH) ;
  \item caractériser des lésions tumorales bénignes ou malignes lorsque cela conditionne la stratégie ;
  \item explorer des atteintes infectieuses hépatiques rares (hépatocultures) ;
  \item réaliser l’examen dans un cadre sécurisé, standardisé et traçable en HDJ.
\end{itemize}

\vspace{0.9em}
\end{minipage}}
\end{center}

\bigskip

% ============================================================
% POPULATION ÉLIGIBLE
% ============================================================

\subsection{Population éligible}

\begin{itemize}[leftmargin=1.1cm]
  \item suspicion de maladie auto-immune ou cholestatique avec bilan discordant ;
  \item toxicité médicamenteuse sévère ou atypique, notamment sous immunothérapie ;
  \item programmes de recherche nécessitant une évaluation histologique (MASLD/NASH) ;
  \item exploration de lésions tumorales primitives ou secondaires ;
  \item suspicion d’hépatite infectieuse atypique ou d’infection profonde (hépatoculture) ;
  \item évaluation pré-allogreffe en hématologie (SMD, aplasie, LAM).
\end{itemize}

\clearpage

% ============================================================
% PARCOURS DE SOINS
% ============================================================

\subsection{Parcours de soins}

\begin{figure}[!ht]
\centering
\caption{Parcours patient — HDJ biopsie hépatique}
\vspace{0.8cm}

\begin{tikzpicture}[
    node distance=1.6cm,
    box/.style={
        rectangle,
        rounded corners=3pt,
        draw=APHPdark,
        thick,
        text width=9.0cm,
        minimum height=1.5cm,
        align=center,
        fill=APHPsoft
    }
]
\node[box] (tri) {Orientation vers l’HDJ (hépatologie / MCO / ville)};
\node[box, below=1.4cm of tri] (e1) {Évaluation initiale : bilan de coagulation, imagerie préalable, consentement};
\node[box, below=1.4cm of e1] (e2) {Biopsie sous guidage échographique \\ (HLHJ003 ou HLHJ006)};
\node[box, below=1.4cm of e2] (e3) {Surveillance spécialisée 4 heures : \\ constantes, douleur, complications};
\node[box, below=1.4cm of e3] (syn) {Compte-rendu et coordination ville–hôpital};

\draw[->, thick, APHPdark] (tri) -- (e1);
\draw[->, thick, APHPdark] (e1) -- (e2);
\draw[->, thick, APHPdark] (e2) -- (e3);
\draw[->, thick, APHPdark] (e3) -- (syn);

\end{tikzpicture}
\end{figure}

\clearpage

% ============================================================
% CODAGE ET GHS ASSOCIÉS
% ============================================================

\subsection{Codage et GHS associés}

\noindent\textbf{Cadre général.}  
Biopsie hépatique codée HLHJ003 (non ciblée) ou HLHJ006 (ciblée) avec surveillance particulière (ATIH AGORA \#226431, 2022).

\medskip

\begin{table}[h!]
\centering
\renewcommand{\arraystretch}{1.25}
\rowcolors{2}{APHPsoft}{white}

\begin{tabular}{p{3.8cm} p{7.5cm} c c c}
\toprule
\rowcolor{APHPsoft}
\textbf{DP} & \textbf{Libellé / DR / DAS} & \textbf{GHM} & \textbf{GHS} & \textbf{Tarif 2025} \\
\midrule

K74.0 / K76.0 &
Fibrose hépatique, NASH \newline DAS : comorbidités majeures &
07M08T &
2538 &
1\,238~€ \\

C22.0 &
CHC \newline DAS : comorbidités &
07M06T &
2528 &
1\,061~€ \\

K70.2 &
Cirrhose alcoolique / fibrose &
07M07T &
2533 &
840~€ \\

K75.4 / D13.4 &
HAI, tumeurs bénignes du foie &
07M04T &
2523 &
911~€ \\

R93.2 &
PBH normale &
07M14T &
2559 &
597~€ \\
\bottomrule
\end{tabular}

\caption{Codage et GHS — HDJ Biopsie hépatique}
\end{table}

\clearpage

% ============================================================
% TRACABILITÉ
% ============================================================

\subsection{Traçabilité minimale}

\begin{table}[h!]
\centering
\renewcommand{\arraystretch}{1.25}
\rowcolors{2}{APHPsoft}{white}

\begin{tabular}{p{5cm} p{9cm}}
\toprule
\rowcolor{APHPsoft}
\textbf{Intervention} & \textbf{Traçabilité requise} \\
\midrule

Biopsie hépatique (HLHJ003/006) &
Fiche d’acte, repérage échographique, nombre de carottes, longueur, calibre \\

Surveillance spécialisée &
Feuille de surveillance pluri-horaire (4h), douleur, tension, saignement local \\

Entretien médical &
Note médicale dédiée : indication, consentement, comorbidités, risque hémorragique \\

Examens complémentaires &
Traçabilité des imageries et bilans pré-biopsie \\

Coordination &
Compte-rendu structuré, consignes post-examen, liaison ville–hôpital \\

\bottomrule
\end{tabular}

\caption{Traçabilité — HDJ Biopsie hépatique}
\end{table}

\clearpage

% ============================================================
% VOLUMÉTRIE 2024
% ============================================================

\subsection{Volumétrie 2024 (référence HDJ hépatologie Cochin)}

\begin{center}
\begin{tabular}{lccc}
\toprule
\textbf{Type de séance} & \textbf{Volume 2024} & \textbf{Tarif unitaire moyen} & \textbf{Recette estimée} \\
\midrule
Biopsies hépatiques (PBH) & 34 & \textasciitilde 830~€ & \textasciitilde 28\,200~€ \\
\bottomrule
\end{tabular}
\end{center}

\clearpage

% ============================================================
% PROJECTIONS
% ============================================================

\subsection{Projections d’activité et recettes prévisionnelles}

\noindent Hypothèses : référence 2024 = 34 PBH/an ; montée progressive par intégration infectiologie–hématologie (2027).

\begin{center}
\begin{tabular}{lccc}
\toprule
\textbf{Année} & \textbf{Volume estimé} & \textbf{Tarif moyen} & \textbf{Recette brute} \\
\midrule
2026 & 40--45 & \textasciitilde830~€ & 33\,000--37\,000~€ \\
2027 & 50--60 & \textasciitilde830~€ & 41\,500--49\,800~€ \\
2028 & 60--70 & \textasciitilde830~€ & 49\,800--58\,100~€ \\
\bottomrule
\end{tabular}
\end{center}

\clearpage

% ============================================================
% CONCLUSION
% ============================================================

\subsection{Conclusion}

La biopsie hépatique en Hôpital de Jour s’intègre dans un parcours sécurisé et standardisé, essentiel pour la prise en charge des maladies hépatiques rares, des toxicités médicamenteuses complexes, des programmes de recherche sur la MASLD/NASH et des besoins croissants des services d’oncologie, d’hématologie et d’infectiologie. Elle constitue un geste clé pour le diagnostic, l’orientation thérapeutique et l’évaluation pronostique.

\clearpage
