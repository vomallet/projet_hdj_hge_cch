% ============================================================
% HÔPITAL DE JOUR — CHIMIOTHÉRAPIE
% ============================================================

\setcounter{table}{0}
\setcounter{figure}{0}

% ============================================================
% RATIONNEL MÉDICAL
% ============================================================

\subsection{Rationnel médical}
\needspace{8\baselineskip}
\begin{spacing}{1.30}

Dans le cadre du socle organisationnel commun des hôpitaux de jour digestifs, la chimiothérapie intraveineuse des cancers digestifs a vocation à être organisée en hôpital de jour. Les référentiels nationaux et institutionnels — notamment le document AP–HP de sécurisation des chimiothérapies injectables et le référentiel 
organisationnel de l’INCa — définissent l’hôpital de jour comme la modalité de référence pour l’administration des anticancéreux, en raison de la structuration des parcours, de la coordination interprofessionnelle et de la maîtrise du risque iatrogène \cite{APHP2020,ONC2021,INCa2025_SECUMED}.

Ce modèle permettrait la mise en œuvre sécurisée des principaux protocoles utilisés en oncologie digestive (FOLFOX, FOLFIRI, FOLFIRINOX, CAPOX, gemcitabine–platine, immunothérapies, thérapies ciblées injectables). Il repose sur une évaluation pré-cycle systématique, l’analyse des toxicités antérieures, la conciliation médicamenteuse et une traçabilité complète du processus de prescription, de préparation pharmaceutique et d’administration.

La sélection des patients repose sur la stabilité clinique (ECOG~0–2), l’absence  de complication aiguë (sepsis, déshydratation, occlusion), la faisabilité thérapeutique en ambulatoire et des conditions sociales et organisationnelles compatibles avec un retour sécurisé à domicile. L’utilisation de pompes externes pour les perfusions prolongées et les stratégies de prophylaxie antiémétique et hématologique contribuent à la sécurisation du parcours.

Les données internationales confirment que l’administration ambulatoire des chimiothérapies digestives est sûre lorsque les filières sont structurées. Les taux de recours non programmé (urgences ou réhospitalisations) varient entre 4 et 20\,\% selon les protocoles, tout en restant maîtrisés dans les équipes spécialisées \cite{Prince2019}.

\end{spacing}

\clearpage

% ============================================================
% OBJECTIFS
% ============================================================

\subsection{Objectifs}
\needspace{6\baselineskip}

\begin{center}
\fcolorbox{APHPdark}{APHPsoft}{
\begin{minipage}{0.95\textwidth}
\vspace{0.9em}
\begin{itemize}[leftmargin=1.1cm]
  \item sécuriser l’administration ambulatoire des chimiothérapies injectables ;
  \item standardiser l’évaluation pré-cycle et la surveillance immédiate ;
  \item réduire les hospitalisations non programmées liées aux toxicités aiguës ;
  \item optimiser l’organisation du parcours et la coordination ville–hôpital ;
  \item améliorer l’expérience patient et la continuité des soins.
\end{itemize}
\vspace{0.9em}
\end{minipage}}
\end{center}

\clearpage

% ============================================================
% POPULATION ÉLIGIBLE
% ============================================================

\subsection{Population éligible}
\needspace{6\baselineskip}

\begin{itemize}[leftmargin=1.1cm]
  \item état général compatible (ECOG~0–2) et stabilité clinique ;
  \item absence de complication aiguë : fièvre, déshydratation, syndrome tumoral évolutif ;
  \item biologie conforme aux seuils décisionnels du protocole ;
  \item protocole réalisable en ambulatoire (durée compatible, absence de surveillance prolongée) ;
  \item conditions sociales compatibles avec une prise en charge ambulatoire sécurisée.
\end{itemize}

\clearpage

% ============================================================
% PARCOURS DE SOINS
% ============================================================

\subsection{Parcours de soins}
\needspace{8\baselineskip}

\begin{figure}[!ht]
\centering
\caption{Parcours patient — HDJ Chimiothérapie}
\vspace{0.8cm}

\begin{tikzpicture}[
node distance=1.6cm,
box/.style={
    rectangle, rounded corners=3pt,
    draw=APHPdark, thick,
    text width=9cm,
    minimum height=1.5cm,
    align=center,
    fill=APHPsoft
}]

\node[box] (tri) {Orientation (oncologie médicale / hématologie / ville)};
\node[box, below=1.4cm of tri] (e1) {Évaluation pré-cycle : examen clinique, biologie, consentement, toxicités antérieures};
\node[box, below=1.4cm of e1] (e2) {Administration : prémédication, préparation pharmaceutique, perfusion supervisée};
\node[box, below=1.4cm of e2] (e3) {Surveillance immédiate : constantes, hypersensibilité, gestion des NV};
\node[box, below=1.4cm of e3] (syn) {Synthèse médicale, éducation thérapeutique, planification du cycle suivant};

\draw[->, thick, APHPdark] (tri) -- (e1);
\draw[->, thick, APHPdark] (e1) -- (e2);
\draw[->, thick, APHPdark] (e2) -- (e3);
\draw[->, thick, APHPdark] (e3) -- (syn);

\end{tikzpicture}
\end{figure}

\clearpage

% ============================================================
% ORGANISATION
% ============================================================

\subsection{Organisation}
\needspace{5\baselineskip}

\begin{itemize}[leftmargin=1.1cm]
  \item Direction : \textbf{Dr Anna Pellat}
  \item Durée : 6--8~heures
  \item Lieu : HDJ — Gastroentérologie et oncologie digestive
  \item Ressources : médecin senior, infirmier expert/IPA, diététicien(ne), psychologue selon besoins
\end{itemize}

\clearpage

% ============================================================
% CODAGE, TARIFS ET VOLUMÉTRIE
% ============================================================

\subsection{Codage, tarifs et volumétrie de référence}
\needspace{10\baselineskip}

\begin{sidewaystable}[p]
\centering
\renewcommand{\arraystretch}{1.25}
\rowcolors{2}{APHPsoft}{white}

\begin{tabularx}{\textwidth}{
p{4.8cm}
>{\raggedright\arraybackslash}X
>{\centering\arraybackslash}p{1.7cm}
>{\centering\arraybackslash}p{1.6cm}
>{\centering\arraybackslash}p{1.9cm}
>{\centering\arraybackslash}p{1.8cm}
>{\centering\arraybackslash}p{2.4cm}
}
\toprule
\rowcolor{APHPsoft}
\textbf{Type de séance} &
\textbf{DP / DAS (cancers digestifs)} &
\textbf{GHM} &
\textbf{GHS} &
\textbf{Tarif 2025} &
\textbf{Volume 2024} &
\textbf{Recette 2024} \\
\midrule

Chimiothérapie cytotoxique IV &
DP : C22.0, C22.1, C25.x, C17–C21, C16.x \newline
DAS : comorbidités, dénutrition, douleur, thrombose, ascite &
28Z07Z & 9616 & 495~€ &
XXX & ZZZZ~€ \\

Immunothérapie IV &
DP : C22.x, C25.x, C17–C21 \newline
DAS : toxicités immunes &
28Z17Z & 9616 & 440~€ &
XXX & ZZZZ~€ \\

Thérapies ciblées IV &
DP : C17–C21, C22.x \newline
DAS : mutation RAS/BRAF, métastases &
28Z17Z & 9616 & 440~€ &
XXX & ZZZZ~€ \\

\midrule
\textbf{Total annuel (base PMSI)} & -- & -- & -- & -- &
\textbf{568} & \textbf{322\,400~€} \\
\bottomrule
\end{tabularx}

\caption{Codage, tarifs et volumétrie — HDJ Chimiothérapie digestive (2024). Les projections financières intègrent un tarif moyen pondéré incluant la structure réelle des séances.}
\end{sidewaystable}

\clearpage

% ============================================================
% TRACABILITÉ
% ============================================================

\subsection{Traçabilité minimale}
\needspace{8\baselineskip}

\begin{table}[h!]
\centering
\renewcommand{\arraystretch}{1.25}
\rowcolors{2}{APHPsoft}{white}

\begin{tabular}{p{4.3cm} p{8.2cm}}
\toprule
\rowcolor{APHPsoft}
\textbf{Intervention} & \textbf{Éléments requis} \\
\midrule

Évaluation pré-cycle &
Examen clinique structuré, biologie validée, toxicités CTCAE, décision thérapeutique, conciliation médicamenteuse. \\

Préparation et administration &
Identité, protocole, doses, lots, double contrôle IDE/pharmacie, prémédications, incidents éventuels. \\

Surveillance immédiate &
Constantes, hypersensibilité, tolérance clinique, gestion des NV. \\

Voie veineuse / PAC &
État du site, perméabilité, incidents locaux, protocoles de retrait/flush. \\

Synthèse médicale &
Tolérance, adaptations posologiques, planification du cycle suivant, messages ville–hôpital. \\
\bottomrule
\end{tabular}

\caption{Traçabilité — HDJ Chimiothérapie digestive}
\end{table}

\clearpage

% ============================================================
% PROJECTIONS D’ACTIVITÉ
% ============================================================

\subsection{Projections d’activité et recettes prévisionnelles}

\noindent Référence 2024 : \textbf{568 séances}, soit \textbf{352\,160~€}.\\
Tarif moyen pondéré retenu : \textbf{620~€ / séance}.\\
Hypothèse de croissance : \textbf{+50 séances / an}.

\begin{table}[h!]
\centering
\renewcommand{\arraystretch}{1.20}
\rowcolors{2}{APHPsoft}{white}

\begin{tabular}{
p{4.3cm}
>{\centering\arraybackslash}p{2.2cm}
>{\centering\arraybackslash}p{2.3cm}
>{\centering\arraybackslash}p{3.0cm}
}
\toprule
\rowcolor{APHPsoft}
\textbf{Phase} & \textbf{Volume estimé} & \textbf{Tarif moyen} & \textbf{Recette brute} \\
\midrule
Amorce    & 568 & 620~€ & 352\,160~€ \\
Montée    & 618 & 620~€ & 383\,160~€ \\
Croisière & 668 & 620~€ & 413\,160~€ \\
\bottomrule
\end{tabular}

\caption{Projections d’activité et recettes — HDJ Chimiothérapie digestive}
\end{table}

\clearpage

% ============================================================
% CONCLUSION
% ============================================================

\subsection{Conclusion}

L’HDJ de chimiothérapie digestive permet une administration sécurisée et standardisée des traitements anticancéreux injectables, conforme aux référentiels INCa et AP–HP. Il structure l’évaluation pré-cycle, la gestion des toxicités et la coordination ville–hôpital, tout en réduisant le recours à l’hospitalisation complète. Ce dispositif constitue un levier central de qualité, de sécurité et d’efficience des parcours oncologiques digestifs.

\clearpage
