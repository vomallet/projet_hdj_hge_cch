% ============================================================
% HÔPITAL DE JOUR — CHIMIOTHÉRAPIE
% ============================================================

\setcounter{table}{0} % Réinitialisation des tableaux
\setcounter{figure}{0} % Réinitialisation des tableaux

% ============================================================
% RATIONNEL MÉDICAL
% ============================================================

\subsection{Rationnel médical}
\needspace{8\baselineskip}
\begin{spacing}{1.30}

La chimiothérapie intraveineuse des cancers digestifs est aujourd’hui majoritairement administrée en HDJ, conformément aux référentiels organisationnels nationaux. Le document de l’AP–HP dédié à la sécurisation des chimiothérapies injectables et le référentiel organisationnel de l’INCa définissent l’HDJ comme la modalité privilégiée d’administration des anticancéreux en raison de la structuration, de la coordination interprofessionnelle et de la maîtrise du risque iatrogène \cite{APHP2020, ONC2021, INCa2025_SECUMED}. 

Ce modèle permet la mise en œuvre sécurisée des principaux protocoles utilisés en oncologie digestive (FOLFOX, FOLFIRI, FOLFIRINOX, CAPOX, gemcitabine–platine, immunothérapies, thérapies ciblées injectables) en intégrant une évaluation pré-cycle systématique, le contrôle des toxicités précédentes, la conciliation médicamenteuse et la traçabilité complète du processus de préparation, délivrance et administration.

La sélection des patients repose sur la stabilité clinique (ECOG~0–2), l’absence de complication aiguë (sepsis, déshydratation, occlusion), la faisabilité thérapeutique en ambulatoire et des conditions sociales compatibles. Les pompes externes pour les perfusions prolongées et la prophylaxie antiémétique/hématologique renforcent la sécurité du parcours.

Les données internationales confirment que l’administration ambulatoire des chimiothérapies digestives est sûre lorsque les filières sont structurées. Les taux de recours non programmés (urgences ou réhospitalisations) varient entre 4 et 20\,\%, selon les protocoles, tout en restant maîtrisés dans les équipes spécialisées \cite{Prince2019}. 

L’HDJ de chimiothérapie digestive assure ainsi une continuité thérapeutique optimale, réduit les hospitalisations conventionnelles, améliore l’expérience patient et garantit un haut niveau de qualité et de sécurité grâce à l’expertise pluridisciplinaire et aux circuits d’escalade en cas de complication.

\end{spacing}

\clearpage

% ============================================================
% OBJECTIFS
% ============================================================

\subsection{Objectifs}
\needspace{6\baselineskip}

\begin{center}
\fcolorbox{APHPdark}{APHPsoft}{
\begin{minipage}{0.95\textwidth}
\vspace{0.9em}
\begin{itemize}[leftmargin=1.1cm]
\item sécuriser l’administration ambulatoire des chimiothérapies injectables ;
\item standardiser l’évaluation pré-cycle et la surveillance immédiate ;
\item réduire les hospitalisations non programmées liées aux toxicités aiguës ;
\item optimiser l’organisation du parcours et la coordination ville–hôpital ;
\item améliorer l’expérience patient et la continuité des soins.
\end{itemize}
\vspace{0.9em}
\end{minipage}}
\end{center}

\clearpage

% ============================================================
% POPULATION ÉLIGIBLE
% ============================================================

\subsection{Population éligible}
\needspace{6\baselineskip}

\begin{itemize}[leftmargin=1.1cm]
\item état général compatible (ECOG~0–2), stabilité clinique ;
\item absence de complication aiguë : fièvre, déshydratation, syndrome tumoral évolutif ;
\item biologie conforme aux seuils décisionnels du protocole ;
\item protocole réalisable en ambulatoire (durée compatible, absence de surveillance prolongée) ;
\item autonomie suffisante et présence d’un accompagnant pour le retour.
\end{itemize}

\clearpage

% ============================================================
% PARCOURS DE SOINS
% ============================================================

\subsection{Parcours de soins}
\needspace{8\baselineskip}

\begin{figure}[!ht]
\centering
\caption{Parcours patient — HDJ Chimiothérapie}
\vspace{0.8cm}

\begin{tikzpicture}[
node distance=1.6cm,
box/.style={
    rectangle, rounded corners=3pt,
    draw=APHPdark, thick,
    text width=9cm,
    minimum height=1.5cm,
    align=center,
    fill=APHPsoft
}]

\node[box] (tri) {Orientation (oncologie médicale / hématologie / ville)};
\node[box, below=1.4cm of tri] (e1) {Évaluation pré-cycle : examen clinique, biologie, consentement, toxicités antérieures};
\node[box, below=1.4cm of e1] (e2) {Administration : prémédication, préparation pharmaceutique, perfusion supervisée};
\node[box, below=1.4cm of e2] (e3) {Surveillance immédiate : constantes, hypersensibilité, gestion des NV};
\node[box, below=1.4cm of e3] (syn) {Synthèse médicale, éducation thérapeutique, planification du cycle suivant};

\draw[->, thick, APHPdark] (tri) -- (e1);
\draw[->, thick, APHPdark] (e1) -- (e2);
\draw[->, thick, APHPdark] (e2) -- (e3);
\draw[->, thick, APHPdark] (e3) -- (syn);

\end{tikzpicture}
\end{figure}

\clearpage

% ============================================================
% ORGANISATION
% ============================================================

\subsection{Organisation}
\needspace{5\baselineskip}

\begin{itemize}[leftmargin=1.1cm]
\item Direction : \textbf{Dr Anna Pellat}
\item Durée : 6--8~heures
\item Lieu : HDJ — Gastroentérologie et oncologie digestive
\item Ressources : médecin sénior, infirmier expert/IPA, psychologue/nutritionniste
\end{itemize}

\clearpage

% ============================================================
% CODAGE, TARIFS ET VOLUMÉTRIE
% ============================================================

\subsection{Codage, tarifs et volumétrie de référence}
\needspace{10\baselineskip}

\begin{sidewaystable}[p]

\centering
\renewcommand{\arraystretch}{1.25}
\rowcolors{2}{APHPsoft}{white}

\begin{tabularx}{\textwidth}{
p{4.8cm}
>{\raggedright\arraybackslash}X
>{\centering\arraybackslash}p{1.7cm}
>{\centering\arraybackslash}p{1.6cm}
>{\centering\arraybackslash}p{1.9cm}
>{\centering\arraybackslash}p{1.8cm}
>{\centering\arraybackslash}p{2.4cm}
}
\toprule
\rowcolor{APHPsoft}
\textbf{Type de séance} &
\textbf{DP / DAS (cancers digestifs)} &
\textbf{GHM} &
\textbf{GHS} &
\textbf{Tarif 2025} &
\textbf{Volume 2024} &
\textbf{Recette 2024} \\
\midrule

Chimiothérapie cytotoxique IV &
DP : C22.0 (CHC), C22.1 (cholangioK), C25.x (pancréas), C17–C21 (intestin), C16.x (œsogastre)\newline
DAS : comorbidités, dénutrition, douleur, thrombose, ascite &
28Z07Z & 9616 & 495~€ &
XXX & ZZZZ~€ \\

Immunothérapie (anti-PD1/PD-L1) &
DP : C22.x, C25.x, C17–C21\newline
DAS : toxicités immunes (colite, hépatite, endocrinopathies) &
28Z17Z & 9616 & 440~€ &
XXX & ZZZZ~€ \\

Thérapies ciblées IV / Anticorps monoclonaux &
DP : C17–C21, C22.x\newline
DAS : mutation RAS/BRAF, métastases, complications locales &
28Z17Z & 9616 & 440~€ &
XXX & ZZZZ~€ \\

\midrule
\textbf{Total annuel} & -- & -- & -- & -- &
\textbf{568} & \textbf{322\,400~€} \\
\bottomrule
\end{tabularx}

\caption{Codage, tarifs et volumétrie — HDJ Chimiothérapie digestive (2024)}
\end{sidewaystable}

\clearpage

% ============================================================
% TRACABILITÉ
% ============================================================

\subsection{Traçabilité minimale}
\needspace{8\baselineskip}

\begin{table}[h!]
\centering
\renewcommand{\arraystretch}{1.25}
\rowcolors{2}{APHPsoft}{white}

\begin{tabular}{p{4.3cm} p{8.2cm}}
\toprule
\rowcolor{APHPsoft}
\textbf{Intervention} & \textbf{Éléments requis} \\
\midrule

Évaluation pré-cycle &
Examen clinique structuré, biologie validée, note des toxicités CTCAE, décision de poursuite/modification, conciliation médicamenteuse. \\

Préparation et administration &
Vérification des identités, protocole, doses, lots, double contrôle IDE/pharmacie, prémédications, durée d’infusion, incidents éventuels. \\

Surveillance immédiate &
Constantes avant/pendant/après perfusion, dépistage des réactions d’hypersensibilité, gestion des nausées/vomissements, tolérance clinique. \\

Voie veineuse / PAC &
État du site, perméabilité, incidents locaux, hémostase, retrait/flush selon protocole. \\

Synthèse médicale &
Tolérance du cycle, points de vigilance, adaptations posologiques, recommandations personnalisées, planification du cycle suivant, messages ville–hôpital. \\
\bottomrule
\end{tabular}

\caption{Traçabilité — HDJ Chimiothérapie digestive}
\end{table}

\clearpage


% ============================================================
% PROJECTIONS D’ACTIVITÉ — CHIMIOTHÉRAPIE DIGESTIVE
% ============================================================

\subsection{Projections d’activité et recettes prévisionnelles}

\noindent Référence 2024 : \textbf{568 séances}, soit \textbf{352\,160~€} (tarif moyen \textbf{620~€}). \\
Hypothèse de croissance : \textbf{+50 séances / an}. \\
Tarif moyen stabilisé : \textbf{620~€ / séance}.

\begin{table}[h!]
\centering
\renewcommand{\arraystretch}{1.20}
\rowcolors{2}{APHPsoft}{white}
\begin{tabular}{
p{4.3cm}
>{\centering\arraybackslash}p{2.2cm}
>{\centering\arraybackslash}p{2.3cm}
>{\centering\arraybackslash}p{3.0cm}
}
\toprule
\rowcolor{APHPsoft}
\textbf{Phase} & \textbf{Volume estimé} & \textbf{Tarif moyen} & \textbf{Recette brute} \\
\midrule
Amorce        & 568 & 620~€ & 352\,160~€ \\
Montée        & 618 & 620~€ & 383\,160~€ \\
Croisière     & 668 & 620~€ & 413\,160~€ \\
\bottomrule
\end{tabular}
\caption{Projections d’activité et recettes prévisionnelles — HDJ Chimiothérapie digestive}
\end{table}

\clearpage




% ============================================================
% CONCLUSION
% ============================================================

\subsection{Conclusion}

L’HDJ de chimiothérapie constitue un pilier de l’organisation oncologique moderne. Il permet une administration sécurisée et standardisée des traitements anticancéreux, réduit l’hospitalisation complète et optimise l’expérience patient. La structuration du parcours — évaluation pré-cycle, gestion des toxicités, surveillance immédiate et coordination ville–hôpital — garantit un haut niveau de qualité et d’efficience.

\clearpage
