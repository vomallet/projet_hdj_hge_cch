% ============================================================
% HÔPITAL DE JOUR HÉPATOMÉTABOLIQUE
% ============================================================

\subsection{Rationnel médical}
\needspace{8\baselineskip} % Empêche que le titre reste seul

\begin{spacing}{1.30}

L’augmentation rapide de l’obésité, du diabète de type 2 (DT2) et de la MASLD/MASH constitue aujourd’hui un enjeu de santé publique majeur. Chez les patients vivant avec un DT2, la prévalence de la MASLD dépasse 60\,\%, et 15--20\,\% présentent une fibrose hépatique cliniquement significative (F$\geq$2), les exposant à un risque accru de progression vers la cirrhose, ses complications et le carcinome hépatocellulaire \cite{RN565}.

\medskip

En France, 3,5 à 4 millions de personnes vivent avec un DT2 selon les données récentes de Santé publique France et de l’Assurance Maladie (SNDS) \cite{SPF2021Diabete}, constituant ainsi un réservoir très important de patients susceptibles d’évoluer vers une maladie hépatique avancée.

\medskip

La détection précoce est néanmoins complexe : les complications sévères sont rares, ce qui rend le repérage ciblé d’autant plus essentiel. Dans la cohorte DT2 de l’Entrepôt de Données de Santé (EDS) de l’AP-HP (77\,368 patients), l’incidence des événements hépatiques graves n’est que de 1{,}31 pour 1\,000 patients-années.

\medskip

Pourtant, des interventions brèves mais ciblées — réduction de la consommation d’alcool, perte pondérale modérée, amélioration de la qualité alimentaire \cite{RN597} — démontrent un impact réel sur l’histoire naturelle de la maladie.

\medskip

L’arrivée de nouvelles thérapeutiques pour la MASLD/MASH (agonistes du GLP-1, resmétirom et autres agents en développement) renforce l’urgence de structurer un parcours permettant d’identifier précocement les patients éligibles à un traitement ou à un protocole de recherche.

\medskip

C'est pourquoi il faut des outils simples de triages des patients à rique sou pas. Les biomarqueurs non invasifs constituent aujourd’hui le pilier du tri diagnostique dans la population DT2. Le score FIB-4 — simple, disponible dans Orbis et largement utilisé en médecine de ville — est validé et recommandé par l’AFEF, l’EASL et l’ADA \cite{RN567}. Il présente une excellente valeur prédictive négative pour exclure une fibrose avancée. À l’inverse, un FIB-4 $\geq$ 1{,}3 justifie une évaluation spécialisée incluant FibroScan et imagerie \cite{RN597}.

\medskip

Dans le périmètre Cochin, les patients présentant un FIB-4 élevé — identifiés via la diabétologie, la cardiologie, ou les CPTS — sont déjà orientés vers le service des maladies du foie. Les actions de sensibilisation locales par les médecins du service et territoriales par les sociétés savantes et les tutelles renforcent ce flux et témoignent d’une dynamique organisée autour de la MASLD.

\medskip

Dans ce contexte, la création d’un Hôpital de Jour hépatométabolique apparaît pleinement justifiée. Il offre une évaluation intégrée de la fibrose et des déterminants comportementaux, associant imagerie, nutrition, activité physique et psychologie.

\end{spacing}


% ============================================================
\subsection{Objectifs}
\needspace{5\baselineskip}

\begin{center}
\fcolorbox{APHPdark}{APHPsoft}{
\begin{minipage}{0.92\textwidth}
\begin{itemize}[leftmargin=1.1cm]
    \item Dépister précocement la fibrose hépatique significative ou avancée.
    \item Structurer une évaluation intégrée : biomarqueurs, imagerie, diététique, psychologie.
    \item Initier une prise en charge hygiéno-diététique et comportementale.
    \item Identifier les patients éligibles aux thérapeutiques MASLD/MASH et aux protocoles de recherche.
\end{itemize}
\end{minipage}}
\end{center}

% ============================================================
\subsection{Population éligible}
\needspace{5\baselineskip}

\begin{itemize}[leftmargin=1.1cm]
    \item Diabète de type 2 ou syndrome métabolique.
    \item FIB-4 $\geq$ 1{,}3.
    \item Suspicion clinique ou échographique de MASLD/MASH.
\end{itemize}

\clearpage

% ============================================================
% \subsection{Parcours de soins (3--4 heures)}
\needspace{6\baselineskip}

\begin{figure}[p]
\centering
\caption{Parcours patient — HDJ Hépatométabolique}
\vspace{0.8cm}

% --- figure TikZ inchangée ---
\begin{tikzpicture}[
    node distance=1.6cm,
    box/.style={
        rectangle,
        rounded corners=3pt,
        draw=APHPdark,
        thick,
        text width=5.0cm,
        minimum height=1cm,
        align=center,
        fill=APHPsoft
    }
]
\node[box] (tri) {Tri initial \\ FIB-4 $\geq$ 1{,}3};
\node[box, below=1.4cm of tri] (entree) {Entrée en HDJ hépatométabolique};
\node[box, below=1.4cm of entree] (echo) {Échographie abdominale + Doppler};
\node[box, below=1.4cm of echo] (fibro) {FibroScan / élastographie};
\node[box, below=1.4cm of fibro] (diet) {Consultation diététique};
\node[box, below=1.4cm of diet] (psy) {Évaluation psychologique \\ (alcool / TCA)};
\node[box, below=1.4cm of psy] (synth) {Synthèse médicale \\ Plan thérapeutique};

\draw[->, thick, APHPdark] (tri) -- (entree);
\draw[->, thick, APHPdark] (entree) -- (echo);
\draw[->, thick, APHPdark] (echo) -- (fibro);
\draw[->, thick, APHPdark] (fibro) -- (diet);
\draw[->, thick, APHPdark] (diet) -- (psy);
\draw[->, thick, APHPdark] (psy) -- (synth);

\end{tikzpicture}

\end{figure}

\clearpage

% =======================
% ORGANISATION
% =======================
\subsection{Organisation}

\begin{itemize}[leftmargin=1.1cm]
    \item Durée : 3--4 heures.
    \item Lieu : Secteur HDJ — Service des maladies du foie.
    \item Ressources : médecin sénior ou PH, infirmier expert/IPA, diététicien(ne), psychologue/addictologue.
\end{itemize}

% =======================
% CODAGE
% =======================
\subsection{Codage et recettes prévisionnelles}

\begin{center}
\begin{tabular}{lcc}
\textbf{Nombre d’actes} & \textbf{GHS} & \textbf{Tarif 2025} \\
\hline
$\geq$ 4 actes & 2558 & 941 € \\
3 actes        & 2583 & 421 € \\
\end{tabular}
\end{center}

Actes CCAM typiques : HLQM002 (élastographie), ZCQM004 (échographie + Doppler), consultations (médicale, diététique, psychologique).

\medskip

La structure tarifaire des GHS permet d’adosser le financement du HDJ à une activité combinant systématiquement au moins quatre actes (imagerie, élastographie, consultation médicale, évaluation psychologique ou diététique). Le tarif de 941~€ par passage constitue ainsi la base du modèle économique.

\medskip

\textbf{Hypothèses volumétriques}.  
L’activité prévue repose sur une montée en charge progressive intégrant les capacités médicales, paramédicales et logistiques :

\begin{itemize}
    \item \textbf{2025 (année 1)} : 2--3 patients/semaine, soit \textasciitilde150 patients/an.
    \item \textbf{2026 (année 2)} : 6 patients/semaine, soit \textasciitilde300 patients/an.
    \item \textbf{2027 (année 3)} : montée en charge jusqu’à \textbf{10 patients/semaine}, soit \textasciitilde500 patients/an.
\end{itemize}

\textbf{Projection financière (GHS 2558 à 941~€).}

\begin{center}
\begin{tabular}{lcc}
\textbf{Année} & \textbf{Volume estimé} & \textbf{Recette brute} \\
\hline
2025 & 150 patients & \textasciitilde141\,000 € \\
2026 & 300 patients & \textasciitilde282\,000 € \\
2027 & 500 patients & \textasciitilde470\,500 € \\
\end{tabular}
\end{center}

Ces projections reposent sur un taux de réalisation $\geq$4~actes par patient, cohérent avec la séquence standardisée du parcours (échographie + Doppler, élastographie, consultation médicale, évaluation diététique ou psychologique). Ce format garantit une soutenabilité financière et une capacité d’autofinancement du renforcement des équipes paramédicales (IPA, diététique, psychologie).


% =======================
% PRÉVISIONS
% =======================
\subsection{Prévisions d’activité}

\begin{itemize}[leftmargin=1.1cm]
    \item Année 1 : 2--3 patients/semaine → $\sim$150/an.
    \item Année 2 : 6 patients/semaine → $\sim$300/an.
    \item Année 3 : 12 patients/semaine → $\sim$600/an.
\end{itemize}

Recette brute estimée : 941 € × nombre de patients/an.

% =======================
% CONCLUSION
% =======================
\subsection{Conclusion}

L’HDJ hépatométabolique constitue un dispositif pertinent, simple à organiser et autosoutenable. Il optimise le dépistage, l’accès aux thérapeutiques et la prise en charge multidisciplinaire des patients MASLD/MASH et DT2.

Pour les supports opérationnels (échographie, évaluation diététique et psychologique), 
voir les annexes~\ref{sec:annexe_echo}, \ref{sec:annexe_diete} et \ref{sec:annexe_psy}.



