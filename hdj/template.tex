% ============================================================
% HÔPITAL DE JOUR — TITRE DE LA FICHE
% ============================================================

\subsection{Rationnel médical}
\needspace{8\baselineskip}

\begin{spacing}{1.30}

% ---- TEXTE À REMPLACER ----
% 10–15 lignes de rationnel pathologique + données épidémiologiques
% Références { } conservées
Votre texte ici.

\end{spacing}

\clearpage


% ============================================================
% OBJECTIFS
% ============================================================

\subsection{Objectifs}
\needspace{6\baselineskip}

\vspace{0.8em}
\begin{center}
\fcolorbox{APHPdark}{APHPsoft}{
\begin{minipage}{0.95\textwidth}
\vspace{0.9em}

\begin{itemize}[leftmargin=1.1cm]
  % ---- OBJECTIFS À ADAPTER ----
  \item objectif 1 ;
  \item objectif 2 ;
  \item objectif 3 ;
  \item objectif 4 ;
  \item objectif 5.
\end{itemize}

\vspace{0.9em}
\end{minipage}}
\end{center}

\bigskip


% ============================================================
% POPULATION ÉLIGIBLE
% ============================================================

\subsection{Population éligible}
\needspace{6\baselineskip}

\begin{itemize}[leftmargin=1.1cm]
  % ---- POPULATION À ADAPTER ----
  \item critère 1 ;
  \item critère 2 ;
  \item critère 3 ;
  \item critère 4 ;
  \item critère 5.
\end{itemize}

\clearpage


% ============================================================
% PARCOURS DE SOINS
% ============================================================

\subsection{Parcours de soins}
\needspace{8\baselineskip}

\begin{figure}[!ht]
\centering
\caption{Parcours patient — HDJ NOM}
\vspace{0.8cm}

\begin{tikzpicture}[
    node distance=1.6cm,
    box/.style={
        rectangle,
        rounded corners=3pt,
        draw=APHPdark,
        thick,
        text width=9.0cm,
        minimum height=1.5cm,
        align=center,
        fill=APHPsoft
    }
]
% ---- ÉTAPES DU PARCOURS À ADAPTER ----
\node[box] (tri) {Orientation vers l'HDJ (source 1 / source 2 / ville)};
\node[box, below=1.4cm of tri] (etape1) {Évaluation initiale (éléments clés)};
\node[box, below=1.4cm of etape1] (etape2) {Acte programmé : liste des actes};
\node[box, below=1.4cm of etape2] (etape3) {Surveillance dédiée};
\node[box, below=1.4cm of etape3] (synth) {Synthèse médicale et planification};

\draw[->, thick, APHPdark] (tri) -- (etape1);
\draw[->, thick, APHPdark] (etape1) -- (etape2);
\draw[->, thick, APHPdark] (etape2) -- (etape3);
\draw[->, thick, APHPdark] (etape3) -- (synth);

\end{tikzpicture}
\end{figure}

\clearpage


% ============================================================
% CODAGE ET GHS ASSOCIÉS
% ============================================================

\subsection{Codage et GHS associés}
\needspace{8\baselineskip}

\noindent\textbf{Cadre général.}  
% ---- TEXTE À ADAPTER ----
Texte générique sur DP / DAS / CCAM + mention éventuelle réserve hospitalière.

\medskip


% --------------------------------------------------------
% TABLEAU 1 — Codage & GHS (FORMAT FIXE QUI NE DÉBORDE PAS)
% --------------------------------------------------------
\begin{table}[h!]
\centering
\renewcommand{\arraystretch}{1.25}
\rowcolors{2}{APHPsoft}{white}

\begin{tabular}{p{4.3cm} p{7.3cm} c c c}
\toprule
\rowcolor{APHPsoft}
\textbf{Type de séance} &
\textbf{DP / DR / DAS} &
\textbf{GHM} &
\textbf{GHS} &
\textbf{Tarif 2025} \\
\midrule

% ---- LIGNES À ADAPTER ----
Acte 1 &
DP / DAS appropriés &
XXXXXX & YYYY & ZZZ~€ \\

Acte 2 &
DP / DAS appropriés &
XXXXXX & YYYY & ZZZ~€ \\

Acte 3 &
DP / DAS appropriés &
XXXXXX & YYYY & ZZZ~€ \\

\bottomrule
\end{tabular}

\caption{Codage et GHS associés — HDJ NOM}
\end{table}

\bigskip


% --------------------------------------------------------
% TABLEAU 2 — Traçabilité
% --------------------------------------------------------
\begin{table}[h!]
\centering
\renewcommand{\arraystretch}{1.25}
\rowcolors{2}{APHPsoft}{white}

\begin{tabular}{p{5cm} p{9cm}}
\toprule
\rowcolor{APHPsoft}
\textbf{Intervention} & \textbf{Traçabilité requise} \\
\midrule

% ---- LIGNES À ADAPTER ----
Acte 1 &
Éléments de traçabilité \\

Acte 2 &
Éléments de traçabilité \\

Acte 3 &
Éléments de traçabilité \\

\bottomrule
\end{tabular}

\caption{Traçabilité des interventions — HDJ NOM}
\end{table}

\clearpage


% ============================================================
% VOLUMÉTRIE RÉFÉRENCE
% ============================================================

\subsection{Volumétrie de référence (année N)}
\needspace{6\baselineskip}

\noindent Activité HDJ de référence :

\begin{center}
\begin{tabular}{lccc}
\toprule
\textbf{Type de séance} & \textbf{Volume N} & \textbf{Tarif unitaire} & \textbf{Recette estimée} \\
\midrule
Acte 1 & XXX & YYY~€ & ZZZZ~€ \\
Acte 2 & XXX & YYY~€ & ZZZZ~€ \\
Acte 3 & XXX & YYY~€ & ZZZZ~€ \\
\midrule
\textbf{Total} & \textbf{XXX} & -- & \textbf{ZZZZ~€} \\
\bottomrule
\end{tabular}
\end{center}

\clearpage


% ============================================================
% PROJECTIONS D’ACTIVITÉ
% ============================================================

\subsection{Projections d’activité et recettes prévisionnelles}
\needspace{8\baselineskip}

\noindent\textbf{Hypothèses.}  
(Volume N → montée en charge N+1, N+2…)

\medskip

\begin{center}
\begin{tabular}{lccc}
\toprule
\textbf{Année} & \textbf{Volume} & \textbf{Tarif moyen} & \textbf{Recette brute} \\
\midrule
Année 1 & XXX & YYY~€ & ZZZZ~€ \\
Année 2 & XXX & YYY~€ & ZZZZ~€ \\
Année 3 & XXX & YYY~€ & ZZZZ~€ \\
\bottomrule
\end{tabular}
\end{center}

\clearpage


% ============================================================
% CONCLUSION
% ============================================================

\subsection{Conclusion}
\needspace{6\baselineskip}

% ---- TEXTE À REMPLACER ----
Texte conclusif (organisation, pertinence, continuité, efficience, prévention, etc.).

\clearpage

Les modalités techniques détaillées figurent dans les Annexes X.X à X.X.


% ============================================================
% VALIDATION DE LA FICHE
% ============================================================

\begin{center}
\begin{tabular}{p{4cm} p{7cm} p{4cm}}
\toprule
\rowcolor{APHPsoft}
\textbf{Date de relecture} & \textbf{Nom du relecteur} & \textbf{Date de validation} \\
\midrule
01/12/2025 & Pr V. Abitbol & NA \\
\bottomrule
\end{tabular}
\end{center}



\clearpage

% ============================================================
% CODAGE ET GHS ASSOCIÉS — HDJ ADDICTOLOGIE
% ============================================================

\subsection{Codage et GHS associés}

\noindent\textbf{Cadre général.}  
Le codage repose sur un Diagnostic Principal (DP) défini par le trouble lié à l’usage d’alcool, complété par des Diagnostics Associés Significatifs (DAS) pour les comorbidités somatiques ou psychiatriques.  
Les Séances HDJ sont valorisées selon les Groupes Homogènes de Séjours (GHS) applicables en très courte durée.  
Les actes techniques (CCAM) sont intégrés selon la réglementation en vigueur (imagerie, EFR, biologie).

\medskip


\begin{table}[h!]
\centering
\renewcommand{\arraystretch}{1.25}
\rowcolors{2}{APHPsoft}{white}

\begin{tabular}{p{4.3cm} p{7.3cm} c c c}
\toprule
\rowcolor{APHPsoft}
\textbf{Type de séance} &
\textbf{DP / DAS} &
\textbf{GHM} &
\textbf{GHS} &
\textbf{Tarif 2025} \\
\midrule

HDJ Évaluation somatique + addictologique &
DP: F101 (usage nocif) \newline DAS: comorbidités majeures &
20Z051 &
7272 &
774~€ \\

HDJ Réduction des risques et dommages &
DP: Z714 (conseil et surveillance pour alcoolisme) \newline DAS: comorbidités majeures &
23M06T &
7967 &
701~€ \\

HDJ Sevrage ambulatoire &
DP: Z502 (sevrage) ou F102 (dépendance) \newline DAS: comorbidités majeures &
20Z04T &
7271 &
541~€ \\

\bottomrule
\end{tabular}
\caption{Codage et GHS associés — HDJ Addictologie}
\end{table}


\begin{table}[h!]
\centering
\renewcommand{\arraystretch}{1.25}
\rowcolors{2}{APHPsoft}{white}

\begin{tabular}{p{5cm} p{9cm}}
\toprule
\rowcolor{APHPsoft}
\textbf{Intervention} & \textbf{Traçabilité requise} \\
\midrule

Évaluation somatique et addictologique &
Anamnèse, scores (AUDIT, CIWA), comorbidités, bilans préthérapeutiques, consentement, vaccination \\

Réduction des risques et dommages &
Motifs, stratégies personnalisées, entretiens motivationnels, plan d’actions, objectifs, suivi \\

Sevrage ambulatoire &
Score CIWA quotidien, adaptation thérapeutique, incidents, paramètres vitaux, supervision médicamenteuse \\

Consultations spécialisées (psychologue, diététique, AS) &
Synthèse écrite, recommandations, suivi programmé \\

Synthèse médicale &
Plan thérapeutique, orientation, continuité ville–hôpital, alertes \\

\bottomrule
\end{tabular}

\caption{Traçabilité des interventions — HDJ Addictologie}
\end{table}


\subsection{Volumétrie de référence (année N)}

\noindent Activité HDJ addictologie — base observée : 5 patients/jour (amorce), 7 (montée), 10 (croisière).

\begin{center}
\begin{tabular}{lccc}
\toprule
\textbf{Type de séance} & \textbf{Volume N} & \textbf{Tarif unitaire} & \textbf{Recette estimée} \\
\midrule

HDJ Évaluation &
893 &
774~€ &
693\,942~€ \\

HDJ Réduction des risques &
809 &
701~€ &
566\,309~€ \\

HDJ Sevrage ambulatoire &
625 &
541~€ &
338\,125~€ \\

\midrule
\textbf{Total} & \textbf{2\,327} & -- & \textbf{1\,598\,376~€} \\
\bottomrule
\end{tabular}
\end{center}


\subsection{Projections d’activité et recettes prévisionnelles}

\noindent\textbf{Hypothèses.}  
Montée en charge progressive : +30 % / +60 %.  
Tarifs constants 2025.

\begin{center}
\begin{tabular}{lccc}
\toprule
\textbf{Année} & \textbf{Volume} & \textbf{Tarif moyen} & \textbf{Recette brute} \\
\midrule

Année 1 (amorce) & 2\,327 & 675~€ & 1\,598\,376~€ \\
Année 2 (montée) & 3\,260 & 675~€ & 2\,200\,500~€ \\
Année 3 (croisière) & 4\,655 & 675~€ & 3\,139\,125~€ \\
\bottomrule
\end{tabular}
\end{center}


\subsection{Conclusion}

L’HDJ addictologie permet une prise en charge structurée, intensive et coordonnée des troubles liés à l’usage d’alcool.  
Il sécurise les sevrages, optimise la réduction des risques, renforce la continuité des soins ville–hôpital et contribue directement à prévenir complications hépatiques, rechutes et hospitalisations complètes.  
Son articulation avec l’hépatologie, la psychiatrie et la médecine interne en fait un dispositif central dans la stratégie de santé publique du GHU.

\clearpage


\begin{center}
\begin{tabular}{p{4cm} p{7cm} p{4cm}}
\toprule
\rowcolor{APHPsoft}
\textbf{Date de relecture} & \textbf{Nom du relecteur} & \textbf{Date de validation} \\
\midrule
01/12/2025 & Pr V. Abitbol & NA \\
\bottomrule
\end{tabular}
\end{center}

\clearpage
