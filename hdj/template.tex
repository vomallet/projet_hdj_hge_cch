% ============================================================
% HÔPITAL DE JOUR — TITRE DE LA FICHE
% ============================================================

\subsection{Rationnel médical}
\needspace{8\baselineskip}

\begin{spacing}{1.30}
% ---- TEXTE À REMPLACER ----
Votre texte ici.
\end{spacing}

\clearpage

% ============================================================
% OBJECTIFS
% ============================================================

\subsection{Objectifs}
\needspace{6\baselineskip}

\begin{center}
\fcolorbox{APHPdark}{APHPsoft}{
\begin{minipage}{0.95\textwidth}
\vspace{0.9em}
\begin{itemize}[leftmargin=1.1cm]
  \item objectif 1 ;
  \item objectif 2 ;
  \item objectif 3 ;
  \item objectif 4 ;
  \item objectif 5.
\end{itemize}
\vspace{0.9em}
\end{minipage}}
\end{center}

\clearpage

% ============================================================
% POPULATION ÉLIGIBLE
% ============================================================

\subsection{Population éligible}
\needspace{6\baselineskip}

\begin{itemize}[leftmargin=1.1cm]
  \item critère 1 ;
  \item critère 2 ;
  \item critère 3 ;
  \item critère 4 ;
  \item critère 5.
\end{itemize}

\clearpage

% ============================================================
% PARCOURS DE SOINS
% ============================================================

\subsection{Parcours de soins}
\needspace{8\baselineskip}

\begin{figure}[!ht]
\centering
\caption{Parcours patient — HDJ NOM}
\vspace{0.8cm}

\begin{tikzpicture}[
    node distance=1.6cm,
    box/.style={
        rectangle,
        rounded corners=3pt,
        draw=APHPdark,
        thick,
        text width=9cm,
        minimum height=1.5cm,
        align=center,
        fill=APHPsoft
    }
]
\node[box] (tri) {Orientation vers l'HDJ (source 1 / source 2 / ville)};
\node[box, below=1.4cm of tri] (etape1) {Évaluation initiale : éléments clés};
\node[box, below=1.4cm of etape1] (etape2) {Acte programmé : liste des actes};
\node[box, below=1.4cm of etape2] (etape3) {Surveillance dédiée et critères de sécurité};
\node[box, below=1.4cm of etape3] (synth) {Synthèse médicale et coordination};

\draw[->, thick, APHPdark] (tri) -- (etape1);
\draw[->, thick, APHPdark] (etape1) -- (etape2);
\draw[->, thick, APHPdark] (etape2) -- (etape3);
\draw[->, thick, APHPdark] (etape3) -- (synth);

\end{tikzpicture}
\end{figure}

\clearpage

% ============================================================
% CODAGE — VUE SYNTHÉTIQUE
% ============================================================

\subsection{Codage et GHS associés}
\needspace{8\baselineskip}

\noindent\textbf{Cadre général.}  
Préciser les DP / DR / DAS, les actes CCAM et les règles spécifiques si applicables.

\medskip

\begin{table}[h!]
\centering
\renewcommand{\arraystretch}{1.25}
\rowcolors{2}{APHPsoft}{white}

\begin{tabular}{p{4.3cm} p{7.3cm} c c c}
\toprule
\rowcolor{APHPsoft}
\textbf{Type de séance} & \textbf{DP / DR / DAS} & \textbf{GHM} & \textbf{GHS} & \textbf{Tarif} \\
\midrule
Acte 1 & DP/DAS & XXX & YYY & ZZZ~€ \\
Acte 2 & DP/DAS & XXX & YYY & ZZZ~€ \\
Acte 3 & DP/DAS & XXX & YYY & ZZZ~€ \\
\bottomrule
\end{tabular}

\caption{Codage et GHS associés — HDJ NOM}
\end{table}

\clearpage

% ============================================================
% TRACABILITÉ
% ============================================================

\subsection{Traçabilité minimale}

\begin{table}[h!]
\centering
\renewcommand{\arraystretch}{1.25}
\rowcolors{2}{APHPsoft}{white}

\begin{tabular}{p{5cm} p{9cm}}
\toprule
\rowcolor{APHPsoft}
\textbf{Intervention} & \textbf{Traçabilité requise} \\
\midrule
Acte 1 & éléments de traçabilité \\
Acte 2 & éléments de traçabilité \\
Acte 3 & éléments de traçabilité \\
\bottomrule
\end{tabular}

\caption{Traçabilité — HDJ NOM}
\end{table}

\clearpage

% ============================================================
% VOLUMÉTRIE DE RÉFÉRENCE
% ============================================================

\subsection{Volumétrie de référence (année N)}
\needspace{6\baselineskip}

\begin{table}[h!]
\centering
\renewcommand{\arraystretch}{1.20}
\rowcolors{2}{APHPsoft}{white}

\begin{tabular}{lccc}
\toprule
\textbf{Type de séance} & \textbf{Volume} & \textbf{Tarif unitaire} & \textbf{Recette} \\
\midrule
Acte 1 & XXX & YYY~€ & ZZZZ~€ \\
Acte 2 & XXX & YYY~€ & ZZZZ~€ \\
Acte 3 & XXX & YYY~€ & ZZZZ~€ \\
\midrule
\textbf{Total} & \textbf{XXX} & -- & \textbf{ZZZZ~€} \\
\bottomrule
\end{tabular}

\caption{Volumétrie HDJ — Année N}
\end{table}

\clearpage

% ============================================================
% PROJECTIONS D’ACTIVITÉ
% ============================================================

\subsection{Projections d’activité et recettes prévisionnelles}

\begin{table}[h!]
\centering
\renewcommand{\arraystretch}{1.20}
\rowcolors{2}{APHPsoft}{white}

\begin{tabular}{lccc}
\toprule
\textbf{Année} & \textbf{Volume} & \textbf{Tarif moyen} & \textbf{Recette brute} \\
\midrule
Année 1 & XXX & YYY~€ & ZZZZ~€ \\
Année 2 & XXX & YYY~€ & ZZZZ~€ \\
Année 3 & XXX & YYY~€ & ZZZZ~€ \\
\bottomrule
\end{tabular}

\caption{Projections d’activité et recettes prévisionnelles — HDJ NOM}
\end{table}

\clearpage

% ============================================================
% VALIDATION
% ============================================================

\begin{center}
\begin{tabular}{p{4cm} p{7cm} p{4cm}}
\toprule
\rowcolor{APHPsoft}
\textbf{Date de relecture} & \textbf{Relecteur} & \textbf{Date de validation} \\
\midrule
01/01/2025 & Nom & NA \\
\bottomrule
\end{tabular}
\end{center}

\clearpage
