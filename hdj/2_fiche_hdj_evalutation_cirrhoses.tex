% ============================================================
% HÔPITAL DE JOUR — ÉVALUATION DES CIRRHOSES ET HTP CLINIQUE
% ============================================================

\subsection{Rationnel médical}
\needspace{8\baselineskip}
\setcounter{table}{0}
\setcounter{figure}{0}

\begin{spacing}{1.28}

L’incidence des cirr\-hoses continue de progresser en Europe, portée par l’augmentation des hépatopathies métaboliques (MASLD), du diabète de type~2, de l’obésité et de la consommation d’alcool. La prévalence des formes compensées croît en moyenne de 3~à~5\,\% par an \cite{EASL_DecompCirrhosis_2018}. En France, les données PMSI et SNDS confirment également une hausse régulière des hospitalisations liées aux maladies hépatiques chroniques \cite{RN597}.

L’évaluation annuelle standardisée des cirr\-hoses avancées constitue un enjeu majeur de prévention des décompensations et d’optimisation du parcours patient. L’évolution de la fibrose vers la cirr\-hose s’accompagne d’une élévation du gradient porto-systémique (HVPG). Une valeur $>$\,10\,mmHg définit l’hypertension portale cliniquement significative (CSPH), étape pivot associée à la première décompensation \cite{EASL_DecompCirrhosis_2018}. Les recommandations de Bav\-eno~VII permettent d’identifier la CSPH par des critères non invasifs :
\begin{itemize}
  \item élasticité hépatique (LSM) $\geq 25$\,kPa, ou
  \item LSM de 20--25\,kPa avec plaquettes $<$\,150\,G/L \cite{RN597}.
\end{itemize}

Chez ces patients, l’introduction précoce d’un bêtabloquant non sélectif (carvédi\-lol) réduit l’HVPG, prévient la première décompensation et améliore la survie, y compris en l’absence de varices œsophagiennes.

Plusieurs facteurs aggravants doivent être systématiquement évalués :
\begin{itemize}
  \item consommation d’alcool ; l’abstinence améliore la survie \cite{Loomba_AlcoholAbstinence_2020,Addolorato_AlcoholCirrhosis_2016} ;
  \item sarco\-pénie (30--70\,\%) \cite{Tantai_Sarcopenia_2022} ;
  \item dénutrition, risque infectieux et complications extra-hépatiques.
\end{itemize}

L’évaluation annuelle d’une cirr\-hose compensée avancée requiert une approche multidimensionnelle combinant LSM hépato-splénique, imagerie, évaluation nutritionnelle et psycho\-logique/addicto\-logique, bilan vaccinal et décision endoscopique selon Bav\-eno~VII.

Un HDJ dédié permet de regrouper ces évaluations en une séance unique, standardisée et conforme aux recommandations tout en garantissant la réalisation d’au moins trois interventions valorisantes.

\end{spacing}

\clearpage

% ============================================================
% OBJECTIFS
% ============================================================

\subsection{Objectifs}
\needspace{6\baselineskip}

\begin{center}
\fcolorbox{APHPdark}{APHPsoft}{
\begin{minipage}{0.95\textwidth}
\vspace{0.9em}

\begin{itemize}[leftmargin=1cm]
  \item identifier la CSPH par critères non invasifs (Bav\-eno~VII) ;
  \item concentrer en une séance l’ensemble des évaluations clés : LSM, échographie, nutrition, psycho\-logue/addicto\-logue ;
  \item initier ou ajuster le carvédi\-lol en conditions sécurisées ;
  \item structurer un parcours annuel de prévention (HTP, alcool, sarco\-pénie) ;
  \item actualiser le statut vaccinal ;
  \item produire une synthèse médicale facilitant la coordination ville–hôpital.
\end{itemize}

\vspace{0.9em}
\end{minipage}}
\end{center}

\clearpage

% ============================================================
% POPULATION ÉLIGIBLE
% ============================================================

\subsection{Population éligible}
\needspace{6\baselineskip}

\begin{itemize}[leftmargin=1cm]
  \item cirr\-hose compensée (Child~A) ou stabilité post-décompensation ;
  \item LSM $\geq$\,15\,kPa, thrombopénie $<$\,150\,G/L ou facteurs de risque de CSPH ;
  \item consommation d’alcool active ou récente ;
  \item sarco\-pénie ou risque nutritionnel ;
  \item besoin d’une requalification annuelle selon Bav\-eno~VII.
\end{itemize}

\clearpage

% ============================================================
% PARCOURS DE SOINS
% ============================================================

\subsection{Parcours de soins}
\needspace{8\baselineskip}

\begin{figure}[!ht]
\centering
\caption{Parcours patient — HDJ cirr\-hose / HTP clinique}
\vspace{0.7cm}

\begin{tikzpicture}[
    node distance=1.5cm,
    box/.style={
        rectangle,
        rounded corners=3pt,
        draw=APHPdark,
        thick,
        text width=8.6cm,
        minimum height=1.4cm,
        align=center,
        fill=APHPsoft
    }
]

\node[box] (tri) {Orientation vers l’HDJ : hépa\-tologie, ville, MCO};
\node[box, below=1.3cm of tri] (e1) {Évaluation initiale : LSM hépato-splénique, biologie, plaquettes};
\node[box, below=1.3cm of e1] (e2) {Échographie hépatique, évaluation nutritionnelle, psycho\-logue/addicto\-logue, vaccinations};
\node[box, below=1.3cm of e2] (e3) {Décisions : indication FOGD (Bav\-eno~VII), initiation du carvédi\-lol si LSM $\geq 25$\,kPa};
\node[box, below=1.3cm of e3] (syn) {Synthèse médicale structurée et plan annuel};

\draw[->, thick, APHPdark] (tri) -- (e1);
\draw[->, thick, APHPdark] (e1) -- (e2);
\draw[->, thick, APHPdark] (e2) -- (e3);
\draw[->, thick, APHPdark] (e3) -- (syn);

\end{tikzpicture}
\end{figure}

\clearpage

% ============================================================
% CODAGE & SÉANCES — VERSION OPÉRATIONNELLE
% ============================================================

\subsection{Codage, tarifs et séances types}
\needspace{10\baselineskip}

\begin{table}[h!]
\centering
\renewcommand{\arraystretch}{1.25}
\rowcolors{2}{APHPsoft}{white}

\begin{tabularx}{\textwidth}{
p{4.6cm}
X
>{\centering\arraybackslash}p{1.4cm}
>{\centering\arraybackslash}p{1.6cm}
>{\centering\arraybackslash}p{1.8cm}
}
\toprule
\rowcolor{APHPsoft}
\textbf{Séance HDJ cirrhose} &
\textbf{Actes comptabilisés dans le gradient} &
\textbf{Nb actes} &
\textbf{GHM} &
\textbf{Tarif} \\
\midrule

\textbf{Évaluation complète multidisciplinaire} &
LSM/SSM ; échographie abdominale CCAM ; évaluation nutritionnelle ; entretien psychologique \newline
DP : Z098 \quad DR : K74.6 \quad DAS : R18, K76.6, E44.x, F10.x &
$\geq$ 4 &
07M13Z &
941~€ \\

\textbf{Évaluation intermédiaire} &
LSM/SSM ; échographie abdominale CCAM ; évaluation nutritionnelle \newline
DP : Z098 \quad DR : K74.6 \quad DAS : R18, K76.6, E44.x &
3 &
07M13Z &
420~€ \\

\textbf{LSM/SSM + échographie + synthèse médicale} &
LSM/SSM ; échographie CCAM ; consultation médicale structurée \newline
DP : Z098 \quad DR : K74.6 \quad DAS : R16.1, R18 &
3--4 &
07M13Z &
420--941~€ \\

\textbf{Évaluation nutritionnelle et psychologique} &
Évaluation nutritionnelle ; entretien psychologue ; synthèse médicale \newline
DP : Z098 \quad DR : K74.6 \quad DAS : E43--E46, F10.x, R18 &
3--4 &
07M13Z &
420--941~€ \\

\textbf{Séance incluant vaccinologie} &
LSM/SSM $\pm$ échographie CCAM ; vaccination (HAV/HBV, pneumocoque) ; synthèse médicale \newline
DP : Z098 \quad DR : K74.6 &
3--4 &
07M13Z &
420--941~€ \\

\bottomrule
\end{tabularx}

\caption{Séances d’hôpital de jour dédiées à la cirrhose : actes pris en compte dans le gradient, codage PMSI et tarifs 2025.}
\end{table}

\vspace{0.4cm}


\footnotetext{
Abréviations — LSM/SSM : Liver / Spleen Stiffness Measurement (élastométrie) ;  
échographie CCAM : échographie diagnostique codée CCAM ;  
le gradient distingue 3 actes (420~€) et $\geq$4 actes (941~€).
}


\clearpage

% ============================================================
% ORGANISATION
% ============================================================

\subsection{Organisation}
\needspace{5\baselineskip}

\begin{itemize}[leftmargin=1.1cm]
    \item Direction : \textbf{Dr Valérie D’Halluin-Venier}
    \item Durée : 4--6~heures
    \item Lieu : Secteur HDJ — Service des maladies du foie
    \item Ressources : médecin senior, infirmier expert/IPA, diététicien(ne), psychologue/addictologue
\end{itemize}

\clearpage

% ============================================================
% TRACABILITÉ
% ============================================================

\subsection{Traçabilité minimale}
\needspace{8\baselineskip}

\begin{table}[h!]
\centering
\renewcommand{\arraystretch}{1.18}
\rowcolors{2}{APHPsoft}{white}

\begin{tabularx}{\textwidth}{p{4.3cm} X}
\toprule
\rowcolor{APHPsoft}
\textbf{Intervention} & \textbf{Éléments requis} \\
\midrule
LSM hépato-splénique & Critères qualité, IQR/med, validation, seuils Bav\-eno~VII, message décisionnel \\
Échographie hépatique & CHC, HTP, flux portal, signes indirects, mesure splé\-nique \\
Évaluation nutritionnelle / sarco\-pénie & IMC, perte pondérale, dynamométrie, plan nutritionnel \\
Psychologie / addictologie & Évaluation motivationnelle, repérage, orientation \\
FOGD & Indications Bav\-eno~VII, résultats, calendrier \\
Décision carvédi\-lol & Dose initiale, titration, objectifs tensionnels, suivi IDE \\
Vaccinations & Pneumocoque, grippe, COVID, VHA/VHB \\
Synthèse médicale & Classification Bav\-eno, plan thérapeutique, coordination \\
\bottomrule
\end{tabularx}

\caption{Traçabilité — HDJ cirr\-hose}
\end{table}

\clearpage

% ============================================================
% PROJECTIONS
% ============================================================

\subsection{Projections d’activité}
\needspace{6\baselineskip}

\noindent Hypothèse : montée en charge progressive avec extension de 1~à~5~HDJ par jour.

\begin{table}[h!]
\centering
\renewcommand{\arraystretch}{1.18}
\rowcolors{2}{APHPsoft}{white}

\begin{tabular}{lcccc}
\toprule
\rowcolor{APHPsoft}
\textbf{Année} & \textbf{HDJ} & \textbf{Volume} & \textbf{Tarif} & \textbf{Recette} \\
\midrule
Amorce (1) & 1 & 90  & 690~€ & 62\,100~€ \\
Montée     & 3 & 270 & 690~€ & 186\,300~€ \\
Régime     & 5 & 450 & 690~€ & 310\,500~€ \\
\bottomrule
\end{tabular}

\caption{Prévisions d’activité (1, 3 puis 5~HDJ)}
\end{table}

\clearpage

% ============================================================
% CONCLUSION
% ============================================================

\subsection{Conclusion}

L’HDJ dédié à l’évaluation des cirr\-hoses constitue un modèle intégré conforme à Bav\-eno~VII, combinant LSM, imagerie, évaluation nutritionnelle, repérage addicto\-logique et initiation du carvédi\-lol lorsque nécessaire. Cette organisation standardisée améliore la prévention des décompensations, renforce le dépistage du CHC et optimise la coordination entre hépa\-tologie hospitalière et médecine de ville.


\clearpage

% ============================================================
% VALIDATION
% ============================================================

\begin{center}
\begin{tabular}{p{4cm} p{7cm} p{4cm}}
\toprule
\rowcolor{APHPsoft}
\textbf{Date} & \textbf{Relecteur} & \textbf{Validation} \\
\midrule

03/12/2025   & Pr V.\,Mallet   & 08/12/2025 \\
NA           & Dr L.\,Parlati & NA \\
03/12/2025   & Dr S.\,Bouam   & NA \\
NA           & Pr R.\,Coriat  & NA \\
\bottomrule
\end{tabular}
\end{center}

\clearpage
