% ============================================================
% HÔPITAL DE JOUR — BIOPSIE HÉPATIQUE (PBH)
% ============================================================

\setcounter{table}{0}
\setcounter{figure}{0}
% ============================================================
% 1. RATIONNEL MÉDICAL
% ============================================================

\subsection{Rationnel médical}
\needspace{8\baselineskip}

\begin{spacing}{1.30}

La biopsie hépatique (PBH) demeure l’examen de référence pour le diagnostic, la classification
et le suivi histologique de nombreuses maladies du foie. Malgré les progrès des outils non
invasifs, elle reste indispensable dans plusieurs situations cliniques complexes où seule
l’analyse tissulaire permet une caractérisation fine et reproductible des lésions.

Elle est essentielle dans les maladies hépatiques rares (hépatites auto-immunes, cholangites,
granulomatoses), notamment en cas de présentations atypiques ou discordantes. Dans les
toxicités médicamenteuses — en particulier sous immunothérapies ou thérapies ciblées —
la PBH permet de distinguer des lésions auto-immunes, granulomateuses, cholestatiques ou
toxiques, et occupe une place centrale dans les algorithmes décisionnels
\cite{Peeraphatdit_ImmuneHepatitis_2020, DeMartin_IOHepatitis_2020}.

La PBH est également incontournable pour documenter la réponse thérapeutique dans les
maladies chroniques du foie. Aucune méthode non invasive ne permet aujourd’hui de
caractériser de manière fiable la réorganisation architecturale hépatique ; la PBH demeure
donc l’examen de référence pour l’évaluation de la régression de la fibrose, notamment dans
les essais MASLD/MASH \cite{Friedman_FibrosisRegression_2023}.

Enfin, la PBH conserve une utilité en oncologie hépatique, en maladies infectieuses rares et
en hématologie, notamment dans l’évaluation pronostique pré-allogreffe.

L’Hôpital de Jour constitue le cadre organisationnel optimal pour la réalisation de la PBH,
associant sélection rigoureuse, guidage échographique, surveillance spécialisée et continuité
ville–hôpital.

\end{spacing}

\clearpage


% ============================================================
% OBJECTIFS
% ============================================================

\subsection{Objectifs}
\needspace{6\baselineskip}

\vspace{0.8em}
\begin{center}
\fcolorbox{APHPdark}{APHPsoft}{
\begin{minipage}{0.95\textwidth}
\vspace{0.9em}

\begin{itemize}[leftmargin=1.1cm]
  \item obtenir un diagnostic histologique précis dans les maladies du foie complexes ou atypiques ;
  \item documenter les toxicités médicamenteuses, notamment sous immunothérapie ;
  \item évaluer la fibrose et la réversion structurale dans les protocoles thérapeutiques ;
  \item caractériser des lésions tumorales bénignes ou malignes lorsque cela conditionne la stratégie ;
  \item explorer des atteintes infectieuses hépatiques rares (hépatocultures) ;
  \item réaliser l’examen dans un cadre sécurisé, standardisé et traçable en HDJ.
\end{itemize}

\vspace{0.9em}
\end{minipage}}
\end{center}

\clearpage

% ============================================================
% 2. POPULATION CIBLE
% ============================================================

\subsection{Population éligible}
\needspace{6\baselineskip}

Patients adultes relevant d’une PBH programmée :
\begin{itemize}[leftmargin=1.1cm]
  \item maladies chroniques du foie complexes ou atypiques ;
  \item suspicion de pathologie auto-immune ou cholestatique discordante ;
  \item toxicités médicamenteuses sévères, notamment sous immunothérapie ;
  \item protocoles de recherche nécessitant une évaluation histologique ;
  \item caractérisation de lésions tumorales hépatiques ;
  \item suspicion d’infection hépatique rare ;
  \item évaluation pré-allogreffe en hématologie.
\end{itemize}

\clearpage


% ============================================================
% 3. PARCOURS PATIENT
% ============================================================

\subsection{Parcours de soins}
\needspace{8\baselineskip}

\begin{figure}[!ht]
\centering
\caption{Parcours patient — PBH en Hôpital de Jour}
\vspace{0.8cm}

\begin{tikzpicture}[
    node distance=1.6cm,
    box/.style={
        rectangle,
        rounded corners=3pt,
        draw=APHPdark,
        thick,
        text width=8.6cm,
        minimum height=1.5cm,
        align=center,
        fill=APHPsoft
    }
]
\node[box] (tri) {Orientation vers l’HDJ (hépatologie / MCO / ville)};
\node[box, below=1.4cm of tri] (e1) {Évaluation initiale : coagulation, imagerie, consentement};
\node[box, below=1.4cm of e1] (e2) {Biopsie sous guidage échographique (HLHJ003/HLHJ006)};
\node[box, below=1.4cm of e2] (e3) {Surveillance spécialisée 6 heures : constantes, douleur, saignement};
\node[box, below=1.4cm of e3] (syn) {Compte-rendu, consignes et coordination ville–hôpital};

\draw[->, thick, APHPdark] (tri) -- (e1);
\draw[->, thick, APHPdark] (e1) -- (e2);
\draw[->, thick, APHPdark] (e2) -- (e3);
\draw[->, thick, APHPdark] (e3) -- (syn);

\end{tikzpicture}
\end{figure}

\clearpage

% ============================================================
% 4. ACTES RÉALISÉS
% ============================================================

\subsection{Actes réalisés}

\begin{itemize}[leftmargin=1.1cm]
  \item biopsie hépatique percutanée sous guidage échographique ;
  \item surveillance post-procédure spécialisée (6 heures) ;
  \item évaluation clinique et biologique pré- et post-acte ;
  \item information patient et coordination du suivi.
\end{itemize}


% ============================================================
% CODAGE & SÉANCES — PBH EN HDJ
% ============================================================

\subsection{Codage, tarifs et séances types}
\needspace{10\baselineskip}

\begin{table}[h!]
\centering
\renewcommand{\arraystretch}{1.25}
\rowcolors{2}{APHPsoft}{white}

\begin{tabularx}{\textwidth}{
p{4.8cm}
X
>{\centering\arraybackslash}p{1.8cm}
>{\centering\arraybackslash}p{2.0cm}
>{\centering\arraybackslash}p{2.4cm}
}
\toprule
\rowcolor{APHPsoft}
\textbf{Séance} &
\textbf{Indication et codage} &
\textbf{GHM} &
\textbf{Tarif 2025} &
\textbf{Vol. 2024 / Recette} \\
\midrule

\textbf{Fibrose|MASLD} &
DP : K74.0|K76.0 \newline
DAS : comorbidités majeures &
07M08T &
1\,238~€ &
16 \newline
19\,808~€ \\

\textbf{CHC|tumeur du foie} &
DP : C22.0 \newline
DAS : comorbidités associées &
07M06T &
1\,061~€ &
9 \newline
9\,549~€ \\

\textbf{Cirrhose alcoolique} &
DP : K70.2 &
07M07T &
840~€ &
2 \newline
1\,680~€ \\

\textbf{HAI|tumeur bénigne} &
DP : K75.4|D13.4 &
07M04T &
919~€ &
13 \newline
11\,947~€ \\

\textbf{PBH normale|NA} &
DP : R93.2 &
07M14T &
603~€ &
57 \newline
34\,371~€ \\

\midrule
\textbf{Total activité PBH HDJ} &
— &
— &
— &
\textbf{97 séances} \newline
\textbf{77\,355~€} \\
\bottomrule
\end{tabularx}

\caption{Séances d’hôpital de jour pour biopsie hépatique : indications cliniques, codage PMSI, volumétrie et recettes observées en 2024.}
\end{table}

\footnotetext{
Abréviations : PBH, biopsie hépatique percutanée ; MASLD, metabolic dysfunction–associated steatotic liver disease ;
CHC, carcinome hépatocellulaire ; HAI, hépatite auto-immune ; NA, non applicable ;
DP, diagnostic principal ; DAS, diagnostic associé significatif ; GHM, groupe homogène de malades ;
HDJ, hôpital de jour.
}


\clearpage


% ============================================================
% 6. PROJECTIONS D’ACTIVITÉ
% ============================================================

\subsection{Projections d’activité et recettes prévisionnelles}

\noindent Référence 2024 : \textbf{97 PBH}, soit \textbf{77\,355~€} (tarif moyen \textbf{798~€}). \\
Hypothèse de croissance : \textbf{+15 actes / an}. \\
Tarif moyen stabilisé : \textbf{800~€ / séance}.

\begin{table}[h!]
\centering
\renewcommand{\arraystretch}{1.20}
\rowcolors{2}{APHPsoft}{white}
\begin{tabular}{
p{4.3cm}
>{\centering\arraybackslash}p{2.2cm}
>{\centering\arraybackslash}p{2.3cm}
>{\centering\arraybackslash}p{3.0cm}
}
\toprule
\rowcolor{APHPsoft}
\textbf{Phase} & \textbf{Volume estimé} & \textbf{Tarif moyen} & \textbf{Recette brute} \\
\midrule
Amorce        & 112 & 800~€ & 89\,600~€ \\
Montée        & 127 & 800~€ & 101\,600~€ \\
Croisière     & 142 & 800~€ & 113\,600~€ \\
\bottomrule
\end{tabular}
\caption{Projections d’activité et recettes prévisionnelles — PBH HDJ (à partir de 2024)}
\end{table}

% ============================================================
% 7. RESSOURCES NÉCESSAIRES
% ============================================================

\subsection{Ressources nécessaires}

\textbf{Ressources humaines}
\begin{itemize}[leftmargin=1.1cm]
  \item hépatologue senior formé à la PBH ;
  \item IDE formée à la surveillance post-biopsie ;
  \item accès radiologue / échographie.
\end{itemize}

\textbf{Ressources matérielles et organisationnelles}
\begin{itemize}[leftmargin=1.1cm]
  \item échographe avec sonde adaptée ;
  \item matériel de biopsie stérile ;
  \item protocole de surveillance tracé (6h) ;
  \item circuit de rendu anatomopathologique.
\end{itemize}

\clearpage

\medskip
\textbf{Traçabilité et sécurité des soins}

\begin{table}[h!]
\centering
\renewcommand{\arraystretch}{1.25}
\rowcolors{2}{APHPsoft}{white}

\begin{tabular}{p{5cm} p{9cm}}
\toprule
\rowcolor{APHPsoft}
\textbf{Intervention} & \textbf{Traçabilité requise} \\
\midrule
Biopsie hépatique &
Fiche d’acte, repérage échographique, nombre de carottes, longueur, calibre \\
Surveillance spécialisée &
Feuille pluri-horaire (6h), douleur, tension, saignement local \\
Entretien médical &
Indication, consentement, comorbidités, risque hémorragique \\
Examens complémentaires &
Imageries et bilans pré-biopsie \\
Coordination &
Compte-rendu structuré, consignes, liaison ville–hôpital \\
\bottomrule
\end{tabular}

\caption{Traçabilité minimale requise — PBH en Hôpital de Jour}
\end{table}

\clearpage



% ============================================================
% 8. CONCLUSION SYNTHÉTIQUE
% ============================================================

\subsection{Conclusion}

Les HDJ de biopsie hépatique s’inscrivent dans une logique de \textbf{plateforme ambulatoire spécialisée}, structurante pour l’ensemble des services MCO du site. Elle constitue un point d’articulation central autour duquel s’organisent principalement les prises en charge d’hépatologie, mais également des parcours transversaux impliquant l’oncologie, l’immunologie et l’hématologie.
Au-delà de l’acte technique, ce modèle repose sur une organisation intégrée associant expertise médicale, plateau d’imagerie, surveillance spécialisée et coordination étroite avec les services prescripteurs. Il garantit une réponse rapide, sécurisée et standardisée à des indications complexes, tout en optimisant l’utilisation des ressources hospitalières et en limitant les hospitalisations conventionnelles.
L’activité sur site est appelée à se développer de manière significative avec l’implantation et la montée en charge des filières d’hématologie et de maladies infectieuses à CCH. Ces évolutions structurelles, en lien étroit avec les programmes de greffe de moelle osseuse et de traitements hématologiques de pointe (notamment CAR T cells) portés par l’hôpital Necker, renforceront la demande d’évaluations hépatiques spécialisées, tant diagnostiques que pronostiques.
Dans ce contexte, l’HDJ PBH apparaît comme un dispositif stratégique, adaptable et scalable, au service des filières d’excellence actuelles et futures, contribuant à la cohérence du parcours patient et à l’attractivité médico-scientifique du site.


Pour le support opérationnel,
voir l'annexe~\ref{annexe:pbh}

\clearpage


% ============================================================
% VALIDATION
% ============================================================

\begin{center}
\begin{tabular}{p{4cm} p{7cm} p{4cm}}
\toprule
\rowcolor{APHPsoft}
\textbf{Date d'envoi} & \textbf{Relecteur} & \textbf{Validation} \\
\midrule
03/12/2025 & Pr V.\,Mallet & 08/12/2025 \\
03/12/2025 & Dr S.\,Bouam & 07/12/2025 \\
NA & Dr V.\,D'Halluin & NA \\
NA & Pr R.\,Coriat & NA \\


\bottomrule
\end{tabular}
\end{center}

\clearpage
