% ============================================================
% HÔPITAL DE JOUR — ADDICTOLOGIE (HDJA)
% ============================================================
\subsection{Rationnel médical}
\begin{spacing}{1.30}
L’alcool constitue en France un déterminant majeur de morbi-mortalité évitable, responsable d’environ 41\,000 décès annuels\cite{Guerin2013AlcoolMortalite} — incluant 16\,000 cancers, 9\,900 maladies cardiovasculaires, 6\,800 maladies digestives et 5\,400 causes externes — pour un coût social estimé à 118 milliards d’euros par an.\cite{Kopp2015CoutSocial} Il représente la première cause identifiable de démence précoce (<65 ans)\cite{Schwarzinger2018AlcoholDementia}, et les troubles d’usage d’alcool réduisent l’espérance de vie de 10 à 13 ans dans leurs formes les plus sévères.\cite{Schwarzinger2017ChronicDisease}

Dans la population générale, 23,6\,\% des adultes dépassent les repères de consommation à faible risque\cite{Richard2019AlcoolSPF}, avec une prévalence particulièrement élevée dans les filières hospitalières exposées (hépatologie, oncologie, diabétologie, psychiatrie). L’alcool potentialise la progression de toutes les maladies chroniques du foie — y compris à faible niveau de consommation — et constitue le principal déterminant évolutif des complications sévères chez les patients atteints de diabète de type~2.\cite{Mallet2022T2DLiverBurden}

L’abstinence est le levier pronostique le plus puissant dans les maladies hépatiques liées à l’alcool : elle améliore la survie après hépatite alcoolique sévère\cite{Louvet2008AHA, Parlati2025RehabAH} et diminue le risque de décompensation chez les patients cirrhotiques.\cite{Addolorato_Baclofen_2007, Loomba_AlcoholAbstinence_2020} Les prises en charge intensives du sevrage réduisent également l’incidence des cancers attribuables à l’alcool et améliorent la survie globale.\cite{Schwarzinger2024RehabCancer}

Dans ce contexte, l’AP-HP a un rôle structurant à jouer dans l’organisation des parcours dédiés. Un HDJ addictologique intégré au sein du DMU DIGEST répondrait à un besoin clairement identifié : filière institutionnelle lisible, interventions spécialisées, continuité des soins et réduction des hospitalisations évitables. Ainsi, la structuration d’un HDJ addictologie s’inscrit pleinement dans les missions de santé publique de l’AP-HP.

\end{spacing}
% Réinitialisation des compteurs (doit être placée ici)

\clearpage
% ============================================================
% OBJECTIFS
% ============================================================

\subsection{Objectifs}

\begin{center}
\fcolorbox{APHPdark}{APHPsoft}{
\begin{minipage}{0.95\textwidth}
\vspace{0.7em}
\begin{itemize}[leftmargin=1.1cm]
\item Réaliser une évaluation somatique et addictologique complète ;
\item Réduire les risques et dommages liés à l’alcool ;
\item Stabiliser la trajectoire hépatologique et prévenir les décompensations ;
\item Accompagner la réduction ou l’arrêt des consommations ;
\item Renforcer l’autonomie et la continuité des soins ville–hôpital.
\end{itemize}
\vspace{0.7em}
\end{minipage}}
\end{center}

\clearpage

% ============================================================
% POPULATION ÉLIGIBLE
% ============================================================

\subsection{Population éligible}

\begin{itemize}[leftmargin=1.1cm]
\item Trouble de l’usage d’alcool (usage nocif ou dépendance) ;
\item Pathologies hépatiques alcool-attribuables (HAA, cirrhoses, MetALD) ;
\item Indication d’une évaluation somatique–addictologique conjointe ;
\item Objectif d’abstinence, de réduction des risques ou de stabilisation ;
\item Fragilités psychosociales et/ou cognitives nécessitant un suivi structuré.
\end{itemize}

\footnotetext{
\textbf{HAA} : hépatite alcoolique aiguë ; 
\textbf{MetALD} : metabolic dysfunction–associated alcohol-related liver disease.
}

\clearpage


% ============================================================
% PARCOURS DE SOINS
% ============================================================

\subsection{Parcours de soins}

% --- PAGE DES ACRONYMES (nouvelle page blanche dédiée) ---
\textbf{Acronymes utilisés dans les parcours HDJ Addictologie}
\vspace{1cm}

\begin{itemize}[leftmargin=1.2cm]
    \item \textbf{ELSA} : Équipe de liaison et de soins en addictologie
    \item \textbf{CSAPA} : Centre de soins, d’accompagnement et de prévention en addictologie
    \item \textbf{IDE} : Infirmier diplômé d’État
    \item \textbf{MOCA} : Montreal Cognitive Assessment
    \item \textbf{BEARNI} : Brief Evaluation of Alcohol-Related Neuropsychological Impairments
    \item \textbf{ECG} : Électrocardiogramme
    \item \textbf{EFR} : Explorations fonctionnelles respiratoires
    \item \textbf{CI} : Contre-indication
    \item \textbf{CIWA} : Clinical Institute Withdrawal Assessment
    \item \textbf{APA} : Activité physique adaptée
    \item \textbf{SSR} : Soins de suite et de réadaptation
\end{itemize}

\clearpage
% --- FIN PAGE ACRONYMES / DÉBUT DES FIGURES ---


% === FIGURE 1 — Évaluation somatique et addictologique ========================

\begin{figure}[h!]
\centering
\caption{Parcours évaluation addictologique et somatique}
\vspace{0.8cm}

\begin{tikzpicture}[
    node distance=1.8cm,
    box/.style={
        rectangle, rounded corners=3pt,
        draw=APHPdark, thick,
        text width=10cm, minimum height=1.7cm,
        align=center, fill=APHPsoft
    }
]

\node[box] (ori) {Orientation vers l’HDJ \\
\small médecin traitant, ELSA, CSAPA, psychiatrie, hépatologie, consultations d’addictologie};

\node[box, below=1.5cm of ori] (eval) {Évaluation médicale et infirmière \\
\small addictologie, hépatologie, entretien IDE};

\node[box, below=1.5cm of eval] (neuro) {Évaluations psychologique et neurocognitive \\
\small psychologue ; MOCA / BEARNI};

\node[box, below=1.5cm of neuro] (som) {Bilan somatique \\
\small ECG, EFR, scanner thoracique, biologie, échographie, Fibroscan};

\node[box, below=1.5cm of som] (synth) {Synthèse pluridisciplinaire \\
\small orientation : HDJ réduction / HDJ sevrage / hospitalisation complète};

\draw[->, thick, APHPdark] (ori) -- (eval);
\draw[->, thick, APHPdark] (eval) -- (neuro);
\draw[->, thick, APHPdark] (neuro) -- (som);
\draw[->, thick, APHPdark] (som) -- (synth);

\end{tikzpicture}
\end{figure}

% === FIGURE 2 — Réduction des risques ========================================

\begin{figure}[h!]
\centering
\caption{Parcours réduction des risques et dommages}
\vspace{0.8cm}

\begin{tikzpicture}[
    node distance=1.8cm,
    box/.style={rectangle, rounded corners=3pt, draw=APHPdark, thick,
    text width=10cm, minimum height=1.7cm, align=center, fill=APHPsoft}
]

\node[box] (ori) {Orientation vers l’HDJ \\
\small patients souhaitant réduire leur consommation sans objectif de sevrage complet};

\node[box, below=1.5cm of ori] (med)
{Suivi médical et paramédical renforcé \\ \small addictologie, IDE, psychologue};

\node[box, below=1.5cm of med] (medias)
{Thérapies par médiation \\ \small APA, socio-esthétique, art-thérapie, revue de presse, écriture, groupes};

\node[box, below=1.5cm of medias] (social)
{Accompagnement social \\ \small démarches administratives, insertion, gestion de crise sociale};

\node[box, below=1.5cm of social] (synth)
{Synthèse pluridisciplinaire \\ \small adaptation du programme, continuité des soins};

\draw[->, thick, APHPdark] (ori) -- (med);
\draw[->, thick, APHPdark] (med) -- (medias);
\draw[->, thick, APHPdark] (medias) -- (social);
\draw[->, thick, APHPdark] (social) -- (synth);

\end{tikzpicture}
\end{figure}

% === FIGURE 3 — Sevrage ambulatoire ========================================

\begin{figure}[h!]
\centering
\caption{Parcours Sevrage ambulatoire}
\vspace{0.8cm}

\begin{tikzpicture}[
    node distance=1.8cm,
    box/.style={rectangle, rounded corners=3pt, draw=APHPdark, thick,
    text width=10cm, minimum height=1.7cm, align=center, fill=APHPsoft}
]

\node[box] (indi)
{Indication de sevrage ambulatoire \\ \small critères de sécurité, absence de CI, environnement compatible};

\node[box, below=1.5cm of indi] (med)
{Évaluation et suivi médical \\ \small addictologie, IDE, monitorage CIWA quotidien};

\node[box, below=1.5cm of med] (psy)
{Suivi psychologique \\ \small soutien, renforcement motivationnel};

\node[box, below=1.5cm of psy] (medias)
{Thérapies par médiation \\ \small APA, art-thérapie, socio-esthétique, revue de presse, groupes};

\node[box, below=1.5cm of medias] (soc)
{Accompagnement social \\ \small démarches, stabilisation du cadre de vie};

\node[box, below=1.5cm of soc] (synth)
{Synthèse pluridisciplinaire \\ \small plan de continuité, prévention des rechutes};

\draw[->, thick, APHPdark] (indi) -- (med);
\draw[->, thick, APHPdark] (med) -- (psy);
\draw[->, thick, APHPdark] (psy) -- (medias);
\draw[->, thick, APHPdark] (medias) -- (soc);
\draw[->, thick, APHPdark] (soc) -- (synth);

\end{tikzpicture}
\end{figure}


% === FIGURE 4 — Consolidation de sevrage ====================================

\begin{figure}[h!]
\centering
\caption{Parcours Consolidation de sevrage}
\vspace{0.8cm}

\begin{tikzpicture}[
    node distance=1.8cm,
    box/.style={rectangle, rounded corners=3pt, draw=APHPdark, thick,
    text width=10cm, minimum height=1.7cm, align=center, fill=APHPsoft}
]

\node[box] (ori)
{Public concerné \\ \small post-sevrage ambulatoire ou résidentiel ; attente SSR ; retours de SSR};

\node[box, below=1.5cm of ori] (med)
{Suivi médical et paramédical structuré \\ \small addictologie ; IDE ; psychologue};

\node[box, below=1.5cm of med] (medias)
{Thérapies par médiation \\ \small APA, socio-esthétique, art-thérapie, revue de presse, écriture, groupes};

\node[box, below=1.5cm of medias] (reinser)
{Accompagnement social et réinsertion \\ \small démarches ; retour au domicile ; activités};

\node[box, below=1.5cm of reinser] (synth)
{Synthèse pluridisciplinaire \\ \small consolidation du sevrage ; prévention des rechutes};

\draw[->, thick, APHPdark] (ori) -- (med);
\draw[->, thick, APHPdark] (med) -- (medias);
\draw[->, thick, APHPdark] (medias) -- (reinser);
\draw[->, thick, APHPdark] (reinser) -- (synth);

\end{tikzpicture}
\end{figure}



% ============================================================
% TRAÇABILITÉ
% ============================================================
\clearpage
\subsection{Traçabilité des interventions}

\begin{table}[h!]
\centering
\renewcommand{\arraystretch}{1.25}
\rowcolors{2}{APHPsoft}{white}

\begin{tabular}{p{5cm} p{9.2cm}}
\toprule
\rowcolor{APHPsoft}
\textbf{Intervention} & \textbf{Traçabilité requise} \\
\midrule

Évaluation somatique et addictologique &
Anamnèse ; scores AUDIT/CIWA ; comorbidités ; bilans pré-thérapeutiques ; évaluations IDE, psychologue, neurocognition ; consultations spécialisées ; synthèse médicale. \\

Réduction des risques &
Objectifs personnalisés ; prévention ; entretiens motivationnels ; ateliers ; activités (Annexe~C.2) ; réévaluation hebdomadaire. \\

Sevrage ambulatoire &
Scores CIWA répétés ; adaptation thérapeutique ; surveillance ; activités psychologiques/sociales/médiations (Annexe~C.2) ; incidents ; synthèse médicale de fin de cure. \\

Consultations spécialisées &
Synthèse écrite obligatoire : objectifs, évolution, recommandations. \\

Synthèse médico-psycho-sociale &
Plan intégré ; orientation ; coordination ville–hôpital. \\
\bottomrule
\end{tabular}

\caption{Traçabilité des interventions — HDJ Addictologie}
\end{table}

\clearpage

% ============================================================
% ORGANISATION
% ============================================================

\subsection{Organisation}

\begin{itemize}[leftmargin=1.1cm]
    \item Direction : \textbf{Docteur Marion Corouge}
    \item Durée : 6--8~heures
    \item Lieu : Secteur HDJ — Service des maladies du foie
    \item Ressources : médecin senior, infirmier expert/IPA, psychologue/addictologue
\end{itemize}

\clearpage

% ============================================================
% SYNTHÈSE — CODAGE, VOLUMÉTRIE ET RECETTES
% ============================================================

\subsection{Synthèse codage–volumétrie–recettes}

\begin{sidewaystable}[p]
\centering
\renewcommand{\arraystretch}{1.22}
\rowcolors{2}{APHPsoft}{white}

\begin{tabularx}{\textwidth}{
p{4.8cm}
X
>{\centering\arraybackslash}p{1.6cm}
>{\centering\arraybackslash}p{1.6cm}
>{\centering\arraybackslash}p{1.7cm}
>{\centering\arraybackslash}p{2cm}
>{\centering\arraybackslash}p{2.7cm}
}
\toprule
\rowcolor{APHPsoft}
\textbf{Type de séance} &
\textbf{DP / DAS attendus} &
\textbf{GHM} &
\textbf{GHS} &
\textbf{Tarif 2025} &
\textbf{Volume annuel projeté} &
\textbf{Recette annuelle projetée} \\
\midrule

HDJ Évaluation somatique + addictologique &
DP : F101\newline
DAS : comorbidités majeures &
20Z051 & 7200 & 774~€ &
47 & 36\,378~€ \\

HDJ Réduction des risques et dommages &
DP : Z714\newline
DAS : comorbidités majeures &
23M06T & 7272 & 701~€ &
47 & 32\,947~€ \\

HDJ Sevrage ambulatoire / consolidation &
DP : F102 ou Z502\newline 
DAS : comorbidités majeures &
20Z04T & 7271 & 541~€ &
47 & 25\,478~€ \\
\midrule

\textbf{Total annuel — Phase d’amorce (1/j)} & -- & -- & -- & -- &
\textbf{141} &
\textbf{96\,703~€} \\
\bottomrule
\end{tabularx}   % ← OBLIGATOIRE : fermeture de tabularx

\caption{Synthèse pivotée — Codage, volumétrie et recettes projetées (HDJ Addictologie)}
\end{sidewaystable}

\clearpage
% ============================================================
% PROJECTIONS D’ACTIVITÉ ET RECETTES PRÉVISIONNELLES
% ============================================================

\subsection{Projections d’activité et recettes prévisionnelles}

\textbf{Hypothèses volumétriques et tarifaires (2025).}
\begin{itemize}
    \item 1, puis 3, puis 5 patients par jour ;
    \item 3 jours d’activité par semaine ;
    \item 47 semaines par an ;
    \item Tarifs GHS 2025 : 774~€ (évaluation), 701~€ (réduction des risques), 541~€ (sevrage) ;
    \item Recettes annuelles = somme pondérée des trois types d’actes selon leur fréquence moyenne observée.
\end{itemize}

\begin{table}[h!]
\centering
\renewcommand{\arraystretch}{1.22}
\rowcolors{2}{APHPsoft}{white}

\begin{tabular}{
p{5.5cm}
>{\centering\arraybackslash}p{2.5cm}
>{\centering\arraybackslash}p{2.5cm}
>{\centering\arraybackslash}p{2.5cm}
}
\toprule
\rowcolor{APHPsoft}
\textbf{Phase d’activité} &
\textbf{Patients/an} &
\textbf{Séances/an} &
\textbf{Recettes/an} \\
\midrule

Amorce (1/jour) &
141 &
141 &
96\,703~€ \\

Montée en charge (3/jour) &
423 &
423 &
289\,936~€ \\

Croisière (5/jour) &
705 &
705 &
483\,269~€ \\

\bottomrule
\end{tabular}

\caption{Projections d’activité et recettes prévisionnelles — HDJ Addictologie}
\end{table}

\clearpage


% ============================================================
% CONCLUSION
% ============================================================

\subsection{Conclusion}

L’HDJ addictologie constitue un dispositif structurant permettant une intervention précoce, intensive et coordonnée pour les patients présentant un trouble de l’usage d’alcool. Il améliore la sécurité du sevrage, la réduction des risques, la stabilité hépatique et limite les hospitalisations non programmées. Son positionnement transversal dans le GHU en fait un outil central de prévention secondaire à fort impact de santé publique.

% ============================================================
% APPELS D’ANNEXES ADDICTOLOGIE
% ============================================================

\subsection*{Annexes associées}
\begin{itemize}
  \item Annexe~\ref{annexe:addicto_activites} — Grille d’évaluation somatique–addictologique
  \item Annexe~\ref{annexe:addicto_tracabilite} — Grille des programmes de médiation (addictologie)
\end{itemize}

\clearpage


% ============================================================
% VALIDATION
% ============================================================

\begin{center}
\begin{tabular}{p{4cm} p{7cm} p{4cm}}
\toprule
\rowcolor{APHPsoft}
\textbf{Date d’envoi} & \textbf{Nom du relecteur} & \textbf{Date de validation} \\
\midrule
03/12/2025 & Pr V.\,Mallet & 08/12/2025 \\
03/12/2025 & Dr S.\,Bouam & 08/12/2025 \\
08/12/2025 & Dr M.\,Corouge & NA \\
08/12/2025 & Dr J.\,Nabarro & NA \\
08/12/2025 & Dr D.\,Karinthi & NA \\
NA          & Pr R.\,Coriat & NA \\
\bottomrule
\end{tabular}
\end{center}

\clearpage

