% ============================================================
% HÔPITAL DE JOUR — ÉVALUATION DES CIRRHOSES ET HTP CLINIQUE
% ============================================================

\subsection{Rationnel médical}
\needspace{8\baselineskip}

\begin{spacing}{1.28}

L’incidence des cirrhoses continue de progresser en Europe, portée par l’augmentation des hépatopathies métaboliques (MASLD), du diabète de type 2, de l’obésité et de la consommation d’alcool. La prévalence des formes compensées croît en moyenne de 3 à 5 % par an \cite{Gines_LancetCirrhosis_2021}. En France, les données PMSI et SNDS confirment également une hausse régulière des hospitalisations liées aux maladies hépatiques chroniques \cite{FrenchHepaticFailure_2020}.
L’évolution de la fibrose vers la cirrhose s’accompagne d’une élévation du gradient porto-systémique (HVPG). Une valeur $>$ 10 mmHg définit l’hypertension portale cliniquement significative (CSPH), étape pivot à partir de laquelle survient la première décompensation \cite{EASL_DecompCirrhosis_2018}. Les recommandations de Baveno VII permettent aujourd’hui d’identifier la CSPH par des critères non invasifs :
\begin{itemize}
\item une élasticité hépatique (LSM) $\geq 25$ kPa, ou
\item une LSM comprise entre 20–25 kPa associée à un taux de plaquettes $<$ 150 G/L \cite{RN597}.
\end{itemize}
Chez les patients répondant à ces critères, l’introduction précoce d’un bêtabloquant non sélectif (carvédilol) réduit l’HVPG, prévient la première décompensation et améliore la survie, y compris en l’absence de varices œsophagiennes.
Plusieurs facteurs aggravants doivent être systématiquement évalués :
\begin{itemize}
\item la consommation d’alcool, dont l’effet accélérateur sur la progression de la maladie est majeur ; l’abstinence améliore nettement la survie \cite{Loomba_AlcoholAbstinence_2020,Addolorato_AlcoholCirrhosis_2016} ;
\item la sarcopénie, présente chez 30–70 % des patients et constituant un déterminant pronostique indépendant \cite{Tantai_Sarcopenia_2022} ;
\item la dénutrition, le risque infectieux et les complications extra-hépatiques.
\end{itemize}
L’évaluation annuelle d’une cirrhose compensée avancée nécessite une prise en charge multidimensionnelle intégrant la mesure de l’élasticité hépatique et splénique, l’évaluation nutritionnelle et psychologique/addictologique, la vérification du statut vaccinal et la décision endoscopique selon Baveno VII.
La structuration d’un hôpital de jour dédié permet de coordonner l’ensemble de ces évaluations en une séance unique, standardisée et conforme aux recommandations internationales, tout en garantissant la réalisation d’au moins trois interventions valorisantes.

\end{spacing}

\clearpage

% ============================================================
% OBJECTIFS
% ============================================================

\subsection{Objectifs}
\needspace{6\baselineskip}

\begin{center}
\fcolorbox{APHPdark}{APHPsoft}{
\begin{minipage}{0.95\textwidth}
\vspace{0.8em}

\begin{itemize}[leftmargin=1cm]
  \item identifier les patients présentant une CSPH selon les critères non invasifs (Baveno~VII) ;
  \item concentrer en une seule séance l’ensemble des évaluations clés : LSM, échographie, nutrition, psychologue/addictologue ;
  \item initier ou adapter le traitement par carvédilol dans des conditions sécurisées ;
  \item structurer un parcours annuel de prévention des décompensations (HTP, alcool, sarcopénie) ;
  \item vérifier et mettre à jour les vaccinations recommandées ;
  \item produire une synthèse médicale standardisée facilitant la coordination ville–hôpital.
\end{itemize}

\vspace{0.8em}
\end{minipage}}
\end{center}

\bigskip

% ============================================================
% POPULATION ÉLIGIBLE
% ============================================================

\subsection{Population éligible}

\begin{itemize}[leftmargin=1cm]
  \item cirrhose compensée (Child~A) ou stabilité après décompensation ;
  \item LSM $\geq$15 kPa, thrombopénie $<$150 G/L ou facteurs de risque de CSPH ;
  \item consommation d’alcool active ou récente ;
  \item sarcopénie ou risque nutritionnel (IMC, perte musculaire, perte pondérale) ;
  \item besoin d’un suivi annuel structuré ou d’une requalification selon Baveno~VII.
\end{itemize}

\clearpage

% ============================================================
% PARCOURS DE SOINS
% ============================================================

\subsection{Parcours de soins}

\begin{figure}[!ht]
\centering
\caption{Parcours patient — HDJ cirrhose / HTP clinique}
\vspace{0.7cm}

\begin{tikzpicture}[
    node distance=1.5cm,
    box/.style={
        rectangle,
        rounded corners=3pt,
        draw=APHPdark,
        thick,
        text width=9cm,
        minimum height=1.4cm,
        align=center,
        fill=APHPsoft
    }
]

\node[box] (tri) {Orientation vers l’HDJ : hépatologie, ville, MCO};
\node[box, below=1.3cm of tri] (e1) {Évaluation initiale : LSM hépato-splénique, biologie, plaquettes};
\node[box, below=1.3cm of e1] (e2) {Échographie hépatique, évaluation nutritionnelle, consultation psychologue/addictologue, vaccinations};
\node[box, below=1.3cm of e2] (e3) {Décisions : indication FOGD (Baveno VII), initiation du carvédilol si LSM $\geq$25 kPa};
\node[box, below=1.3cm of e3] (syn) {Synthèse médicale structurée et plan annuel};

\draw[->, thick, APHPdark] (tri) -- (e1);
\draw[->, thick, APHPdark] (e1) -- (e2);
\draw[->, thick, APHPdark] (e2) -- (e3);
\draw[->, thick, APHPdark] (e3) -- (syn);

\end{tikzpicture}
\end{figure}

\clearpage

% ============================================================
% CODAGE ET GHS ASSOCIÉS
% ============================================================

\subsection{Codage et GHS associés}

\noindent\textbf{Cadre général.}  
HDJ médico-fonctionnel intégrant plusieurs interventions valorisantes (≥3).  
Le **DP recommandé est R18** (ascite / hypertension portale clinique), optimisant la valorisation PMSI.  
Les DR/DAS sont adaptés à l’étiologie : K70–K77, F10.x, E44.x, K76.6.

\medskip

\begin{table}[h!]
\centering
\renewcommand{\arraystretch}{1.2}
\rowcolors{2}{APHPsoft}{white}

\begin{tabular}{p{4.3cm} p{7.3cm} c c c}
\toprule
\rowcolor{APHPsoft}
\textbf{Séance} &
\textbf{DP / DR / DAS} &
\textbf{GHM} &
\textbf{GHS} &
\textbf{Tarif} \\
\midrule

Évaluation complète cirrhose &
DP : R18 \newline DR : K74.6 \newline DAS : K76.6, E44.x, F10.x &
28Z07Z &
9572 &
\textasciitilde 320 € \\

LSM + échographie &
DP : R18 \newline DAS : R16.1, K76.6 &
28Z07Z &
9572 &
\textasciitilde 320 € \\

Évaluation nutrition + psychologue &
DP : R18 \newline DAS : E43–E46, F10.x &
28Z07Z &
9572 &
\textasciitilde 320 € \\
\bottomrule
\end{tabular}

\caption{Codage et GHS associés — HDJ cirrhose / HTP clinique}
\end{table}

\clearpage

% ============================================================
% TRACABILITÉ
% ============================================================

\subsection{Traçabilité minimale}

\begin{table}[h!]
\centering
\renewcommand{\arraystretch}{1.2}
\rowcolors{2}{APHPsoft}{white}

\begin{tabular}{p{5cm} p{8.5cm}}
\toprule
\rowcolor{APHPsoft}
\textbf{Intervention} & \textbf{Éléments requis} \\
\midrule
LSM hépatique/splénique &
Critères qualité, IQR/med, seuils Baveno, message décisionnel \\

Échographie hépatique &
Compte-rendu CHC, HTP, flux portal, signes indirects \\

Nutrition / sarcopénie &
IMC, perte pondérale, force musculaire, plan nutritionnel \\

Psychologie / alcool &
Évaluation motivationnelle, repérage addictologique, orientation \\

FOGD & 
Application Baveno VII, justification, calendrier \\

Décision carvédilol &
Dose initiale, titration, objectifs de PA, surveillance IDE \\

Vaccinations &
Mises à jour : grippe, pneumocoque, covid, hépatites \\

Synthèse médicale &
Résumé standardisé, plan de suivi, coordination ville–hôpital \\
\bottomrule
\end{tabular}

\caption{Traçabilité — HDJ évaluation des cirrhoses}
\end{table}

\clearpage

% ============================================================
% VOLUMÉTRIE
% ============================================================

\subsection{Volumétrie de référence}

\noindent File active annuelle : 7\,500 patients. Prévalence estimée de la cirrhose : 21~\%, soit **1\,575 patients** potentiellement concernés.

\begin{center}
\begin{tabular}{lccc}
\toprule
\textbf{Séance} & \textbf{Volume 2025 (est.)} & \textbf{Tarif unitaire} & \textbf{Recette} \\
\midrule
Évaluation annuelle complète & 450 & 320~€ & 144\,000~€ \\
LSM + échographie & 600 & 320~€ & 192\,000~€ \\
Consultations psycho/diète & 350 & 320~€ & 112\,000~€ \\
\midrule
\textbf{Total} & \textbf{1\,400} & -- & \textbf{448\,000~€} \\
\bottomrule
\end{tabular}
\end{center}

\clearpage

% ============================================================
% PROJECTIONS
% ============================================================

\subsection{Projections d’activité}

\noindent Hypothèse : accroissement progressif du recours à l’HDJ (30~\% → 70~\%).

\begin{center}
\begin{tabular}{lccc}
\toprule
\textbf{Année} & \textbf{Volume estimé} & \textbf{Tarif moyen} & \textbf{Recette brute} \\
\midrule
2026 & 1\,600 & 320~€ & 512\,000~€ \\
2027 & 1\,900 & 320~€ & 608\,000~€ \\
2028 & 2\,200 & 320~€ & 704\,000~€ \\
\bottomrule
\end{tabular}
\end{center}

\clearpage

% ============================================================
% CONCLUSION
% ============================================================

\subsection{Conclusion}

L’HDJ dédié à l’évaluation des cirrhoses offre une organisation intégrée, conforme à Baveno~VII, regroupant LSM, imagerie, évaluation nutritionnelle, repérage addictologique et initiation du carvédilol lorsque indiqué. Cette approche coordonnée renforce la prévention des décompensations, améliore la qualité du dépistage du CHC et fluidifie les parcours entre l’hépatologie et la médecine de ville.
