% ============================================================
% 3. PLATEAU HDJ DIGESTIF MUTUALISÉ — ORGANISATION TRANSVERSALE
% ============================================================

\titleformat{\subsection}[runin]
  {\bfseries\color{APHPblue}}
  {}
  {0pt}
  {}

\label{sec:plateau_hdj_mutualise}

\begin{spacing}{1.25}

% ============================================================
% 3.1 PRINCIPES D’ORGANISATION
% ============================================================

\subsection{Principes d’organisation}

Le plateau d’HDJ digestif mutualisé repose sur une organisation transversale visant à optimiser l’efficience médico-soignante, la lisibilité des parcours et l’utilisation des ressources humaines et matérielles.  
Il constitue un dispositif unique regroupant, sur un même site, l’ensemble des filières ambulatoires digestives et interventionnelles, selon des processus harmonisés et sécurisés.

Le modèle repose sur une \textbf{organisation graduée des prises en charge ambulatoires}, permettant d’adapter les moyens mobilisés à l’intensité réelle des besoins cliniques, tout en garantissant un haut niveau de sécurité.

Les principes structurants sont les suivants :
\begin{itemize}[leftmargin=1.1cm]
    \item mutualisation des espaces, des équipes soignantes et des circuits logistiques ;
    \item standardisation des parcours d’entrée, de prise en charge et de sortie ;
    \item distinction fonctionnelle des zones selon l’intensité de surveillance requise ;
    \item allocation graduée des ressources en fonction du risque et de la durée de prise en charge.
\end{itemize}

% ============================================================
% 3.2 PARCOURS ET FLUX PATIENTS
% ============================================================

\subsection{Parcours et flux patients}

Les patients suivent un parcours organisationnel commun, modulé selon la filière et la complexité clinique, garantissant à la fois fluidité, sécurité et lisibilité.

\begin{itemize}[leftmargin=1.1cm]
    \item \textbf{Orientation} : consultation spécialisée, RCP, urgences, ville ou hospitalisation complète.
    \item \textbf{Programmation} : affectation à un niveau d’HDJ adapté.
    \item \textbf{Accueil centralisé} : admission administrative et paramédicale unique.
    \item \textbf{Prise en charge} : actes médicaux, techniques ou multidisciplinaires.
    \item \textbf{Surveillance post-acte} : proportionnée au niveau de risque.
    \item \textbf{Sortie sécurisée} : synthèse médicale, prescriptions et organisation du suivi.
\end{itemize}

Cette structuration garantit une gestion efficiente des flux tout en sécurisant les parcours complexes.

% ============================================================
% 3.3 NIVEAUX DE PRISE EN CHARGE AMBULATOIRE
% ============================================================

\subsection{Niveaux de prise en charge ambulatoire}

Le plateau HDJ est structuré autour de \textbf{trois niveaux fonctionnels}, définis selon l’intensité de surveillance requise :

\begin{itemize}[leftmargin=1.1cm]
    \item \textbf{Niveau~1 — Lits médicalisés} :  
    patients nécessitant une surveillance continue ou des actes à risque élevé (cirrhoses avancées, procédures interventionnelles, chimiothérapies prolongées).
    
    \item \textbf{Niveau~2 — Fauteuils de soins} :  
    prises en charge intermédiaires avec surveillance infirmière directe (albumine, fer IV, biothérapies, chimiothérapies simples).
    
    \item \textbf{Niveau~3 — Espace non allongé de type \emph{lounge}} :  
    patients autonomes ne nécessitant ni lit ni surveillance continue, mais requérant une évaluation médicale et paramédicale structurée et multidisciplinaire (hépatométabolique, cirrhose compensée, addictologie, MICI sélectionnées).
\end{itemize}

\paragraph{Fonctionnement opérationnel de l’espace \emph{lounge}}

L’espace \emph{lounge} constitue un dispositif ambulatoire spécifique, centré sur la gestion de parcours cliniques à faible intensité de surveillance mais à forte composante évaluative et multidisciplinaire. Il ne s’agit pas d’un espace de soins continus, mais d’un lieu d’ancrage du patient au sein du plateau, autour duquel s’organisent des interventions médicales et paramédicales séquentielles.

Les patients y sont installés sur des fauteuils non allongés pour des durées variables, généralement \textbf{de 180 minutes} (2-3 rotations par jour), intégrant des temps d’attente clinique active, de coordination et d’évaluation. Durant ce temps, les différents professionnels interviennent de manière non simultanée, selon une logique de rotation autour du patient.

Les intervenants concernés incluent notamment :
\begin{itemize}[leftmargin=1.1cm]
    \item médecins spécialistes (hépatologie, MICI, addictologie) ;
    \item diététicien(ne)s et nutritionnistes ;
    \item psychologues et addictologues ;
    \item consultants spécialisés, le cas échéant ;
    \item interface avec le plateau médico-technique, notamment radiologique.
\end{itemize}

Le fauteuil \emph{lounge} constitue ainsi le point fixe du parcours ambulatoire, tandis que les compétences médicales et paramédicales gravitent autour du patient, permettant une optimisation des ressources sans mobilisation prolongée de lits ou de fauteuils de soins.

Cette organisation induit une variabilité intrinsèque des durées de séjour et des temps d’occupation des fauteuils \emph{lounge}. En pratique, le fonctionnement courant repose sur \textbf{une à deux rotations quotidiennes par fauteuil}, avec des possibilités ponctuelles de troisième rotation sur des parcours très courts, sans que celle-ci constitue une cible soutenable à l’échelle annuelle. Ces éléments constituent le cadre de référence organisationnel retenu pour le dimensionnement capacitaire et les hypothèses de productivité présentées dans les sections ultérieures.

% ============================================================
% 3.4 ZONES FONCTIONNELLES
% ============================================================

\subsection{Zones fonctionnelles}

La traduction spatiale de cette organisation repose sur des zones fonctionnelles clairement identifiées :

\begin{itemize}[leftmargin=1.1cm]
    \item zone d’accueil et d’admission centralisée ;
    \item zone de lits HDJ pour surveillance continue ;
    \item zone de fauteuils de soins ;
    \item zone non allongée dédiée aux parcours multidisciplinaires ;
    \item zones de consultations intégrées (diététique, psychologie/addictologie) ;
    \item zone technique mutualisée (échographie, élastographie, préparation des traitements) ;
    \item zone logistique (pharmacie, stockage, circuits propres et sales).
\end{itemize}

% ============================================================
% 3.5 MUTUALISATION DES RESSOURCES
% ============================================================

\subsection{Mutualisation des ressources humaines et des locaux}

Le fonctionnement du plateau repose sur une mutualisation complète des compétences et des espaces :

\begin{itemize}[leftmargin=1.1cm]
    \item IDE expertes formées à l’ensemble des niveaux de prise en charge ;
    \item IPA assurant coordination, évaluation et suivi protocolisé ;
    \item médecins seniors intervenant par vacations ciblées ;
    \item psychologues, diététicien(ne)s et addictologues intégrés au plateau ;
    \item locaux partagés permettant une flexibilité maximale des capacités.
\end{itemize}

\clearpage


% ============================================================
% 3.6 SCHÉMA FONCTIONNEL UNIQUE DU PLATEAU HDJ
% ============================================================

\subsection{Schéma fonctionnel unique du plateau HDJ digestif mutualisé}

\begin{figure}[!ht]
\centering
\vspace{0.6cm}

\begin{tikzpicture}[
    node distance=1.4cm,
    box/.style={
        rectangle,
        rounded corners=3pt,
        draw=APHPdark,
        thick,
        text width=5.4cm,
        minimum height=1.2cm,
        align=center,
        fill=APHPsoft
    },
    arrow/.style={->, thick, APHPdark}
]

\node[box] (orientation) {Orientation \\ Consultation / Ville / Urgences / MCO};
\node[box, below=of orientation] (accueil) {Accueil HDJ centralisé \\ Admission IDE};
\node[box, below=of accueil] (soins) {Zones de soins mutualisées \\ Lits / Fauteuils / Lounge};
\node[box, below=of soins] (surv) {Surveillance post-acte \\ IDE / IPA};
\node[box, below=of surv] (sortie) {Synthèse médicale \\ Sortie sécurisée};

\node[box, right=2.8cm of soins] (ressources) {Ressources mutualisées \\ IDE expertes \\ IPA \\ Locaux \\ Plateaux techniques};

\draw[arrow] (orientation) -- (accueil);
\draw[arrow] (accueil) -- (soins);
\draw[arrow] (soins) -- (surv);
\draw[arrow] (surv) -- (sortie);
\draw[arrow] (ressources) -- (soins);

\end{tikzpicture}

\caption{Organisation fonctionnelle graduée du plateau HDJ digestif mutualisé intégrant lits,
fauteuils et espace \emph{lounge}}
\end{figure}

\end{spacing}

\clearpage

% ============================================================
% 3.7 GOUVERNANCE DU PLATEAU HDJ DIGESTIF MUTUALISÉ
% ============================================================

\subsection{Gouvernance du plateau HDJ digestif mutualisé}

Le plateau mutualisé d’hôpitaux de jour digestifs est placé sous une responsabilité médicale unique, garante de la cohérence des parcours, de la sécurité des prises en charge et de l’harmonisation des pratiques entre filières.

L’organisation repose sur une articulation formalisée entre le DMU Hématologie Cancérologie et Spécialités Médico-Chirurgicales, la radiologie interventionnelle et les plateaux médico-techniques associés.  
L’encadrement soignant transversal assure la coordination opérationnelle, la gestion des flux et la continuité des soins.

Le pilotage médico-économique s’appuie sur un suivi régulier de l’activité, des indicateurs de qualité et de sécurité, en lien avec le DIM et les directions fonctionnelles, permettant une adaptation continue du dispositif.
