% ============================================================
% 2. PLATEAU HDJ DIGESTIF MUTUALISÉ — ORGANISATION TRANSVERSALE
% ============================================================
\titleformat{\subsection}[runin]
  {\bfseries\color{APHPblue}}
  {}
  {0pt}
  {}

\label{sec:plateau_hdj_mutualise}

\begin{spacing}{1.25}

% ============================================================
% 2.1 PRINCIPES D’ORGANISATION
% ============================================================

\subsection{Principes d’organisation}

Le plateau d’HDJ digestif mutualisé repose sur une organisation transversale visant à optimiser l’efficience médico-soignante, la lisibilité des parcours et l’utilisation des ressources humaines et matérielles. Il constitue un dispositif unique regroupant l’ensemble des filières ambulatoires digestives et interventionnelles sur un même site, avec des processus standardisés.

L’originalité du modèle repose sur une \textbf{organisation graduée des prises en charge ambulatoires}, articulée autour de \textbf{trois niveaux fonctionnels distincts}, définis selon l’intensité de surveillance requise et la complexité médicale, permettant d’allouer les ressources au plus près des besoins cliniques réels.

Les principes structurants sont les suivants :
\begin{itemize}[leftmargin=1.1cm]
    \item mutualisation des espaces, des équipes soignantes et des circuits logistiques ;
    \item séparation claire des zones patients (accueil, soins, surveillance) et des zones techniques ;
    \item standardisation des parcours d’entrée, de surveillance et de sortie ;
    \item graduation des capacités d’accueil selon le niveau de risque et la durée de prise en charge.
\end{itemize}

% ============================================================
% 2.2 FLUX PATIENTS
% ============================================================

\subsection{Flux patients}

L’ensemble des patients suit un parcours commun, modulé selon la filière et le niveau de complexité de la prise en charge, garantissant à la fois fluidité organisationnelle et sécurité clinique.

\begin{itemize}[leftmargin=1.1cm]
    \item \textbf{Orientation} : consultation spécialisée, RCP, urgences, ville ou hospitalisation complète.
    \item \textbf{Programmation} : orientation vers un niveau d’HDJ adapté (lit, fauteuil, lounge).
    \item \textbf{Accueil centralisé} : admission administrative et paramédicale unique.
    \item \textbf{Soins et évaluations} : actes médicaux, techniques ou multidisciplinaires.
    \item \textbf{Surveillance post-acte} : adaptée au niveau de risque clinique.
    \item \textbf{Sortie sécurisée} : synthèse médicale, prescriptions et organisation du suivi.
\end{itemize}

Cette organisation graduée permet une gestion efficiente des flux tout en sécurisant les parcours complexes.

% ============================================================
% 2.3 NIVEAUX DE PRISE EN CHARGE AMBULATOIRE
% ============================================================

\subsection{Niveaux de prise en charge ambulatoire}

Le plateau HDJ est structuré autour de \textbf{trois niveaux fonctionnels}, correspondant à des intensités croissantes de surveillance :

\begin{itemize}[leftmargin=1.1cm]
    \item \textbf{Niveau~1 — Lits médicalisés} :  
    patients nécessitant une surveillance continue, un alitement prolongé ou des actes à risque (cirrhoses avancées, procédures interventionnelles, chimiothérapies prolongées).
    
    \item \textbf{Niveau~2 — Fauteuils de soins} :  
    prises en charge intermédiaires avec surveillance infirmière directe (albumine, fer IV, biothérapies, chimiothérapies simples).
    
    \item \textbf{Niveau~3 — Espace non allongé de type \emph{lounge}} :  
    patients autonomes ne nécessitant ni lit ni surveillance continue, mais requérant un temps médical et paramédical structuré, séquentiel et multidisciplinaire (hépatométabolique, cirrhose compensée, addictologie, MICI sélectionnées).
\end{itemize}

L’espace \emph{lounge}, inspiré des standards de confort des salons premium, constitue un levier majeur d’optimisation capacitaire en permettant d’absorber des volumes importants sans surdimensionnement hospitalier.

% ============================================================
% 2.4 ZONES FONCTIONNELLES
% ============================================================

\subsection{Zones fonctionnelles}

Le plateau HDJ est structuré en zones fonctionnelles clairement identifiées :

\begin{itemize}[leftmargin=1.1cm]
    \item \textbf{Zone accueil et admission} : secrétariat, accueil IDE, évaluation initiale.
    \item \textbf{Zone lits HDJ} : lits médicalisés pour surveillance continue.
    \item \textbf{Zone fauteuils HDJ} : soins ambulatoires intermédiaires.
    \item \textbf{Zone lounge (non allongée)} : évaluations multidisciplinaires programmées.
    \item \textbf{Zone consultations intégrées} : diététique, psychologie/addictologie, synthèses médicales.
    \item \textbf{Zone technique mutualisée} : échographie, élastographie, préparation des traitements.
    \item \textbf{Zone logistique} : pharmacie, stockage, circuits propres et sales.
\end{itemize}

% ============================================================
% 2.5 MUTUALISATION DES RESSOURCES HUMAINES ET DES LOCAUX
% ============================================================

\subsection{Mutualisation des ressources humaines et des locaux}

Le fonctionnement du plateau repose sur une mutualisation complète des compétences et des espaces :

\begin{itemize}[leftmargin=1.1cm]
    \item \textbf{IDE expertes} formées aux différents niveaux d’HDJ ;
    \item \textbf{IPA} assurant coordination, évaluation et suivi protocolisé ;
    \item \textbf{Médecins seniors} intervenant par vacations ciblées ;
    \item \textbf{Psychologues, diététicien(ne)s, addictologues} intégrés au plateau ;
    \item \textbf{Locaux partagés} permettant une flexibilité maximale des capacités.
\end{itemize}
\clearpage

% ============================================================
% 2.6 SCHÉMA FONCTIONNEL UNIQUE DU PLATEAU HDJ
% ============================================================

\subsection{Schéma fonctionnel unique du plateau HDJ digestif mutualisé}

\begin{figure}[!ht]
\centering
\vspace{0.6cm}

\begin{tikzpicture}[
    node distance=1.4cm,
    box/.style={
        rectangle,
        rounded corners=3pt,
        draw=APHPdark,
        thick,
        text width=5.4cm,
        minimum height=1.2cm,
        align=center,
        fill=APHPsoft
    },
    arrow/.style={->, thick, APHPdark}
]

\node[box] (orientation) {Orientation \\ Consultation / Ville / Urgences / MCO};
\node[box, below=of orientation] (accueil) {Accueil HDJ centralisé \\ Admission IDE};
\node[box, below=of accueil] (soins) {Zones de soins mutualisées \\ Lits / Fauteuils / Lounge};
\node[box, below=of soins] (surv) {Surveillance post-acte \\ IDE / IPA};
\node[box, below=of surv] (sortie) {Synthèse médicale \\ Sortie sécurisée};

\node[box, right=2.8cm of soins] (ressources) {Ressources mutualisées \\ IDE expertes \\ IPA \\ Locaux \\ Plateaux techniques};

\draw[arrow] (orientation) -- (accueil);
\draw[arrow] (accueil) -- (soins);
\draw[arrow] (soins) -- (surv);
\draw[arrow] (surv) -- (sortie);
\draw[arrow] (ressources) -- (soins);

\end{tikzpicture}

\caption{Organisation fonctionnelle graduée du plateau HDJ digestif mutualisé intégrant lits,
fauteuils et espace \emph{lounge}}
\end{figure}

\end{spacing}

\clearpage

% ============================================================
% GOUVERNANCE DU PLATEAU HDJ DIGESTIF MUTUALISÉ
% ============================================================

\subsection*{Gouvernance du plateau HDJ digestif mutualisé}

\textbf{Gouvernance et pilotage.}  
Le plateau mutualisé d’hôpitaux de jour digestifs sera placé sous une responsabilité médicale unique, garantissant la cohérence des parcours, la sécurité des prises en charge et l’harmonisation des pratiques entre les différentes filières (hépatologie, oncologie digestive, addictologie, MICI, radiologie interventionnelle).

L’organisation reposera sur une articulation formalisée entre le DMU DIGEST, les équipes de radiologie interventionnelle et les plateaux médico-techniques associés. L’encadrement soignant transversal assurera la coordination opérationnelle quotidienne, la gestion des flux patients et la continuité des soins.

Le pilotage médico-économique s’appuiera sur un suivi régulier de l’activité (volumétrie, codage, recettes), des indicateurs de qualité et de sécurité, en lien avec le DIM et les directions fonctionnelles concernées. Une revue périodique du fonctionnement du plateau permettra d’adapter les capacités, les organisations et les parcours en fonction de l’évolution de l’activité et des besoins institutionnels.
