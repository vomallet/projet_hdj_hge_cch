% ============================================================
% 3. PLATEAU HDJ DIGESTIF MUTUALISÉ — ORGANISATION TRANSVERSALE
% ============================================================
\titleformat{\subsection}[runin]
  {\bfseries\color{APHPblue}}
  {}
  {0pt}
  {}
\label{sec:plateau_hdj_mutualise}
\begin{spacing}{1.3}

% ============================================================
% 3.1 PRINCIPES D'ORGANISATION
% ============================================================
\subsection{Principes d'organisation —}
Le plateau d'HDJ digestif mutualisé constitue un dispositif unique regroupant, sur un même site, l'ensemble des filières ambulatoires digestives et interventionnelles. Son organisation repose sur quatre principes structurants :

\begin{itemize}[leftmargin=1.1cm]
    \item \textbf{Graduation des prises en charge} : adaptation des moyens à l'intensité réelle des besoins cliniques ;
    \item \textbf{Mutualisation des ressources} : partage des espaces, des équipes et des circuits logistiques ;
    \item \textbf{Flexibilité capacitaire} : système de vases communicants entre filières pour optimiser l'occupation ;
    \item \textbf{Excellence hôtelière} : environnement de qualité garantissant confort, intimité et dignité des patients.
\end{itemize}

% ============================================================
% 3.2 QUALITÉ HÔTELIÈRE ET INTIMITÉ DES PATIENTS
% ============================================================
\subsection{Qualité hôtelière et intimité des patients —}
Le plateau HDJ est conçu pour offrir un environnement de soins haut de gamme, différenciant et attractif. L'intimité des patients constitue une exigence non négociable, traduite par des choix architecturaux et organisationnels structurants :

\begin{itemize}[leftmargin=1.1cm]
    \item \textbf{Chambres individuelles} pour l'ensemble des lits médicalisés (niveau 1) — pas de chambres doubles ;
    \item \textbf{Boxes individuels ou semi-cloisonnés} pour les fauteuils de soins (niveau 2) ;
    \item \textbf{Espaces lounge} aménagés avec séparations visuelles et acoustiques entre patients ;
    \item \textbf{Circulations différenciées} entre patients entrants, sortants et en cours de soins ;
    \item \textbf{Ambiance soignée} : lumière naturelle, mobilier contemporain, signalétique claire.
\end{itemize}

Cette approche répond à une double exigence : respecter la dignité de patients souvent fragilisés par des pathologies chroniques lourdes, et positionner le plateau comme une référence en termes de qualité perçue au sein du GHU.

\clearpage
% ============================================================
% 3.3 PARCOURS ET FLUX PATIENTS
% ============================================================
\subsection{Parcours patient —}
Les patients suivent un parcours standardisé, modulé selon la filière et la complexité clinique :

\begin{enumerate}[leftmargin=1.1cm]
\begin{spacing}{2}

    \item \textbf{Orientation} — consultation spécialisée, RCP, urgences, ville ou hospitalisation complète.
    \item \textbf{Classification} — attribution d'un niveau de prise en charge (1, 2 ou 3) selon critères médicaux.
    \item \textbf{Programmation} — affectation à une place et un créneau horaire.
    \item \textbf{Accueil centralisé} — admission administrative et paramédicale unique.
    \item \textbf{Prise en charge} — actes médicaux, techniques ou multidisciplinaires.
    \item \textbf{Surveillance} — proportionnée au niveau de risque.
    \item \textbf{Sortie sécurisée} — synthèse médicale, prescriptions et organisation du suivi.
        
\end{spacing}
\end{enumerate}

\clearpage
% ============================================================
% 3.4 NIVEAUX DE PRISE EN CHARGE — VUE D'ENSEMBLE
% ============================================================
\subsection{Trois niveaux de prise en charge —}
Le plateau HDJ est structuré autour de trois niveaux fonctionnels, définis selon l'intensité de surveillance requise et la durée de séjour. Cette graduation permet une allocation optimale des ressources et une rotation adaptée à chaque type de prise en charge.

\begin{table}[!ht]
\centering
\small
\rowcolors{2}{APHPsoft}{white}
\begin{tabular}{@{}p{2.8cm}p{2cm}p{2.2cm}p{2.5cm}p{3.5cm}@{}}
\toprule
\textbf{Niveau} & \textbf{Places} & \textbf{Durée} & \textbf{Rotation} & \textbf{Capacité/an} \\
\midrule
1 — Lits & 6 chambres & 6–8 h & 1 patient/jour & $\sim$1\,120 séances \\
2 — Fauteuils & 6 fauteuils & 3–4 h & 2 patients/jour & $\sim$2\,110 séances \\
3 — Lounge & 5 fauteuils & 2–3 h & 2–3 patients/jour & $\sim$1\,650 séances \\
\midrule
\textbf{Total} & \textbf{17 places} & — & — & \textbf{$\sim$4\,900 séances} \\
\bottomrule
\end{tabular}
\caption{Synthèse des trois niveaux de prise en charge (base 220 jours/an, coefficient 0.85)}
\label{tab:niveaux_synthese}
\end{table}

% ============================================================
% 3.4.1 NIVEAU 1 — LITS MÉDICALISÉS
% ============================================================
\subsubsection*{Niveau 1 — Lits médicalisés en chambre individuelle}

\paragraph{Principe —}
Le niveau 1 accueille les patients nécessitant une surveillance médicale continue et/ou des actes invasifs à risque. Chaque patient occupe une \textbf{chambre individuelle} pour la journée entière, garantissant intimité, repos et conditions optimales de surveillance.

\paragraph{Indications —}
\begin{itemize}[leftmargin=1.1cm]
    \item biopsies hépatiques transpariétales (repos strict 6h post-procédure) ;
    \item ponctions d'ascite grand volume ($>$5 litres, compensation albumine) ;
    \item cirrhoses décompensées (encéphalopathie, ascite réfractaire) ;
    \item chimiothérapies prolongées ou à risque (surveillance continue) ;
    \item surveillance post-radiologie interventionnelle (embolisation, drainage).
\end{itemize}

\paragraph{Organisation —}
\begin{itemize}[leftmargin=1.1cm]
    \item \textbf{Capacité} : 6 chambres individuelles équipées (scope, oxygène, aspiration).
    \item \textbf{Durée de séjour} : 6 à 8 heures (arrivée 8h00, sortie 16h00–18h00).
    \item \textbf{Rotation} : 1 patient par lit par jour — pas de double rotation possible.
    \item \textbf{Ratio IDE} : 1 IDE pour 3 patients (surveillance renforcée).
    \item \textbf{Présence médicale} : médecin senior disponible sur le plateau.
\end{itemize}

\paragraph{Capacité annuelle —}
Avec 6 lits × 220 jours × 1 patient/jour × coefficient 0.85 = \textbf{1\,122 séances/an}.

% ============================================================
% 3.4.2 NIVEAU 2 — FAUTEUILS DE SOINS
% ============================================================
\subsubsection*{Niveau 2 — Fauteuils de soins en box individuel}

\paragraph{Principe —}
Le niveau 2 accueille les patients nécessitant des soins techniques avec surveillance infirmière directe, mais sans nécessité de décubitus prolongé. Les fauteuils sont installés dans des \textbf{boxes individuels ou semi-cloisonnés}, préservant l'intimité tout en permettant une surveillance visuelle.

\paragraph{Indications —}
\begin{itemize}[leftmargin=1.1cm]
    \item perfusions d'albumine (cirrhose, syndrome néphrotique) ;
    \item fer injectable IV (anémie ferriprive, MICI) ;
    \item biothérapies IV (infliximab, vedolizumab, ustekinumab) ;
    \item chimiothérapies digestives simples (durée $<$4h) ;
    \item transfusions programmées ;
    \item traitements IV divers (antibiotiques, immunoglobulines).
\end{itemize}

\paragraph{Organisation —}
\begin{itemize}[leftmargin=1.1cm]
    \item \textbf{Capacité} : 6 fauteuils inclinables en boxes individuels.
    \item \textbf{Durée de séjour} : 3 à 4 heures par passage.
    \item \textbf{Rotation} : 2 patients par fauteuil par jour (matin + après-midi).
    \item \textbf{Ratio IDE} : 1 IDE pour 4 patients.
    \item \textbf{Présence médicale} : médecin joignable, passage systématique si besoin.
\end{itemize}

\paragraph{Capacité annuelle —}
Avec 6 fauteuils × 220 jours × 2 patients/jour × coefficient 0.80 = \textbf{2\,112 séances/an}.

% ============================================================
% 3.4.3 NIVEAU 3 — ESPACE LOUNGE
% ============================================================
\subsubsection*{Niveau 3 — Espace lounge multidisciplinaire}

\paragraph{Principe —}
Le niveau 3 constitue un dispositif ambulatoire innovant, dédié aux parcours à faible intensité de surveillance mais à forte composante évaluative et multidisciplinaire. Les patients sont installés sur des \textbf{fauteuils confortables non allongés}, dans un espace aménagé avec séparations visuelles, autour desquels gravitent les différents professionnels.

Il ne s'agit pas d'un espace de soins continus, mais d'un lieu d'ancrage du patient permettant une coordination efficiente des interventions successives.

\paragraph{Indications —}
\begin{itemize}[leftmargin=1.1cm]
    \item bilans hépatométaboliques (MASLD/MASH, stéatose) ;
    \item évaluations cirrhose compensée (élastographie, bilan nutritionnel) ;
    \item consultations addictologie intégrées (sevrage, suivi, soutien psychologique) ;
    \item MICI stables (biothérapies SC, éducation thérapeutique) ;
    \item évaluations pré-thérapeutiques (avant traitement oral, inclusion essai).
\end{itemize}

\paragraph{Organisation —}
\begin{itemize}[leftmargin=1.1cm]
    \item \textbf{Capacité} : 5 fauteuils lounge avec séparations.
    \item \textbf{Durée de séjour} : 2 à 3 heures par passage.
    \item \textbf{Rotation cible} : 2 à 3 patients par fauteuil par jour.
    \item \textbf{Intervenants} : médecins, IPA, diététiciens, psychologues, addictologues (en rotation).
    \item \textbf{Ratio IDE} : 1 IDE pour 5 patients (surveillance légère).
\end{itemize}

\paragraph{Intervenants et logique de rotation —}
Durant son séjour, le patient reçoit la visite séquentielle (non simultanée) de plusieurs professionnels :
\begin{itemize}[leftmargin=1.1cm]
    \item médecin spécialiste (hépatologie, gastroentérologie, addictologie) ;
    \item diététicien(ne) ou nutritionniste ;
    \item psychologue ou addictologue ;
    \item IPA pour coordination et éducation thérapeutique ;
    \item examens complémentaires si besoin (élastographie, prélèvements).
\end{itemize}

Le fauteuil lounge constitue le point fixe du parcours ; les compétences gravitent autour du patient.

\paragraph{Capacité annuelle —}
Hypothèse conservatrice (2 rotations) : 5 fauteuils × 220 jours × 2 × 0.75 = \textbf{1\,650 séances/an}. \\
Hypothèse optimiste (3 rotations) : 5 fauteuils × 220 jours × 3 × 0.75 = \textbf{2\,475 séances/an}.

\clearpage

% ============================================================
% 3.5 SYSTÈME DE VASES COMMUNICANTS
% ============================================================
\subsection{Flexibilité capacitaire — Système de vases communicants —}
\label{sec:vases_communicants}

L'organisation du plateau repose sur un principe fondamental de \textbf{flexibilité inter-filières} : les places non utilisées par une spécialité sont immédiatement réallouées aux autres filières selon un système de vases communicants.

\paragraph{Principe —}
Chaque filière dispose de créneaux réservés (cf. planning hebdomadaire), mais les places non programmées ou libérées (annulation, sortie anticipée) sont mises à disposition d'un pool commun, accessible à l'ensemble des spécialités contributrices.

\paragraph{Mécanisme d'allocation —}
La gestion des places repose sur un système de priorisation en trois temps :

\begin{enumerate}[leftmargin=1.1cm]
    \item \textbf{J-7 à J-3} — Créneaux réservés par filière selon le planning type. Chaque référent médical confirme ses patients programmés.
    
    \item \textbf{J-2 à J-1} — Les places non confirmées basculent dans le \textbf{pool mutualisé}. L'IPA coordinateur les propose aux autres filières selon une liste d'attente partagée.
    
    \item \textbf{Jour J} — Toute place libérée (annulation tardive, sortie précoce) est immédiatement signalée et réattribuée en temps réel par l'IDE coordinateur.
\end{enumerate}

\paragraph{Critères de priorisation du pool —}
En cas de demandes concurrentes, l'allocation respecte les priorités suivantes :
\begin{enumerate}[leftmargin=1.1cm]
    \item urgences médicales relatives (décompensation, infection, anémie symptomatique) ;
    \item patients en attente depuis plus de 7 jours ;
    \item optimisation du taux d'occupation (remplissage des créneaux matin/après-midi) ;
    \item équité entre filières sur la période glissante (suivi mensuel).
\end{enumerate}

\paragraph{Outils de gestion —}
\begin{itemize}[leftmargin=1.1cm]
    \item tableau de programmation partagé (Orbis ou outil dédié) visible par tous les référents ;
    \item liste d'attente mutualisée avec date de demande et niveau de priorité ;
    \item alerte automatique en cas de place libérée (notification IPA/IDE coordinateur) ;
    \item reporting hebdomadaire du taux d'occupation par filière et par niveau.
\end{itemize}

\paragraph{Bénéfices attendus —}
\begin{itemize}[leftmargin=1.1cm]
    \item optimisation du taux d'occupation global ($>$85\% cible) ;
    \item réduction des délais de programmation pour toutes les filières ;
    \item équité d'accès entre spécialités contributrices ;
    \item absorption des variations saisonnières d'activité.
\end{itemize}

\clearpage

% ============================================================
% 3.6 PLANNING HEBDOMADAIRE TYPE
% ============================================================
\subsection{Planning hebdomadaire type —}
L'organisation des lits médicalisés (niveau 1) repose sur une programmation hebdomadaire structurée. Chaque filière dispose de créneaux réservés ; les places non confirmées à J-2 basculent dans le pool mutualisé.

\vspace{0.4cm}

\begin{table}[!ht]
\centering
\small
\renewcommand{\arraystretch}{1.4}
\rowcolors{2}{APHPsoft}{white}
\begin{tabular}{@{}p{1.8cm}p{5.5cm}p{1.8cm}p{1.8cm}@{}}
\toprule
\textbf{Jour} & \textbf{Activités programmées} & \textbf{Réservé} & \textbf{Pool} \\
\midrule
Lundi & PBH (4-5) + Ascite grand volume (1-2) & 5 lits & 1 lit \\
Mardi & Radiologie interventionnelle (2-3) + Chimio lourde (1) & 4 lits & 2 lits \\
Mercredi & MICI immunothérapies IV (4-5) + Chimio lourde (1) & 5 lits & 1 lit \\
Jeudi & PBH (4-5) + Ascite grand volume (1-2) & 5 lits & 1 lit \\
Vendredi & RI (2-3) + MICI + Chimio lourde (2) & 5 lits & 1 lit \\
\bottomrule
\end{tabular}
\caption{Planning hebdomadaire — Lits médicalisés (niveau 1) : créneaux réservés et pool mutualisé}
\label{tab:planning_hebdo_lits}
\end{table}

\vspace{0.5cm}

\begin{table}[!ht]
\centering
\small
\renewcommand{\arraystretch}{1.4}
\rowcolors{2}{APHPsoft}{white}
\begin{tabular}{@{}p{1.8cm}p{5.5cm}p{5.5cm}@{}}
\toprule
\textbf{Jour} & \textbf{Matin (8h–12h)} & \textbf{Après-midi (13h–17h)} \\
\midrule
Lundi & Fer IV, Albumine & Biothérapies MICI \\
Mardi & Biothérapies MICI & Chimio simples \\
Mercredi & Fer IV, Transfusions & Albumine, Biothérapies \\
Jeudi & Biothérapies MICI & Fer IV, Divers \\
Vendredi & Chimio simples & Biothérapies, Rattrapage \\
\bottomrule
\end{tabular}
\caption{Planning hebdomadaire — Fauteuils de soins (niveau 2)}
\label{tab:planning_hebdo_fauteuils}
\end{table}

\vspace{0.4cm}

\noindent L'espace lounge (niveau 3) fonctionne quotidiennement sans contrainte de jour, avec programmation adaptée aux disponibilités des intervenants (médecins, diététiciens, psychologues).

\clearpage
% ============================================================
% 3.7 ZONES FONCTIONNELLES
% ============================================================
\subsection{Zones fonctionnelles —}
La traduction spatiale de l'organisation repose sur des zones clairement identifiées :

\begin{itemize}[leftmargin=1.1cm]
    \item \textbf{Zone d'accueil} — admission centralisée, attente confortable, orientation.
    \item \textbf{Zone niveau 1} — 6 chambres individuelles équipées (lits médicalisés).
    \item \textbf{Zone niveau 2} — 6 boxes individuels avec fauteuils inclinables.
    \item \textbf{Zone niveau 3} — espace lounge aménagé (5 fauteuils, séparations).
    \item \textbf{Zone consultations} — bureaux intégrés (diététique, psychologie, addictologie).
    \item \textbf{Zone technique} — échographie, élastographie, préparation traitements.
    \item \textbf{Zone logistique} — pharmacie, stockage, circuits propres/sales.
\end{itemize}

\clearpage

% ============================================================
% 3.8 MUTUALISATION DES RESSOURCES HUMAINES
% ============================================================
\subsection{Ressources humaines mutualisées —}
Le fonctionnement du plateau repose sur une équipe dédiée, polyvalente et formée à l'ensemble des niveaux de prise en charge.

\begin{table}[!ht]
\centering
\small
\rowcolors{2}{APHPsoft}{white}
\begin{tabular}{@{}p{4.5cm}p{2.5cm}p{2.5cm}p{3.5cm}@{}}
\toprule
\textbf{Catégorie} & \textbf{Phase pilote} & \textbf{Maturité} & \textbf{Missions principales} \\
\midrule
IDE expertes & 4–5 ETP & 6–7 ETP & Soins, surveillance, coordination \\
IPA & 1 ETP & 1.5 ETP & Coordination, évaluation, ETP \\
Cadre de santé & 0.5 ETP & 1 ETP & Management, organisation \\
Secrétariat médical & 0.5 ETP & 1 ETP & Programmation, accueil, DPI \\
Psychologue & 0.3 ETP & 0.5 ETP & Soutien, addictologie \\
Diététicien(ne) & 0.3 ETP & 0.5 ETP & Bilans nutritionnels, ETP \\
\midrule
\textbf{Total PNM} & \textbf{$\sim$7 ETP} & \textbf{$\sim$11.5 ETP} & — \\
\bottomrule
\end{tabular}
\caption{Effectifs non médicaux (PNM) — Phase pilote et maturité}
\label{tab:effectifs_pnm}
\end{table}

\begin{table}[!ht]
\centering
\small
\rowcolors{2}{APHPsoft}{white}
\begin{tabular}{@{}p{5cm}p{3cm}p{3cm}@{}}
\toprule
\textbf{Spécialité médicale} & \textbf{Phase pilote} & \textbf{Maturité} \\
\midrule
Hépatologie (PBH, cirrhose) & 2 demi-j/sem & 4 demi-j/sem \\
MICI (biothérapies) & 2 demi-j/sem & 3 demi-j/sem \\
Addictologie & 1 demi-j/sem & 2 demi-j/sem \\
Oncologie digestive & 1 demi-j/sem & 2 demi-j/sem \\
Radiologie interventionnelle & — & 1 demi-j/sem \\
\midrule
\textbf{Total vacations/semaine} & \textbf{6 demi-j} & \textbf{12 demi-j} \\
\bottomrule
\end{tabular}
\caption{Vacations médicales hebdomadaires (PM)}
\label{tab:effectifs_pm}
\end{table}

\clearpage

% ============================================================
% 3.9 SCHÉMA FONCTIONNEL
% ============================================================
\subsection{Schéma fonctionnel du plateau —}

\begin{figure}[!ht]
\centering
\vspace{0.5cm}
\begin{tikzpicture}[
    node distance=1.2cm,
    box/.style={
        rectangle,
        rounded corners=3pt,
        draw=APHPdark,
        thick,
        text width=5.2cm,
        minimum height=1.1cm,
        align=center,
        fill=APHPsoft
    },
    levelbox/.style={
        rectangle,
        rounded corners=3pt,
        draw=APHPblue,
        thick,
        text width=3.2cm,
        minimum height=0.9cm,
        align=center,
        fill=white
    },
    arrow/.style={->, thick, APHPdark}
]

\node[box] (orientation) {Orientation \\ Consultation / Ville / Urgences};
\node[box, below=of orientation] (classif) {Classification patient \\ Niveau 1, 2 ou 3};
\node[box, below=of classif] (accueil) {Accueil centralisé \\ Admission unique};

\node[levelbox, below left=1.5cm and -0.5cm of accueil] (n1) {Niveau 1 \\ 6 chambres};
\node[levelbox, below=1.5cm of accueil] (n2) {Niveau 2 \\ 6 fauteuils};
\node[levelbox, below right=1.5cm and -0.5cm of accueil] (n3) {Niveau 3 \\ 5 lounge};

\node[box, below=3.5cm of accueil] (sortie) {Sortie sécurisée \\ Synthèse + Suivi};

\node[box, right=2.5cm of classif] (pool) {Pool mutualisé \\ Vases communicants};

\draw[arrow] (orientation) -- (classif);
\draw[arrow] (classif) -- (accueil);
\draw[arrow] (accueil) -- (n1);
\draw[arrow] (accueil) -- (n2);
\draw[arrow] (accueil) -- (n3);
\draw[arrow] (n1) -- (sortie);
\draw[arrow] (n2) -- (sortie);
\draw[arrow] (n3) -- (sortie);
\draw[arrow, dashed] (pool) -- (accueil);

\end{tikzpicture}
\caption{Schéma fonctionnel du plateau HDJ avec classification par niveau et pool mutualisé}
\end{figure}

\clearpage

% ============================================================
% 3.10 GOUVERNANCE
% ============================================================
\subsection{Gouvernance —}
Le plateau est placé sous une responsabilité médicale unique. La gouvernance s'appuie sur deux instances :

\paragraph{Comité de pilotage stratégique (COPIL) — Trimestriel}
\begin{itemize}[leftmargin=1.1cm]
    \item \textbf{Composition} : responsable médical, référents filières, cadre supérieur, direction des soins, direction médico-économique.
    \item \textbf{Missions} : orientations stratégiques, arbitrages capacitaires, validation clé de redistribution.
\end{itemize}

\paragraph{Comité opérationnel (CODIR) — Mensuel}
\begin{itemize}[leftmargin=1.1cm]
    \item \textbf{Composition} : responsable médical, cadre de santé, IPA, secrétariat, pharmacie.
    \item \textbf{Missions} : suivi opérationnel, gestion des flux, résolution des dysfonctionnements.
\end{itemize}

\paragraph{Tableau de bord —}
\begin{table}[!ht]
\centering
\small
\rowcolors{2}{APHPsoft}{white}
\begin{tabular}{@{}p{3cm}p{7cm}p{2.5cm}@{}}
\toprule
\textbf{Domaine} & \textbf{Indicateurs} & \textbf{Fréquence} \\
\midrule
Activité & Séances/filière/niveau, taux occupation, délais, annulations & Mensuel \\
Médico-économique & Recettes/filière, écart budget, coût/séance & Mensuel \\
Qualité & Complications, réhospitalisations, satisfaction, EI & Trimestriel \\
RH & Absentéisme, vacances postes, heures sup., formations & Trimestriel \\
\bottomrule
\end{tabular}
\caption{Indicateurs de pilotage}
\label{tab:kpi}
\end{table}

% ============================================================
% 3.11 MODÈLE ÉCONOMIQUE UF
% ============================================================
\subsection{Modèle économique de l'unité fonctionnelle —}
\label{sec:modele_economique_uf}

Le plateau HDJ est constitué en \textbf{unité fonctionnelle (UF) distincte}, permettant une traçabilité des recettes et une redistribution équitable vers les UF cliniques contributrices.

\paragraph{Centralisation —}
L'ensemble des recettes T2A (GHS, suppléments, molécules onéreuses) est imputé à l'UF HDJ Digestif Mutualisé.

\paragraph{Redistribution —}
Les recettes sont redistribuées selon un mécanisme mixte validé annuellement en COPIL :
\begin{itemize}[leftmargin=1.1cm]
    \item \textbf{Base forfaitaire (60\%)} : enveloppe fixe par UF clinique ;
    \item \textbf{Part variable (40\%)} : au prorata de l'activité réalisée.
\end{itemize}

\begin{table}[!ht]
\centering
\small
\rowcolors{2}{APHPsoft}{white}
\begin{tabular}{@{}p{5cm}p{4.5cm}p{2.5cm}@{}}
\toprule
\textbf{UF bénéficiaire} & \textbf{Filières} & \textbf{\% indicatif} \\
\midrule
Maladies du Foie & PBH, cirrhose, hépatométab. & 35–40\% \\
Gastroentérologie & MICI, oncologie digestive & 35–40\% \\
Addictologie & HDJ addictologie & 10–15\% \\
Radiologie interventionnelle & Post-procédures RI & 10–15\% \\
\bottomrule
\end{tabular}
\caption{Redistribution indicative des recettes}
\label{tab:redistribution}
\end{table}

\end{spacing}
\clearpage