% ============================================================
% 3. PLATEAU HDJ DIGESTIF MUTUALISÉ — ORGANISATION TRANSVERSALE
% Version 3 — Intégrant les recommandations d'audit
% ============================================================

\titleformat{\subsection}[runin]
  {\bfseries\color{APHPblue}}
  {}
  {0pt}
  {}

\label{sec:plateau_hdj_mutualise}

\begin{spacing}{1.25}

% ============================================================
% 3.1 PRINCIPES D'ORGANISATION
% ============================================================

\subsection{Principes d'organisation}

Le plateau d'HDJ digestif mutualisé repose sur une organisation transversale visant à optimiser l'efficience médico-soignante, la lisibilité des parcours et l'utilisation des ressources humaines et matérielles.  
Il constitue un dispositif unique regroupant, sur un même site, l'ensemble des filières ambulatoires digestives et interventionnelles, selon des processus harmonisés et sécurisés.

Le modèle repose sur une \textbf{organisation graduée des prises en charge ambulatoires}, permettant d'adapter les moyens mobilisés à l'intensité réelle des besoins cliniques, tout en garantissant un haut niveau de sécurité.

Les principes structurants sont les suivants :
\begin{itemize}[leftmargin=1.1cm]
    \item mutualisation des espaces, des équipes soignantes et des circuits logistiques ;
    \item standardisation des parcours d'entrée, de prise en charge et de sortie ;
    \item distinction fonctionnelle des zones selon l'intensité de surveillance requise ;
    \item allocation graduée des ressources en fonction du risque et de la durée de prise en charge.
\end{itemize}

% ============================================================
% 3.2 PARCOURS ET FLUX PATIENTS
% ============================================================

\subsection{Parcours et flux patients}

Les patients suivent un parcours organisationnel commun, modulé selon la filière et la complexité clinique, garantissant à la fois fluidité, sécurité et lisibilité.

\begin{itemize}[leftmargin=1.1cm]
    \item \textbf{Orientation} : consultation spécialisée, RCP, urgences, ville ou hospitalisation complète.
    \item \textbf{Programmation} : affectation à un niveau d'HDJ adapté selon critères d'éligibilité validés.
    \item \textbf{Accueil centralisé} : admission administrative et paramédicale unique, vérification des prérequis.
    \item \textbf{Prise en charge} : actes médicaux, techniques ou multidisciplinaires selon protocoles harmonisés.
    \item \textbf{Surveillance post-acte} : proportionnée au niveau de risque, avec critères de sortie explicites.
    \item \textbf{Sortie sécurisée} : synthèse médicale, prescriptions, organisation du suivi et traçabilité complète.
\end{itemize}

Cette structuration garantit une gestion efficiente des flux tout en sécurisant les parcours complexes.

\textbf{Note opérationnelle} : Les délais de programmation et les taux de remplissage feront l'objet d'un suivi mensuel dans le cadre du pilotage médico-économique (cf. section 3.7).

% ============================================================
% 3.3 NIVEAUX DE PRISE EN CHARGE AMBULATOIRE
% ============================================================

\subsection{Niveaux de prise en charge ambulatoire}

Le plateau HDJ est structuré autour de \textbf{trois niveaux fonctionnels}, définis selon l'intensité de surveillance requise :

\begin{itemize}[leftmargin=1.1cm]
    \item \textbf{Niveau~1 — Lits médicalisés} :  
    patients nécessitant une surveillance continue ou des actes à risque élevé (cirrhoses avancées, procédures interventionnelles, chimiothérapies prolongées).  
    \emph{Productivité cible} : 1 patient/lit/jour, coefficient de réalisation 0,85.
    
    \item \textbf{Niveau~2 — Fauteuils de soins} :  
    prises en charge intermédiaires avec surveillance infirmière directe (albumine, fer IV, biothérapies, chimiothérapies simples).  
    \emph{Productivité cible} : 2 patients/fauteuil/jour, coefficient de réalisation 0,80.
    
    \item \textbf{Niveau~3 — Espace non allongé de type \emph{lounge}} :  
    patients autonomes ne nécessitant ni lit ni surveillance continue, mais requérant une évaluation médicale et paramédicale structurée et multidisciplinaire (hépatométabolique, cirrhose compensée, addictologie, MICI sélectionnées).  
    \emph{Productivité cible} : 1 à 2 rotations/fauteuil/jour (cf. paragraphe détaillé ci-après).
\end{itemize}

\paragraph{Fonctionnement opérationnel de l'espace \emph{lounge}}

L'espace \emph{lounge} constitue un dispositif ambulatoire spécifique, centré sur la gestion de parcours cliniques à faible intensité de surveillance mais à forte composante évaluative et multidisciplinaire. Il ne s'agit pas d'un espace de soins continus, mais d'un lieu d'ancrage du patient au sein du plateau, autour duquel s'organisent des interventions médicales et paramédicales séquentielles.

Les patients y sont installés sur des fauteuils non allongés pour des durées variables, généralement \textbf{de 180 minutes} (soit environ 3 heures), intégrant des temps d'attente clinique active, de coordination et d'évaluation. Durant ce temps, les différents professionnels interviennent de manière non simultanée, selon une logique de rotation autour du patient.

Les intervenants concernés incluent notamment :
\begin{itemize}[leftmargin=1.1cm]
    \item médecins spécialistes (hépatologie, MICI, addictologie) ;
    \item diététicien(ne)s et nutritionnistes ;
    \item psychologues et addictologues ;
    \item consultants spécialisés, le cas échéant ;
    \item interface avec le plateau médico-technique, notamment radiologique.
\end{itemize}

Le fauteuil \emph{lounge} constitue ainsi le point fixe du parcours ambulatoire, tandis que les compétences médicales et paramédicales gravitent autour du patient, permettant une optimisation des ressources sans mobilisation prolongée de lits ou de fauteuils de soins.

Cette organisation induit une variabilité intrinsèque des durées de séjour et des temps d'occupation des fauteuils \emph{lounge}. En pratique, le fonctionnement courant repose sur \textbf{une à deux rotations quotidiennes par fauteuil}, avec des possibilités ponctuelles de troisième rotation sur des parcours très courts, sans que celle-ci constitue une cible soutenable à l'échelle annuelle. 

\textbf{Hypothèse conservatrice retenue pour le dimensionnement} : Les projections capacitaires et budgétaires s'appuient sur une hypothèse prudente de \textbf{1,5 rotation moyenne par fauteuil et par jour ouvré}, correspondant à une productivité intermédiaire entre le scénario minimal (1 rotation) et le scénario optimiste (2 rotations). Cette hypothèse conservatrice permet de limiter le risque de surestimation de la capacité théorique maximale et d'intégrer une marge organisationnelle nécessaire à la gestion des aléas (absences, retards, complexité imprévue des parcours).

Ces éléments constituent le cadre de référence organisationnel retenu pour le dimensionnement capacitaire et les hypothèses de productivité présentées dans les sections ultérieures.

% ============================================================
% 3.4 ZONES FONCTIONNELLES
% ============================================================

\subsection{Zones fonctionnelles}

La traduction spatiale de cette organisation repose sur des zones fonctionnelles clairement identifiées :

\begin{itemize}[leftmargin=1.1cm]
    \item zone d'accueil et d'admission centralisée ;
    \item zone de lits HDJ pour surveillance continue (6 lits prévus) ;
    \item zone de fauteuils de soins (dimensionnement selon activité prévisionnelle) ;
    \item zone non allongée dédiée aux parcours multidisciplinaires (5 fauteuils \emph{lounge} prévus) ;
    \item zones de consultations intégrées (diététique, psychologie/addictologie) ;
    \item zone technique mutualisée (échographie, élastographie, préparation des traitements) ;
    \item zone logistique (pharmacie, stockage, circuits propres et sales) ;
    \item bureau de coordination et d'encadrement soignant.
\end{itemize}

\textbf{Surface estimée nécessaire} : environ 350-400 m² pour un fonctionnement optimal incluant circulations, zones de stockage et locaux techniques annexes. Ce dimensionnement fait l'objet d'une étude architecturale complémentaire en lien avec la Direction des Travaux et de la Logistique.

% ============================================================
% 3.5 ORGANISATION OPÉRATIONNELLE HEBDOMADAIRE
% ============================================================

\subsection{Organisation opérationnelle hebdomadaire}

Le plateau HDJ fonctionne selon une organisation hebdomadaire structurée permettant d'optimiser l'utilisation des ressources tout en garantissant la disponibilité quotidienne pour les activités prioritaires.

\paragraph{Principes d'organisation}

\begin{itemize}[leftmargin=1.1cm]
    \item \textbf{Lits médicalisés} : affectés par filière selon programmation hebdomadaire, avec réserve de capacité pour urgences relatives.
    \item \textbf{Fauteuils de soins} : utilisés quotidiennement pour activités programmées et semi-urgentes.
    \item \textbf{Espace lounge} : fonctionnement continu sans programmation rigide, capacité flexible selon affluence.
\end{itemize}

\paragraph{Planning type hebdomadaire — Lits médicalisés}

\begin{table}[!ht]
\centering
\small
\begin{tabular}{|l|p{10cm}|}
\hline
\textbf{Jour} & \textbf{Activités programmées sur lits médicalisés} \\
\hline
Lundi & \textbf{Biopsies hépatiques percutanées} (4-5 patients) + \textbf{Ponctions d'ascite volumineuses} (1-2 patients) \\
\hline
Mardi & \textbf{Immunothérapies MICI} (4-6 patients nécessitant surveillance prolongée) + Chimio digestive complexe \\
\hline
Mercredi & \textbf{Cirrhoses avancées} (bilan multidisciplinaire, ponctions, albumine) + Gestes interventionnels \\
\hline
Jeudi & \textbf{Biopsies hépatiques percutanées} (4-5 patients) + \textbf{Ponctions d'ascite volumineuses} (1-2 patients) \\
\hline
Vendredi & \textbf{Immunothérapies MICI} (4-6 patients) + Chimio digestive + Radiologie interventionnelle \\
\hline
\end{tabular}
\caption{Répartition hebdomadaire des activités sur lits médicalisés (6 lits disponibles)}
\end{table}

\textbf{Note importante} : Cette répartition constitue un cadre d'organisation et non une contrainte rigide. Des ajustements quotidiens sont possibles selon les besoins cliniques et les disponibilités.

\paragraph{Activités quotidiennes — Fauteuils de soins et gestes interventionnels}

Certaines activités ne sont pas limitées à des jours spécifiques et sont réalisées quotidiennement selon les besoins :

\begin{itemize}[leftmargin=1.1cm]
    \item \textbf{Chimiothérapies digestives} : programmables tous les jours (majoritairement mardi-mercredi-vendredi) ;
    \item \textbf{Gestes de radiologie interventionnelle} : créneaux quotidiens dédiés en coordination avec le plateau de RI ;
    \item \textbf{Perfusions de fer IV, albumine, immunoglobulines} : tous les jours selon disponibilité des fauteuils ;
    \item \textbf{Biothérapies MICI courtes} (< 2h de surveillance) : tous les jours sur fauteuils de soins.
\end{itemize}

\paragraph{Espace lounge — Fonctionnement continu et flexible}

L'espace lounge fonctionne \textbf{quotidiennement sans limitation de jour}, avec une capacité d'absorption variable selon l'affluence. Les 5 fauteuils permettent d'accueillir :

\begin{itemize}[leftmargin=1.1cm]
    \item \textbf{Addictologie} : consultations de sevrage, évaluations multidisciplinaires, groupes thérapeutiques ;
    \item \textbf{MASLD/NASH} : bilans métaboliques complets, éducation thérapeutique, coordination diététique ;
    \item \textbf{Cirrhoses compensées} : bilans de surveillance, élastographie, échographie, dépistage CHC ;
    \item \textbf{MICI (parcours courts)} : consultations infirmières d'éducation, auto-injection, suivi biothérapies SC ;
    \item \textbf{Hépatites virales} : bilans pré-thérapeutiques, suivis de traitement, consultations d'observance ;
    \item \textbf{Explorations diverses} : FibroScan®, élastographie, échographies de dépistage.
\end{itemize}

\textbf{Principe clé} : Le lounge n'est \textbf{pas soumis à une programmation rigide par jour de semaine}. Il absorbe quotidiennement les parcours multidisciplinaires ne nécessitant pas de lit ou de fauteuil de soins. La capacité réelle dépend de la durée moyenne des parcours et du nombre de professionnels mobilisables.

\textbf{Capacité estimée} : 
\begin{itemize}[leftmargin=1.1cm]
    \item Scénario conservateur : 1 rotation/fauteuil/jour = 5 patients/jour = 1 250 séances/an (250 jours ouvrés)
    \item Scénario intermédiaire : 1,5 rotation/fauteuil/jour = 7-8 patients/jour = 1 875 séances/an
    \item Scénario optimiste (non soutenable en continu) : 2 rotations/fauteuil/jour = 10 patients/jour = 2 500 séances/an
\end{itemize}

\textbf{Hypothèse retenue pour dimensionnement} : 1,5 rotation moyenne = environ 1 800-2 000 séances lounge par an.

\paragraph{Coordination quotidienne et régulation des flux}

La coordination quotidienne est assurée par :
\begin{itemize}[leftmargin=1.1cm]
    \item \textbf{Briefing matinal} (8h30) : IDE référente du jour + IPA + médecin présent → revue de la programmation, ajustements ;
    \item \textbf{Régulation en temps réel} : IPA coordinatrice disponible toute la journée pour arbitrages et réaffectations ;
    \item \textbf{Débriefing de fin de journée} (17h) : bilan activité, événements indésirables, préparation J+1.
\end{itemize}

% ============================================================
% 3.6 MUTUALISATION DES RESSOURCES
% ============================================================

\subsection{Mutualisation des ressources humaines et des locaux}

Le fonctionnement du plateau repose sur une mutualisation complète des compétences et des espaces :

\begin{itemize}[leftmargin=1.1cm]
    \item IDE expertes formées à l'ensemble des niveaux de prise en charge ;
    \item IPA assurant coordination, évaluation et suivi protocolisé ;
    \item médecins seniors intervenant par vacations ciblées ;
    \item psychologues, diététicien(ne)s et addictologues intégrés au plateau ;
    \item locaux partagés permettant une flexibilité maximale des capacités.
\end{itemize}

\paragraph{Dimensionnement RH prévisionnel (à affiner)}

\textbf{Phase de montée en charge (Année 1)} :
\begin{itemize}[leftmargin=1.1cm]
    \item 4 à 5 ETP IDE expertes (présence quotidienne 8h-18h) ;
    \item 1 ETP IPA de coordination (8h30-17h30, régulation des flux) ;
    \item 0,5 ETP secrétariat médical dédié ;
    \item Vacations médicales seniors : 
    \begin{itemize}
        \item 2 demi-journées/semaine hépatologie (PBH, cirrhose)
        \item 2 demi-journées/semaine MICI
        \item 1 demi-journée/semaine addictologie
        \item 1 demi-journée/semaine oncologie digestive
    \end{itemize}
    \item 0,3 ETP psychologue ;
    \item 0,3 ETP diététicien(ne).
\end{itemize}

\textbf{À maturité (Année 3)} :
\begin{itemize}[leftmargin=1.1cm]
    \item 6 à 7 ETP IDE expertes (dont 1 référente lits, 1 référente fauteuils/lounge) ;
    \item 1,5 ETP IPA (1 coordination + 0,5 évaluation clinique lounge) ;
    \item 1 ETP secrétariat médical ;
    \item Vacations médicales seniors :
    \begin{itemize}
        \item 4 demi-journées/semaine hépatologie
        \item 3 demi-journées/semaine MICI
        \item 2 demi-journées/semaine addictologie
        \item 2 demi-journées/semaine oncologie digestive
        \item 1 demi-journée/semaine RI (coordination gestes)
    \end{itemize}
    \item 0,5 ETP psychologue ;
    \item 0,5 ETP diététicien(ne) ;
    \item 1 ETP cadre de santé dédié.
\end{itemize}

\textbf{Note} : Ces projections nécessitent une validation par le Département des Ressources Humaines et feront l'objet d'un ajustement progressif selon l'activité réelle observée. Un plan de recrutement et de formation sera établi au moins 6 mois avant l'ouverture du plateau.

\clearpage

% ============================================================
% 3.7 SCHÉMA FONCTIONNEL UNIQUE DU PLATEAU HDJ
% ============================================================

\subsection{Schéma fonctionnel unique du plateau HDJ digestif mutualisé}

\begin{figure}[!ht]
\centering
\vspace{0.6cm}

\begin{tikzpicture}[
    node distance=1.4cm,
    box/.style={
        rectangle,
        rounded corners=3pt,
        draw=APHPdark,
        thick,
        text width=5.4cm,
        minimum height=1.2cm,
        align=center,
        fill=APHPsoft
    },
    arrow/.style={->, thick, APHPdark}
]

\node[box] (orientation) {Orientation \\ Consultation / Ville / Urgences / MCO};
\node[box, below=of orientation] (accueil) {Accueil HDJ centralisé \\ Admission IDE + vérification prérequis};
\node[box, below=of accueil] (soins) {Zones de soins mutualisées \\ Lits (6) / Fauteuils / Lounge (5)};
\node[box, below=of soins] (surv) {Surveillance post-acte \\ IDE / IPA selon protocoles};
\node[box, below=of surv] (sortie) {Synthèse médicale \\ Sortie sécurisée + traçabilité};

\node[box, right=2.8cm of soins] (ressources) {Ressources mutualisées \\ IDE expertes (6-7 ETP) \\ IPA (1,5 ETP) \\ Locaux \\ Plateaux techniques};

\draw[arrow] (orientation) -- (accueil);
\draw[arrow] (accueil) -- (soins);
\draw[arrow] (soins) -- (surv);
\draw[arrow] (surv) -- (sortie);
\draw[arrow] (ressources) -- (soins);

\end{tikzpicture}

\caption{Organisation fonctionnelle graduée du plateau HDJ digestif mutualisé intégrant lits,
fauteuils et espace \emph{lounge} — Dimensionnement prévisionnel à maturité}
\end{figure}

\end{spacing}

\clearpage

% ============================================================
% 3.8 GOUVERNANCE DU PLATEAU HDJ DIGESTIF MUTUALISÉ
% ============================================================

\subsection{Gouvernance du plateau HDJ digestif mutualisé}

Le plateau mutualisé d'hôpitaux de jour digestifs est placé sous une responsabilité médicale unique, garante de la cohérence des parcours, de la sécurité des prises en charge et de l'harmonisation des pratiques entre filières.

L'organisation repose sur une articulation formalisée entre le DMU Hématologie Cancérologie et Spécialités Médico-Chirurgicales, la radiologie interventionnelle et les plateaux médico-techniques associés.  
L'encadrement soignant transversal assure la coordination opérationnelle, la gestion des flux et la continuité des soins.

Le pilotage médico-économique s'appuie sur un suivi régulier de l'activité, des indicateurs de qualité et de sécurité, en lien avec le DIM et les directions fonctionnelles, permettant une adaptation continue du dispositif.

\paragraph{Instances de pilotage}

\textbf{Comité de pilotage stratégique} (trimestriel) :
\begin{itemize}[leftmargin=1.1cm]
    \item Responsable médical du plateau ;
    \item Référents médicaux par filière (hépatologie, MICI, oncologie, addictologie, radiologie interventionnelle) ;
    \item Cadre supérieur de santé du DMU ;
    \item Représentant de la Direction des Soins ;
    \item Représentant de la Direction Médico-Économique ;
    \item Contrôleur de gestion.
\end{itemize}

\textbf{Comité opérationnel} (mensuel) :
\begin{itemize}[leftmargin=1.1cm]
    \item Responsable médical du plateau ou délégué ;
    \item Cadre de santé dédié au plateau HDJ ;
    \item IPA de coordination ;
    \item Secrétariat médical ;
    \item Représentant pharmacie ;
    \item Représentant logistique.
\end{itemize}

\paragraph{Tableaux de bord et indicateurs de suivi}

\textbf{Indicateurs d'activité} (suivi mensuel) :
\begin{itemize}[leftmargin=1.1cm]
    \item Nombre de séances par filière et par niveau (lits / fauteuils / lounge) ;
    \item Taux d'occupation par type de place ;
    \item Délai moyen de programmation par filière ;
    \item File active par filière ;
    \item Nombre de séances annulées ou reportées (avec causes).
\end{itemize}

\textbf{Indicateurs médico-économiques} (suivi mensuel) :
\begin{itemize}[leftmargin=1.1cm]
    \item Recettes par filière (GHS + suppléments + molécules onéreuses) ;
    \item Écart budget/réalisé (cumul glissant sur 12 mois) ;
    \item Coût par séance (masse salariale + consommables + médicaments) ;
    \item Seuil de rentabilité atteint / non atteint par filière.
\end{itemize}

\textbf{Indicateurs qualité et sécurité} (suivi trimestriel) :
\begin{itemize}[leftmargin=1.1cm]
    \item Taux de complications par filière ;
    \item Taux de réhospitalisation à 48h et à 30 jours ;
    \item Satisfaction patient (enquête semestrielle) ;
    \item Événements indésirables déclarés ;
    \item Conformité des parcours aux protocoles (audit interne semestriel).
\end{itemize}

\textbf{Indicateurs RH} (suivi trimestriel) :
\begin{itemize}[leftmargin=1.1cm]
    \item Taux d'absentéisme du personnel dédié ;
    \item Taux de vacances de postes ;
    \item Nombre d'heures supplémentaires ;
    \item Formations suivies par catégorie de personnel.
\end{itemize}

\paragraph{Calendrier de montée en charge}

\textbf{Phase pilote (Mois 1 à 6)} :
\begin{itemize}[leftmargin=1.1cm]
    \item Objectif : 40-50\% de la capacité théorique ;
    \item Focus sur 3 à 4 filières prioritaires (biopsies hépatiques, MICI, cirrhose avancée) ;
    \item Recrutement et formation des équipes ;
    \item Rodage des circuits et des protocoles ;
    \item Évaluation mensuelle et ajustements rapides.
\end{itemize}

\textbf{Phase de consolidation (Mois 7 à 18)} :
\begin{itemize}[leftmargin=1.1cm]
    \item Objectif : 60-70\% de la capacité théorique ;
    \item Déploiement progressif des filières restantes (addictologie, chimiothérapie, radiologie interventionnelle) ;
    \item Stabilisation des équipes et des processus ;
    \item Mise en place des tableaux de bord consolidés.
\end{itemize}

\textbf{Phase de maturité (Mois 19 à 36)} :
\begin{itemize}[leftmargin=1.1cm]
    \item Objectif : 80-90\% de la capacité théorique (4 000 à 4 500 séances/an) ;
    \item Optimisation continue des processus ;
    \item Évaluation médico-économique complète ;
    \item Bilan à 3 ans et perspectives d'évolution.
\end{itemize}

\textbf{Note importante} : La capacité théorique maximale de 4 700-5 000 séances/an correspond à un fonctionnement à pleine charge sans marge organisationnelle. L'objectif réaliste à 3 ans est de stabiliser l'activité entre 4 000 et 4 500 séances/an, garantissant une qualité de prise en charge optimale et une soutenabilité pour les équipes.

\paragraph{Procédures de coordination inter-services}

Des conventions formalisées seront établies avec :
\begin{itemize}[leftmargin=1.1cm]
    \item Service de radiologie interventionnelle (créneaux dédiés, circuits patient) ;
    \item Pharmacie (approvisionnement, préparation des traitements, gestion des stupéfiants) ;
    \item Plateau d'endoscopie (coordination des parcours, gestion des urgences) ;
    \item Laboratoire de biologie médicale (circuits prélèvements, délais de résultats) ;
    \item Urgences et réanimation (protocoles de transfert en cas de complication).
\end{itemize}

Ces conventions préciseront les modalités pratiques de coordination, les circuits d'information, les référents identifiés et les procédures d'escalade en cas de difficulté.

\end{spacing}